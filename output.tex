% LTeX: enabled=false
\documentclass[a4paper,11pt,oneside]{article}

\usepackage{placeins}

\makeatletter
\AtBeginDocument{%
  % Modifying section command to include FloatBarrier
  \expandafter\renewcommand\expandafter\section\expandafter
    {\expandafter\@fb@secFB\section}%
  % Modifying subsection command to include FloatBarrier
  \expandafter\renewcommand\expandafter\subsection\expandafter
    {\expandafter\@fb@secFB\subsection}%
  
  \newcommand\@fb@secFB{\FloatBarrier
    \gdef\@fb@afterHHook{\@fb@topbarrier \gdef\@fb@afterHHook{}}}%
  
  \g@addto@macro\@afterheading{\@fb@afterHHook}%
  \gdef\@fb@afterHHook{}%
}
\makeatother

\usepackage[english]{babel}

\usepackage[T1]{fontenc}

\usepackage[utf8]{inputenc}

\usepackage{graphicx}
\usepackage{cite}
\usepackage{url}
\usepackage{ifthen}
\usepackage{listings}
\usepackage{xcolor}
\usepackage{blindtext} % https://tex.stackexchange.com/questions/111948/what-is-a-overfull-hbox-9-89561pt-too-wide

\usepackage{helvet} % Add sans-serif font package
\renewcommand{\familydefault}{\sfdefault} % Set sans-serif as default

\graphicspath{{images/}}

\usepackage[hidelinks]{hyperref}
\usepackage{url}



\def \lstlistingname {SQL}
\lstset{
  language=SQL,
  tabsize=2,
  numbers=left,
  frame=L,
  floatplacement=hbtp,
  basicstyle=\ttfamily\small,
  keywordstyle=\color{blue},
  stringstyle=\color{red},
  commentstyle=\color{gray},
  captionpos=b, % This sets the caption below the listing
  showstringspaces=false,
  literate={'}{{\textquotesingle}}1
}

% LTeX: enabled=true

\title{IDSV - Old Exam Questions, 2021-present}

\author{\textit{Compiled by Edwin Sundberg}}
  % {\small\textcolor{darkgray}{\href{mailto:edwinsu@dsv.su.se}{edwinsu@dsv.su.se}}} \\

\begin{document}

\maketitle \pagebreak

% https://tex.stackexchange.com/questions/111948/what-is-a-overfull-hbox-9-89561pt-too-wide
\begin{sloppypar}  

\tableofcontents \pagebreak

\section{Preface}
\label{preface}
This document contains old exam questions and answers for the course \textit{IDSV - Introduction to Computer and Systems Sciences} at Stockholm University. The questions and answers are collected from the exams held in the years after 2021. They are intended to serve as a study aid and reference for students preparing for their exams.


\textbf{Note}: This document contains more questions than what is presented in a single exam, see the course syllabus for more information about your specific exam.

\textsuperscript

\subsection{Generated Notes}
\label{generatedNotes}

% <<<TEMPLATEVAR_GENERATED_NOTES>>>


\section{Svenska utan Svar}
\label{svenskaUtanSvar}

\subsection{Ge ett exempel f\"or varje kategori fr\r{a}n maskininstruktionerna ovan (The Machine{\textquoteright}s Language)!}

\label{q:1:sa:sv:False}

\vspace{2cm}

\noindent\makebox[\textwidth]{\hrulefill}

\vspace{1cm}

\textit{Svar}: \autoref{q:1:sa:sv:True}



\subsection{Vilken datalagringsteknik anv\"andes f\"or f\"orsta g\r{a}ngen 1801 av Joseph Jacquard?}

\label{q:3:sa:sv:False}

\vspace{2cm}

\noindent\makebox[\textwidth]{\hrulefill}

\vspace{1cm}

\textit{Svar}: \autoref{q:3:sa:sv:True}



\subsection{I vilket specialregister finns minnesadressen till n\"asta instruktion?}

\label{q:4:sa:sv:False}

\vspace{2cm}

\noindent\makebox[\textwidth]{\hrulefill}

\vspace{1cm}

\textit{Svar}: \autoref{q:4:sa:sv:True}



\subsection{I vilket specialregister finns den maskinkodsinstruktion som skall utf\"oras?}

\label{q:5:sa:sv:False}

\vspace{2cm}

\noindent\makebox[\textwidth]{\hrulefill}

\vspace{1cm}

\textit{Svar}: \autoref{q:5:sa:sv:True}



\subsection{Vilken processorarkitektur har f\r{a}, enkla och snabba maskininstruktioner?}

\label{q:6:sa:sv:False}

\vspace{2cm}

\noindent\makebox[\textwidth]{\hrulefill}

\vspace{1cm}

\textit{Svar}: \autoref{q:6:sa:sv:True}



\subsection{Det finns en s\"arskild typ av maskininstruktion som beh\"ovs f\"or att kunna koordinera olika processers tillg\r{a}ng till gemensamma resurser, vad kallas den?}

\label{q:7:sa:sv:False}

\vspace{2cm}

\noindent\makebox[\textwidth]{\hrulefill}

\vspace{1cm}

\textit{Svar}: \autoref{q:7:sa:sv:True}



\subsection{Vad kallas den del av operativsystemet som uppr\"atth\r{a}ller en processtabell?}

\label{q:8:sa:sv:False}

\vspace{2cm}

\noindent\makebox[\textwidth]{\hrulefill}

\vspace{1cm}

\textit{Svar}: \autoref{q:8:sa:sv:True}



\subsection{Vad kallas det n\"ar en enskild anv\"andare i ett enanv\"andarsystem kan exekvera flera program {\textquotedblright}samtidigt{\textquotedblright}?}

\label{q:10:sa:sv:False}

\vspace{2cm}

\noindent\makebox[\textwidth]{\hrulefill}

\vspace{1cm}

\textit{Svar}: \autoref{q:10:sa:sv:True}



\subsection{Vilket av f\"oljande alternativ \"ar inte en del av operativsystemet?}

\label{q:11:sa:sv:False}

\vspace{2cm}

\noindent\makebox[\textwidth]{\hrulefill}

\vspace{1cm}

\textit{Svar}: \autoref{q:11:sa:sv:True}



\subsection{Vad kallas en flagga som styr \r{a}tkomsten till en kritisk region (critical region) f\"or att garantera att inte flera processer kommer \r{a}t den kritiska regionen samtidigt (mutual exclusion)?}

\label{q:12:sa:sv:False}

\vspace{2cm}

\noindent\makebox[\textwidth]{\hrulefill}

\vspace{1cm}

\textit{Svar}: \autoref{q:12:sa:sv:True}



\subsection{Vad kallas den del av operativsystemet som hanterar data som ligger lagrat som namngivna enheter (named separate groups of data) p\r{a} sekund\"arminne?}

\label{q:13:sa:sv:False}

\vspace{2cm}

\noindent\makebox[\textwidth]{\hrulefill}

\vspace{1cm}

\textit{Svar}: \autoref{q:13:sa:sv:True}



\subsection{En dator kan simulera att den har mer prim\"arminne \"an sitt faktiska fysiska prim\"arminnet. Vad kallas detta simulerade minne?}

\label{q:14:sa:sv:False}

\vspace{2cm}

\noindent\makebox[\textwidth]{\hrulefill}

\vspace{1cm}

\textit{Svar}: \autoref{q:14:sa:sv:True}



\subsection{Vad kallas den s\"arskilda process som beh\"ovs f\"or att starta en dator?}

\label{q:15:sa:sv:False}

\vspace{2cm}

\noindent\makebox[\textwidth]{\hrulefill}

\vspace{1cm}

\textit{Svar}: \autoref{q:15:sa:sv:True}



\subsection{Vad kallas den del av operativsystemet som allokerar (allocates) och avallokerar (deallocates) prim\"arminne (main memory) till olika processer?}

\label{q:16:sa:sv:False}

\vspace{2cm}

\noindent\makebox[\textwidth]{\hrulefill}

\vspace{1cm}

\textit{Svar}: \autoref{q:16:sa:sv:True}



\subsection{Vad kallas den del av operativsystemet som tilldelar processortid (time slices) till olika processer?}

\label{q:17:sa:sv:False}

\vspace{2cm}

\noindent\makebox[\textwidth]{\hrulefill}

\vspace{1cm}

\textit{Svar}: \autoref{q:17:sa:sv:True}



\subsection{Vad heter protokollet som anv\"ands av applikationen World Wide Web?}

\label{q:18:sa:sv:False}

\vspace{2cm}

\noindent\makebox[\textwidth]{\hrulefill}

\vspace{1cm}

\textit{Svar}: \autoref{q:18:sa:sv:True}



\subsection{Vilket Internet-mjukvarulager (Internet software layer) tillh\"or protokollet UDP (user datagram protocol)?}

\label{q:19:sa:sv:False}

\vspace{2cm}

\noindent\makebox[\textwidth]{\hrulefill}

\vspace{1cm}

\textit{Svar}: \autoref{q:19:sa:sv:True}



\subsection{Vilket Internet-mjukvarulager (Internet software layer) tillh\"or protokollet FTP (file transfer protocol)?}

\label{q:20:sa:sv:False}

\vspace{2cm}

\noindent\makebox[\textwidth]{\hrulefill}

\vspace{1cm}

\textit{Svar}: \autoref{q:20:sa:sv:True}



\subsection{Vad \"ar ett LAN?}

\label{q:21:sa:sv:False}

\vspace{2cm}

\noindent\makebox[\textwidth]{\hrulefill}

\vspace{1cm}

\textit{Svar}: \autoref{q:21:sa:sv:True}



\subsection{Vilket Internet-mjukvarulager (Internet software layer) tillh\"or protokollet TCP?}

\label{q:22:sa:sv:False}

\vspace{2cm}

\noindent\makebox[\textwidth]{\hrulefill}

\vspace{1cm}

\textit{Svar}: \autoref{q:22:sa:sv:True}



\subsection{Vad heter den organisation som ansvarar f\"or tilldelning av IP-nummer (det r\"acker med f\"orkortningen)?}

\label{q:23:sa:sv:False}

\vspace{2cm}

\noindent\makebox[\textwidth]{\hrulefill}

\vspace{1cm}

\textit{Svar}: \autoref{q:23:sa:sv:True}



\subsection{Till vilket Internet-mjukvarulager (Internet software layer) h\"or protokollet IPv6?}

\label{q:24:sa:sv:False}

\vspace{2cm}

\noindent\makebox[\textwidth]{\hrulefill}

\vspace{1cm}

\textit{Svar}: \autoref{q:24:sa:sv:True}



\subsection{Vad kallas det n\"ar en webbklient st\"aller en fr\r{a}ga till en s\"arskild typ av server f\"or att \"overs\"atta ett dom\"annamn till ett IP-nummer?}

\label{q:25:sa:sv:False}

\vspace{2cm}

\noindent\makebox[\textwidth]{\hrulefill}

\vspace{1cm}

\textit{Svar}: \autoref{q:25:sa:sv:True}



\subsection{Vilket Internet-protokoll f\"or transport-lagret \"ar mest tillf\"orlitligt?}

\label{q:26:sa:sv:False}

\vspace{2cm}

\noindent\makebox[\textwidth]{\hrulefill}

\vspace{1cm}

\textit{Svar}: \autoref{q:26:sa:sv:True}



\subsection{Vad kallas det spr\r{a}k som man beskriver webbsidor med?}

\label{q:27:sa:sv:False}

\vspace{2cm}

\noindent\makebox[\textwidth]{\hrulefill}

\vspace{1cm}

\textit{Svar}: \autoref{q:27:sa:sv:True}



\subsection{F\"or att skydda en dator eller ett n\"atverk av datorer anv\"ands ofta ett system, best\r{a}ende av programvara och eventuellt \"aven h\r{a}rdvara, som kan inspektera, blockera och filtrera inkommande och utg\r{a}ende n\"atverkstrafik. Vad kallas ett s\r{a}dant system?}

\label{q:28:sa:sv:False}

\vspace{2cm}

\noindent\makebox[\textwidth]{\hrulefill}

\vspace{1cm}

\textit{Svar}: \autoref{q:28:sa:sv:True}



\subsection{F\"or att b\"attre kunna skydda ett n\"atverk av datorer anv\"ands ofta en mellanliggande dator som g\"or att kommunikation inte g\r{a}r direkt mellan en klient p\r{a} n\"atverket och en extern server. Vad kallas en s\r{a}dan dator?}

\label{q:29:sa:sv:False}

\vspace{2cm}

\noindent\makebox[\textwidth]{\hrulefill}

\vspace{1cm}

\textit{Svar}: \autoref{q:29:sa:sv:True}



\subsection{Vad kallas s\"attet att \r{a}stadkomma repetition i kod som kr\"aver mer utrymme i prim\"arminnet f\"or varje repetition?}

\label{q:31:sa:sv:False}

\vspace{2cm}

\noindent\makebox[\textwidth]{\hrulefill}

\vspace{1cm}

\textit{Svar}: \autoref{q:31:sa:sv:True}



\subsection{Vad kallas s\"attet att \r{a}stadkomma repetition i kod som inte kr\"aver mer utrymme i prim\"arminnet f\"or varje repetition?}

\label{q:32:sa:sv:False}

\vspace{2cm}

\noindent\makebox[\textwidth]{\hrulefill}

\vspace{1cm}

\textit{Svar}: \autoref{q:32:sa:sv:True}



\subsection{Vilken \"ar den vanligaste metoden f\"or att verifiera att ett program fungerar korrekt?}

\label{q:33:sa:sv:False}

\vspace{2cm}

\noindent\makebox[\textwidth]{\hrulefill}

\vspace{1cm}

\textit{Svar}: \autoref{q:33:sa:sv:True}



\subsection{Vad kallas den grundl\"aggande byggstenen i imperativa programmeringsspr\r{a}k?}

\label{q:34:sa:sv:False}

\vspace{2cm}

\noindent\makebox[\textwidth]{\hrulefill}

\vspace{1cm}

\textit{Svar}: \autoref{q:34:sa:sv:True}



\subsection{Vad kallas den logiska h\"arledningsteknik som anv\"ands i logikprogrammering?}

\label{q:35:sa:sv:False}

\vspace{2cm}

\noindent\makebox[\textwidth]{\hrulefill}

\vspace{1cm}

\textit{Svar}: \autoref{q:35:sa:sv:True}



\subsection{I objektorienterad programmering, vad kallas mallarna fr\r{a}n vilka objekt skapas?}

\label{q:36:sa:sv:False}

\vspace{2cm}

\noindent\makebox[\textwidth]{\hrulefill}

\vspace{1cm}

\textit{Svar}: \autoref{q:36:sa:sv:True}



\subsection{Vad kallas det programmeringsparadigm d\"ar man beskriver vad som skall utf\"oras ist\"allet f\"or hur det skall utf\"oras?}

\label{q:37:sa:sv:False}

\vspace{2cm}

\noindent\makebox[\textwidth]{\hrulefill}

\vspace{1cm}

\textit{Svar}: \autoref{q:37:sa:sv:True}



\subsection{Vad kallas ett program som \"overs\"atter k\"allkod till maskinkod?}

\label{q:38:sa:sv:False}

\vspace{2cm}

\noindent\makebox[\textwidth]{\hrulefill}

\vspace{1cm}

\textit{Svar}: \autoref{q:38:sa:sv:True}



\subsection{Vad kallas den typ av programmering som besvarar fr\r{a}gor huruvida ett faktum \"ar h\"arledbart fr\r{a}n ett program eller inte?}

\label{q:39:sa:sv:False}

\vspace{2cm}

\noindent\makebox[\textwidth]{\hrulefill}

\vspace{1cm}

\textit{Svar}: \autoref{q:39:sa:sv:True}



\subsection{Ge ett exempel p\r{a} en l\"attr\"orlig utvecklingsmodell (agile development model)?}

\label{q:40:sa:sv:False}

\vspace{2cm}

\noindent\makebox[\textwidth]{\hrulefill}

\vspace{1cm}

\textit{Svar}: \autoref{q:40:sa:sv:True}



\subsection{Vad kallas den roll i Scrum som uppr\"atth\r{a}ller en lista med krav och prioriterar mellan dessa krav?}

\label{q:41:sa:sv:False}

\vspace{2cm}

\noindent\makebox[\textwidth]{\hrulefill}

\vspace{1cm}

\textit{Svar}: \autoref{q:41:sa:sv:True}



\subsection{Vad kallas de korta iterationer (2 {\textendash} 4 veckor) i Scrum, som skall resultera i n\r{a}gonting levererbart till kund/best\"allare?}

\label{q:42:sa:sv:False}

\vspace{2cm}

\noindent\makebox[\textwidth]{\hrulefill}

\vspace{1cm}

\textit{Svar}: \autoref{q:42:sa:sv:True}



\subsection{Vad kallas den roll i Scrum som skall s\"akerst\"alla att Scrum-ramverket f\"oljs?}

\label{q:43:sa:sv:False}

\vspace{2cm}

\noindent\makebox[\textwidth]{\hrulefill}

\vspace{1cm}

\textit{Svar}: \autoref{q:43:sa:sv:True}



\subsection{Vad kallas i Scrum de korta dagliga m\"oten d\r{a} varje projektdeltagare skall svara p\r{a} tre fr\r{a}gor?}

\label{q:44:sa:sv:False}

\vspace{2cm}

\noindent\makebox[\textwidth]{\hrulefill}

\vspace{1cm}

\textit{Svar}: \autoref{q:44:sa:sv:True}



\subsection{Vad kallas i Scrum de m\"oten d\r{a} man diskuterar vad som har g\r{a}tt bra denna iteration och vad som kan f\"orb\"attras i n\"asta iteration?}

\label{q:45:sa:sv:False}

\vspace{2cm}

\noindent\makebox[\textwidth]{\hrulefill}

\vspace{1cm}

\textit{Svar}: \autoref{q:45:sa:sv:True}



\subsection{Vad kallas rollen i ett team som \"ar ansvarig f\"or att team:et f\"oljer Scrum-metodiken?}

\label{q:46:sa:sv:False}

\vspace{2cm}

\noindent\makebox[\textwidth]{\hrulefill}

\vspace{1cm}

\textit{Svar}: \autoref{q:46:sa:sv:True}



\subsection{Vad st\r{a}r f\"orkortningen CASE f\"or avseende programvaruutveckling?}

\label{q:47:sa:sv:False}

\vspace{2cm}

\noindent\makebox[\textwidth]{\hrulefill}

\vspace{1cm}

\textit{Svar}: \autoref{q:47:sa:sv:True}



\subsection{Vad st\r{a}r f\"orkortningen IDE f\"or avseende programvaruutveckling?}

\label{q:48:sa:sv:False}

\vspace{2cm}

\noindent\makebox[\textwidth]{\hrulefill}

\vspace{1cm}

\textit{Svar}: \autoref{q:48:sa:sv:True}



\subsection{Vad heter den roll i Scum som \"ar ansvarig f\"or att prioritera vilken utveckling som skall utf\"oras under n\"asta sprint?}

\label{q:49:sa:sv:False}

\vspace{2cm}

\noindent\makebox[\textwidth]{\hrulefill}

\vspace{1cm}

\textit{Svar}: \autoref{q:49:sa:sv:True}



\subsection{Vad kallas Scrum-m\"otet, i slutet av en sprint d\"ar det avslutade arbetet med sprinten utv\"arderas med avseende p\r{a} sprintm\r{a}len?}

\label{q:50:sa:sv:False}

\vspace{2cm}

\noindent\makebox[\textwidth]{\hrulefill}

\vspace{1cm}

\textit{Svar}: \autoref{q:50:sa:sv:True}



\subsection{Vad kallas den grundl\"aggande datastruktur som best\r{a}r av ett block av dataelement av samma datatyp och storlek, och d\"ar varje dataelement direkt n\r{a}s via ett index?}

\label{q:51:sa:sv:False}

\vspace{2cm}

\noindent\makebox[\textwidth]{\hrulefill}

\vspace{1cm}

\textit{Svar}: \autoref{q:51:sa:sv:True}



\subsection{Vad kallas den grundl\"aggande datastruktur som best\r{a}r av ett block av dataelement av vanligtvis olika datatyper och storlek, och d\"ar de enskilda dataelementen n\r{a}s via namn?}

\label{q:52:sa:sv:False}

\vspace{2cm}

\noindent\makebox[\textwidth]{\hrulefill}

\vspace{1cm}

\textit{Svar}: \autoref{q:52:sa:sv:True}



\subsection{Vad kallas en variabel som inneh\r{a}ller en minnesadress ist\"allet f\"or data (anv\"ands i dynamiska datastrukturer)?}

\label{q:53:sa:sv:False}

\vspace{2cm}

\noindent\makebox[\textwidth]{\hrulefill}

\vspace{1cm}

\textit{Svar}: \autoref{q:53:sa:sv:True}



\subsection{Vad heter det dominerande fr\r{a}gespr\r{a}ket som anv\"ands f\"or att h\"amta data fr\r{a}n och manipulera data i en databas?}

\label{q:54:sa:sv:False}

\vspace{2cm}

\noindent\makebox[\textwidth]{\hrulefill}

\vspace{1cm}

\textit{Svar}: \autoref{q:54:sa:sv:True}



\subsection{Vad kallas i databassammanhang, en sekvens av operationer som paketeras ihop och d\"ar antingen alla operationer lyckas (utf\"ors) eller alla misslyckas (ingen utf\"ors) (all operations together either succeed or fail)?}

\label{q:55:sa:sv:False}

\vspace{2cm}

\noindent\makebox[\textwidth]{\hrulefill}

\vspace{1cm}

\textit{Svar}: \autoref{q:55:sa:sv:True}



\subsection{Vad kallas den typ av data mining som har gjort webbshopen Amazon s\r{a} framg\r{a}ngsrik?}

\label{q:56:sa:sv:False}

\vspace{2cm}

\noindent\makebox[\textwidth]{\hrulefill}

\vspace{1cm}

\textit{Svar}: \autoref{q:56:sa:sv:True}



\subsection{Vad kallas den typ av analys inom data-mining, som f\"ors\"oker uppt\"acka klasser genom att gruppera objekt i ett antal separata grupper (i motsats till klassbeskrivning, som f\"ors\"oker uppt\"acka egenskaper hos medlemmar inom klasser som redan \"ar identifierade)?}

\label{q:57:sa:sv:False}

\vspace{2cm}

\noindent\makebox[\textwidth]{\hrulefill}

\vspace{1cm}

\textit{Svar}: \autoref{q:57:sa:sv:True}



\subsection{Vad kallas den typ av analys inom data-mining, som f\"ors\"oker identifiera beteendem\"onster \"over tid, till exempel trender i ekonomiska system som aktiemarknader eller i milj\"osystem som klimatf\"orh\r{a}llanden?}

\label{q:58:sa:sv:False}

\vspace{2cm}

\noindent\makebox[\textwidth]{\hrulefill}

\vspace{1cm}

\textit{Svar}: \autoref{q:58:sa:sv:True}



\subsection{Vid rendrering s\r{a} skall en trediminsionell modell \"overf\"oras till en platt yta. Vad kallas denna platta yta?}

\label{q:59:sa:sv:False}

\vspace{2cm}

\noindent\makebox[\textwidth]{\hrulefill}

\vspace{1cm}

\textit{Svar}: \autoref{q:59:sa:sv:True}



\subsection{Vid rendering av 3D-grafik s\r{a} skall en tredimensionell modell \"overf\"oras till en platt yta, vad kallas denna platta yta?}

\label{q:60:sa:sv:False}

\vspace{2cm}

\noindent\makebox[\textwidth]{\hrulefill}

\vspace{1cm}

\textit{Svar}: \autoref{q:60:sa:sv:True}



\subsection{Vad kallas det n\"ar man till\"ampar fysikens lagar f\"or att best\"amma objekts positioner, t.ex. biljardbollars positioner efter en biljardst\"ot?}

\label{q:61:sa:sv:False}

\vspace{2cm}

\noindent\makebox[\textwidth]{\hrulefill}

\vspace{1cm}

\textit{Svar}: \autoref{q:61:sa:sv:True}



\subsection{Vad kallas den del av maskininl\"arning (machine learning) d\"ar en m\"anniska beskriver det korrekta svaret f\"or ett antal exempel och agenten (maskininl\"arningsalgoritmen) generaliserar utifr\r{a}n dessa exempel?}

\label{q:62:sa:sv:False}

\vspace{2cm}

\noindent\makebox[\textwidth]{\hrulefill}

\vspace{1cm}

\textit{Svar}: \autoref{q:62:sa:sv:True}



\subsection{Ge ett exempel p\r{a} en icke-ber\"akningsbar funktion?}

\label{q:63:sa:sv:False}

\vspace{2cm}

\noindent\makebox[\textwidth]{\hrulefill}

\vspace{1cm}

\textit{Svar}: \autoref{q:63:sa:sv:True}



\subsection{Vad kallas den maskin som utg\"or den enklast t\"ankbara modellen av en dator?}

\label{q:64:sa:sv:False}

\vspace{2cm}

\noindent\makebox[\textwidth]{\hrulefill}

\vspace{1cm}

\textit{Svar}: \autoref{q:64:sa:sv:True}



\subsection{En ljudfil i CD-kvalitet inneb\"ar en samplingsfrekvens (sampling frequency) om 44100 per sekund, och ett samplingsdjup (sampling depth) om 16 bitar per kanal. Hur stor plats i kilobyte (kB) tar en okomprimerad ljudfil i stereo (2 kanaler) i CD-kvalitet med en l\"angd p\r{a} 3 minuter?}

\label{q:65:sa:sv:False}

\vspace{2cm}

\noindent\makebox[\textwidth]{\hrulefill}

\vspace{1cm}

\textit{Svar}: \autoref{q:65:sa:sv:True}



\subsection{Antag att vi tidigare har lagrat digitala bilder med f\"argdjupet 12 bitar per pixel (color depth 12 bits per pixel). Om vi nu vill kunna representera h\"alften s\r{a} m\r{a}nga olika f\"arger j\"amf\"ort med tidigare, vilket f\"argdjup skall vi anv\"anda d\r{a}?}

\label{q:66:sa:sv:False}

\vspace{2cm}

\noindent\makebox[\textwidth]{\hrulefill}

\vspace{1cm}

\textit{Svar}: \autoref{q:66:sa:sv:True}



\subsection{Antag att vi tidigare har lagrat digitala bilder med f\"argdjupet 12 bitar per pixel (color depth 12 bits per pixel). Om vi nu vill kunna representera dubbelt s\r{a} m\r{a}nga olika f\"arger j\"amf\"ort med tidigare, vilket f\"argdjup skall vi anv\"anda d\r{a}?}

\label{q:67:sa:sv:False}

\vspace{2cm}

\noindent\makebox[\textwidth]{\hrulefill}

\vspace{1cm}

\textit{Svar}: \autoref{q:67:sa:sv:True}



\subsection{Vad \"ar en teckenkodning (character encoding)?}

\label{q:68:sa:sv:False}

\vspace{2cm}

\noindent\makebox[\textwidth]{\hrulefill}

\vspace{1cm}

\textit{Svar}: \autoref{q:68:sa:sv:True}



\subsection{Om 6A38 \"ar den hexadecimala notationen f\"or ett bitm\"onster som representerar en ljud-sample (one sound sample), vad har denna ljud-sample f\"or samplingsdjup (sampling depth)?}

\label{q:71:sa:sv:False}

\vspace{2cm}

\noindent\makebox[\textwidth]{\hrulefill}

\vspace{1cm}

\textit{Svar}: \autoref{q:71:sa:sv:True}



\subsection{Om 6A36B3 \"ar den hexadecimala notationen f\"or ett bitm\"onster som representerar en RGB-kodad pixel, vad har denna pixel f\"or f\"argdjup (colour depth)?}

\label{q:72:sa:sv:False}

\vspace{2cm}

\noindent\makebox[\textwidth]{\hrulefill}

\vspace{1cm}

\textit{Svar}: \autoref{q:72:sa:sv:True}



\subsection{Vad \"ar f\"argdjup (color depth) i samband med lagring av bilder?}

\label{q:73:sa:sv:False}

\vspace{2cm}

\noindent\makebox[\textwidth]{\hrulefill}

\vspace{1cm}

\textit{Svar}: \autoref{q:73:sa:sv:True}



\subsection{Vilket decimaltal (basen 10) motsvarar det hexadecimala talet 15?}

\label{q:74:sa:sv:False}

\vspace{2cm}

\noindent\makebox[\textwidth]{\hrulefill}

\vspace{1cm}

\textit{Svar}: \autoref{q:74:sa:sv:True}



\subsection{Antag att vi tidigare har lagrat digitala bilder med f\"argdjupet 8 bitar per pixel (color depth 8 bits per pixel). Om vi nu vill kunna representera dubbelt s\r{a} m\r{a}nga olika f\"arger j\"amf\"ort med tidigare, vilket f\"argdjup skall vi anv\"anda d\r{a}?}

\label{q:76:sa:sv:False}

\vspace{2cm}

\noindent\makebox[\textwidth]{\hrulefill}

\vspace{1cm}

\textit{Svar}: \autoref{q:76:sa:sv:True}



\subsection{Vad \"ar samplingsfrekvens (sample rate) i samband med digital lagring av ljud?}

\label{q:77:sa:sv:False}

\vspace{2cm}

\noindent\makebox[\textwidth]{\hrulefill}

\vspace{1cm}

\textit{Svar}: \autoref{q:77:sa:sv:True}



\subsection{Vilket decimaltal (basen 10) motsvarar det hexadecimala talet 3F?}

\label{q:78:sa:sv:False}

\vspace{2cm}

\noindent\makebox[\textwidth]{\hrulefill}

\vspace{1cm}

\textit{Svar}: \autoref{q:78:sa:sv:True}



\subsection{Antag att vi tidigare har lagrat digitala bilder med f\"argdjupet 8 bitar per pixel (color depth 8 bits per pixel). Om vi nu vill kunna representera h\"alften s\r{a} m\r{a}nga olika f\"arger j\"amf\"ort med tidigare, vilket f\"argdjup skall vi anv\"anda d\r{a}?}

\label{q:79:sa:sv:False}

\vspace{2cm}

\noindent\makebox[\textwidth]{\hrulefill}

\vspace{1cm}

\textit{Svar}: \autoref{q:79:sa:sv:True}



\subsection{Hur m\r{a}nga bitar (f\"argdjup) beh\"ovs f\"or att representera 16 olika f\"arger?}

\label{q:80:sa:sv:False}

\vspace{2cm}

\noindent\makebox[\textwidth]{\hrulefill}

\vspace{1cm}

\textit{Svar}: \autoref{q:80:sa:sv:True}



\subsection{Vad \"ar en ljudfils samplingsdjup (sample depth)?}

\label{q:81:sa:sv:False}

\vspace{2cm}

\noindent\makebox[\textwidth]{\hrulefill}

\vspace{1cm}

\textit{Svar}: \autoref{q:81:sa:sv:True}



\subsection{Vad \"ar en ljudfils samplingsfrekvens (sample rate)?}

\label{q:82:sa:sv:False}

\vspace{2cm}

\noindent\makebox[\textwidth]{\hrulefill}

\vspace{1cm}

\textit{Svar}: \autoref{q:82:sa:sv:True}



\subsection{Vad anger en ljudfils samplingsfrekvens?}

\label{q:83:sa:sv:False}

\vspace{2cm}

\noindent\makebox[\textwidth]{\hrulefill}

\vspace{1cm}

\textit{Svar}: \autoref{q:83:sa:sv:True}



\subsection{Antag att vi tidigare har lagrat digitala bilder med f\"argdjupet 5 bitar per pixel (color depth 5 bits per pixel). Om vi nu vill kunna representera dubbelt s\r{a} m\r{a}nga olika f\"arger j\"amf\"ort med tidigare, vilket f\"argdjup skall vi anv\"anda d\r{a}?}

\label{q:84:sa:sv:False}

\vspace{2cm}

\noindent\makebox[\textwidth]{\hrulefill}

\vspace{1cm}

\textit{Svar}: \autoref{q:84:sa:sv:True}



\subsection{Hur m\r{a}nga bitar (f\"argdjup) beh\"ovs f\"or att representera 24 olika f\"arger?}

\label{q:85:sa:sv:False}

\vspace{2cm}

\noindent\makebox[\textwidth]{\hrulefill}

\vspace{1cm}

\textit{Svar}: \autoref{q:85:sa:sv:True}



\subsection{Hur m\r{a}nga bitar (f\"argdjup) beh\"ovs f\"or att representera 12 olika f\"arger?}

\label{q:86:sa:sv:False}

\vspace{2cm}

\noindent\makebox[\textwidth]{\hrulefill}

\vspace{1cm}

\textit{Svar}: \autoref{q:86:sa:sv:True}



\subsection{Hur m\r{a}nga bitar kr\"avs det f\"or att representera ett boolskt v\"arde?}

\label{q:87:sa:sv:False}

\vspace{2cm}

\noindent\makebox[\textwidth]{\hrulefill}

\vspace{1cm}

\textit{Svar}: \autoref{q:87:sa:sv:True}



\subsection{Ge exempel p\r{a} tv\r{a} olika logiska operationer som kan utf\"oras p\r{a} boolska v\"arden?}

\label{q:88:sa:sv:False}

\vspace{2cm}

\noindent\makebox[\textwidth]{\hrulefill}

\vspace{1cm}

\textit{Svar}: \autoref{q:88:sa:sv:True}



\subsection{Hur m\r{a}nga bitar (f\"argdjup) beh\"ovs f\"or att representera 9 olika f\"arger?}

\label{q:89:sa:sv:False}

\vspace{2cm}

\noindent\makebox[\textwidth]{\hrulefill}

\vspace{1cm}

\textit{Svar}: \autoref{q:89:sa:sv:True}



\subsection{Hur m\r{a}nga bitar (f\"argdjup) beh\"ovs f\"or att representera 15 olika f\"arger?}

\label{q:90:sa:sv:False}

\vspace{2cm}

\noindent\makebox[\textwidth]{\hrulefill}

\vspace{1cm}

\textit{Svar}: \autoref{q:90:sa:sv:True}



\subsection{Vad \"ar en byte?}

\label{q:91:sa:sv:False}

\vspace{2cm}

\noindent\makebox[\textwidth]{\hrulefill}

\vspace{1cm}

\textit{Svar}: \autoref{q:91:sa:sv:True}



\subsection{Vad kallas 8 bitar?}

\label{q:92:sa:sv:False}

\vspace{2cm}

\noindent\makebox[\textwidth]{\hrulefill}

\vspace{1cm}

\textit{Svar}: \autoref{q:92:sa:sv:True}



\subsection{Vilka \"ar de tre olika kategorierna av maskininstruktioner (machine instruction categories)?}

\label{q:93:sa:sv:False}

\vspace{2cm}

\noindent\makebox[\textwidth]{\hrulefill}

\vspace{1cm}

\textit{Svar}: \autoref{q:93:sa:sv:True}



\subsection{Vad \"ar ett maskinspr\r{a}k (machine language)?}

\label{q:94:sa:sv:False}

\vspace{2cm}

\noindent\makebox[\textwidth]{\hrulefill}

\vspace{1cm}

\textit{Svar}: \autoref{q:94:sa:sv:True}



\subsection{Vad lagras i programr\"aknaren (program counter)?}

\label{q:95:sa:sv:False}

\vspace{2cm}

\noindent\makebox[\textwidth]{\hrulefill}

\vspace{1cm}

\textit{Svar}: \autoref{q:95:sa:sv:True}



\subsection{Vad lagras i instruktionsregistret (instruction register)?}

\label{q:96:sa:sv:False}

\vspace{2cm}

\noindent\makebox[\textwidth]{\hrulefill}

\vspace{1cm}

\textit{Svar}: \autoref{q:96:sa:sv:True}



\subsection{Beskriv skillnaden mellan RISC- och CISC-processorer.}

\label{q:97:sa:sv:False}

\vspace{2cm}

\noindent\makebox[\textwidth]{\hrulefill}

\vspace{1cm}

\textit{Svar}: \autoref{q:97:sa:sv:True}



\subsection{Vilka olika steg ing\r{a}r i en maskincykel (machine cycle)? Ange stegen i den ordning de utf\"ors.}

\label{q:98:sa:sv:False}

\vspace{2cm}

\noindent\makebox[\textwidth]{\hrulefill}

\vspace{1cm}

\textit{Svar}: \autoref{q:98:sa:sv:True}



\subsection{Vilket bitm\"onster erh\r{a}ller vi om vi utf\"or operationen ADD p\r{a} bitm\"onstren 1011 0011 och 0010 0110?}

\label{q:99:sa:sv:False}

\vspace{2cm}

\noindent\makebox[\textwidth]{\hrulefill}

\vspace{1cm}

\textit{Svar}: \autoref{q:99:sa:sv:True}



\subsection{Vilka \"ar de tre huvudsakliga delar som en processor (CPU) best\r{a}r av?}

\label{q:100:sa:sv:False}

\vspace{2cm}

\noindent\makebox[\textwidth]{\hrulefill}

\vspace{1cm}

\textit{Svar}: \autoref{q:100:sa:sv:True}



\subsection{Vilket bitm\"onster erh\r{a}ller vi om vi utf\"or operationen XOR p\r{a} bitm\"onstren 1010 0011 och 0010 0110?}

\label{q:101:sa:sv:False}

\vspace{2cm}

\noindent\makebox[\textwidth]{\hrulefill}

\vspace{1cm}

\textit{Svar}: \autoref{q:101:sa:sv:True}



\subsection{Vad kr\"avs f\"or att man ska kunna tolka ett bitm\"onster som ett tecken? What is required to be able to interpret a bit pattern as a character?}

\label{q:102:sa:sv:False}

\vspace{2cm}

\noindent\makebox[\textwidth]{\hrulefill}

\vspace{1cm}

\textit{Svar}: \autoref{q:102:sa:sv:True}



\subsection{Hur ser man till att processer inte kan utf\"ora operationer som \"ar destruktiva f\"or andra processer p\r{a} en dator, t.ex. att skriva data i andra processers delar av prim\"arminnet?}

\label{q:104:sa:sv:False}

\vspace{2cm}

\noindent\makebox[\textwidth]{\hrulefill}

\vspace{1cm}

\textit{Svar}: \autoref{q:104:sa:sv:True}



\subsection{Vad inneb\"ar boot strapping (booting) och varf\"or beh\"ovs det?}

\label{q:105:sa:sv:False}

\vspace{2cm}

\noindent\makebox[\textwidth]{\hrulefill}

\vspace{1cm}

\textit{Svar}: \autoref{q:105:sa:sv:True}



\subsection{Vad inneb\"ar realtidsbehandling (real time processing)?}

\label{q:106:sa:sv:False}

\vspace{2cm}

\noindent\makebox[\textwidth]{\hrulefill}

\vspace{1cm}

\textit{Svar}: \autoref{q:106:sa:sv:True}



\subsection{Vad inneb\"ar virtuellt minne (virtual memory)?}

\label{q:108:sa:sv:False}

\vspace{2cm}

\noindent\makebox[\textwidth]{\hrulefill}

\vspace{1cm}

\textit{Svar}: \autoref{q:108:sa:sv:True}



\subsection{Vilken huvudsaklig funktion har ett operativsystem?}

\label{q:109:sa:sv:False}

\vspace{2cm}

\noindent\makebox[\textwidth]{\hrulefill}

\vspace{1cm}

\textit{Svar}: \autoref{q:109:sa:sv:True}



\subsection{Vad inneb\"ar interaktiv bearbetning (interactive processing)?}

\label{q:110:sa:sv:False}

\vspace{2cm}

\noindent\makebox[\textwidth]{\hrulefill}

\vspace{1cm}

\textit{Svar}: \autoref{q:110:sa:sv:True}



\subsection{Vad inneb\"ar realtidsbearbetning (real time processing)?}

\label{q:111:sa:sv:False}

\vspace{2cm}

\noindent\makebox[\textwidth]{\hrulefill}

\vspace{1cm}

\textit{Svar}: \autoref{q:111:sa:sv:True}



\subsection{Vad \"ar skillnaden mellan batch-bearbetning (batch processing) och interaktiv-bearbetning (interactive processing)?}

\label{q:112:sa:sv:False}

\vspace{2cm}

\noindent\makebox[\textwidth]{\hrulefill}

\vspace{1cm}

\textit{Svar}: \autoref{q:112:sa:sv:True}



\subsection{Vad \"ar virtuellt minne och vad kan det vara bra f\"or?}

\label{q:113:sa:sv:False}

\vspace{2cm}

\noindent\makebox[\textwidth]{\hrulefill}

\vspace{1cm}

\textit{Svar}: \autoref{q:113:sa:sv:True}



\subsection{Ange fyra olika komponenter i ett operativsystems k\"arna (operating system kernel)?}

\label{q:114:sa:sv:False}

\vspace{2cm}

\noindent\makebox[\textwidth]{\hrulefill}

\vspace{1cm}

\textit{Svar}: \autoref{q:114:sa:sv:True}



\subsection{Vad \"ar en fil (file) i ett filhanteringssystem (file management system)?}

\label{q:115:sa:sv:False}

\vspace{2cm}

\noindent\makebox[\textwidth]{\hrulefill}

\vspace{1cm}

\textit{Svar}: \autoref{q:115:sa:sv:True}



\subsection{Vilka \"ar de fyra grundl\"aggande funktionerna f\"or ett operativsystem (functions of operating systems)?}

\label{q:116:sa:sv:False}

\vspace{2cm}

\noindent\makebox[\textwidth]{\hrulefill}

\vspace{1cm}

\textit{Svar}: \autoref{q:116:sa:sv:True}



\subsection{Ett operativsystem best\r{a}r av tv\r{a} huvudsakliga komponenter (operating system components), vilka?}

\label{q:118:sa:sv:False}

\vspace{2cm}

\noindent\makebox[\textwidth]{\hrulefill}

\vspace{1cm}

\textit{Svar}: \autoref{q:118:sa:sv:True}



\subsection{Vad kr\"avs f\"or att en deadlock skall kunna uppst\r{a} (conditions required for deadlock)?}

\label{q:119:sa:sv:False}

\vspace{2cm}

\noindent\makebox[\textwidth]{\hrulefill}

\vspace{1cm}

\textit{Svar}: \autoref{q:119:sa:sv:True}



\subsection{En process aktuella tillst\r{a}nd (state) kan beskrivas av en m\"angd data, vilket data?}

\label{q:120:sa:sv:False}

\vspace{2cm}

\noindent\makebox[\textwidth]{\hrulefill}

\vspace{1cm}

\textit{Svar}: \autoref{q:120:sa:sv:True}



\subsection{Vad \"ar ett program och vad \"ar en process?}

\label{q:121:sa:sv:False}

\vspace{2cm}

\noindent\makebox[\textwidth]{\hrulefill}

\vspace{1cm}

\textit{Svar}: \autoref{q:121:sa:sv:True}



\subsection{Vad \"ar en fil?}

\label{q:122:sa:sv:False}

\vspace{2cm}

\noindent\makebox[\textwidth]{\hrulefill}

\vspace{1cm}

\textit{Svar}: \autoref{q:122:sa:sv:True}



\subsection{Vad \"ar en katalog (directory)?}

\label{q:123:sa:sv:False}

\vspace{2cm}

\noindent\makebox[\textwidth]{\hrulefill}

\vspace{1cm}

\textit{Svar}: \autoref{q:123:sa:sv:True}



\subsection{Vad inneb\"ar paging?}

\label{q:124:sa:sv:False}

\vspace{2cm}

\noindent\makebox[\textwidth]{\hrulefill}

\vspace{1cm}

\textit{Svar}: \autoref{q:124:sa:sv:True}



\subsection{Vad \"ar och vad g\"or en boot loader?}

\label{q:125:sa:sv:False}

\vspace{2cm}

\noindent\makebox[\textwidth]{\hrulefill}

\vspace{1cm}

\textit{Svar}: \autoref{q:125:sa:sv:True}



\subsection{Vad inneb\"ar interaktiv bearbetning (interactive processing)?}

\label{q:126:sa:sv:False}

\vspace{2cm}

\noindent\makebox[\textwidth]{\hrulefill}

\vspace{1cm}

\textit{Svar}: \autoref{q:126:sa:sv:True}



\subsection{Vad inneb\"ar batch-bearbetning (batch processing)?}

\label{q:127:sa:sv:False}

\vspace{2cm}

\noindent\makebox[\textwidth]{\hrulefill}

\vspace{1cm}

\textit{Svar}: \autoref{q:127:sa:sv:True}



\subsection{Vad inneb\"ar begreppet deadlock?}

\label{q:129:sa:sv:False}

\vspace{2cm}

\noindent\makebox[\textwidth]{\hrulefill}

\vspace{1cm}

\textit{Svar}: \autoref{q:129:sa:sv:True}



\subsection{Vad \"ar ett job i samband med batch-processing?}

\label{q:130:sa:sv:False}

\vspace{2cm}

\noindent\makebox[\textwidth]{\hrulefill}

\vspace{1cm}

\textit{Svar}: \autoref{q:130:sa:sv:True}



\subsection{Vad inneb\"ar multitasking?}

\label{q:132:sa:sv:False}

\vspace{2cm}

\noindent\makebox[\textwidth]{\hrulefill}

\vspace{1cm}

\textit{Svar}: \autoref{q:132:sa:sv:True}



\subsection{Vad inneb\"ar "paging"?}

\label{q:133:sa:sv:False}

\vspace{2cm}

\noindent\makebox[\textwidth]{\hrulefill}

\vspace{1cm}

\textit{Svar}: \autoref{q:133:sa:sv:True}



\subsection{Vad \"ar skillnaden mellan en switch och en router?}

\label{q:134:sa:sv:False}

\vspace{2cm}

\noindent\makebox[\textwidth]{\hrulefill}

\vspace{1cm}

\textit{Svar}: \autoref{q:134:sa:sv:True}



\subsection{Vilka \"ar de tv\r{a} modellerna f\"or inter-process-kommunikation?}

\label{q:135:sa:sv:False}

\vspace{2cm}

\noindent\makebox[\textwidth]{\hrulefill}

\vspace{1cm}

\textit{Svar}: \autoref{q:135:sa:sv:True}



\subsection{Vad \"ar en IP-adress?}

\label{q:136:sa:sv:False}

\vspace{2cm}

\noindent\makebox[\textwidth]{\hrulefill}

\vspace{1cm}

\textit{Svar}: \autoref{q:136:sa:sv:True}



\subsection{Vad \"ar DNS?}

\label{q:137:sa:sv:False}

\vspace{2cm}

\noindent\makebox[\textwidth]{\hrulefill}

\vspace{1cm}

\textit{Svar}: \autoref{q:137:sa:sv:True}



\subsection{Vad inneb\"ar bus och star n\"ar det handlar om n\"attopologi?}

\label{q:138:sa:sv:False}

\vspace{2cm}

\noindent\makebox[\textwidth]{\hrulefill}

\vspace{1cm}

\textit{Svar}: \autoref{q:138:sa:sv:True}



\subsection{Vad inneb\"ar cloud computing?}

\label{q:139:sa:sv:False}

\vspace{2cm}

\noindent\makebox[\textwidth]{\hrulefill}

\vspace{1cm}

\textit{Svar}: \autoref{q:139:sa:sv:True}



\subsection{Vad \"ar den huvudsakliga skillnaden mellan IPv4 (IP version 4) och IPv6 (IP version 6)?}

\label{q:140:sa:sv:False}

\vspace{2cm}

\noindent\makebox[\textwidth]{\hrulefill}

\vspace{1cm}

\textit{Svar}: \autoref{q:140:sa:sv:True}



\subsection{Vad \"ar ett certifikat (certificate) i samband med public-key-kryptering (public key encryption)?}

\label{q:141:sa:sv:False}

\vspace{2cm}

\noindent\makebox[\textwidth]{\hrulefill}

\vspace{1cm}

\textit{Svar}: \autoref{q:141:sa:sv:True}



\subsection{Ge ett exempel p\r{a} en typ av malware?}

\label{q:142:sa:sv:False}

\vspace{2cm}

\noindent\makebox[\textwidth]{\hrulefill}

\vspace{1cm}

\textit{Svar}: \autoref{q:142:sa:sv:True}



\subsection{Vad inneb\"ar DNS lookup?}

\label{q:143:sa:sv:False}

\vspace{2cm}

\noindent\makebox[\textwidth]{\hrulefill}

\vspace{1cm}

\textit{Svar}: \autoref{q:143:sa:sv:True}



\subsection{Vad g\"or en (n\"atverks-) hub?}

\label{q:144:sa:sv:False}

\vspace{2cm}

\noindent\makebox[\textwidth]{\hrulefill}

\vspace{1cm}

\textit{Svar}: \autoref{q:144:sa:sv:True}



\subsection{Till vilket Internet-mjukvarulager (Internet software layer) h\"or protokollet SMTP?}

\label{q:145:sa:sv:False}

\vspace{2cm}

\noindent\makebox[\textwidth]{\hrulefill}

\vspace{1cm}

\textit{Svar}: \autoref{q:145:sa:sv:True}



\subsection{Vad g\"or en webbserver (webserver)?}

\label{q:146:sa:sv:False}

\vspace{2cm}

\noindent\makebox[\textwidth]{\hrulefill}

\vspace{1cm}

\textit{Svar}: \autoref{q:146:sa:sv:True}



\subsection{Vad \"ar syftet med en URL/URI?}

\label{q:147:sa:sv:False}

\vspace{2cm}

\noindent\makebox[\textwidth]{\hrulefill}

\vspace{1cm}

\textit{Svar}: \autoref{q:147:sa:sv:True}



\subsection{Vilka \"ar de tv\r{a} vanliga Internet-protokollen f\"or transport-lagret (transport layer)?}

\label{q:148:sa:sv:False}

\vspace{2cm}

\noindent\makebox[\textwidth]{\hrulefill}

\vspace{1cm}

\textit{Svar}: \autoref{q:148:sa:sv:True}



\subsection{Vad kallas den krypteringsteknik som anv\"ands mycket p\r{a} Internet och som inneb\"ar att parterna inte i f\"orv\"ag beh\"over ha tillg\r{a}ng till en gemensam nyckel?}

\label{q:149:sa:sv:False}

\vspace{2cm}

\noindent\makebox[\textwidth]{\hrulefill}

\vspace{1cm}

\textit{Svar}: \autoref{q:149:sa:sv:True}



\subsection{Vad \"ar Internet-dom\"aner (Internet domains) och vad \"ar syftet med dem?}

\label{q:150:sa:sv:False}

\vspace{2cm}

\noindent\makebox[\textwidth]{\hrulefill}

\vspace{1cm}

\textit{Svar}: \autoref{q:150:sa:sv:True}



\subsection{Ge tv\r{a} exempel p\r{a} Internet-applikationer med \"oppna (allm\"ant tillg\"angliga) protokoll?}

\label{q:151:sa:sv:False}

\vspace{2cm}

\noindent\makebox[\textwidth]{\hrulefill}

\vspace{1cm}

\textit{Svar}: \autoref{q:151:sa:sv:True}



\subsection{Vad \"ar det f\"or skillnad mellan protokollen HTTP och HTTPS?}

\label{q:152:sa:sv:False}

\vspace{2cm}

\noindent\makebox[\textwidth]{\hrulefill}

\vspace{1cm}

\textit{Svar}: \autoref{q:152:sa:sv:True}



\subsection{F\"orklara kortfattat skillnaden mellan n\"atverkskomponenterna hub, switch och router?}

\label{q:153:sa:sv:False}

\vspace{2cm}

\noindent\makebox[\textwidth]{\hrulefill}

\vspace{1cm}

\textit{Svar}: \autoref{q:153:sa:sv:True}



\subsection{Vad \"overf\"ors med de olika protokollen FTP, HTTP, SMTP?}

\label{q:154:sa:sv:False}

\vspace{2cm}

\noindent\makebox[\textwidth]{\hrulefill}

\vspace{1cm}

\textit{Svar}: \autoref{q:154:sa:sv:True}



\subsection{Vad \"ar ett certifikat? Kan man lita lika mycket p\r{a} alla certifikat? Motivera ditt svar!}

\label{q:155:sa:sv:False}

\vspace{2cm}

\noindent\makebox[\textwidth]{\hrulefill}

\vspace{1cm}

\textit{Svar}: \autoref{q:155:sa:sv:True}



\subsection{Vad inneb\"ar en digital signatur (digital signature) vid publik-nyckel-kryptering (public key encryption), d.v.s. att vid \"overf\"oringen av en fil s\r{a} kan man garantera avs\"andarens identitet?}

\label{q:156:sa:sv:False}

\vspace{2cm}

\noindent\makebox[\textwidth]{\hrulefill}

\vspace{1cm}

\textit{Svar}: \autoref{q:156:sa:sv:True}



\subsection{Vad k\"annetecknar ett distribuerat system (distributed system)?}

\label{q:157:sa:sv:False}

\vspace{2cm}

\noindent\makebox[\textwidth]{\hrulefill}

\vspace{1cm}

\textit{Svar}: \autoref{q:157:sa:sv:True}



\subsection{Hur m\r{a}nga g\r{a}nger fler adresser kan representeras med IPv6 j\"amf\"ort med IPv4 (som vanligt s\r{a} beh\"over ni inte r\"akna ut ett v\"arde utan det r\"acker med att st\"alla upp en korrekt utr\"akning)?}

\label{q:158:sa:sv:False}

\vspace{2cm}

\noindent\makebox[\textwidth]{\hrulefill}

\vspace{1cm}

\textit{Svar}: \autoref{q:158:sa:sv:True}



\subsection{Vad \"ar ett distribuerat system (distributed system)?}

\label{q:159:sa:sv:False}

\vspace{2cm}

\noindent\makebox[\textwidth]{\hrulefill}

\vspace{1cm}

\textit{Svar}: \autoref{q:159:sa:sv:True}



\subsection{Vad anv\"ands HTML till?}

\label{q:160:sa:sv:False}

\vspace{2cm}

\noindent\makebox[\textwidth]{\hrulefill}

\vspace{1cm}

\textit{Svar}: \autoref{q:160:sa:sv:True}



\subsection{Inom public key encryption anv\"ands begreppet certifikat, vad \"ar det?}

\label{q:161:sa:sv:False}

\vspace{2cm}

\noindent\makebox[\textwidth]{\hrulefill}

\vspace{1cm}

\textit{Svar}: \autoref{q:161:sa:sv:True}



\subsection{Inom public key encryption anv\"ands begreppet certificate authority, vad \"ar det?}

\label{q:162:sa:sv:False}

\vspace{2cm}

\noindent\makebox[\textwidth]{\hrulefill}

\vspace{1cm}

\textit{Svar}: \autoref{q:162:sa:sv:True}



\subsection{Vad \"ar en f\"ordel med att anv\"anda TCP ist\"allet f\"or UDP? Vad \"ar en nackdel?}

\label{q:163:sa:sv:False}

\vspace{2cm}

\noindent\makebox[\textwidth]{\hrulefill}

\vspace{1cm}

\textit{Svar}: \autoref{q:163:sa:sv:True}



\subsection{Vad \"ar en f\"ordel med att anv\"anda UDP ist\"allet f\"or TCP? Vad \"ar en nackdel?}

\label{q:164:sa:sv:False}

\vspace{2cm}

\noindent\makebox[\textwidth]{\hrulefill}

\vspace{1cm}

\textit{Svar}: \autoref{q:164:sa:sv:True}



\subsection{Vilka \"ar de fyra Internet-mjukvarulagren?}

\label{q:165:sa:sv:False}

\vspace{2cm}

\noindent\makebox[\textwidth]{\hrulefill}

\vspace{1cm}

\textit{Svar}: \autoref{q:165:sa:sv:True}



\subsection{F\"orklara kortfattat begreppet client-server!}

\label{q:166:sa:sv:False}

\vspace{2cm}

\noindent\makebox[\textwidth]{\hrulefill}

\vspace{1cm}

\textit{Svar}: \autoref{q:166:sa:sv:True}



\subsection{Vad anv\"ands SMTP till?}

\label{q:167:sa:sv:False}

\vspace{2cm}

\noindent\makebox[\textwidth]{\hrulefill}

\vspace{1cm}

\textit{Svar}: \autoref{q:167:sa:sv:True}



\subsection{Om person A vill skicka ett meddelande till person B, krypterat enligt public-key-encryption, s\r{a} att ingen annan \"an B kan l\"asa meddelandet. Vad beh\"over meddelandet d\r{a} krypteras med innan meddelandet skickas fr\r{a}n A?}

\label{q:168:sa:sv:False}

\vspace{2cm}

\noindent\makebox[\textwidth]{\hrulefill}

\vspace{1cm}

\textit{Svar}: \autoref{q:168:sa:sv:True}



\subsection{Om person A vill skicka ett meddelande till person B, krypterat enligt public-key-encryption, s\r{a} att ingen annan \"an A kan ha skickat meddelandet. Vad beh\"over meddelandet d\r{a} krypteras med innan meddelandet skickas fr\r{a}n A?}

\label{q:169:sa:sv:False}

\vspace{2cm}

\noindent\makebox[\textwidth]{\hrulefill}

\vspace{1cm}

\textit{Svar}: \autoref{q:169:sa:sv:True}



\subsection{N\"amn en f\"ordel med publik-nyckel-kryptering j\"amf\"ort med symmetriska krypteringstekniker.}

\label{q:170:sa:sv:False}

\vspace{2cm}

\noindent\makebox[\textwidth]{\hrulefill}

\vspace{1cm}

\textit{Svar}: \autoref{q:170:sa:sv:True}



\subsection{Vad anv\"ands en hub respektive en router f\"or?}

\label{q:171:sa:sv:False}

\vspace{2cm}

\noindent\makebox[\textwidth]{\hrulefill}

\vspace{1cm}

\textit{Svar}: \autoref{q:171:sa:sv:True}



\subsection{N\"amn en f\"ordel med UDP framf\"or TCP, och en f\"ordel med TCP framf\"or UDP.}

\label{q:172:sa:sv:False}

\vspace{2cm}

\noindent\makebox[\textwidth]{\hrulefill}

\vspace{1cm}

\textit{Svar}: \autoref{q:172:sa:sv:True}



\subsection{Vad \"ar rekursion?}

\label{q:173:sa:sv:False}

\vspace{2cm}

\noindent\makebox[\textwidth]{\hrulefill}

\vspace{1cm}

\textit{Svar}: \autoref{q:173:sa:sv:True}



\subsection{Varf\"or \"ar bin\"ar s\"okning b\"attre \"an sekvensiell s\"okning p\r{a} sorterat data?}

\label{q:174:sa:sv:False}

\vspace{2cm}

\noindent\makebox[\textwidth]{\hrulefill}

\vspace{1cm}

\textit{Svar}: \autoref{q:174:sa:sv:True}



\subsection{\"A\ensuremath{\ddot{}}r det n\r{a}gon skillnad mellan iteration och rekursion n\"ar det g\"aller anv\"andningen av minne?}

\label{q:175:sa:sv:False}

\vspace{2cm}

\noindent\makebox[\textwidth]{\hrulefill}

\vspace{1cm}

\textit{Svar}: \autoref{q:175:sa:sv:True}



\subsection{Vad \"ar skillnaden mellan en algoritm och ett program?}

\label{q:177:sa:sv:False}

\vspace{2cm}

\noindent\makebox[\textwidth]{\hrulefill}

\vspace{1cm}

\textit{Svar}: \autoref{q:177:sa:sv:True}



\subsection{Vilka tv\r{a} olika metoder anv\"ands f\"or att verifiera att ett program \"ar korrekt (software verification)?}

\label{q:178:sa:sv:False}

\vspace{2cm}

\noindent\makebox[\textwidth]{\hrulefill}

\vspace{1cm}

\textit{Svar}: \autoref{q:178:sa:sv:True}



\subsection{Vad \"ar ett program i f\"orh\r{a}llande till en algoritm?}

\label{q:179:sa:sv:False}

\vspace{2cm}

\noindent\makebox[\textwidth]{\hrulefill}

\vspace{1cm}

\textit{Svar}: \autoref{q:179:sa:sv:True}



\subsection{Beskriv hur bin\"ars\"okning g\r{a}r till! Vilka krav finns p\r{a} den data som man s\"oker i?}

\label{q:180:sa:sv:False}

\vspace{2cm}

\noindent\makebox[\textwidth]{\hrulefill}

\vspace{1cm}

\textit{Svar}: \autoref{q:180:sa:sv:True}



\subsection{Vilka metoder kan anv\"andas f\"or att verifiera ett programs korrekthet?}

\label{q:181:sa:sv:False}

\vspace{2cm}

\noindent\makebox[\textwidth]{\hrulefill}

\vspace{1cm}

\textit{Svar}: \autoref{q:181:sa:sv:True}



\subsection{P\r{a} vilka tv\r{a} grundl\"aggande olika s\"att kan man \r{a}stadkomma repetition i en algoritm?}

\label{q:182:sa:sv:False}

\vspace{2cm}

\noindent\makebox[\textwidth]{\hrulefill}

\vspace{1cm}

\textit{Svar}: \autoref{q:182:sa:sv:True}



\subsection{N\"ar \"ar sekventiell s\"okning att f\"oredra framf\"or bin\"ars\"okning?}

\label{q:183:sa:sv:False}

\vspace{2cm}

\noindent\makebox[\textwidth]{\hrulefill}

\vspace{1cm}

\textit{Svar}: \autoref{q:183:sa:sv:True}



\subsection{Vad \"ar en f\"oruts\"attning f\"or att bin\"ars\"okning (binary search) ska fungera? Motivera ditt svar.}

\label{q:184:sa:sv:False}

\vspace{2cm}

\noindent\makebox[\textwidth]{\hrulefill}

\vspace{1cm}

\textit{Svar}: \autoref{q:184:sa:sv:True}



\subsection{\"Ar bin\"ars\"okning ett bra val f\"or att s\"oka i osorterad data? Motivera ditt svar.}

\label{q:185:sa:sv:False}

\vspace{2cm}

\noindent\makebox[\textwidth]{\hrulefill}

\vspace{1cm}

\textit{Svar}: \autoref{q:185:sa:sv:True}



\subsection{Definiera begreppet algoritm (algorithm)!}

\label{q:186:sa:sv:False}

\vspace{2cm}

\noindent\makebox[\textwidth]{\hrulefill}

\vspace{1cm}

\textit{Svar}: \autoref{q:186:sa:sv:True}



\subsection{Kan alla algoritmer beskrivas som ett fl\"odes-schema (flow chart)? Motivera ditt svar!}

\label{q:187:sa:sv:False}

\vspace{2cm}

\noindent\makebox[\textwidth]{\hrulefill}

\vspace{1cm}

\textit{Svar}: \autoref{q:187:sa:sv:True}



\subsection{\"Ar ett programmeringsspr\r{a}k, t.ex. Python, l\"ampligt f\"or att beskriva algoritmer? Motivera ditt svar!}

\label{q:188:sa:sv:False}

\vspace{2cm}

\noindent\makebox[\textwidth]{\hrulefill}

\vspace{1cm}

\textit{Svar}: \autoref{q:188:sa:sv:True}



\subsection{Vad inneb\"ar top-down metodologin n\"ar man utvecklar (eller uppt\"acker) algoritmer?}

\label{q:189:sa:sv:False}

\vspace{2cm}

\noindent\makebox[\textwidth]{\hrulefill}

\vspace{1cm}

\textit{Svar}: \autoref{q:189:sa:sv:True}



\subsection{Varf\"or \"ar det inte s\r{a} viktigt att f\"olja en strikt syntax i pseudokod?}

\label{q:190:sa:sv:False}

\vspace{2cm}

\noindent\makebox[\textwidth]{\hrulefill}

\vspace{1cm}

\textit{Svar}: \autoref{q:190:sa:sv:True}



\subsection{Varf\"or \"ar det n\"odv\"andigt att veta vilken datatyp en variabel har?}

\label{q:191:sa:sv:False}

\vspace{2cm}

\noindent\makebox[\textwidth]{\hrulefill}

\vspace{1cm}

\textit{Svar}: \autoref{q:191:sa:sv:True}



\subsection{Vad \"ar skillnaden mellan k\"allkod och objektkod?}

\label{q:192:sa:sv:False}

\vspace{2cm}

\noindent\makebox[\textwidth]{\hrulefill}

\vspace{1cm}

\textit{Svar}: \autoref{q:192:sa:sv:True}



\subsection{Producerar ett syntaktiskt korrekt program alltid korrekta resultat? Motivera ditt svar.}

\label{q:193:sa:sv:False}

\vspace{2cm}

\noindent\makebox[\textwidth]{\hrulefill}

\vspace{1cm}

\textit{Svar}: \autoref{q:193:sa:sv:True}



\subsection{Vad k\"annetecknar en datastruktur av typen struct/record (aggregate type)?}

\label{q:194:sa:sv:False}

\vspace{2cm}

\noindent\makebox[\textwidth]{\hrulefill}

\vspace{1cm}

\textit{Svar}: \autoref{q:194:sa:sv:True}



\subsection{Vad inneb\"ar det att en parameter till en subrutin \"overf\"ors som v\"arde (passed by value)?}

\label{q:195:sa:sv:False}

\vspace{2cm}

\noindent\makebox[\textwidth]{\hrulefill}

\vspace{1cm}

\textit{Svar}: \autoref{q:195:sa:sv:True}



\subsection{Vad inneb\"ar det att en parameter till en subrutin \"overf\"ors som referens (passed by reference)?}

\label{q:196:sa:sv:False}

\vspace{2cm}

\noindent\makebox[\textwidth]{\hrulefill}

\vspace{1cm}

\textit{Svar}: \autoref{q:196:sa:sv:True}



\subsection{Vad g\"or en assemblator/assemblerare (assembler)?}

\label{q:197:sa:sv:False}

\vspace{2cm}

\noindent\makebox[\textwidth]{\hrulefill}

\vspace{1cm}

\textit{Svar}: \autoref{q:197:sa:sv:True}



\subsection{Vad k\"annetecknar en datastruktur av typen array?}

\label{q:198:sa:sv:False}

\vspace{2cm}

\noindent\makebox[\textwidth]{\hrulefill}

\vspace{1cm}

\textit{Svar}: \autoref{q:198:sa:sv:True}



\subsection{Vilka \"ar de fyra stora programmeringsparadigmerna (programming paradigms)?}

\label{q:199:sa:sv:False}

\vspace{2cm}

\noindent\makebox[\textwidth]{\hrulefill}

\vspace{1cm}

\textit{Svar}: \autoref{q:199:sa:sv:True}



\subsection{Ange fyra vanliga primitiva datatyper.}

\label{q:200:sa:sv:False}

\vspace{2cm}

\noindent\makebox[\textwidth]{\hrulefill}

\vspace{1cm}

\textit{Svar}: \autoref{q:200:sa:sv:True}



\subsection{Vad g\"or en kompilator (compiler)?}

\label{q:201:sa:sv:False}

\vspace{2cm}

\noindent\makebox[\textwidth]{\hrulefill}

\vspace{1cm}

\textit{Svar}: \autoref{q:201:sa:sv:True}



\subsection{Ett program kan ge upphov till tre olika typer av fel: syntaktiska fel (syntactic errors), exekveringsfel (runtime errors) och logiska fel (logic errors). Vilken typ av fel \"ar mest allvarliga och varf\"or?}

\label{q:202:sa:sv:False}

\vspace{2cm}

\noindent\makebox[\textwidth]{\hrulefill}

\vspace{1cm}

\textit{Svar}: \autoref{q:202:sa:sv:True}



\subsection{Ett program kan ge upphov till tre olika typer av fel: syntaktiska fel (syntactic errors), exekveringsfel (runtime errors) och logiska fel (logic errors). Vilket typ av fel \"ar minst allvarliga och varf\"or?}

\label{q:203:sa:sv:False}

\vspace{2cm}

\noindent\makebox[\textwidth]{\hrulefill}

\vspace{1cm}

\textit{Svar}: \autoref{q:203:sa:sv:True}



\subsection{Vad \"ar concurrent programming?}

\label{q:204:sa:sv:False}

\vspace{2cm}

\noindent\makebox[\textwidth]{\hrulefill}

\vspace{1cm}

\textit{Svar}: \autoref{q:204:sa:sv:True}



\subsection{Beskriv kortfattat begreppen sekvens, selektion och iteration.}

\label{q:205:sa:sv:False}

\vspace{2cm}

\noindent\makebox[\textwidth]{\hrulefill}

\vspace{1cm}

\textit{Svar}: \autoref{q:205:sa:sv:True}



\subsection{En variabel pekar p\r{a} ett bitm\"onster i lagrat i minnet; vad beh\"over vi veta f\"or att kunna tolka bitm\"onstret p\r{a} r\"att s\"att?}

\label{q:206:sa:sv:False}

\vspace{2cm}

\noindent\makebox[\textwidth]{\hrulefill}

\vspace{1cm}

\textit{Svar}: \autoref{q:206:sa:sv:True}



\subsection{Vad inneb\"ar begreppen sekvens, selektion och iteration?}

\label{q:207:sa:sv:False}

\vspace{2cm}

\noindent\makebox[\textwidth]{\hrulefill}

\vspace{1cm}

\textit{Svar}: \autoref{q:207:sa:sv:True}



\subsection{Ge exempel p\r{a} tv\r{a} olika s\"att att beskriva algoritmer.}

\label{q:208:sa:sv:False}

\vspace{2cm}

\noindent\makebox[\textwidth]{\hrulefill}

\vspace{1cm}

\textit{Svar}: \autoref{q:208:sa:sv:True}



\subsection{Vilken generation av programmeringsspr\r{a}k k\"annetecknas av:- ett-till-ett-f\"orh\r{a}llande mellan spr\r{a}kinstruktioner och maskininstruktioner; - inneboende maskin-beroende?}

\label{q:209:sa:sv:False}

\vspace{2cm}

\noindent\makebox[\textwidth]{\hrulefill}

\vspace{1cm}

\textit{Svar}: \autoref{q:209:sa:sv:True}



\subsection{Vilken generation av programmeringsspr\r{a}k k\"annetecknas av:- maskinoberoende (vanligtvis);- varje primitiv motsvarar en sekvens av maskinspr\r{a}ksinstruktioner?}

\label{q:210:sa:sv:False}

\vspace{2cm}

\noindent\makebox[\textwidth]{\hrulefill}

\vspace{1cm}

\textit{Svar}: \autoref{q:210:sa:sv:True}



\subsection{Vad \"ar en literal i ett programmeringsspr\r{a}k?}

\label{q:211:sa:sv:False}

\vspace{2cm}

\noindent\makebox[\textwidth]{\hrulefill}

\vspace{1cm}

\textit{Svar}: \autoref{q:211:sa:sv:True}



\subsection{Vad \"ar en konstant i ett programmeringsspr\r{a}k?}

\label{q:212:sa:sv:False}

\vspace{2cm}

\noindent\makebox[\textwidth]{\hrulefill}

\vspace{1cm}

\textit{Svar}: \autoref{q:212:sa:sv:True}



\subsection{I objektorienterad programmering har man klasser och objekt. Ut\"over detta s\r{a} finns det tre egenskaper som k\"annetecknar objektorienterad programmering, vilka?}

\label{q:213:sa:sv:False}

\vspace{2cm}

\noindent\makebox[\textwidth]{\hrulefill}

\vspace{1cm}

\textit{Svar}: \autoref{q:213:sa:sv:True}



\subsection{\"Overs\"attningen fr\r{a}n k\"allkod till maskinkod sker i tre steg av tre olika enheter i \"overs\"attaren; vad kallas dessa tre enheter?}

\label{q:214:sa:sv:False}

\vspace{2cm}

\noindent\makebox[\textwidth]{\hrulefill}

\vspace{1cm}

\textit{Svar}: \autoref{q:214:sa:sv:True}



\subsection{Vad \"ar en tr\r{a}d i concurrent programmering?}

\label{q:215:sa:sv:False}

\vspace{2cm}

\noindent\makebox[\textwidth]{\hrulefill}

\vspace{1cm}

\textit{Svar}: \autoref{q:215:sa:sv:True}



\subsection{Vad \"ar den grundl\"aggande byggstenen i logikprogrammeringsspr\r{a}k?}

\label{q:216:sa:sv:False}

\vspace{2cm}

\noindent\makebox[\textwidth]{\hrulefill}

\vspace{1cm}

\textit{Svar}: \autoref{q:216:sa:sv:True}



\subsection{Vad \"ar en variabel i ett programmeringsspr\r{a}k?}

\label{q:217:sa:sv:False}

\vspace{2cm}

\noindent\makebox[\textwidth]{\hrulefill}

\vspace{1cm}

\textit{Svar}: \autoref{q:217:sa:sv:True}



\subsection{Vad \"ar syftet med att anv\"anda procedurenheter (subprogram, subrutin, procedur, funktion, metod, predikat etc.) vid programmering?}

\label{q:218:sa:sv:False}

\vspace{2cm}

\noindent\makebox[\textwidth]{\hrulefill}

\vspace{1cm}

\textit{Svar}: \autoref{q:218:sa:sv:True}



\subsection{Vad inneb\"ar arv i objektorienterad programmering?}

\label{q:219:sa:sv:False}

\vspace{2cm}

\noindent\makebox[\textwidth]{\hrulefill}

\vspace{1cm}

\textit{Svar}: \autoref{q:219:sa:sv:True}



\subsection{Vad \"ar skillnaden mellan en kompilator (compiler) och en interpretator (interpreter)?}

\label{q:220:sa:sv:False}

\vspace{2cm}

\noindent\makebox[\textwidth]{\hrulefill}

\vspace{1cm}

\textit{Svar}: \autoref{q:220:sa:sv:True}



\subsection{Alla programmeringsspr\r{a}k har tre typer av styrning av programfl\"odet, vilka?}

\label{q:221:sa:sv:False}

\vspace{2cm}

\noindent\makebox[\textwidth]{\hrulefill}

\vspace{1cm}

\textit{Svar}: \autoref{q:221:sa:sv:True}



\subsection{Vilka tre saker k\"annetecknar l\"attr\"orliga utvecklingsmodeller (agile development models)?}

\label{q:222:sa:sv:False}

\vspace{2cm}

\noindent\makebox[\textwidth]{\hrulefill}

\vspace{1cm}

\textit{Svar}: \autoref{q:222:sa:sv:True}



\subsection{Vad \"ar design patterns?}

\label{q:223:sa:sv:False}

\vspace{2cm}

\noindent\makebox[\textwidth]{\hrulefill}

\vspace{1cm}

\textit{Svar}: \autoref{q:223:sa:sv:True}



\subsection{Vad \"ar syftet med use case diagram?}

\label{q:224:sa:sv:False}

\vspace{2cm}

\noindent\makebox[\textwidth]{\hrulefill}

\vspace{1cm}

\textit{Svar}: \autoref{q:224:sa:sv:True}



\subsection{Vad \"ar syftet med klassdiagram (class diagrams)?}

\label{q:225:sa:sv:False}

\vspace{2cm}

\noindent\makebox[\textwidth]{\hrulefill}

\vspace{1cm}

\textit{Svar}: \autoref{q:225:sa:sv:True}



\subsection{Vilka \"ar de fyra traditionella utvecklingsfaserna vid programvaruutveckling (the traditional development phases of the software life cycle)?}

\label{q:226:sa:sv:False}

\vspace{2cm}

\noindent\makebox[\textwidth]{\hrulefill}

\vspace{1cm}

\textit{Svar}: \autoref{q:226:sa:sv:True}



\subsection{Vad \"ar huvudsyftet med att dela upp en programvara i moduler?}

\label{q:227:sa:sv:False}

\vspace{2cm}

\noindent\makebox[\textwidth]{\hrulefill}

\vspace{1cm}

\textit{Svar}: \autoref{q:227:sa:sv:True}



\subsection{Vilka \"ar de tre \"onskv\"arda egenskaperna f\"or moduler som man vill uppn\r{a} n\"ar man delar upp en programvara i moduler?}

\label{q:228:sa:sv:False}

\vspace{2cm}

\noindent\makebox[\textwidth]{\hrulefill}

\vspace{1cm}

\textit{Svar}: \autoref{q:228:sa:sv:True}



\subsection{Vad \"ar det f\"or skillnad p\r{a} glass-box-testning (glass-box testing) och black-box-testning (black-box testing)?}

\label{q:229:sa:sv:False}

\vspace{2cm}

\noindent\makebox[\textwidth]{\hrulefill}

\vspace{1cm}

\textit{Svar}: \autoref{q:229:sa:sv:True}



\subsection{Beskriv skillnaderna mellan en-till-en- (one-to-one), en-till-m\r{a}nga- (one-to-many) och m\r{a}nga-till- m\r{a}nga- (many-to-many) relationer, g\"arna med hj\"alp av exempel.}

\label{q:230:sa:sv:False}

\vspace{2cm}

\noindent\makebox[\textwidth]{\hrulefill}

\vspace{1cm}

\textit{Svar}: \autoref{q:230:sa:sv:True}



\subsection{Vad kallas programvarutekniken som bygger p\r{a} att konstruera programvara genom att kombinera olika f\"ardiga komponenter (ist\"allet f\"or att utveckla egna komponenter)?}

\label{q:231:sa:sv:False}

\vspace{2cm}

\noindent\makebox[\textwidth]{\hrulefill}

\vspace{1cm}

\textit{Svar}: \autoref{q:231:sa:sv:True}



\subsection{Beskriv ett exempel p\r{a} var och en av de olika typerna av relation: en-till-en (one-to-one), en-till- m\r{a}nga (one-to-many) och m\r{a}nga-till-m\r{a}nga (many-to-many)!}

\label{q:232:sa:sv:False}

\vspace{2cm}

\noindent\makebox[\textwidth]{\hrulefill}

\vspace{1cm}

\textit{Svar}: \autoref{q:232:sa:sv:True}



\subsection{Vilka \"ar de fyra stegen i traditionell mjukvaruutveckling (med t ex vattenfallsmodellen)?}

\label{q:233:sa:sv:False}

\vspace{2cm}

\noindent\makebox[\textwidth]{\hrulefill}

\vspace{1cm}

\textit{Svar}: \autoref{q:233:sa:sv:True}



\subsection{Beskriv kortfattat n\r{a}gra f\"ordelar med att dela upp program i moduler?}

\label{q:234:sa:sv:False}

\vspace{2cm}

\noindent\makebox[\textwidth]{\hrulefill}

\vspace{1cm}

\textit{Svar}: \autoref{q:234:sa:sv:True}



\subsection{Vad inneb\"ar prototyping?}

\label{q:235:sa:sv:False}

\vspace{2cm}

\noindent\makebox[\textwidth]{\hrulefill}

\vspace{1cm}

\textit{Svar}: \autoref{q:235:sa:sv:True}



\subsection{Beskriv vad en sprint inom agil utveckling med Scrum \"ar?}

\label{q:236:sa:sv:False}

\vspace{2cm}

\noindent\makebox[\textwidth]{\hrulefill}

\vspace{1cm}

\textit{Svar}: \autoref{q:236:sa:sv:True}



\subsection{Vad utm\"arker black-box-testning (black-box testing)?}

\label{q:237:sa:sv:False}

\vspace{2cm}

\noindent\makebox[\textwidth]{\hrulefill}

\vspace{1cm}

\textit{Svar}: \autoref{q:237:sa:sv:True}



\subsection{Vad \"ar design patterns (designm\"onster) och vad \"ar de bra f\"or?}

\label{q:238:sa:sv:False}

\vspace{2cm}

\noindent\makebox[\textwidth]{\hrulefill}

\vspace{1cm}

\textit{Svar}: \autoref{q:238:sa:sv:True}



\subsection{F\"orklara begreppen koppling (coupling) och sammanh\r{a}llning (cohesion)?}

\label{q:239:sa:sv:False}

\vspace{2cm}

\noindent\makebox[\textwidth]{\hrulefill}

\vspace{1cm}

\textit{Svar}: \autoref{q:239:sa:sv:True}



\subsection{Vilka tre olika typer av relationer mellan entiteter \"ar viktiga att skilja p\r{a} vid programvaruutveckling?}

\label{q:240:sa:sv:False}

\vspace{2cm}

\noindent\makebox[\textwidth]{\hrulefill}

\vspace{1cm}

\textit{Svar}: \autoref{q:240:sa:sv:True}



\subsection{Vad kallas programvaruutvecklingsmetoder som v\"ardes\"atter:- individer och interaktioner mer \"an processer och verktyg;- fungerande programvara mer \"an omfattande dokumentation; - kundsamarbete mer \"an kontraktsf\"orhandlingar.- lyh\"ordhet f\"or f\"or\"andring mer \"an att f\"olja en plan.}

\label{q:241:sa:sv:False}

\vspace{2cm}

\noindent\makebox[\textwidth]{\hrulefill}

\vspace{1cm}

\textit{Svar}: \autoref{q:241:sa:sv:True}



\subsection{Vad \"ar en programvarumodul?}

\label{q:242:sa:sv:False}

\vspace{2cm}

\noindent\makebox[\textwidth]{\hrulefill}

\vspace{1cm}

\textit{Svar}: \autoref{q:242:sa:sv:True}



\subsection{Vad \"ar syftet med Scrum-m\"otet {\textquotedblleft}sprint retrospective{\textquotedblright}?}

\label{q:243:sa:sv:False}

\vspace{2cm}

\noindent\makebox[\textwidth]{\hrulefill}

\vspace{1cm}

\textit{Svar}: \autoref{q:243:sa:sv:True}



\subsection{Hur m\r{a}nga medlemmar b\"or ett utvecklingsteam ha enligt Scrum?}

\label{q:244:sa:sv:False}

\vspace{2cm}

\noindent\makebox[\textwidth]{\hrulefill}

\vspace{1cm}

\textit{Svar}: \autoref{q:244:sa:sv:True}



\subsection{Ge tv\r{a} exempel p\r{a} diagram som anv\"ands vid modellering (vid programvaruutveckling).}

\label{q:245:sa:sv:False}

\vspace{2cm}

\noindent\makebox[\textwidth]{\hrulefill}

\vspace{1cm}

\textit{Svar}: \autoref{q:245:sa:sv:True}



\subsection{Vad \"ar ett designm\"onster (vid programvaruutveckling)?}

\label{q:246:sa:sv:False}

\vspace{2cm}

\noindent\makebox[\textwidth]{\hrulefill}

\vspace{1cm}

\textit{Svar}: \autoref{q:246:sa:sv:True}



\subsection{Vad utm\"arker glass-box-testning?}

\label{q:247:sa:sv:False}

\vspace{2cm}

\noindent\makebox[\textwidth]{\hrulefill}

\vspace{1cm}

\textit{Svar}: \autoref{q:247:sa:sv:True}



\subsection{Vilka 3 fr\r{a}gor skall varje team-medlem kort besvara vid Daily Scrum-m\"otena?}

\label{q:248:sa:sv:False}

\vspace{2cm}

\noindent\makebox[\textwidth]{\hrulefill}

\vspace{1cm}

\textit{Svar}: \autoref{q:248:sa:sv:True}



\subsection{Vad kallas Scrum-m\"otet, i slutet av en sprint d\"ar ni diskuterar vad som gick bra under den tidigare sprintprocessen och vad som kan f\"orb\"attras inf\"or n\"asta sprint?}

\label{q:249:sa:sv:False}

\vspace{2cm}

\noindent\makebox[\textwidth]{\hrulefill}

\vspace{1cm}

\textit{Svar}: \autoref{q:249:sa:sv:True}



\subsection{Utvecklingsmetoden Scrum har tre olika roller definierade, vilka?}

\label{q:250:sa:sv:False}

\vspace{2cm}

\noindent\makebox[\textwidth]{\hrulefill}

\vspace{1cm}

\textit{Svar}: \autoref{q:250:sa:sv:True}



\subsection{Vad \"ar en abstrakt datatyp (abstract data type)?}

\label{q:251:sa:sv:False}

\vspace{2cm}

\noindent\makebox[\textwidth]{\hrulefill}

\vspace{1cm}

\textit{Svar}: \autoref{q:251:sa:sv:True}



\subsection{Vad k\"annetecknar ett sorterat bin\"art tr\"ad (sorted binary tree) ({\textquotedblright}bin\"art s\"oktr\"ad{\textquotedblright})?}

\label{q:252:sa:sv:False}

\vspace{2cm}

\noindent\makebox[\textwidth]{\hrulefill}

\vspace{1cm}

\textit{Svar}: \autoref{q:252:sa:sv:True}



\subsection{Vilka \"ar de fyra grundl\"aggande datastrukturerna (basic data structures) ut\"over arrayer?}

\label{q:253:sa:sv:False}

\vspace{2cm}

\noindent\makebox[\textwidth]{\hrulefill}

\vspace{1cm}

\textit{Svar}: \autoref{q:253:sa:sv:True}



\subsection{Vad \"ar skillnaden mellan en dynamisk och en statisk datastruktur?}

\label{q:254:sa:sv:False}

\vspace{2cm}

\noindent\makebox[\textwidth]{\hrulefill}

\vspace{1cm}

\textit{Svar}: \autoref{q:254:sa:sv:True}



\subsection{Vad k\"annetecknar ett bin\"art tr\"ad?}

\label{q:255:sa:sv:False}

\vspace{2cm}

\noindent\makebox[\textwidth]{\hrulefill}

\vspace{1cm}

\textit{Svar}: \autoref{q:255:sa:sv:True}



\subsection{Vad \"ar skillnaden mellan en statisk (static) och en dynamisk (dynamic) datastruktur (data structure)?}

\label{q:256:sa:sv:False}

\vspace{2cm}

\noindent\makebox[\textwidth]{\hrulefill}

\vspace{1cm}

\textit{Svar}: \autoref{q:256:sa:sv:True}



\subsection{Kan en lista implementeras som en statisk eller dynamisk datastruktur, b\r{a}de och, eller varken eller? Motivera ditt svar!}

\label{q:257:sa:sv:False}

\vspace{2cm}

\noindent\makebox[\textwidth]{\hrulefill}

\vspace{1cm}

\textit{Svar}: \autoref{q:257:sa:sv:True}



\subsection{Beskriv de grundl\"aggande datastrukturerna stack (stack) och k\"o (queue)?}

\label{q:258:sa:sv:False}

\vspace{2cm}

\noindent\makebox[\textwidth]{\hrulefill}

\vspace{1cm}

\textit{Svar}: \autoref{q:258:sa:sv:True}



\subsection{Kan l\r{a}gniv\r{a}-datastrukturen array anv\"andas f\"or att implementera en k\"o (queue)? Motivera ditt svar!}

\label{q:259:sa:sv:False}

\vspace{2cm}

\noindent\makebox[\textwidth]{\hrulefill}

\vspace{1cm}

\textit{Svar}: \autoref{q:259:sa:sv:True}



\subsection{Vad \"ar en abstrakt datastruktur? Vad \"ar skillnaden mot en record/struct?}

\label{q:260:sa:sv:False}

\vspace{2cm}

\noindent\makebox[\textwidth]{\hrulefill}

\vspace{1cm}

\textit{Svar}: \autoref{q:260:sa:sv:True}



\subsection{Ge ett exempel p\r{a} en datastruktur som anv\"ander principen LIFO och en datastruktur som anv\"ander principen FIFO?}

\label{q:261:sa:sv:False}

\vspace{2cm}

\noindent\makebox[\textwidth]{\hrulefill}

\vspace{1cm}

\textit{Svar}: \autoref{q:261:sa:sv:True}



\subsection{Listor kan lagras antingen i sammanh\"angande block i minnet, eller i form av l\"ankade listor. Vilket \"ar att f\"oredra f\"or statiska listor, och vilket \"ar b\"attre f\"or dynamiska listor?}

\label{q:262:sa:sv:False}

\vspace{2cm}

\noindent\makebox[\textwidth]{\hrulefill}

\vspace{1cm}

\textit{Svar}: \autoref{q:262:sa:sv:True}



\subsection{F\"orklara vad en pekare (pointer) \"ar?}

\label{q:263:sa:sv:False}

\vspace{2cm}

\noindent\makebox[\textwidth]{\hrulefill}

\vspace{1cm}

\textit{Svar}: \autoref{q:263:sa:sv:True}



\subsection{Tv\r{a} typer av specialiserade listor \"ar stack och k\"o, beskriv hur de skiljer sig fr\r{a}n varandra!}

\label{q:264:sa:sv:False}

\vspace{2cm}

\noindent\makebox[\textwidth]{\hrulefill}

\vspace{1cm}

\textit{Svar}: \autoref{q:264:sa:sv:True}



\subsection{Vad skiljer en abstrakt datatyp (abstract data type) fr\r{a}n en sammansatt datatyp (aggregate type / struct / record)?}

\label{q:265:sa:sv:False}

\vspace{2cm}

\noindent\makebox[\textwidth]{\hrulefill}

\vspace{1cm}

\textit{Svar}: \autoref{q:265:sa:sv:True}



\subsection{I en variant av listor l\"agger man till och tar bort element i samma \"ande, vad kallas den datastrukturen? I en annan variant l\"agger man till element i ena \"anden och tar bort i den andra, vad kallas den datastrukturen?}

\label{q:266:sa:sv:False}

\vspace{2cm}

\noindent\makebox[\textwidth]{\hrulefill}

\vspace{1cm}

\textit{Svar}: \autoref{q:266:sa:sv:True}



\subsection{Vad k\"annetecknar en aggregattyp (struct/record)?}

\label{q:267:sa:sv:False}

\vspace{2cm}

\noindent\makebox[\textwidth]{\hrulefill}

\vspace{1cm}

\textit{Svar}: \autoref{q:267:sa:sv:True}



\subsection{Kan en k\"o implementeras som en statisk eller dynamisk datastruktur, b\r{a}de och, eller varken eller? Motivera ditt svar!}

\label{q:268:sa:sv:False}

\vspace{2cm}

\noindent\makebox[\textwidth]{\hrulefill}

\vspace{1cm}

\textit{Svar}: \autoref{q:268:sa:sv:True}



\subsection{Vad k\"annetecknar en dynamisk datastruktur till skillnad fr\r{a}n en statiskt datastruktur?}

\label{q:269:sa:sv:False}

\vspace{2cm}

\noindent\makebox[\textwidth]{\hrulefill}

\vspace{1cm}

\textit{Svar}: \autoref{q:269:sa:sv:True}



\subsection{Vad k\"annetecknar datastrukturen bin\"art tr\"ad?}

\label{q:270:sa:sv:False}

\vspace{2cm}

\noindent\makebox[\textwidth]{\hrulefill}

\vspace{1cm}

\textit{Svar}: \autoref{q:270:sa:sv:True}



\subsection{Vad k\"annetecknar rotnoden i en tr\"ad-datastruktur?}

\label{q:271:sa:sv:False}

\vspace{2cm}

\noindent\makebox[\textwidth]{\hrulefill}

\vspace{1cm}

\textit{Svar}: \autoref{q:271:sa:sv:True}



\subsection{Beskriv en f\"ordel och en nackdel med att lagra en aggregattyp (struct/record) i ett sammanh\"angande block ist\"allet f\"or de olika delarna p\r{a} separata platser utpekade av pekare.}

\label{q:272:sa:sv:False}

\vspace{2cm}

\noindent\makebox[\textwidth]{\hrulefill}

\vspace{1cm}

\textit{Svar}: \autoref{q:272:sa:sv:True}



\subsection{Vad k\"annetecknar en array?}

\label{q:273:sa:sv:False}

\vspace{2cm}

\noindent\makebox[\textwidth]{\hrulefill}

\vspace{1cm}

\textit{Svar}: \autoref{q:273:sa:sv:True}



\subsection{Vad k\"annetecknar en statiskt datastruktur?}

\label{q:274:sa:sv:False}

\vspace{2cm}

\noindent\makebox[\textwidth]{\hrulefill}

\vspace{1cm}

\textit{Svar}: \autoref{q:274:sa:sv:True}



\subsection{Vad k\"annetecknar en dynamisk datastruktur?}

\label{q:275:sa:sv:False}

\vspace{2cm}

\noindent\makebox[\textwidth]{\hrulefill}

\vspace{1cm}

\textit{Svar}: \autoref{q:275:sa:sv:True}



\subsection{Beskriv en f\"ordel och en nackdel med att lagra de olika delarna av en aggregattyp (struct/record) p\r{a} separata platser utpekade av pekare ist\"allet f\"or i ett sammanh\"angande block.}

\label{q:276:sa:sv:False}

\vspace{2cm}

\noindent\makebox[\textwidth]{\hrulefill}

\vspace{1cm}

\textit{Svar}: \autoref{q:276:sa:sv:True}



\subsection{Vad \"ar ett databashanteringssystem (database management system)?}

\label{q:277:sa:sv:False}

\vspace{2cm}

\noindent\makebox[\textwidth]{\hrulefill}

\vspace{1cm}

\textit{Svar}: \autoref{q:277:sa:sv:True}



\subsection{Vad inneb\"ar commit och rollback i databas-sammanhang?}

\label{q:278:sa:sv:False}

\vspace{2cm}

\noindent\makebox[\textwidth]{\hrulefill}

\vspace{1cm}

\textit{Svar}: \autoref{q:278:sa:sv:True}



\subsection{F\"or relationsdabaser finns det tre (3) operationer (relational operations), med vars hj\"alp man kan skapa nya tabeller som utg\"or delm\"angder och/eller kombinationer av befintliga tabeller. Vilka operationer?}

\label{q:279:sa:sv:False}

\vspace{2cm}

\noindent\makebox[\textwidth]{\hrulefill}

\vspace{1cm}

\textit{Svar}: \autoref{q:279:sa:sv:True}



\subsection{Vad inneb\"ar data mining?}

\label{q:280:sa:sv:False}

\vspace{2cm}

\noindent\makebox[\textwidth]{\hrulefill}

\vspace{1cm}

\textit{Svar}: \autoref{q:280:sa:sv:True}



\subsection{Vad \"ar ett data warehouse?}

\label{q:281:sa:sv:False}

\vspace{2cm}

\noindent\makebox[\textwidth]{\hrulefill}

\vspace{1cm}

\textit{Svar}: \autoref{q:281:sa:sv:True}



\subsection{Vad \"ar en databas (database) i f\"orh\r{a}llande till ett databashanteringssystem (DBMS {\textendash} database management system)?}

\label{q:282:sa:sv:False}

\vspace{2cm}

\noindent\makebox[\textwidth]{\hrulefill}

\vspace{1cm}

\textit{Svar}: \autoref{q:282:sa:sv:True}



\subsection{N\"amn ett vanligt problem som kan uppst\r{a} vid t ex \"overf\"oringar mellan konton som transaktioner skyddar mot.}

\label{q:283:sa:sv:False}

\vspace{2cm}

\noindent\makebox[\textwidth]{\hrulefill}

\vspace{1cm}

\textit{Svar}: \autoref{q:283:sa:sv:True}



\subsection{Vad \"ar SQL?}

\label{q:284:sa:sv:False}

\vspace{2cm}

\noindent\makebox[\textwidth]{\hrulefill}

\vspace{1cm}

\textit{Svar}: \autoref{q:284:sa:sv:True}



\subsection{P\r{a} vilka tv\r{a} s\"att kan en transaktion avslutas?}

\label{q:285:sa:sv:False}

\vspace{2cm}

\noindent\makebox[\textwidth]{\hrulefill}

\vspace{1cm}

\textit{Svar}: \autoref{q:285:sa:sv:True}



\subsection{Vad \"ar en transaktion?}

\label{q:286:sa:sv:False}

\vspace{2cm}

\noindent\makebox[\textwidth]{\hrulefill}

\vspace{1cm}

\textit{Svar}: \autoref{q:286:sa:sv:True}



\subsection{En transaktion kan avslutas p\r{a} tv\r{a} olika s\"att, vilka?}

\label{q:287:sa:sv:False}

\vspace{2cm}

\noindent\makebox[\textwidth]{\hrulefill}

\vspace{1cm}

\textit{Svar}: \autoref{q:287:sa:sv:True}



\subsection{Vad \"ar ett databas-schema?}

\label{q:288:sa:sv:False}

\vspace{2cm}

\noindent\makebox[\textwidth]{\hrulefill}

\vspace{1cm}

\textit{Svar}: \autoref{q:288:sa:sv:True}



\subsection{Vad \"ar en databas?}

\label{q:289:sa:sv:False}

\vspace{2cm}

\noindent\makebox[\textwidth]{\hrulefill}

\vspace{1cm}

\textit{Svar}: \autoref{q:289:sa:sv:True}



\subsection{Vad \"ar en databasmodell?}

\label{q:290:sa:sv:False}

\vspace{2cm}

\noindent\makebox[\textwidth]{\hrulefill}

\vspace{1cm}

\textit{Svar}: \autoref{q:290:sa:sv:True}



\subsection{N\"amn tv\r{a} saker som skiljer en objektorienterad databas fr\r{a}n en relationsdatabas?}

\label{q:291:sa:sv:False}

\vspace{2cm}

\noindent\makebox[\textwidth]{\hrulefill}

\vspace{1cm}

\textit{Svar}: \autoref{q:291:sa:sv:True}



\subsection{Vad \"ar data mining?}

\label{q:292:sa:sv:False}

\vspace{2cm}

\noindent\makebox[\textwidth]{\hrulefill}

\vspace{1cm}

\textit{Svar}: \autoref{q:292:sa:sv:True}



\subsection{Vad representerar en tabell i relationsmodellen f\"or databaser?}

\label{q:293:sa:sv:False}

\vspace{2cm}

\noindent\makebox[\textwidth]{\hrulefill}

\vspace{1cm}

\textit{Svar}: \autoref{q:293:sa:sv:True}



\subsection{Vad representerar en kolumn i en tabell i relationsmodellen f\"or databaser?}

\label{q:294:sa:sv:False}

\vspace{2cm}

\noindent\makebox[\textwidth]{\hrulefill}

\vspace{1cm}

\textit{Svar}: \autoref{q:294:sa:sv:True}



\subsection{Vad representerar en rad i en tabell i relationsmodellen f\"or databaser?}

\label{q:295:sa:sv:False}

\vspace{2cm}

\noindent\makebox[\textwidth]{\hrulefill}

\vspace{1cm}

\textit{Svar}: \autoref{q:295:sa:sv:True}



\subsection{Vilka \"ar de tre relationsoperationerna i relationsmodellen f\"or databaser?}

\label{q:296:sa:sv:False}

\vspace{2cm}

\noindent\makebox[\textwidth]{\hrulefill}

\vspace{1cm}

\textit{Svar}: \autoref{q:296:sa:sv:True}



\subsection{Vilka \"ar de tre grundl\"aggande relationsoperationerna f\"or att ta fram efterfr\r{a}gat data fr\r{a}n en relationsdatabas?}

\label{q:297:sa:sv:False}

\vspace{2cm}

\noindent\makebox[\textwidth]{\hrulefill}

\vspace{1cm}

\textit{Svar}: \autoref{q:297:sa:sv:True}



\subsection{Vad \"ar ett schema i samband med ett databassystem?}

\label{q:298:sa:sv:False}

\vspace{2cm}

\noindent\makebox[\textwidth]{\hrulefill}

\vspace{1cm}

\textit{Svar}: \autoref{q:298:sa:sv:True}



\subsection{Till vilken programmeringsparadigm h\"or databasfr\r{a}gespr\r{a}ket SQL (structured query language)?}

\label{q:299:sa:sv:False}

\vspace{2cm}

\noindent\makebox[\textwidth]{\hrulefill}

\vspace{1cm}

\textit{Svar}: \autoref{q:299:sa:sv:True}



\subsection{Varf\"or \"ar det av intresse att k\"anna till en algoritms effektivitetsklass/komplexitetsklass?}

\label{q:300:sa:sv:False}

\vspace{2cm}

\noindent\makebox[\textwidth]{\hrulefill}

\vspace{1cm}

\textit{Svar}: \autoref{q:300:sa:sv:True}



\subsection{Processen att skapa 3D-grafik best\r{a}r av tre steg, varav det f\"orsta \"ar 3D-modellering (3D modeling), och det tredje \"ar bildvisning (display). Vad kallas det andra steget, och vad g\"ors i det steget?}

\label{q:301:sa:sv:False}

\vspace{2cm}

\noindent\makebox[\textwidth]{\hrulefill}

\vspace{1cm}

\textit{Svar}: \autoref{q:301:sa:sv:True}



\subsection{I animationsprojekt utf\"or man arbetet vanligtvis i tre steg, vilka?}

\label{q:302:sa:sv:False}

\vspace{2cm}

\noindent\makebox[\textwidth]{\hrulefill}

\vspace{1cm}

\textit{Svar}: \autoref{q:302:sa:sv:True}



\subsection{Tv\r{a} grenar inom omr\r{a}det mekanik har visat sig s\"arskilt anv\"andbara vid simulering av naturliga r\"orelser, vilka?}

\label{q:303:sa:sv:False}

\vspace{2cm}

\noindent\makebox[\textwidth]{\hrulefill}

\vspace{1cm}

\textit{Svar}: \autoref{q:303:sa:sv:True}



\subsection{N\"amn ett s\"att att ta fram s.k. polygonal meshes vid 3D-modellering!}

\label{q:304:sa:sv:False}

\vspace{2cm}

\noindent\makebox[\textwidth]{\hrulefill}

\vspace{1cm}

\textit{Svar}: \autoref{q:304:sa:sv:True}



\subsection{F\"orklara kortfattat skillnaden mellan lokala ljusmodeller (local lightning model) och globala ljusmodeller (global lightning model). Vilken modell ger mest realistiskt resultat? F\"ordelen med den andra?}

\label{q:305:sa:sv:False}

\vspace{2cm}

\noindent\makebox[\textwidth]{\hrulefill}

\vspace{1cm}

\textit{Svar}: \autoref{q:305:sa:sv:True}



\subsection{Inom datorgrafik spelar ljus en viktig roll. Ljus brukar delas in i tre (3) olika sorter, vilka? Vad skiljer dem \r{a}t?}

\label{q:306:sa:sv:False}

\vspace{2cm}

\noindent\makebox[\textwidth]{\hrulefill}

\vspace{1cm}

\textit{Svar}: \autoref{q:306:sa:sv:True}



\subsection{F\"orklara hur begreppen frame, key frame och in-betweening som anv\"ands inom animation h\"anger ihop?}

\label{q:307:sa:sv:False}

\vspace{2cm}

\noindent\makebox[\textwidth]{\hrulefill}

\vspace{1cm}

\textit{Svar}: \autoref{q:307:sa:sv:True}



\subsection{Processen att skapa 3D-grafik best\r{a}r av tv\r{a} huvudsteg, vilka?}

\label{q:308:sa:sv:False}

\vspace{2cm}

\noindent\makebox[\textwidth]{\hrulefill}

\vspace{1cm}

\textit{Svar}: \autoref{q:308:sa:sv:True}



\subsection{Vad k\"annetecknar en lokal belysningsmodell (local lighting model) inom datorgrafik?}

\label{q:309:sa:sv:False}

\vspace{2cm}

\noindent\makebox[\textwidth]{\hrulefill}

\vspace{1cm}

\textit{Svar}: \autoref{q:309:sa:sv:True}



\subsection{Vad k\"annetecknar en global belysningsmodell (global lighting model) inom datorgrafik?}

\label{q:310:sa:sv:False}

\vspace{2cm}

\noindent\makebox[\textwidth]{\hrulefill}

\vspace{1cm}

\textit{Svar}: \autoref{q:310:sa:sv:True}



\subsection{M\r{a}nga sv\r{a}ra problem kan beskrivas som s\"okproblem, vilket inneb\"ar att man s\"oker efter en l\"osning i ett s\"oktr\"ad. F\"or att v\"alja s\"okv\"ag i s\"oktr\"adet anv\"ander man sig av {\textquotedblright}tumregler{\textquotedblright} (rules of thumb). Vad kallas s\r{a}dana tumregler och varf\"or beh\"ovs de?}

\label{q:311:sa:sv:False}

\vspace{2cm}

\noindent\makebox[\textwidth]{\hrulefill}

\vspace{1cm}

\textit{Svar}: \autoref{q:311:sa:sv:True}



\subsection{Vad \"ar skillnaden mellan svag (weak) AI och stark (strong) AI?}

\label{q:312:sa:sv:False}

\vspace{2cm}

\noindent\makebox[\textwidth]{\hrulefill}

\vspace{1cm}

\textit{Svar}: \autoref{q:312:sa:sv:True}



\subsection{Ett s\"att att klassificera maskininl\"arningansatser (machine/computer learning approaches) \"ar genom i vilken grad de kr\"aver m\"ansklig inblandning. Vilka tre s\r{a}dana klasser brukar man prata om?}

\label{q:313:sa:sv:False}

\vspace{2cm}

\noindent\makebox[\textwidth]{\hrulefill}

\vspace{1cm}

\textit{Svar}: \autoref{q:313:sa:sv:True}



\subsection{Vad \"ar ett artificiellt neuralt n\"atverk (artificial neural network) och hur f\"or\"andras ett s\r{a}dant n\"atverk under inl\"arning?}

\label{q:314:sa:sv:False}

\vspace{2cm}

\noindent\makebox[\textwidth]{\hrulefill}

\vspace{1cm}

\textit{Svar}: \autoref{q:314:sa:sv:True}



\subsection{Vad \"ar skillnaden mellan \"overvakad inl\"arning (supervised learning) och o\"overvakad inl\"arning (unsupervised learning)?}

\label{q:315:sa:sv:False}

\vspace{2cm}

\noindent\makebox[\textwidth]{\hrulefill}

\vspace{1cm}

\textit{Svar}: \autoref{q:315:sa:sv:True}



\subsection{\"Ar reinforcement learning en typ av \"overvakad inl\"arning (supervised learning) eller inte? Varf\"or?}

\label{q:316:sa:sv:False}

\vspace{2cm}

\noindent\makebox[\textwidth]{\hrulefill}

\vspace{1cm}

\textit{Svar}: \autoref{q:316:sa:sv:True}



\subsection{Ett neuralt n\"atverk \"ar en ber\"akningsmodell som inspirerats av hur den m\"anskliga hj\"arnan fungerar. Hur l\"ar sig ett neuralt n\"atverk fr\r{a}n exempeldata?}

\label{q:317:sa:sv:False}

\vspace{2cm}

\noindent\makebox[\textwidth]{\hrulefill}

\vspace{1cm}

\textit{Svar}: \autoref{q:317:sa:sv:True}



\subsection{F\"orklara kortfattat begreppen {\textquotedblright}information retrieval{\textquotedblright} och {\textquotedblright}information extraction{\textquotedblright} inom spr\r{a}kanalys (natural language processing)!}

\label{q:318:sa:sv:False}

\vspace{2cm}

\noindent\makebox[\textwidth]{\hrulefill}

\vspace{1cm}

\textit{Svar}: \autoref{q:318:sa:sv:True}



\subsection{Vilka tre typer av lager (layers) finns i ett neuronn\"atverks (neural network) topologi?}

\label{q:319:sa:sv:False}

\vspace{2cm}

\noindent\makebox[\textwidth]{\hrulefill}

\vspace{1cm}

\textit{Svar}: \autoref{q:319:sa:sv:True}



\subsection{Vad \"ar ett s\"oktr\"ad inom AI?}

\label{q:320:sa:sv:False}

\vspace{2cm}

\noindent\makebox[\textwidth]{\hrulefill}

\vspace{1cm}

\textit{Svar}: \autoref{q:320:sa:sv:True}



\subsection{Vid behandling av naturligt spr\r{a}k utf\"ors tre olika typer av analyser, vilka?}

\label{q:321:sa:sv:False}

\vspace{2cm}

\noindent\makebox[\textwidth]{\hrulefill}

\vspace{1cm}

\textit{Svar}: \autoref{q:321:sa:sv:True}



\subsection{Vad \"ar Turing-testet?}

\label{q:322:sa:sv:False}

\vspace{2cm}

\noindent\makebox[\textwidth]{\hrulefill}

\vspace{1cm}

\textit{Svar}: \autoref{q:322:sa:sv:True}



\subsection{Vad \"ar definitionen av en intelligent agent (inom AI)?}

\label{q:323:sa:sv:False}

\vspace{2cm}

\noindent\makebox[\textwidth]{\hrulefill}

\vspace{1cm}

\textit{Svar}: \autoref{q:323:sa:sv:True}



\subsection{Vad k\"annetecknar supervised (machine) learning?}

\label{q:324:sa:sv:False}

\vspace{2cm}

\noindent\makebox[\textwidth]{\hrulefill}

\vspace{1cm}

\textit{Svar}: \autoref{q:324:sa:sv:True}



\subsection{Vad k\"annetecknar (machine) learning by reinforcement?}

\label{q:325:sa:sv:False}

\vspace{2cm}

\noindent\makebox[\textwidth]{\hrulefill}

\vspace{1cm}

\textit{Svar}: \autoref{q:325:sa:sv:True}



\subsection{Vad \"ar definitionen av en intelligent agent?}

\label{q:326:sa:sv:False}

\vspace{2cm}

\noindent\makebox[\textwidth]{\hrulefill}

\vspace{1cm}

\textit{Svar}: \autoref{q:326:sa:sv:True}



\subsection{Vad \"ar definitionen av en intelligent agent?}

\label{q:327:sa:sv:False}

\vspace{2cm}

\noindent\makebox[\textwidth]{\hrulefill}

\vspace{1cm}

\textit{Svar}: \autoref{q:327:sa:sv:True}



\subsection{Vad \"ar s\"ok-heuristik (search heuristics), och vad k\"annetecknar bra s\"ok-heuristik?}

\label{q:328:sa:sv:False}

\vspace{2cm}

\noindent\makebox[\textwidth]{\hrulefill}

\vspace{1cm}

\textit{Svar}: \autoref{q:328:sa:sv:True}



\subsection{Varf\"or beh\"ovs s\"ok-heurestik n\"ar man s\"oker i ett s\"ok-tr\"ad?}

\label{q:329:sa:sv:False}

\vspace{2cm}

\noindent\makebox[\textwidth]{\hrulefill}

\vspace{1cm}

\textit{Svar}: \autoref{q:329:sa:sv:True}



\subsection{Vad \"ar skillnaden mellan en tillst\r{a}ndsgraf och ett s\"oktr\"ad?}

\label{q:330:sa:sv:False}

\vspace{2cm}

\noindent\makebox[\textwidth]{\hrulefill}

\vspace{1cm}

\textit{Svar}: \autoref{q:330:sa:sv:True}



\subsection{Vad \"ar stopp-problemet (the halting problem), och varf\"or \"ar det intressant ur ett ber\"akningsteoretiskt perspektiv?}

\label{q:331:sa:sv:False}

\vspace{2cm}

\noindent\makebox[\textwidth]{\hrulefill}

\vspace{1cm}

\textit{Svar}: \autoref{q:331:sa:sv:True}



\subsection{Ordna f\"oljande komplexitets-/effektivitetsklasser (complexity/efficiency classes) fr\r{a}n den mest effektiva till den minst effektiva: \ensuremath{\Theta}(n^10), \ensuremath{\Theta}(log n), \ensuremath{\Theta}(n), \ensuremath{\Theta}(2^n).}

\label{q:332:sa:sv:False}

\vspace{2cm}

\noindent\makebox[\textwidth]{\hrulefill}

\vspace{1cm}

\textit{Svar}: \autoref{q:332:sa:sv:True}



\subsection{Vad \"ar en Turing-maskin och vad \"ar dess syfte?}

\label{q:333:sa:sv:False}

\vspace{2cm}

\noindent\makebox[\textwidth]{\hrulefill}

\vspace{1cm}

\textit{Svar}: \autoref{q:333:sa:sv:True}



\subsection{Ordna f\"oljande komplexitets-/effektivitetsklasser (complexity/efficiency classes) fr\r{a}n den mest effektiva till den minst effektiva: \ensuremath{\Theta}(n^4), \ensuremath{\Theta}(n), \ensuremath{\Theta}(2^n), \ensuremath{\Theta}(log n).}

\label{q:334:sa:sv:False}

\vspace{2cm}

\noindent\makebox[\textwidth]{\hrulefill}

\vspace{1cm}

\textit{Svar}: \autoref{q:334:sa:sv:True}



\subsection{Vad inneb\"ar det att ett problem \"ar ett polynomiellt problem (polynomial problem) (tillh\"or klassen polynomiella problem)?}

\label{q:335:sa:sv:False}

\vspace{2cm}

\noindent\makebox[\textwidth]{\hrulefill}

\vspace{1cm}

\textit{Svar}: \autoref{q:335:sa:sv:True}



\subsection{\"Ar klassen av polynomiella problem P mindre eller lika med klassen av icke-deterministiskt polynomiella problem NP? Motivera ditt svar!}

\label{q:336:sa:sv:False}

\vspace{2cm}

\noindent\makebox[\textwidth]{\hrulefill}

\vspace{1cm}

\textit{Svar}: \autoref{q:336:sa:sv:True}



\subsection{Givet att komplexiteten f\"or algoritm A \"ar O(n), algoritm B \"ar O(log n), algoritm C \"ar O(n2) och algoritm D \"ar O(n log n2), lista algoritmerna i ordning fr\r{a}n den mest effektiva till den minst effektiva!}

\label{q:337:sa:sv:False}

\vspace{2cm}

\noindent\makebox[\textwidth]{\hrulefill}

\vspace{1cm}

\textit{Svar}: \autoref{q:337:sa:sv:True}



\subsection{Ge exempel p\r{a} tre komplexitetsklasser i O-notation och ordna dessa fr\r{a}n mest effektiv till minst effektiv!}

\label{q:338:sa:sv:False}

\vspace{2cm}

\noindent\makebox[\textwidth]{\hrulefill}

\vspace{1cm}

\textit{Svar}: \autoref{q:338:sa:sv:True}



\subsection{Varf\"or \"ar stopp-problemet (the halting problem) intressant ur ett ber\"akningsteoretiskt perspektiv?}

\label{q:339:sa:sv:False}

\vspace{2cm}

\noindent\makebox[\textwidth]{\hrulefill}

\vspace{1cm}

\textit{Svar}: \autoref{q:339:sa:sv:True}



\subsection{Vad k\"annetecknar stopp-problemet (inom ber\"akningsteori).}

\label{q:340:sa:sv:False}

\vspace{2cm}

\noindent\makebox[\textwidth]{\hrulefill}

\vspace{1cm}

\textit{Svar}: \autoref{q:340:sa:sv:True}



\subsection{Vad inneb\"ar det att en funktion \"ar ber\"akningsbar?}

\label{q:341:sa:sv:False}

\vspace{2cm}

\noindent\makebox[\textwidth]{\hrulefill}

\vspace{1cm}

\textit{Svar}: \autoref{q:341:sa:sv:True}



\subsection{Uppfyller icke-detministiska algoritmer definitionen av en algoritm? Motivera ditt svar!}

\label{q:342:sa:sv:False}

\vspace{2cm}

\noindent\makebox[\textwidth]{\hrulefill}

\vspace{1cm}

\textit{Svar}: \autoref{q:342:sa:sv:True}



\subsection{Vad vet vi om f\"orh\r{a}llandet mellan polynomiella problem P och icke-deterministiskt polynomiella problem NP?}

\label{q:343:sa:sv:False}

\vspace{2cm}

\noindent\makebox[\textwidth]{\hrulefill}

\vspace{1cm}

\textit{Svar}: \autoref{q:343:sa:sv:True}



\subsection{Vad skiljer en deterministisk algoritm fr\r{a}n en icke-deterministisk?}

\label{q:344:sa:sv:False}

\vspace{2cm}

\noindent\makebox[\textwidth]{\hrulefill}

\vspace{1cm}

\textit{Svar}: \autoref{q:344:sa:sv:True}



\subsection{Vad \"ar syftet med Turing-maskiner?}

\label{q:345:sa:sv:False}

\vspace{2cm}

\noindent\makebox[\textwidth]{\hrulefill}

\vspace{1cm}

\textit{Svar}: \autoref{q:345:sa:sv:True}



\subsection{P\r{a} vilka tv\r{a} s\"att kan en transaktion avslutas?}

\label{q:346:sa:sv:False}

\vspace{2cm}

\noindent\makebox[\textwidth]{\hrulefill}

\vspace{1cm}

\textit{Svar}: \autoref{q:346:sa:sv:True}



\subsection{Antag att vi har f\"oljande bitm\"onster och att de representerar heltal enligt tv\r{a}komplementsnotation (two{\textquoteright}s complement notation): "0111 1111, 1111 1001, 1011 1111, 0010 0100, 1000 0001" - Vilket av dessa bitm\"onster representerar det minsta heltalet?}

\label{q:34800:mc:sv:False}

\begin{itemize}
  \item[$\bigcirc$] 0111 1111
  \item[$\bigcirc$] 1111 1001
  \item[$\bigcirc$] 1011 1111
  \item[$\bigcirc$] 1000 0001
\end{itemize}

\vspace{1cm}

\textit{Svar}: \autoref{q:34800:mc:sv:True}

\subsection{Antag att vi har f\"oljande bitm\"onster och att de representerar heltal enligt tv\r{a}komplementsnotation (two{\textquoteright}s complement notation): "0111 1111, 1111 1001, 1011 1111, 0010 0100, 1000 0001" - Vilket av dessa bitm\"onster representerar det st\"orsta heltalet?}

\label{q:3480001:mc:sv:False}

\begin{itemize}
  \item[$\bigcirc$] 1111 1001
  \item[$\bigcirc$] 1011 1111
  \item[$\bigcirc$] 1000 0001
  \item[$\bigcirc$] 0010 0100
\end{itemize}

\vspace{1cm}

\textit{Svar}: \autoref{q:3480001:mc:sv:True}



\subsection{Antag att 00FF00 \"ar den hexadecimala notationen f\"or ett bitm\"onster som representerar en pixel enligt RGB-standarden. - Vad har denna pixel f\"or f\"argdjup (color depth)?}

\label{q:34900:sa:sv:False}

\vspace{2cm}

\noindent\makebox[\textwidth]{\hrulefill}

\vspace{1cm}

\textit{Svar}: \autoref{q:34900:sa:sv:True}

\subsection{Antag att 00FF00 \"ar den hexadecimala notationen f\"or ett bitm\"onster som representerar en pixel enligt RGB-standarden. - Vilken av f\"oljande f\"arger har den pixeln?}

\label{q:3490001:mc:sv:False}

\begin{itemize}
  \item[$\bigcirc$] Vit
  \item[$\bigcirc$] Svart
  \item[$\bigcirc$] R\"od
  \item[$\bigcirc$] Bl\r{a}
  \item[$\bigcirc$] Gul
  \item[$\bigcirc$] Cyan
  \item[$\bigcirc$] Magenta
\end{itemize}

\vspace{1cm}

\textit{Svar}: \autoref{q:3490001:mc:sv:True}



\subsection{Antag att vi har f\"oljande bitm\"onster och att de representerar heltal enligt tv\r{a}komplementsnotation (two{\textquoteright}s complement notation):1111 1110 0111 1111 0000 0000 0000 0001 1000 0000 1111 1111 - Vilket av dessa bitm\"onster representerar talet -1 (minus ett)?}

\label{q:35000:mc:sv:False}

\begin{itemize}
  \item[$\bigcirc$] 1111 1110
  \item[$\bigcirc$] 0111 1111
  \item[$\bigcirc$] 0000 0000
  \item[$\bigcirc$] 0000 0001
  \item[$\bigcirc$] 1000 0000
\end{itemize}

\vspace{1cm}

\textit{Svar}: \autoref{q:35000:mc:sv:True}

\subsection{Antag att vi har f\"oljande bitm\"onster och att de representerar heltal enligt tv\r{a}komplementsnotation (two{\textquoteright}s complement notation):1111 1110 0111 1111 0000 0000 0000 0001 1000 0000 1111 1111 - Vilket av dessa bitm\"onster representerar talet 1 (ett)?}

\label{q:3500001:mc:sv:False}

\begin{itemize}
  \item[$\bigcirc$] 1111 1110
  \item[$\bigcirc$] 0111 1111
  \item[$\bigcirc$] 0000 0000
  \item[$\bigcirc$] 1000 0000
  \item[$\bigcirc$] 1111 1111
\end{itemize}

\vspace{1cm}

\textit{Svar}: \autoref{q:3500001:mc:sv:True}



\subsection{Antag att vi har f\"oljande bitm\"onster och att de representerar heltal enligt tv\r{a}komplementsnotation (two{\textquoteright}s complement notation):1111 0100 0111 0101 0000 1010 0000 1011 1000 1010 1111 0101 - Vilket av dessa bitm\"onster representerar det st\"orsta talet?}

\label{q:35100:mc:sv:False}

\begin{itemize}
  \item[$\bigcirc$] 0111 0011
  \item[$\bigcirc$] 0111 0001
  \item[$\bigcirc$] 0110 1111
\end{itemize}

\vspace{1cm}

\textit{Svar}: \autoref{q:35100:mc:sv:True}

\subsection{Antag att vi har f\"oljande bitm\"onster och att de representerar heltal enligt tv\r{a}komplementsnotation (two{\textquoteright}s complement notation):1111 0100 0111 0101 0000 1010 0000 1011 1000 1010 1111 0101 - Vilket av dessa bitm\"onster representerar det minsta talet?}

\label{q:3510001:mc:sv:False}

\begin{itemize}
  \item[$\bigcirc$] 1000 1011
  \item[$\bigcirc$] 1001 0010
  \item[$\bigcirc$] 1011 1101
\end{itemize}

\vspace{1cm}

\textit{Svar}: \autoref{q:3510001:mc:sv:True}



\subsection{Antag att vi har f\"oljande bitm\"onster: 1000 0011. - Vilket decimalt naturligt tal (noll eller positivt heltal) (unsigned integer) representerar bitm\"ontret ovan?}

\label{q:35200:sa:sv:False}

\vspace{2cm}

\noindent\makebox[\textwidth]{\hrulefill}

\vspace{1cm}

\textit{Svar}: \autoref{q:35200:sa:sv:True}

\subsection{Antag att vi har f\"oljande bitm\"onster: 1000 0011. - Vilket decimalt heltal (signed integer) representerar bitm\"ontret ovan enligt tv\r{a}komplementsnotation (two{\textquoteright}s complement notation)?}

\label{q:3520001:sa:sv:False}

\vspace{2cm}

\noindent\makebox[\textwidth]{\hrulefill}

\vspace{1cm}

\textit{Svar}: \autoref{q:3520001:sa:sv:True}



\subsection{Antag att vi har f\"oljande bitm\"onster och att de representerar heltal enligt tv\r{a}komplementsnotation (two{\textquoteright}s complement notation):0111 0100, 0010 1001, 1100 0010, 1100 0100, 0011 0001 - Vilket av dessa bitm\"onster representerar det st\"orsta talet?}

\label{q:35300:mc:sv:False}

\begin{itemize}
  \item[$\bigcirc$] 0010 1001
  \item[$\bigcirc$] 1100 0010
  \item[$\bigcirc$] 1100 0100
  \item[$\bigcirc$] 0011 0001
\end{itemize}

\vspace{1cm}

\textit{Svar}: \autoref{q:35300:mc:sv:True}

\subsection{Antag att vi har f\"oljande bitm\"onster och att de representerar heltal enligt tv\r{a}komplementsnotation (two{\textquoteright}s complement notation):0111 0100, 0010 1001, 1100 0010, 1100 0100, 0011 0001 - Vilket av dessa bitm\"onster representerar det minsta talet?}

\label{q:3530001:mc:sv:False}

\begin{itemize}
  \item[$\bigcirc$] 0010 1001
  \item[$\bigcirc$] 1100 0010
  \item[$\bigcirc$] 1100 0100
  \item[$\bigcirc$] 0011 0001
\end{itemize}

\vspace{1cm}

\textit{Svar}: \autoref{q:3530001:mc:sv:True}



\subsection{Antag att RGB-f\"argkoden f\"or en pixel \"ar CC3300 p\r{a} hexadecimal form (basen 16), att pixeln ing\r{a}r i ett foto taget med en 6 megapixel-kamera, och att fotot \"ar lagrat som en bitmap (d.v.s. okomprimerad). - Ange pixelns f\"argv\"arden f\"or R (r\"ott), G (gr\"ont) och B (bl\r{a}tt) p\r{a} decimal form (basen 10)?}

\label{q:35400:sa:sv:False}

\vspace{2cm}

\noindent\makebox[\textwidth]{\hrulefill}

\vspace{1cm}

\textit{Svar}: \autoref{q:35400:sa:sv:True}

\subsection{Antag att RGB-f\"argkoden f\"or en pixel \"ar CC3300 p\r{a} hexadecimal form (basen 16), att pixeln ing\r{a}r i ett foto taget med en 6 megapixel-kamera, och att fotot \"ar lagrat som en bitmap (d.v.s. okomprimerad). - Vad \"ar pixelns f\"argdjup?}

\label{q:3540001:sa:sv:False}

\vspace{2cm}

\noindent\makebox[\textwidth]{\hrulefill}

\vspace{1cm}

\textit{Svar}: \autoref{q:3540001:sa:sv:True}

\subsection{Antag att RGB-f\"argkoden f\"or en pixel \"ar CC3300 p\r{a} hexadecimal form (basen 16), att pixeln ing\r{a}r i ett foto taget med en 6 megapixel-kamera, och att fotot \"ar lagrat som en bitmap (d.v.s. okomprimerad). - Hur stor plats tar lagringen av fotot i MB (mega-byte)?}

\label{q:354000102:sa:sv:False}

\vspace{2cm}

\noindent\makebox[\textwidth]{\hrulefill}

\vspace{1cm}

\textit{Svar}: \autoref{q:354000102:sa:sv:True}



\subsection{Antag att vi har f\"oljande bitm\"onster: 1010 1010, 1100 1100, 1001 0000 och 1001 1111. - Om bitm\"onstren ovan representerar naturliga tal (unsigned integers), vilket bitm\"onster representerar d\r{a} det minsta talet?}

\label{q:35500:mc:sv:False}

\begin{itemize}
  \item[$\bigcirc$] 1010 1010
  \item[$\bigcirc$] 1100 1100
  \item[$\bigcirc$] 1001 1111
\end{itemize}

\vspace{1cm}

\textit{Svar}: \autoref{q:35500:mc:sv:True}

\subsection{Antag att vi har f\"oljande bitm\"onster: 1010 1010, 1100 1100, 1001 0000 och 1001 1111. - Om bitm\"onstren ovan representerar heltal enligt tv\r{a}komplementsnotation (two{\textquoteright}s complement notation), vilket bitm\"onster representerar d\r{a} det minsta talet?}

\label{q:3550001:mc:sv:False}

\begin{itemize}
  \item[$\bigcirc$] 1010 1010
  \item[$\bigcirc$] 1100 1100
  \item[$\bigcirc$] 1001 1111
\end{itemize}

\vspace{1cm}

\textit{Svar}: \autoref{q:3550001:mc:sv:True}





\subsection{Beskriv det decimala talet 9 som ett bin\"art tal representerat med 8-bitar (8 bit unsigned integer).}

\label{q:357:sa:sv:False}

\vspace{2cm}

\noindent\makebox[\textwidth]{\hrulefill}

\vspace{1cm}

\textit{Svar}: \autoref{q:357:sa:sv:True}



\subsection{Beskriv talet -1 (minus ett) som ett 8-bitars bitm\"onster enligt tv\r{a}komplementsnotation (two{\textquoteright}s complement notation).}

\label{q:358:sa:sv:False}

\vspace{2cm}

\noindent\makebox[\textwidth]{\hrulefill}

\vspace{1cm}

\textit{Svar}: \autoref{q:358:sa:sv:True}



\subsection{Vilket bitm\"onster motsvarar det hexidecimala uttrycket 7F?}

\label{q:359:sa:sv:False}

\vspace{2cm}

\noindent\makebox[\textwidth]{\hrulefill}

\vspace{1cm}

\textit{Svar}: \autoref{q:359:sa:sv:True}



\subsection{Beskriv det decimala talet 3 som ett bin\"art tal representerat med 8-bitar (8 bit unsigned integer).}

\label{q:360:sa:sv:False}

\vspace{2cm}

\noindent\makebox[\textwidth]{\hrulefill}

\vspace{1cm}

\textit{Svar}: \autoref{q:360:sa:sv:True}



\subsection{Vilket bitm\"onster motsvarar det hexidecimala uttrycket AB?}

\label{q:361:sa:sv:False}

\vspace{2cm}

\noindent\makebox[\textwidth]{\hrulefill}

\vspace{1cm}

\textit{Svar}: \autoref{q:361:sa:sv:True}



\subsection{Vilket decimaltal (basen 10) motsvarar det hexadecimala talet A2?}

\label{q:362:sa:sv:False}

\vspace{2cm}

\noindent\makebox[\textwidth]{\hrulefill}

\vspace{1cm}

\textit{Svar}: \autoref{q:362:sa:sv:True}



\subsection{Beskriv talet -3 (minus tre) som ett 8-bitars bitm\"onster enligt tv\r{a}komplementsnotation (two{\textquoteright}s complement notation).}

\label{q:363:sa:sv:False}

\vspace{2cm}

\noindent\makebox[\textwidth]{\hrulefill}

\vspace{1cm}

\textit{Svar}: \autoref{q:363:sa:sv:True}



\subsection{Vilket bitm\"onster motsvarar det hexadecimala talet 8F?}

\label{q:364:sa:sv:False}

\vspace{2cm}

\noindent\makebox[\textwidth]{\hrulefill}

\vspace{1cm}

\textit{Svar}: \autoref{q:364:sa:sv:True}



\subsection{Vilket decimaltal (basen 10) motsvarar det hexadecimala talet B3 ?}

\label{q:365:sa:sv:False}

\vspace{2cm}

\noindent\makebox[\textwidth]{\hrulefill}

\vspace{1cm}

\textit{Svar}: \autoref{q:365:sa:sv:True}



\subsection{Beskriv talet 3 (tre) som ett 8-bitars bitm\"onster enligt tv\r{a}komplementsnotation (two{\textquoteright}s complement notation)!}

\label{q:366:sa:sv:False}

\vspace{2cm}

\noindent\makebox[\textwidth]{\hrulefill}

\vspace{1cm}

\textit{Svar}: \autoref{q:366:sa:sv:True}



\subsection{Beskriv talet 3 (tre) med tv\r{a} tecken i hexadecimal form!}

\label{q:367:sa:sv:False}

\vspace{2cm}

\noindent\makebox[\textwidth]{\hrulefill}

\vspace{1cm}

\textit{Svar}: \autoref{q:367:sa:sv:True}



\subsection{Beskriv talet -4 (minus fyra) som ett 8-bitars bitm\"onster enligt tv\r{a}komplementsnotation (two{\textquoteright}s complement notation).}

\label{q:368:sa:sv:False}

\vspace{2cm}

\noindent\makebox[\textwidth]{\hrulefill}

\vspace{1cm}

\textit{Svar}: \autoref{q:368:sa:sv:True}



\subsection{Beskriv talet 2 (tv\r{a}) som ett 8-bitars bitm\"onster enligt tv\r{a}komplementsnotation (two{\textquoteright}s complement notation)!}

\label{q:369:sa:sv:False}

\vspace{2cm}

\noindent\makebox[\textwidth]{\hrulefill}

\vspace{1cm}

\textit{Svar}: \autoref{q:369:sa:sv:True}



\subsection{Beskriv talet \ensuremath{-}2 (minus tv\r{a}) som ett 8-bitars bitm\"onster enligt tv\r{a}komplementsnotation (two{\textquoteright}s complement notation)!}

\label{q:370:sa:sv:False}

\vspace{2cm}

\noindent\makebox[\textwidth]{\hrulefill}

\vspace{1cm}

\textit{Svar}: \autoref{q:370:sa:sv:True}



\subsection{Vad \"ar det positiva decimala heltalet 127 som ett bin\"art tal representerat med 8-bitar enligt tv\r{a}komplementsnotation (two{\textquoteright}s complement notation)?}

\label{q:371:sa:sv:False}

\vspace{2cm}

\noindent\makebox[\textwidth]{\hrulefill}

\vspace{1cm}

\textit{Svar}: \autoref{q:371:sa:sv:True}



\subsection{Vad \"ar det negativa decimala heltalet \ensuremath{-}127 som ett bin\"art tal representerat med 8-bitar enligt tv\r{a}komplementsnotation (two{\textquoteright}s complement notation)?}

\label{q:372:sa:sv:False}

\vspace{2cm}

\noindent\makebox[\textwidth]{\hrulefill}

\vspace{1cm}

\textit{Svar}: \autoref{q:372:sa:sv:True}



\subsection{Vilket decimalt naturligt tal (noll eller positivt heltal) (unsigned integer) representerar bitm\"onstret 1010 1010?}

\label{q:373:sa:sv:False}

\vspace{2cm}

\noindent\makebox[\textwidth]{\hrulefill}

\vspace{1cm}

\textit{Svar}: \autoref{q:373:sa:sv:True}



\subsection{Vilket decimalt naturligt tal (noll eller positivt heltal) (unsigned integer) representerar bitm\"onstret 1011 1011?}

\label{q:374:sa:sv:False}

\vspace{2cm}

\noindent\makebox[\textwidth]{\hrulefill}

\vspace{1cm}

\textit{Svar}: \autoref{q:374:sa:sv:True}



\subsection{Vilket bitm\"onster motsvarar det hexadecimala talet C4?}

\label{q:375:sa:sv:False}

\vspace{2cm}

\noindent\makebox[\textwidth]{\hrulefill}

\vspace{1cm}

\textit{Svar}: \autoref{q:375:sa:sv:True}



\subsection{Vilket bitm\"onster motsvarar det hexadecimala talet B3?}

\label{q:376:sa:sv:False}

\vspace{2cm}

\noindent\makebox[\textwidth]{\hrulefill}

\vspace{1cm}

\textit{Svar}: \autoref{q:376:sa:sv:True}



\subsection{Vilket decimalt heltal (signed integer) representerar bitm\"onstret 1010 enligt tv\r{a}komplementsnotation?}

\label{q:377:sa:sv:False}

\vspace{2cm}

\noindent\makebox[\textwidth]{\hrulefill}

\vspace{1cm}

\textit{Svar}: \autoref{q:377:sa:sv:True}



\subsection{Vilket decimalt heltal (signed integer) representerar bitm\"onstret 1011 enligt tv\r{a}komplementsnotation?}

\label{q:378:sa:sv:False}

\vspace{2cm}

\noindent\makebox[\textwidth]{\hrulefill}

\vspace{1cm}

\textit{Svar}: \autoref{q:378:sa:sv:True}



\subsection{Vilket decimalt naturligt tal (noll eller positivt heltal) (unsigned integer) representerar bitm\"onstret 1101 1011?}

\label{q:379:sa:sv:False}

\vspace{2cm}

\noindent\makebox[\textwidth]{\hrulefill}

\vspace{1cm}

\textit{Svar}: \autoref{q:379:sa:sv:True}



\subsection{Vilket bitm\"onster motsvarar det hexadecimala talet D2?}

\label{q:380:sa:sv:False}

\vspace{2cm}

\noindent\makebox[\textwidth]{\hrulefill}

\vspace{1cm}

\textit{Svar}: \autoref{q:380:sa:sv:True}



\subsection{Vilket decimalt heltal (signed integer) representerar bitm\"onstret 1101 enligt tv\r{a}komplementsnotation?}

\label{q:381:sa:sv:False}

\vspace{2cm}

\noindent\makebox[\textwidth]{\hrulefill}

\vspace{1cm}

\textit{Svar}: \autoref{q:381:sa:sv:True}



\subsection{Vilket bitm\"onster motsvarar det hexadecimala talet A5}

\label{q:382:sa:sv:False}

\vspace{2cm}

\noindent\makebox[\textwidth]{\hrulefill}

\vspace{1cm}

\textit{Svar}: \autoref{q:382:sa:sv:True}



\subsection{Vilket bitm\"onster motsvarar det hexadecimala talet B4}

\label{q:383:sa:sv:False}

\vspace{2cm}

\noindent\makebox[\textwidth]{\hrulefill}

\vspace{1cm}

\textit{Svar}: \autoref{q:383:sa:sv:True}



\subsection{Vilket decimaltal (basen 10) motsvarar det hexadecimala talet 4D ?}

\label{q:384:sa:sv:False}

\vspace{2cm}

\noindent\makebox[\textwidth]{\hrulefill}

\vspace{1cm}

\textit{Svar}: \autoref{q:384:sa:sv:True}



\subsection{Vilket decimaltal (basen 10) motsvarar det hexadecimala talet 5C ?}

\label{q:385:sa:sv:False}

\vspace{2cm}

\noindent\makebox[\textwidth]{\hrulefill}

\vspace{1cm}

\textit{Svar}: \autoref{q:385:sa:sv:True}



\subsection{Beskriv talet 4 (fyra) som ett 8-bitars bitm\"onster enligt tv\r{a}komplementsnotation (two{\textquoteright}s complement notation)!}

\label{q:386:sa:sv:False}

\vspace{2cm}

\noindent\makebox[\textwidth]{\hrulefill}

\vspace{1cm}

\textit{Svar}: \autoref{q:386:sa:sv:True}



\subsection{Beskriv talet \ensuremath{-}5 (minus fem) som ett 8-bitars bitm\"onster enligt tv\r{a}komplementsnotation (two{\textquoteright}s complement notation)!}

\label{q:387:sa:sv:False}

\vspace{2cm}

\noindent\makebox[\textwidth]{\hrulefill}

\vspace{1cm}

\textit{Svar}: \autoref{q:387:sa:sv:True}



\subsection{Vilket bitm\"onster motsvarar det hexadecimala talet C3?}

\label{q:388:sa:sv:False}

\vspace{2cm}

\noindent\makebox[\textwidth]{\hrulefill}

\vspace{1cm}

\textit{Svar}: \autoref{q:388:sa:sv:True}



\subsection{Vilket bitm\"onster motsvarar det hexadecimala talet 3C?}

\label{q:389:sa:sv:False}

\vspace{2cm}

\noindent\makebox[\textwidth]{\hrulefill}

\vspace{1cm}

\textit{Svar}: \autoref{q:389:sa:sv:True}



\subsection{Vilket decimaltal (basen 10) motsvarar det hexadecimala talet 3C?}

\label{q:390:sa:sv:False}

\vspace{2cm}

\noindent\makebox[\textwidth]{\hrulefill}

\vspace{1cm}

\textit{Svar}: \autoref{q:390:sa:sv:True}



\subsection{Vilket decimaltal (basen 10) motsvarar det hexadecimala talet C3?}

\label{q:391:sa:sv:False}

\vspace{2cm}

\noindent\makebox[\textwidth]{\hrulefill}

\vspace{1cm}

\textit{Svar}: \autoref{q:391:sa:sv:True}



\subsection{Beskriv talet -3 (minus tre) som ett 8-bitars bitm\"onster enligt tv\r{a}komplementsnotation (two{\textquoteright}s complement notation)!}

\label{q:392:sa:sv:False}

\vspace{2cm}

\noindent\makebox[\textwidth]{\hrulefill}

\vspace{1cm}

\textit{Svar}: \autoref{q:392:sa:sv:True}



\subsection{Beskriv talet -4 (minus fyra) som ett 8-bitars bitm\"onster enligt tv\r{a}komplementsnotation (two{\textquoteright}s complement notation)!}

\label{q:393:sa:sv:False}

\vspace{2cm}

\noindent\makebox[\textwidth]{\hrulefill}

\vspace{1cm}

\textit{Svar}: \autoref{q:393:sa:sv:True}



\subsection{Vilket bitm\"onster motsvarar det hexadecimala talet BE?}

\label{q:394:sa:sv:False}

\vspace{2cm}

\noindent\makebox[\textwidth]{\hrulefill}

\vspace{1cm}

\textit{Svar}: \autoref{q:394:sa:sv:True}



\subsection{Vilket decimaltal (basen 10) motsvarar det hexadecimala talet 2D?}

\label{q:395:sa:sv:False}

\vspace{2cm}

\noindent\makebox[\textwidth]{\hrulefill}

\vspace{1cm}

\textit{Svar}: \autoref{q:395:sa:sv:True}



\subsection{Vilket bitm\"onster motsvarar det hexadecimala talet A2?}

\label{q:396:sa:sv:False}

\vspace{2cm}

\noindent\makebox[\textwidth]{\hrulefill}

\vspace{1cm}

\textit{Svar}: \autoref{q:396:sa:sv:True}



\subsection{Vilket decimaltal (basen 10) motsvarar det hexadecimala talet D2?}

\label{q:397:sa:sv:False}

\vspace{2cm}

\noindent\makebox[\textwidth]{\hrulefill}

\vspace{1cm}

\textit{Svar}: \autoref{q:397:sa:sv:True}



\subsection{Vilket bitm\"onster motsvarar det hexadecimala talet B1?}

\label{q:398:sa:sv:False}

\vspace{2cm}

\noindent\makebox[\textwidth]{\hrulefill}

\vspace{1cm}

\textit{Svar}: \autoref{q:398:sa:sv:True}



\subsection{Vilket decimaltal (basen 10) motsvarar det hexadecimala talet 5E ?}

\label{q:399:sa:sv:False}

\vspace{2cm}

\noindent\makebox[\textwidth]{\hrulefill}

\vspace{1cm}

\textit{Svar}: \autoref{q:399:sa:sv:True}



\subsection{Vilket bitm\"onster motsvarar det hexadecimala talet 7F?}

\label{q:400:sa:sv:False}

\vspace{2cm}

\noindent\makebox[\textwidth]{\hrulefill}

\vspace{1cm}

\textit{Svar}: \autoref{q:400:sa:sv:True}



\subsection{Vilket decimaltal (basen 10) motsvarar det hexadecimala talet A6?}

\label{q:401:sa:sv:False}

\vspace{2cm}

\noindent\makebox[\textwidth]{\hrulefill}

\vspace{1cm}

\textit{Svar}: \autoref{q:401:sa:sv:True}



\subsection{Beskriv talet -6 (minus sex) som ett 8-bitars bitm\"onster enligt tv\r{a}komplementsnotation (two{\textquoteright}s complement notation).}

\label{q:402:sa:sv:False}

\vspace{2cm}

\noindent\makebox[\textwidth]{\hrulefill}

\vspace{1cm}

\textit{Svar}: \autoref{q:402:sa:sv:True}



\subsection{Antag att vi har f\"oljande tv\r{a} bitm\"onster 10000001 och 01111110. Vilket bitm\"onster erh\r{a}ller vi om vi utf\"or den logiska operationen AND p\r{a} dessa bitm\"onster?}

\label{q:403:sa:sv:False}

\vspace{2cm}

\noindent\makebox[\textwidth]{\hrulefill}

\vspace{1cm}

\textit{Svar}: \autoref{q:403:sa:sv:True}



\subsection{Antag att vi har f\"oljande tv\r{a} bitm\"onster 10000001 och 01111110. Vilket bitm\"onster erh\r{a}ller vi om vi utf\"or den aritmetiska operationen ADD enligt tv\r{a}komplementsnotation (two{\textquoteright}s complement notation) p\r{a} dessa bitm\"onster som d\r{a} representerar tv\r{a} heltal (signed integers)?}

\label{q:404:sa:sv:False}

\vspace{2cm}

\noindent\makebox[\textwidth]{\hrulefill}

\vspace{1cm}

\textit{Svar}: \autoref{q:404:sa:sv:True}



\subsection{Vilket bitm\"onster erh\r{a}ller vi om vi utf\"or operationen OR p\r{a} bitm\"onstren 1011 0011 och 0010 0110?}

\label{q:405:sa:sv:False}

\vspace{2cm}

\noindent\makebox[\textwidth]{\hrulefill}

\vspace{1cm}

\textit{Svar}: \autoref{q:405:sa:sv:True}



\subsection{Vilket bitm\"onster erh\r{a}ller vi om vi utf\"or operationen XOR p\r{a} bitm\"onstren 1011 0011 och 0010 0110?}

\label{q:406:sa:sv:False}

\vspace{2cm}

\noindent\makebox[\textwidth]{\hrulefill}

\vspace{1cm}

\textit{Svar}: \autoref{q:406:sa:sv:True}



\subsection{Vilket bitm\"onster erh\r{a}ller vi om vi utf\"or operationen AND p\r{a} bitm\"onstren 1001 1011 och 1000 1110?}

\label{q:407:sa:sv:False}

\vspace{2cm}

\noindent\makebox[\textwidth]{\hrulefill}

\vspace{1cm}

\textit{Svar}: \autoref{q:407:sa:sv:True}



\subsection{Vilket bitm\"onster erh\r{a}ller vi om vi utf\"or operationen OR p\r{a} bitm\"onstren 1001 1011 och 1000 1110?}

\label{q:408:sa:sv:False}

\vspace{2cm}

\noindent\makebox[\textwidth]{\hrulefill}

\vspace{1cm}

\textit{Svar}: \autoref{q:408:sa:sv:True}



\subsection{Vad blir resultatet av den logiska operationen AND med bitm\"onstren 1010 0101 och 0111 1110? Ange svaret som ett bitm\"onster.}

\label{q:409:sa:sv:False}

\vspace{2cm}

\noindent\makebox[\textwidth]{\hrulefill}

\vspace{1cm}

\textit{Svar}: \autoref{q:409:sa:sv:True}



\subsection{Vad blir resultatet av den logiska operationen XOR med bitm\"onstren 10100101 och 01111110? Ange svaret som ett bitm\"onster.}

\label{q:410:sa:sv:False}

\vspace{2cm}

\noindent\makebox[\textwidth]{\hrulefill}

\vspace{1cm}

\textit{Svar}: \autoref{q:410:sa:sv:True}



\subsection{Vad blir resultatet av den logiska operationen XOR med bitm\"onstren 10100001 och 01101010? Ange svaret som ett bitm\"onster.}

\label{q:411:sa:sv:False}

\vspace{2cm}

\noindent\makebox[\textwidth]{\hrulefill}

\vspace{1cm}

\textit{Svar}: \autoref{q:411:sa:sv:True}



\subsection{Vilket bitm\"onster erh\r{a}ller vi om vi utf\"or operationen OR p\r{a} bitm\"onstren 110011 och 101000 ?}

\label{q:412:sa:sv:False}

\vspace{2cm}

\noindent\makebox[\textwidth]{\hrulefill}

\vspace{1cm}

\textit{Svar}: \autoref{q:412:sa:sv:True}



\subsection{Vilket bitm\"onster erh\r{a}ller vi om vi utf\"or operationen XOR p\r{a} bitm\"onstren 0110 0011 och 0101 0000?}

\label{q:413:sa:sv:False}

\vspace{2cm}

\noindent\makebox[\textwidth]{\hrulefill}

\vspace{1cm}

\textit{Svar}: \autoref{q:413:sa:sv:True}



\subsection{Vilket bitm\"onster erh\r{a}ller vi om vi utf\"or operationen OR p\r{a} bitm\"onstren 101011 och 010011?}

\label{q:414:sa:sv:False}

\vspace{2cm}

\noindent\makebox[\textwidth]{\hrulefill}

\vspace{1cm}

\textit{Svar}: \autoref{q:414:sa:sv:True}



\subsection{Vilket bitm\"onster erh\r{a}ller vi om vi utf\"or operationen AND p\r{a} bitm\"onstren 110011 och 101001?}

\label{q:415:sa:sv:False}

\vspace{2cm}

\noindent\makebox[\textwidth]{\hrulefill}

\vspace{1cm}

\textit{Svar}: \autoref{q:415:sa:sv:True}



\subsection{Vilket bitm\"onster erh\r{a}ller vi om vi utf\"or operationen XOR p\r{a} bitm\"onstren 0110 0011 och 0101 0001?}

\label{q:416:sa:sv:False}

\vspace{2cm}

\noindent\makebox[\textwidth]{\hrulefill}

\vspace{1cm}

\textit{Svar}: \autoref{q:416:sa:sv:True}



\subsection{Vilket bitm\"onster erh\r{a}ller vi om vi utf\"or operationen AND p\r{a} bitm\"onstren 1101 1101 och 1111 1001?}

\label{q:417:sa:sv:False}

\vspace{2cm}

\noindent\makebox[\textwidth]{\hrulefill}

\vspace{1cm}

\textit{Svar}: \autoref{q:417:sa:sv:True}



\subsection{Vilket bitm\"onster erh\r{a}ller vi om vi utf\"or operationen XOR p\r{a} bitm\"onstren 0101 0101 och 1000 1100?}

\label{q:418:sa:sv:False}

\vspace{2cm}

\noindent\makebox[\textwidth]{\hrulefill}

\vspace{1cm}

\textit{Svar}: \autoref{q:418:sa:sv:True}



\subsection{Vilket bitm\"onster erh\r{a}ller vi om vi utf\"or operationen OR p\r{a} bitm\"onstren 01001000 och 10011001?}

\label{q:419:sa:sv:False}

\vspace{2cm}

\noindent\makebox[\textwidth]{\hrulefill}

\vspace{1cm}

\textit{Svar}: \autoref{q:419:sa:sv:True}



\subsection{Vad \"ar det minsta antalet g\r{a}nger som satserna i en loop-kropp (loop body) utf\"ors i en iteration med pre-test-villkor?}

\label{q:420:sa:sv:False}

\vspace{2cm}

\noindent\makebox[\textwidth]{\hrulefill}

\vspace{1cm}

\textit{Svar}: \autoref{q:420:sa:sv:True}



\subsection{Vilket bitm\"onster motsvarar det hexadecimala talet B7?}

\label{q:421:sa:sv:False}

\vspace{2cm}

\noindent\makebox[\textwidth]{\hrulefill}

\vspace{1cm}

\textit{Svar}: \autoref{q:421:sa:sv:True}



\subsection{Vilket bitm\"onster motsvarar det hexadecimala talet C1?}

\label{q:422:sa:sv:False}

\vspace{2cm}

\noindent\makebox[\textwidth]{\hrulefill}

\vspace{1cm}

\textit{Svar}: \autoref{q:422:sa:sv:True}



\subsection{Vilket bitm\"onster motsvarar det hexadecimala talet E3?}

\label{q:423:sa:sv:False}

\vspace{2cm}

\noindent\makebox[\textwidth]{\hrulefill}

\vspace{1cm}

\textit{Svar}: \autoref{q:423:sa:sv:True}



\subsection{Vilket hexadecimalt tal motsvarar bitm\"onstret 10010101?}

\label{q:424:sa:sv:False}

\vspace{2cm}

\noindent\makebox[\textwidth]{\hrulefill}

\vspace{1cm}

\textit{Svar}: \autoref{q:424:sa:sv:True}



\subsection{F\"argen magenta \"ar en blandning av maximalt r\"ott och maximalt bl\r{a}tt. Vilket bitm\"onster representerar en magentaf\"argad pixel kodad enligt RGB-standarden med bitdjupet 24 bitar/pixel? Ange svaret i hexadecimal notation.}

\label{q:425:sa:sv:False}

\vspace{2cm}

\noindent\makebox[\textwidth]{\hrulefill}

\vspace{1cm}

\textit{Svar}: \autoref{q:425:sa:sv:True}



\subsection{F\"argen gul \"ar en blandning av maximalt r\"ott och maximalt gr\"ont. Vilket bitm\"onster representerar en gulf\"argad pixel kodad enligt RGB-standarden med bitdjupet 24 bitar/pixel? Ange svaret i hexadecimal notation.}

\label{q:426:sa:sv:False}

\vspace{2cm}

\noindent\makebox[\textwidth]{\hrulefill}

\vspace{1cm}

\textit{Svar}: \autoref{q:426:sa:sv:True}



\subsection{Vilket hexadecimalt tal motsvarar bitm\"onstret 1110 0101?}

\label{q:427:sa:sv:False}

\vspace{2cm}

\noindent\makebox[\textwidth]{\hrulefill}

\vspace{1cm}

\textit{Svar}: \autoref{q:427:sa:sv:True}



\subsection{Vilket hexadecimalt tal motsvarar bitm\"onstret 10101101?}

\label{q:428:sa:sv:False}

\vspace{2cm}

\noindent\makebox[\textwidth]{\hrulefill}

\vspace{1cm}

\textit{Svar}: \autoref{q:428:sa:sv:True}



\subsection{Vilket hexadecimalt tal motsvarar bitm\"onstret 11010100?}

\label{q:429:sa:sv:False}

\vspace{2cm}

\noindent\makebox[\textwidth]{\hrulefill}

\vspace{1cm}

\textit{Svar}: \autoref{q:429:sa:sv:True}



\subsection{Vilket v\"arde kommer register 0 att ha efter tre (3) maskincykler? Ange bitm\"onstret p\r{a} hexadecimalform.}

\label{q:430:sa:sv:False}

\vspace{2cm}

\noindent\makebox[\textwidth]{\hrulefill}

\vspace{1cm}

\textit{Svar}: \autoref{q:430:sa:sv:True}



\subsection{Vilket v\"arde kommer register 1 att ha efter tre (3) maskincykler? Ange bitm\"onstret p\r{a} hexadecimalform.}

\label{q:431:sa:sv:False}

\vspace{2cm}

\noindent\makebox[\textwidth]{\hrulefill}

\vspace{1cm}

\textit{Svar}: \autoref{q:431:sa:sv:True}



\subsection{Vilket v\"arde kommer register 2 att ha efter tre (3) maskincykler? Ange bitm\"onstret p\r{a} hexadecimalform.}

\label{q:432:sa:sv:False}

\vspace{2cm}

\noindent\makebox[\textwidth]{\hrulefill}

\vspace{1cm}

\textit{Svar}: \autoref{q:432:sa:sv:True}



\subsection{Vilket v\"arde kommer programr\"aknaren (program counter) att ha efter tre (3) maskincykler? Ange bitm\"onstret p\r{a} hexadecimalform.}

\label{q:433:sa:sv:False}

\vspace{2cm}

\noindent\makebox[\textwidth]{\hrulefill}

\vspace{1cm}

\textit{Svar}: \autoref{q:433:sa:sv:True}



\subsection{Vem skrev program f\"or "the Analytical Engine" och d\"armed kan betraktas som v\"arldens f\"orsta programmerare?}

\label{q:434:mc:sv:False}

\begin{itemize}
  \item[$\bigcirc$] Charles Babbage, Joseph Marie Jacquard, Alonzo Church, Kurt G\"odel, John von Neumann, Blaise Pascal, Alan Turing
\end{itemize}

\vspace{1cm}

\textit{Svar}: \autoref{q:434:mc:sv:True}



\subsection{Vem designade "the Analytical Engine" - v\"arldens f\"orsta programmerbara ber\"akningsmaskin?}

\label{q:435:mc:sv:False}

\begin{itemize}
  \item[$\bigcirc$] Joseph Marie Jacquard, Ada Byron (Lovelace), Alonzo Church, Kurt G\"odel, John von Neumann, Blaise Pascal, Alan Turing
\end{itemize}

\vspace{1cm}

\textit{Svar}: \autoref{q:435:mc:sv:True}



\subsection{Vem var den f\"orste att anv\"anda h\r{a}lkort (anv\"andes f\"or att lagra tygm\"onster till automatiska v\"avstolar)?}

\label{q:436:mc:sv:False}

\begin{itemize}
  \item[$\bigcirc$] Charles Babbage, Ada Byron (Lovelace), Alonzo Church, Kurt G\"odel, John von Neumann, Blaise Pascal, Alan Turing
\end{itemize}

\vspace{1cm}

\textit{Svar}: \autoref{q:436:mc:sv:True}



\subsection{Vem utvecklade den f\"orsta kugghjulsbaserade maskinen f\"or att utf\"ora addition?}

\label{q:437:mc:sv:False}

\begin{itemize}
  \item[$\bigcirc$] Charles Babbage, Joseph Marie Jacquard, Ada Byron (Lovelace), Alonzo Church, Kurt G\"odel, John von Neumann, Alan Turing
\end{itemize}

\vspace{1cm}

\textit{Svar}: \autoref{q:437:mc:sv:True}



\subsection{Vem har publicerat ett ofullst\"andighetsteorem som s\"ager att det i alla matematiska teorier som omfattar v\r{a}rt traditionella aritmetiska system finns p\r{a}st\r{a}enden vars sanning eller falskhet inte kan fastst\"allas med hj\"alp av en algoritm?}

\label{q:438:mc:sv:False}

\begin{itemize}
  \item[$\bigcirc$] Alan Turing, Blaise Pascal, Alonzo Church, Charles Babbage, Tim Berners-Lee, Ada Byron (Lovelace), Joseph Jacquard
\end{itemize}

\vspace{1cm}

\textit{Svar}: \autoref{q:438:mc:sv:True}



\subsection{Vem f\"oreslog ett system genom vilket dokument som lagras p\r{a} datorer p\r{a} hela Internet kan l\"ankas samman och producera ett n\"at av l\"ankad information (World Wide Web)?}

\label{q:439:mc:sv:False}

\begin{itemize}
  \item[$\bigcirc$] Alan Turing, Blaise Pascal, Alonzo Church, Kurt G\"odel, Charles Babbage, Ada Byron (Lovelace), Joseph Jacquard
\end{itemize}

\vspace{1cm}

\textit{Svar}: \autoref{q:439:mc:sv:True}



\subsection{Vem har gett upphov till namnet p\r{a} den datorarkitektur d\"ar CPU h\"amtar instruktioner fr\r{a}n minne \"over en central bus?}

\label{q:440:mc:sv:False}

\begin{itemize}
  \item[$\bigcirc$] Charles Babbage, Joseph Marie Jacquard, Ada Byron (Lovelace), Alonzo Church, Kurt G\"odel, Blaise Pascal, Alan Turing
\end{itemize}

\vspace{1cm}

\textit{Svar}: \autoref{q:440:mc:sv:True}



\subsection{Vem gav upphov till namnet p\r{a} den matematiska modellen f\"or en dator som anv\"ands f\"or att studera kraften i algoritmisk bearbetning?}

\label{q:441:mc:sv:False}

\begin{itemize}
  \item[$\bigcirc$] Charles Babbage, Joseph Marie Jacquard, Ada Byron (Lovelace), Alonzo Church, Kurt G\"odel, John von Neumann, Blaise Pascal
\end{itemize}

\vspace{1cm}

\textit{Svar}: \autoref{q:441:mc:sv:True}



\subsection{Tesen att de funktioner som kan ber\"aknas av en Turing-maskin \"ar samma som alla ber\"akningsbara funktioner, \"ar namngiven efter Turing och ytterligare en matematiker som bidragit till tesen, vilken?}

\label{q:442:mc:sv:False}

\begin{itemize}
  \item[$\bigcirc$] Charles Babbage, Joseph Marie Jacquard, Ada Byron (Lovelace), Kurt G\"odel, John von Neumann, Blaise Pascal, Tim Berners-Lee
\end{itemize}

\vspace{1cm}

\textit{Svar}: \autoref{q:442:mc:sv:True}



\subsection{Vilket bitm\"onster motsvarar det hexadecimala talet D5?}

\label{q:443:mc:sv:False}

\begin{itemize}
  \item[$\bigcirc$] 1010 0010, 1111 0101, 1101 0101
\end{itemize}

\vspace{1cm}

\textit{Svar}: \autoref{q:443:mc:sv:True}



\subsection{Vilket hexadecimalt tal motsvarar bitm\"onstret 10001111?}

\label{q:444:mc:sv:False}

\begin{itemize}
  \item[$\bigcirc$] 8F, 7F, 8C
\end{itemize}

\vspace{1cm}

\textit{Svar}: \autoref{q:444:mc:sv:True}

\section{Svenska med Svar}
\label{svenska}

\subsection{Ge ett exempel f\"or varje kategori fr\r{a}n maskininstruktionerna ovan (The Machine{\textquoteright}s Language)!}

\label{q:1:sa:sv:True}

\textbf{Svar}: Data\"overf\"oringsinstruktioner: op-koderna 1, 2, 3 och 4 (det r\"acker att ha angett en); aritmetiska/logiska instruktioner: op-koderna 5, 6, 7, 8, 9 och A (det r\"acker att ha angett en); styrinstruktioner: op-koderna B och C (det r\"acker att ha angett en).



\subsection{Vilken datalagringsteknik anv\"andes f\"or f\"orsta g\r{a}ngen 1801 av Joseph Jacquard?}

\label{q:3:sa:sv:True}

\textbf{Svar}: H\r{a}lkort



\subsection{I vilket specialregister finns minnesadressen till n\"asta instruktion?}

\label{q:4:sa:sv:True}

\textbf{Svar}: Programr\"aknaren



\subsection{I vilket specialregister finns den maskinkodsinstruktion som skall utf\"oras?}

\label{q:5:sa:sv:True}

\textbf{Svar}: Instruktionsregistret



\subsection{Vilken processorarkitektur har f\r{a}, enkla och snabba maskininstruktioner?}

\label{q:6:sa:sv:True}

\textbf{Svar}: RISC



\subsection{Det finns en s\"arskild typ av maskininstruktion som beh\"ovs f\"or att kunna koordinera olika processers tillg\r{a}ng till gemensamma resurser, vad kallas den?}

\label{q:7:sa:sv:True}

\textbf{Svar}: Test-and-set (eller compare-and-swap)



\subsection{Vad kallas den del av operativsystemet som uppr\"atth\r{a}ller en processtabell?}

\label{q:8:sa:sv:True}

\textbf{Svar}: Scheduler



\subsection{Vad kallas det n\"ar en enskild anv\"andare i ett enanv\"andarsystem kan exekvera flera program {\textquotedblright}samtidigt{\textquotedblright}?}

\label{q:10:sa:sv:True}

\textbf{Svar}: Multitasking



\subsection{Vilket av f\"oljande alternativ \"ar inte en del av operativsystemet?}

\label{q:11:sa:sv:True}

\textbf{Svar}: Styrenhet



\subsection{Vad kallas en flagga som styr \r{a}tkomsten till en kritisk region (critical region) f\"or att garantera att inte flera processer kommer \r{a}t den kritiska regionen samtidigt (mutual exclusion)?}

\label{q:12:sa:sv:True}

\textbf{Svar}: Semafor



\subsection{Vad kallas den del av operativsystemet som hanterar data som ligger lagrat som namngivna enheter (named separate groups of data) p\r{a} sekund\"arminne?}

\label{q:13:sa:sv:True}

\textbf{Svar}: Filhanteraren



\subsection{En dator kan simulera att den har mer prim\"arminne \"an sitt faktiska fysiska prim\"arminnet. Vad kallas detta simulerade minne?}

\label{q:14:sa:sv:True}

\textbf{Svar}: Virtuellt minne



\subsection{Vad kallas den s\"arskilda process som beh\"ovs f\"or att starta en dator?}

\label{q:15:sa:sv:True}

\textbf{Svar}: Boot strapping (booting)



\subsection{Vad kallas den del av operativsystemet som allokerar (allocates) och avallokerar (deallocates) prim\"arminne (main memory) till olika processer?}

\label{q:16:sa:sv:True}

\textbf{Svar}: Minneshanterare



\subsection{Vad kallas den del av operativsystemet som tilldelar processortid (time slices) till olika processer?}

\label{q:17:sa:sv:True}

\textbf{Svar}: Dispatcher



\subsection{Vad heter protokollet som anv\"ands av applikationen World Wide Web?}

\label{q:18:sa:sv:True}

\textbf{Svar}: Hypertext Transfer Protocol (HTTP)



\subsection{Vilket Internet-mjukvarulager (Internet software layer) tillh\"or protokollet UDP (user datagram protocol)?}

\label{q:19:sa:sv:True}

\textbf{Svar}: Transportlager



\subsection{Vilket Internet-mjukvarulager (Internet software layer) tillh\"or protokollet FTP (file transfer protocol)?}

\label{q:20:sa:sv:True}

\textbf{Svar}: Applikationslagret



\subsection{Vad \"ar ett LAN?}

\label{q:21:sa:sv:True}

\textbf{Svar}: Lokalt n\"atverk (local area network)



\subsection{Vilket Internet-mjukvarulager (Internet software layer) tillh\"or protokollet TCP?}

\label{q:22:sa:sv:True}

\textbf{Svar}: Transportlagret (transport layer)



\subsection{Vad heter den organisation som ansvarar f\"or tilldelning av IP-nummer (det r\"acker med f\"orkortningen)?}

\label{q:23:sa:sv:True}

\textbf{Svar}: ICANN (Internet Corporation for Assigned Names and Numbers)



\subsection{Till vilket Internet-mjukvarulager (Internet software layer) h\"or protokollet IPv6?}

\label{q:24:sa:sv:True}

\textbf{Svar}: Network layer



\subsection{Vad kallas det n\"ar en webbklient st\"aller en fr\r{a}ga till en s\"arskild typ av server f\"or att \"overs\"atta ett dom\"annamn till ett IP-nummer?}

\label{q:25:sa:sv:True}

\textbf{Svar}: DNS-lookup



\subsection{Vilket Internet-protokoll f\"or transport-lagret \"ar mest tillf\"orlitligt?}

\label{q:26:sa:sv:True}

\textbf{Svar}: Transmission Control Protocol (TCP)



\subsection{Vad kallas det spr\r{a}k som man beskriver webbsidor med?}

\label{q:27:sa:sv:True}

\textbf{Svar}: HyperText Markup Language (HTML)



\subsection{F\"or att skydda en dator eller ett n\"atverk av datorer anv\"ands ofta ett system, best\r{a}ende av programvara och eventuellt \"aven h\r{a}rdvara, som kan inspektera, blockera och filtrera inkommande och utg\r{a}ende n\"atverkstrafik. Vad kallas ett s\r{a}dant system?}

\label{q:28:sa:sv:True}

\textbf{Svar}: Brandv\"agg



\subsection{F\"or att b\"attre kunna skydda ett n\"atverk av datorer anv\"ands ofta en mellanliggande dator som g\"or att kommunikation inte g\r{a}r direkt mellan en klient p\r{a} n\"atverket och en extern server. Vad kallas en s\r{a}dan dator?}

\label{q:29:sa:sv:True}

\textbf{Svar}: Proxyserver



\subsection{Vad kallas s\"attet att \r{a}stadkomma repetition i kod som kr\"aver mer utrymme i prim\"arminnet f\"or varje repetition?}

\label{q:31:sa:sv:True}

\textbf{Svar}: Rekursion



\subsection{Vad kallas s\"attet att \r{a}stadkomma repetition i kod som inte kr\"aver mer utrymme i prim\"arminnet f\"or varje repetition?}

\label{q:32:sa:sv:True}

\textbf{Svar}: Iteration



\subsection{Vilken \"ar den vanligaste metoden f\"or att verifiera att ett program fungerar korrekt?}

\label{q:33:sa:sv:True}

\textbf{Svar}: Testning



\subsection{Vad kallas den grundl\"aggande byggstenen i imperativa programmeringsspr\r{a}k?}

\label{q:34:sa:sv:True}

\textbf{Svar}: Procedur



\subsection{Vad kallas den logiska h\"arledningsteknik som anv\"ands i logikprogrammering?}

\label{q:35:sa:sv:True}

\textbf{Svar}: Resolution



\subsection{I objektorienterad programmering, vad kallas mallarna fr\r{a}n vilka objekt skapas?}

\label{q:36:sa:sv:True}

\textbf{Svar}: Klass



\subsection{Vad kallas det programmeringsparadigm d\"ar man beskriver vad som skall utf\"oras ist\"allet f\"or hur det skall utf\"oras?}

\label{q:37:sa:sv:True}

\textbf{Svar}: Deklarativa programmeringsspr\r{a}k



\subsection{Vad kallas ett program som \"overs\"atter k\"allkod till maskinkod?}

\label{q:38:sa:sv:True}

\textbf{Svar}: Kompilator



\subsection{Vad kallas den typ av programmering som besvarar fr\r{a}gor huruvida ett faktum \"ar h\"arledbart fr\r{a}n ett program eller inte?}

\label{q:39:sa:sv:True}

\textbf{Svar}: Logikprogrammeringsspr\r{a}k



\subsection{Ge ett exempel p\r{a} en l\"attr\"orlig utvecklingsmodell (agile development model)?}

\label{q:40:sa:sv:True}

\textbf{Svar}: Scrum



\subsection{Vad kallas den roll i Scrum som uppr\"atth\r{a}ller en lista med krav och prioriterar mellan dessa krav?}

\label{q:41:sa:sv:True}

\textbf{Svar}: Produkt\"agare



\subsection{Vad kallas de korta iterationer (2 {\textendash} 4 veckor) i Scrum, som skall resultera i n\r{a}gonting levererbart till kund/best\"allare?}

\label{q:42:sa:sv:True}

\textbf{Svar}: Sprint



\subsection{Vad kallas den roll i Scrum som skall s\"akerst\"alla att Scrum-ramverket f\"oljs?}

\label{q:43:sa:sv:True}

\textbf{Svar}: Scrum master



\subsection{Vad kallas i Scrum de korta dagliga m\"oten d\r{a} varje projektdeltagare skall svara p\r{a} tre fr\r{a}gor?}

\label{q:44:sa:sv:True}

\textbf{Svar}: Daily scrum (stand-up)



\subsection{Vad kallas i Scrum de m\"oten d\r{a} man diskuterar vad som har g\r{a}tt bra denna iteration och vad som kan f\"orb\"attras i n\"asta iteration?}

\label{q:45:sa:sv:True}

\textbf{Svar}: Sprint retrospective



\subsection{Vad kallas rollen i ett team som \"ar ansvarig f\"or att team:et f\"oljer Scrum-metodiken?}

\label{q:46:sa:sv:True}

\textbf{Svar}: Scrum master



\subsection{Vad st\r{a}r f\"orkortningen CASE f\"or avseende programvaruutveckling?}

\label{q:47:sa:sv:True}

\textbf{Svar}: Computer Aided Software Engineering (CASE)



\subsection{Vad st\r{a}r f\"orkortningen IDE f\"or avseende programvaruutveckling?}

\label{q:48:sa:sv:True}

\textbf{Svar}: Integrated Development Environment (IDE)



\subsection{Vad heter den roll i Scum som \"ar ansvarig f\"or att prioritera vilken utveckling som skall utf\"oras under n\"asta sprint?}

\label{q:49:sa:sv:True}

\textbf{Svar}: Produkt\"agare



\subsection{Vad kallas Scrum-m\"otet, i slutet av en sprint d\"ar det avslutade arbetet med sprinten utv\"arderas med avseende p\r{a} sprintm\r{a}len?}

\label{q:50:sa:sv:True}

\textbf{Svar}: Sprint review



\subsection{Vad kallas den grundl\"aggande datastruktur som best\r{a}r av ett block av dataelement av samma datatyp och storlek, och d\"ar varje dataelement direkt n\r{a}s via ett index?}

\label{q:51:sa:sv:True}

\textbf{Svar}: Array



\subsection{Vad kallas den grundl\"aggande datastruktur som best\r{a}r av ett block av dataelement av vanligtvis olika datatyper och storlek, och d\"ar de enskilda dataelementen n\r{a}s via namn?}

\label{q:52:sa:sv:True}

\textbf{Svar}: Aggregattyp



\subsection{Vad kallas en variabel som inneh\r{a}ller en minnesadress ist\"allet f\"or data (anv\"ands i dynamiska datastrukturer)?}

\label{q:53:sa:sv:True}

\textbf{Svar}: Pekare



\subsection{Vad heter det dominerande fr\r{a}gespr\r{a}ket som anv\"ands f\"or att h\"amta data fr\r{a}n och manipulera data i en databas?}

\label{q:54:sa:sv:True}

\textbf{Svar}: Structured Query Language (SQL)



\subsection{Vad kallas i databassammanhang, en sekvens av operationer som paketeras ihop och d\"ar antingen alla operationer lyckas (utf\"ors) eller alla misslyckas (ingen utf\"ors) (all operations together either succeed or fail)?}

\label{q:55:sa:sv:True}

\textbf{Svar}: Transaktion



\subsection{Vad kallas den typ av data mining som har gjort webbshopen Amazon s\r{a} framg\r{a}ngsrik?}

\label{q:56:sa:sv:True}

\textbf{Svar}: Association analysis



\subsection{Vad kallas den typ av analys inom data-mining, som f\"ors\"oker uppt\"acka klasser genom att gruppera objekt i ett antal separata grupper (i motsats till klassbeskrivning, som f\"ors\"oker uppt\"acka egenskaper hos medlemmar inom klasser som redan \"ar identifierade)?}

\label{q:57:sa:sv:True}

\textbf{Svar}: Klusteranalys



\subsection{Vad kallas den typ av analys inom data-mining, som f\"ors\"oker identifiera beteendem\"onster \"over tid, till exempel trender i ekonomiska system som aktiemarknader eller i milj\"osystem som klimatf\"orh\r{a}llanden?}

\label{q:58:sa:sv:True}

\textbf{Svar}: Sekventiell m\"onsteranalys



\subsection{Vid rendrering s\r{a} skall en trediminsionell modell \"overf\"oras till en platt yta. Vad kallas denna platta yta?}

\label{q:59:sa:sv:True}

\textbf{Svar}: Projiceringsplan/projiceringsyta



\subsection{Vid rendering av 3D-grafik s\r{a} skall en tredimensionell modell \"overf\"oras till en platt yta, vad kallas denna platta yta?}

\label{q:60:sa:sv:True}

\textbf{Svar}: Projiceringsplan/projiceringsyta



\subsection{Vad kallas det n\"ar man till\"ampar fysikens lagar f\"or att best\"amma objekts positioner, t.ex. biljardbollars positioner efter en biljardst\"ot?}

\label{q:61:sa:sv:True}

\textbf{Svar}: Dynamik



\subsection{Vad kallas den del av maskininl\"arning (machine learning) d\"ar en m\"anniska beskriver det korrekta svaret f\"or ett antal exempel och agenten (maskininl\"arningsalgoritmen) generaliserar utifr\r{a}n dessa exempel?}

\label{q:62:sa:sv:True}

\textbf{Svar}: Supervised Learning



\subsection{Ge ett exempel p\r{a} en icke-ber\"akningsbar funktion?}

\label{q:63:sa:sv:True}

\textbf{Svar}: Stopp-problemet



\subsection{Vad kallas den maskin som utg\"or den enklast t\"ankbara modellen av en dator?}

\label{q:64:sa:sv:True}

\textbf{Svar}: Turing-maskin



\subsection{En ljudfil i CD-kvalitet inneb\"ar en samplingsfrekvens (sampling frequency) om 44100 per sekund, och ett samplingsdjup (sampling depth) om 16 bitar per kanal. Hur stor plats i kilobyte (kB) tar en okomprimerad ljudfil i stereo (2 kanaler) i CD-kvalitet med en l\"angd p\r{a} 3 minuter?}

\label{q:65:sa:sv:True}

\textbf{Svar}: 44100 * 16 * 2 * 180 / 8000 = 31 752 kB



\subsection{Antag att vi tidigare har lagrat digitala bilder med f\"argdjupet 12 bitar per pixel (color depth 12 bits per pixel). Om vi nu vill kunna representera h\"alften s\r{a} m\r{a}nga olika f\"arger j\"amf\"ort med tidigare, vilket f\"argdjup skall vi anv\"anda d\r{a}?}

\label{q:66:sa:sv:True}

\textbf{Svar}: 11 bitar (212 = 4096, 211 = 2048)



\subsection{Antag att vi tidigare har lagrat digitala bilder med f\"argdjupet 12 bitar per pixel (color depth 12 bits per pixel). Om vi nu vill kunna representera dubbelt s\r{a} m\r{a}nga olika f\"arger j\"amf\"ort med tidigare, vilket f\"argdjup skall vi anv\"anda d\r{a}?}

\label{q:67:sa:sv:True}

\textbf{Svar}: 13 bitar (212 = 4096, 213 = 8192)



\subsection{Vad \"ar en teckenkodning (character encoding)?}

\label{q:68:sa:sv:True}

\textbf{Svar}: En beskrivning hur bitm\"onster \"overs\"atts till tecken och tv\"artom. A mapping from bit patterns to characters and vice versa.



\subsection{Om 6A38 \"ar den hexadecimala notationen f\"or ett bitm\"onster som representerar en ljud-sample (one sound sample), vad har denna ljud-sample f\"or samplingsdjup (sampling depth)?}

\label{q:71:sa:sv:True}

\textbf{Svar}: 16 bitar (bits).



\subsection{Om 6A36B3 \"ar den hexadecimala notationen f\"or ett bitm\"onster som representerar en RGB-kodad pixel, vad har denna pixel f\"or f\"argdjup (colour depth)?}

\label{q:72:sa:sv:True}

\textbf{Svar}: 24-bitars f\"argdjup.



\subsection{Vad \"ar f\"argdjup (color depth) i samband med lagring av bilder?}

\label{q:73:sa:sv:True}

\textbf{Svar}: Antalet bitar per pixel som anv\"ands f\"or att koda f\"argen av pixeln.



\subsection{Vilket decimaltal (basen 10) motsvarar det hexadecimala talet 15?}

\label{q:74:sa:sv:True}

\textbf{Svar}: 1 * 16 + 5 * 1 = 21



\subsection{Antag att vi tidigare har lagrat digitala bilder med f\"argdjupet 8 bitar per pixel (color depth 8 bits per pixel). Om vi nu vill kunna representera dubbelt s\r{a} m\r{a}nga olika f\"arger j\"amf\"ort med tidigare, vilket f\"argdjup skall vi anv\"anda d\r{a}?}

\label{q:76:sa:sv:True}

\textbf{Svar}: 9 bitar per pixel (28 = 256 och 29 = 512).



\subsection{Vad \"ar samplingsfrekvens (sample rate) i samband med digital lagring av ljud?}

\label{q:77:sa:sv:True}

\textbf{Svar}: Samplingsfrekvensen beskriver antalet samplingar (avl\"asningar av ljudv\r{a}gen) som g\"ors per tidsenhet (sekund).



\subsection{Vilket decimaltal (basen 10) motsvarar det hexadecimala talet 3F?}

\label{q:78:sa:sv:True}

\textbf{Svar}: Den hexadecimala (bas 16) talserien:0,1,2,3,4,5,6,7,8,9,A,B,C,D,E,F (d\"ar A=10, B=11, C=12, D=13, E=14, F=15). 3*161 +F*160 =48+15=63



\subsection{Antag att vi tidigare har lagrat digitala bilder med f\"argdjupet 8 bitar per pixel (color depth 8 bits per pixel). Om vi nu vill kunna representera h\"alften s\r{a} m\r{a}nga olika f\"arger j\"amf\"ort med tidigare, vilket f\"argdjup skall vi anv\"anda d\r{a}?}

\label{q:79:sa:sv:True}

\textbf{Svar}: 7 bitar per pixel; (8 bitar kan representera 256 v\"arden, f\"or att representera h\"alften (128) beh\"ovs 7 bitar eftersom 27 = 128).



\subsection{Hur m\r{a}nga bitar (f\"argdjup) beh\"ovs f\"or att representera 16 olika f\"arger?}

\label{q:80:sa:sv:True}

\textbf{Svar}: 4 bitar kan representera 16 v\"arden.



\subsection{Vad \"ar en ljudfils samplingsdjup (sample depth)?}

\label{q:81:sa:sv:True}

\textbf{Svar}: Beskriver hur m\r{a}nga bitar som anv\"ands f\"or att representera informationen av en sampling (m\"atpunkt).



\subsection{Vad \"ar en ljudfils samplingsfrekvens (sample rate)?}

\label{q:82:sa:sv:True}

\textbf{Svar}: Beskriver antalet samplingar (m\"atpunkter) per tidsenhet.



\subsection{Vad anger en ljudfils samplingsfrekvens?}

\label{q:83:sa:sv:True}

\textbf{Svar}: Samplingsfrekvensen beskriver antalet samplingar (avl\"asningar av ljudv\r{a}gen) som g\"ors per tidsenhet(sekund).



\subsection{Antag att vi tidigare har lagrat digitala bilder med f\"argdjupet 5 bitar per pixel (color depth 5 bits per pixel). Om vi nu vill kunna representera dubbelt s\r{a} m\r{a}nga olika f\"arger j\"amf\"ort med tidigare, vilket f\"argdjup skall vi anv\"anda d\r{a}?}

\label{q:84:sa:sv:True}

\textbf{Svar}: 6 bitar per pixel



\subsection{Hur m\r{a}nga bitar (f\"argdjup) beh\"ovs f\"or att representera 24 olika f\"arger?}

\label{q:85:sa:sv:True}

\textbf{Svar}: 5 bitar kan representera 32 v\"arden.



\subsection{Hur m\r{a}nga bitar (f\"argdjup) beh\"ovs f\"or att representera 12 olika f\"arger?}

\label{q:86:sa:sv:True}

\textbf{Svar}: 4 bitar kan representera 16 v\"arden.



\subsection{Hur m\r{a}nga bitar kr\"avs det f\"or att representera ett boolskt v\"arde?}

\label{q:87:sa:sv:True}

\textbf{Svar}: En bit.



\subsection{Ge exempel p\r{a} tv\r{a} olika logiska operationer som kan utf\"oras p\r{a} boolska v\"arden?}

\label{q:88:sa:sv:True}

\textbf{Svar}: Tv\r{a} stycken av AND, OR, XOR och NOT.



\subsection{Hur m\r{a}nga bitar (f\"argdjup) beh\"ovs f\"or att representera 9 olika f\"arger?}

\label{q:89:sa:sv:True}

\textbf{Svar}: 4 bitar kan representera 16 v\"arden.



\subsection{Hur m\r{a}nga bitar (f\"argdjup) beh\"ovs f\"or att representera 15 olika f\"arger?}

\label{q:90:sa:sv:True}

\textbf{Svar}: 4 bitar kan representera 16 v\"arden.



\subsection{Vad \"ar en byte?}

\label{q:91:sa:sv:True}

\textbf{Svar}: En samling av 8 bitar.



\subsection{Vad kallas 8 bitar?}

\label{q:92:sa:sv:True}

\textbf{Svar}: En byte.



\subsection{Vilka \"ar de tre olika kategorierna av maskininstruktioner (machine instruction categories)?}

\label{q:93:sa:sv:True}

\textbf{Svar}: Data\"overf\"oringsinstruktioner (data transfer instructions), aritmetiska/logiska instruktioner (arithmetic/logic instructions) och styrinstruktioner (control instructions).



\subsection{Vad \"ar ett maskinspr\r{a}k (machine language)?}

\label{q:94:sa:sv:True}

\textbf{Svar}: Ett maskinspr\r{a}k \"ar m\"angden av alla maskininstruktioner som k\"anns igen av en viss CPU (a machine language is the set of all machine instructions recognized by the CPU).



\subsection{Vad lagras i programr\"aknaren (program counter)?}

\label{q:95:sa:sv:True}

\textbf{Svar}: Minnesadressen till n\"asta instruktion.



\subsection{Vad lagras i instruktionsregistret (instruction register)?}

\label{q:96:sa:sv:True}

\textbf{Svar}: Den maskinkodsinstruktion som skall utf\"oras.



\subsection{Beskriv skillnaden mellan RISC- och CISC-processorer.}

\label{q:97:sa:sv:True}

\textbf{Svar}: RISC-processorer har f\r{a}, enkla och snabba maskininstruktioner, och CISC-processorer har m\r{a}nga och kraftfulla maskininstruktioner.



\subsection{Vilka olika steg ing\r{a}r i en maskincykel (machine cycle)? Ange stegen i den ordning de utf\"ors.}

\label{q:98:sa:sv:True}

\textbf{Svar}: Fetch, Decode och Execute.



\subsection{Vilket bitm\"onster erh\r{a}ller vi om vi utf\"or operationen ADD p\r{a} bitm\"onstren 1011 0011 och 0010 0110?}

\label{q:99:sa:sv:True}

\textbf{Svar}: 1101 1001 (1011 0011 (= 179) ADD 0010 0110 (= 38) gives 1101 1001 (= 217))



\subsection{Vilka \"ar de tre huvudsakliga delar som en processor (CPU) best\r{a}r av?}

\label{q:100:sa:sv:True}

\textbf{Svar}: Aritmetisk/logisk enhet (arithmetic/logic unit), styrenhet (control unit) och register (registers).



\subsection{Vilket bitm\"onster erh\r{a}ller vi om vi utf\"or operationen XOR p\r{a} bitm\"onstren 1010 0011 och 0010 0110?}

\label{q:101:sa:sv:True}

\textbf{Svar}: 1010 00110010 0110 =1000 0101



\subsection{Vad kr\"avs f\"or att man ska kunna tolka ett bitm\"onster som ett tecken? What is required to be able to interpret a bit pattern as a character?}

\label{q:102:sa:sv:True}

\textbf{Svar}: Att man k\"anner till teckenkodningen.



\subsection{Hur ser man till att processer inte kan utf\"ora operationer som \"ar destruktiva f\"or andra processer p\r{a} en dator, t.ex. att skriva data i andra processers delar av prim\"arminnet?}

\label{q:104:sa:sv:True}

\textbf{Svar}: Genom att vissa maskininstruktioner, s.k. privileged instructions, endast f\r{a}r utf\"oras av processer som \"ar i privileged mode, vilket endast operativsystemprocesser skall vara.



\subsection{Vad inneb\"ar boot strapping (booting) och varf\"or beh\"ovs det?}

\label{q:105:sa:sv:True}

\textbf{Svar}: En s\"arskild process f\"or att starta en dator, som inneb\"ar att operativsystemet l\"ases in i prim\"arminnet och b\"orjar exekveras. Det beh\"ovs f\"or att n\"ar en dator startas \"ar prim\"arminnet helt tomt, och d\r{a} har processorn inga instruktioner att f\"olja.



\subsection{Vad inneb\"ar realtidsbehandling (real time processing)?}

\label{q:106:sa:sv:True}

\textbf{Svar}: Utf\"orande av uppgifter i enlighet med deadlines i den omgivande verkligheten.



\subsection{Vad inneb\"ar virtuellt minne (virtual memory)?}

\label{q:108:sa:sv:True}

\textbf{Svar}: Datorn simulerar att den har mer prim\"arminne (genom paging) \"an det faktiska fysiska prim\"arminnet.



\subsection{Vilken huvudsaklig funktion har ett operativsystem?}

\label{q:109:sa:sv:True}

\textbf{Svar}: Att administrera en dators resurser (to manage the resources of a computer), vilket bl.a. inneb\"ar.:i) att \"overvaka driften av datorn (to oversee the operation of the computer);ii) att spara och h\"amta filer (to store and retrieve files);iii) att schemal\"agga program f\"or exekvering (to schedule programs for execution);iv) att koordinera exekveringen av program (to ccordinate the execution of programs).



\subsection{Vad inneb\"ar interaktiv bearbetning (interactive processing)?}

\label{q:110:sa:sv:True}

\textbf{Svar}: Att programexekveringen st\"oder interaktion med anv\"andaren.



\subsection{Vad inneb\"ar realtidsbearbetning (real time processing)?}

\label{q:111:sa:sv:True}

\textbf{Svar}: Programexekvering i enlighet med deadlines i den omgivande verkligheten.



\subsection{Vad \"ar skillnaden mellan batch-bearbetning (batch processing) och interaktiv-bearbetning (interactive processing)?}

\label{q:112:sa:sv:True}

\textbf{Svar}: Batch-bearbetning \"ar exekvering av program (eller mer exakt: av en batch av jobb) utan n\r{a}gon interaktion med en anv\"andare. Interaktiv-bearbetning \"ar exekvering av program med n\r{a}gon typ av interaktion med en anv\"andare.



\subsection{Vad \"ar virtuellt minne och vad kan det vara bra f\"or?}

\label{q:113:sa:sv:True}

\textbf{Svar}: Virtuellt minne \"ar en minneshanteringsteknik som anv\"ands f\"or att ut\"oka datorns tillg\"angliga minne ut\"over prim\"arminnet. Normalt skapas det virtuella minnet p\r{a} en sekund\"ar lagringsenhet, t ex en h\r{a}rddisk, och f\"ordelen \"ar att man kan arbeta med program och data som kr\"aver mer minne \"an det fysiska prim\"arminne man har. Nackdelen \"ar att det normalt \"ar l\r{a}ngsammare (fast det beror ju p\r{a} vilken typ av media det sparas p\r{a}).



\subsection{Ange fyra olika komponenter i ett operativsystems k\"arna (operating system kernel)?}

\label{q:114:sa:sv:True}

\textbf{Svar}: File manager, device drivers, memory manager, scheduler, dispatcher.



\subsection{Vad \"ar en fil (file) i ett filhanteringssystem (file management system)?}

\label{q:115:sa:sv:True}

\textbf{Svar}: En namngiven separat grupp av data.



\subsection{Vilka \"ar de fyra grundl\"aggande funktionerna f\"or ett operativsystem (functions of operating systems)?}

\label{q:116:sa:sv:True}

\textbf{Svar}: Oversee the operation of a computer; store and retrieve files; schedule programs for execution; coordinate the execution of programs.



\subsection{Ett operativsystem best\r{a}r av tv\r{a} huvudsakliga komponenter (operating system components), vilka?}

\label{q:118:sa:sv:True}

\textbf{Svar}: Anv\"andargr\"anssnitt (user interface) och k\"arna (kernel).



\subsection{Vad kr\"avs f\"or att en deadlock skall kunna uppst\r{a} (conditions required for deadlock)?}

\label{q:119:sa:sv:True}

\textbf{Svar}: Competition for non-sharable resources; resources requested on a partial basis; allocated resources cannot be forcibly retrieved.



\subsection{En process aktuella tillst\r{a}nd (state) kan beskrivas av en m\"angd data, vilket data?}

\label{q:120:sa:sv:True}

\textbf{Svar}: Inneh\r{a}llet i programr\"aknaren, inneh\r{a}llet i general purpose registren, och till processen tillh\"orande delar av prim\"arminnet.



\subsection{Vad \"ar ett program och vad \"ar en process?}

\label{q:121:sa:sv:True}

\textbf{Svar}: Ett program \"ar en samling instruktioner som \"ar utf\"orbara av en dator (en exekverbar algoritm), och en process \"ar aktiviteten att utf\"ora dessa instruktioner.



\subsection{Vad \"ar en fil?}

\label{q:122:sa:sv:True}

\textbf{Svar}: En namngiven grupp data.



\subsection{Vad \"ar en katalog (directory)?}

\label{q:123:sa:sv:True}

\textbf{Svar}: En namngiven samling filer och (under-)kataloger.



\subsection{Vad inneb\"ar paging?}

\label{q:124:sa:sv:True}

\textbf{Svar}: Att program och data roteras fram och tillbaka mellan prim\"ar- och sekund\"arminne.



\subsection{Vad \"ar och vad g\"or en boot loader?}

\label{q:125:sa:sv:True}

\textbf{Svar}: Ett program lagrat i ROM, som k\"ors n\"ar datorn startas och laddar in operativsystemet i prim\"arminnet och sedan \"overf\"or kontrollen till operativsystemet.



\subsection{Vad inneb\"ar interaktiv bearbetning (interactive processing)?}

\label{q:126:sa:sv:True}

\textbf{Svar}: Interaktiv-bearbetning \"ar exekvering av program med n\r{a}gon typ av interaktion med en anv\"andare.



\subsection{Vad inneb\"ar batch-bearbetning (batch processing)?}

\label{q:127:sa:sv:True}

\textbf{Svar}: Batch-bearbetning \"ar exekvering av program (eller mer exakt: av en batch av jobb) utan n\r{a}gon interaktion med en anv\"andare.



\subsection{Vad inneb\"ar begreppet deadlock?}

\label{q:129:sa:sv:True}

\textbf{Svar}: Att processer blockerar (hindrar) varandra fr\r{a}n att forts\"atta.



\subsection{Vad \"ar ett job i samband med batch-processing?}

\label{q:130:sa:sv:True}

\textbf{Svar}: Ett program som ska exekveras tillsammans med dess indata och utdata.



\subsection{Vad inneb\"ar multitasking?}

\label{q:132:sa:sv:True}

\textbf{Svar}: Att flera program kan k\"oras "p\r{a} samma g\r{a}ng" av en enda anv\"andare.



\subsection{Vad inneb\"ar "paging"?}

\label{q:133:sa:sv:True}

\textbf{Svar}: Att program och data roteras fram och tillbaka mellan prim\"ar- och sekund\"arminne.



\subsection{Vad \"ar skillnaden mellan en switch och en router?}

\label{q:134:sa:sv:True}

\textbf{Svar}: En switch kopplar samman flera {\textquotedblright}bussar{\textquotedblright} (buses) (och/eller datorer) till ett n\"atverk. En router koppar samman flera olika n\"atverk till ett n\"atverk av n\"atverk (internet).



\subsection{Vilka \"ar de tv\r{a} modellerna f\"or inter-process-kommunikation?}

\label{q:135:sa:sv:True}

\textbf{Svar}: Client/server och peer-to-peer.



\subsection{Vad \"ar en IP-adress?}

\label{q:136:sa:sv:True}

\textbf{Svar}: En unik numerisk adress till en dator uppkopplad p\r{a} Internet.



\subsection{Vad \"ar DNS?}

\label{q:137:sa:sv:True}

\textbf{Svar}: Domain Name System \"overs\"atter dom\"annamn till IP-adresser.



\subsection{Vad inneb\"ar bus och star n\"ar det handlar om n\"attopologi?}

\label{q:138:sa:sv:True}

\textbf{Svar}: Bus-topologi inneb\"ar att alla enheter \"ar kopplade till en gemensam kommunikationslina, s.k. bus. Star-topologin inneb\"ar att alla \"ovriga enheter \"ar kopplade till en central enhet, s.k. accesspunkt (access point).



\subsection{Vad inneb\"ar cloud computing?}

\label{q:139:sa:sv:True}

\textbf{Svar}: Stora pooler av delade datorer som kan tilldelas f\"or anv\"andning utifr\r{a}n behov.



\subsection{Vad \"ar den huvudsakliga skillnaden mellan IPv4 (IP version 4) och IPv6 (IP version 6)?}

\label{q:140:sa:sv:True}

\textbf{Svar}: IPv4-adresser \"ar 32 bitar och IPv6-adresser \"ar 128 bitar, vilket inneb\"ar att det finns v\"aldigt m\r{a}nga fler IPv6-adresser \"an IPv4-adresser.



\subsection{Vad \"ar ett certifikat (certificate) i samband med public-key-kryptering (public key encryption)?}

\label{q:141:sa:sv:True}

\textbf{Svar}: Ett paket best\r{a}ende av namn/identitet och publik nyckel (a package of name/identity and public key), vilket intygar att man \"ar den man utger sig f\"or att vara.



\subsection{Ge ett exempel p\r{a} en typ av malware?}

\label{q:142:sa:sv:True}

\textbf{Svar}: Virus, maskar, trojaner, spion-programvara, n\"atfiske-programvara (viruses, worms, Trojan horses, spyware, phishing software) (det r\"acker att ha angett en).



\subsection{Vad inneb\"ar DNS lookup?}

\label{q:143:sa:sv:True}

\textbf{Svar}: Anv\"andandet av DNS (domain name system) f\"or att \"overs\"atta fr\r{a}n ett dom\"annamn till en IP-adress.



\subsection{Vad g\"or en (n\"atverks-) hub?}

\label{q:144:sa:sv:True}

\textbf{Svar}: Den kopplar ihop datorer till ett n\"atverk.



\subsection{Till vilket Internet-mjukvarulager (Internet software layer) h\"or protokollet SMTP?}

\label{q:145:sa:sv:True}

\textbf{Svar}: Applikationslagret (the application layer) (SMTP = simple mail transfer protocol).



\subsection{Vad g\"or en webbserver (webserver)?}

\label{q:146:sa:sv:True}

\textbf{Svar}: Ger tillg\r{a}ng till olika webbresurser, som t.ex. webbsidor.



\subsection{Vad \"ar syftet med en URL/URI?}

\label{q:147:sa:sv:True}

\textbf{Svar}: Att unikt identifiera en webbresurs, t.ex. en webbsida.



\subsection{Vilka \"ar de tv\r{a} vanliga Internet-protokollen f\"or transport-lagret (transport layer)?}

\label{q:148:sa:sv:True}

\textbf{Svar}: TCP (transmission control protocol) och UDP (user datagram protocol).



\subsection{Vad kallas den krypteringsteknik som anv\"ands mycket p\r{a} Internet och som inneb\"ar att parterna inte i f\"orv\"ag beh\"over ha tillg\r{a}ng till en gemensam nyckel?}

\label{q:149:sa:sv:True}

\textbf{Svar}: Public key encryption (t.ex. RSA-algoritmen).



\subsection{Vad \"ar Internet-dom\"aner (Internet domains) och vad \"ar syftet med dem?}

\label{q:150:sa:sv:True}

\textbf{Svar}: Mnemoniska namn f\"or numeriska IP-adresser, vilket \"ar l\"attare f\"or m\"anniskor att minnas och inneb\"ar att man kan byta IP-adress men \"and\r{a} beh\r{a}lla samma mnemoniska adress.



\subsection{Ge tv\r{a} exempel p\r{a} Internet-applikationer med \"oppna (allm\"ant tillg\"angliga) protokoll?}

\label{q:151:sa:sv:True}

\textbf{Svar}: T.ex. HTTP (hypertext transfer protocol) och FTP (file transfer protocol).



\subsection{Vad \"ar det f\"or skillnad mellan protokollen HTTP och HTTPS?}

\label{q:152:sa:sv:True}

\textbf{Svar}: Trafiken \"over HTTP \"ar inte krypterad, medan trafiken \"over HTTPS \"ar krypterad (public key encryption).



\subsection{F\"orklara kortfattat skillnaden mellan n\"atverkskomponenterna hub, switch och router?}

\label{q:153:sa:sv:True}

\textbf{Svar}: En hub kopplar ihop maskiner/n\"atverksenheter till ett n\"atverk, och skickar all trafik till alla anslutna enheter.En switch \"ar en typ av smartare hub som ansluter n\"atverksenheter p\r{a} samma s\"att som en hub, men skickar bara trafik mellan de enheter/portar som ska kommunicera. En router ansluter flera n\"arverk till varandra och skickar trafik vidare mellan n\"atverk.



\subsection{Vad \"overf\"ors med de olika protokollen FTP, HTTP, SMTP?}

\label{q:154:sa:sv:True}

\textbf{Svar}: FTP \"overf\"or filer. HTTP \"overf\"or olika typer av resurser f\"or t ex webbsidor mm (t ex html-dokument, bilder, ljud mm). SMTP \"overf\"or elektronisk post.



\subsection{Vad \"ar ett certifikat? Kan man lita lika mycket p\r{a} alla certifikat? Motivera ditt svar!}

\label{q:155:sa:sv:True}

\textbf{Svar}: Ett certifikat \"ar ett elektroniskt dokument som visar vem som \"ager en viss identitet eller krypteringsnyckel. Hur mycket man litar p\r{a} ett certifikat beror p\r{a} tilliten till den som har utf\"ardat certifikatet. Vissa certifikatutf\"ardare, t ex en del myndigheter, \r{a}tnjuter en h\"ogre tillit, medans ett certifikat utf\"ardat av en sj\"alv eller en mindre trov\"ardig eller seri\"os akt\"or \"ar mindre v\"art tillit. J\"amf\"or med vanliga identitetshandlingar {\textendash} ett pass utf\"ardat av staten genom Polismyndigheten \"ar v\"art betydligt mer \"an ett pass som man har tillverkat hemma.



\subsection{Vad inneb\"ar en digital signatur (digital signature) vid publik-nyckel-kryptering (public key encryption), d.v.s. att vid \"overf\"oringen av en fil s\r{a} kan man garantera avs\"andarens identitet?}

\label{q:156:sa:sv:True}

\textbf{Svar}: Att filen \"ar krypterad med avs\"andarens privata nyckel (private key).



\subsection{Vad k\"annetecknar ett distribuerat system (distributed system)?}

\label{q:157:sa:sv:True}

\textbf{Svar}: Consists of software units that execute on several different computers.



\subsection{Hur m\r{a}nga g\r{a}nger fler adresser kan representeras med IPv6 j\"amf\"ort med IPv4 (som vanligt s\r{a} beh\"over ni inte r\"akna ut ett v\"arde utan det r\"acker med att st\"alla upp en korrekt utr\"akning)?}

\label{q:158:sa:sv:True}

\textbf{Svar}: 2^128 / 2^32 = 2^96



\subsection{Vad \"ar ett distribuerat system (distributed system)?}

\label{q:159:sa:sv:True}

\textbf{Svar}: Best\r{a}r av programvaruenheter som exekverar p\r{a} flera olika datorer.



\subsection{Vad anv\"ands HTML till?}

\label{q:160:sa:sv:True}

\textbf{Svar}: HTML (hyper text markup language) \"ar ett spr\r{a}k som anv\"ands f\"or att skapa/beskriva webbsidor.



\subsection{Inom public key encryption anv\"ands begreppet certifikat, vad \"ar det?}

\label{q:161:sa:sv:True}

\textbf{Svar}: Ett paket best\r{a}ende av namn/identitet och publik nyckel, vilket, om det \"ar utf\"ardat av en certificate authority, intygar att man \"ar den man utger sig f\"or att vara.



\subsection{Inom public key encryption anv\"ands begreppet certificate authority, vad \"ar det?}

\label{q:162:sa:sv:True}

\textbf{Svar}: En organisation som ger ut certifikat (ett paket best\r{a}ende av namn/identitet och publik nyckel), vilket intygar att man \"ar den man utger sig f\"or att vara.



\subsection{Vad \"ar en f\"ordel med att anv\"anda TCP ist\"allet f\"or UDP? Vad \"ar en nackdel?}

\label{q:163:sa:sv:True}

\textbf{Svar}: En f\"ordel \"ar att det \"ar mer p\r{a}litligt, en nackdel \"ar att det \"ar l\r{a}ngsammare.



\subsection{Vad \"ar en f\"ordel med att anv\"anda UDP ist\"allet f\"or TCP? Vad \"ar en nackdel?}

\label{q:164:sa:sv:True}

\textbf{Svar}: En f\"ordel \"ar att det \"ar snabbare, en nackdel \"ar att det \"ar mindre p\r{a}litligt. An advantage is that it is faster, a disadvantage is that it is less reliable.



\subsection{Vilka \"ar de fyra Internet-mjukvarulagren?}

\label{q:165:sa:sv:True}

\textbf{Svar}: Application, transport, network, link.



\subsection{F\"orklara kortfattat begreppet client-server!}

\label{q:166:sa:sv:True}

\textbf{Svar}: En modell d\"ar klienter beg\"ar att tj\"anster ska utf\"oras av servrar som tillhandah\r{a}ller tj\"ansterna.



\subsection{Vad anv\"ands SMTP till?}

\label{q:167:sa:sv:True}

\textbf{Svar}: SMTP \"ar ett protokoll f\"or att hantera elektronisk post.



\subsection{Om person A vill skicka ett meddelande till person B, krypterat enligt public-key-encryption, s\r{a} att ingen annan \"an B kan l\"asa meddelandet. Vad beh\"over meddelandet d\r{a} krypteras med innan meddelandet skickas fr\r{a}n A?}

\label{q:168:sa:sv:True}

\textbf{Svar}: Person B:s publika nyckel.



\subsection{Om person A vill skicka ett meddelande till person B, krypterat enligt public-key-encryption, s\r{a} att ingen annan \"an A kan ha skickat meddelandet. Vad beh\"over meddelandet d\r{a} krypteras med innan meddelandet skickas fr\r{a}n A?}

\label{q:169:sa:sv:True}

\textbf{Svar}: Person A:s privata nyckel



\subsection{N\"amn en f\"ordel med publik-nyckel-kryptering j\"amf\"ort med symmetriska krypteringstekniker.}

\label{q:170:sa:sv:True}

\textbf{Svar}: Du kan anv\"anda en kombination av publik och privat nyckel f\"or kryptering och dekryptering p\r{a} ett s\r{a}dant s\"att att du inte beh\"over utbyta n\r{a}gon nyckel i f\"orv\"ag.



\subsection{Vad anv\"ands en hub respektive en router f\"or?}

\label{q:171:sa:sv:True}

\textbf{Svar}: En hub kopplar ihop maskiner/n\"atverksenheter till ett n\"atverk. En router kopplar ihop flera n\"atverk till ett internet (n\"atverk av n\"atverk).



\subsection{N\"amn en f\"ordel med UDP framf\"or TCP, och en f\"ordel med TCP framf\"or UDP.}

\label{q:172:sa:sv:True}

\textbf{Svar}: UDP \"ar effektivare, och TCP \"ar mer tillf\"orlitligt.



\subsection{Vad \"ar rekursion?}

\label{q:173:sa:sv:True}

\textbf{Svar}: Alternativ 1: Rekursion inneb\"ar en repetition genom att en subrutin/funktion anropar sig sj\"alv. Alternativ 2: En repetition d\"ar varje steg (i repetitionen) l\"oser en deluppgift av tidigare steg (i repetitionen).



\subsection{Varf\"or \"ar bin\"ar s\"okning b\"attre \"an sekvensiell s\"okning p\r{a} sorterat data?}

\label{q:174:sa:sv:True}

\textbf{Svar}: D\"arf\"or att med bin\"ars\"okning s\r{a} v\"axer antalet steg i s\"okprocessen logaritmiskt med antalet poster, medan med sekvensiell s\"okning s\r{a} v\"axer antalet steg linj\"art med antalet poster, vilket inneb\"ar att bin\"ars\"okning \"ar betydligt effektivare.



\subsection{\"A\ensuremath{\ddot{}}r det n\r{a}gon skillnad mellan iteration och rekursion n\"ar det g\"aller anv\"andningen av minne?}

\label{q:175:sa:sv:True}

\textbf{Svar}: Ja, varje rekursivt anrop i en rekursion kr\"aver extra minne, till skillnad fr\r{a}n en iteration d\"ar varje varv inte kr\"aver n\r{a}got extra minne.



\subsection{Vad \"ar skillnaden mellan en algoritm och ett program?}

\label{q:177:sa:sv:True}

\textbf{Svar}: Ett program \"ar en algoritm kodad p\r{a} ett s\r{a}dant s\"att att en dator kan exekvera den.



\subsection{Vilka tv\r{a} olika metoder anv\"ands f\"or att verifiera att ett program \"ar korrekt (software verification)?}

\label{q:178:sa:sv:True}

\textbf{Svar}: Statisk verifiering (static verification) eller kodanalys (code analysis), och testning (testing).



\subsection{Vad \"ar ett program i f\"orh\r{a}llande till en algoritm?}

\label{q:179:sa:sv:True}

\textbf{Svar}: Ett program \"ar en algoritm kodad i ett programmeringsspr\r{a}k, d.v.s. p\r{a} ett s\r{a}dant s\"att att en dator kan exekvera den.



\subsection{Beskriv hur bin\"ars\"okning g\r{a}r till! Vilka krav finns p\r{a} den data som man s\"oker i?}

\label{q:180:sa:sv:True}

\textbf{Svar}: Bin\"ars\"okning kr\"aver att den data man s\"oker i \"ar sorterad. Vid varje repetition i s\"okningen halveras antalet poster. F\"or varje repetition s\r{a} unders\"ok posten i mittenpositionen: om posten som efters\"oks ordnas f\"ore posten i mittenpositionen s\r{a} forts\"att s\"okningen i f\"orsta halvan; om posten som efters\"oks ordnas efter posten i mittenpositionen s\r{a} forts\"att s\"okningen i andra halvan. Forts\"att p\r{a} liknande s\"att och avsluta s\"okningen n\"ar posten antingen hittats eller den kvarvarande halvan \"ar tom.



\subsection{Vilka metoder kan anv\"andas f\"or att verifiera ett programs korrekthet?}

\label{q:181:sa:sv:True}

\textbf{Svar}: Statisk verifiering (static verification) eller kodanalys (code analysis), och testning (testing).



\subsection{P\r{a} vilka tv\r{a} grundl\"aggande olika s\"att kan man \r{a}stadkomma repetition i en algoritm?}

\label{q:182:sa:sv:True}

\textbf{Svar}: Rekursion och iteration.



\subsection{N\"ar \"ar sekventiell s\"okning att f\"oredra framf\"or bin\"ars\"okning?}

\label{q:183:sa:sv:True}

\textbf{Svar}: Sekventiell s\"okning \"ar att f\"oredra f\"or mycket korta listor och n\"ar data inte \"ar sorterat eftersom bin\"ars\"okning kr\"aver sorterad data.



\subsection{Vad \"ar en f\"oruts\"attning f\"or att bin\"ars\"okning (binary search) ska fungera? Motivera ditt svar.}

\label{q:184:sa:sv:True}

\textbf{Svar}: Sorterad data.



\subsection{\"Ar bin\"ars\"okning ett bra val f\"or att s\"oka i osorterad data? Motivera ditt svar.}

\label{q:185:sa:sv:True}

\textbf{Svar}: Nej, det fungerar inte med osorterad data.



\subsection{Definiera begreppet algoritm (algorithm)!}

\label{q:186:sa:sv:True}

\textbf{Svar}: En algoritm \"ar en ordnad m\"angd av otvetydiga, exekverbara steg som beskriver en avslutande process.



\subsection{Kan alla algoritmer beskrivas som ett fl\"odes-schema (flow chart)? Motivera ditt svar!}

\label{q:187:sa:sv:True}

\textbf{Svar}: Ja, rektanglar och romber beskriver exekverbara steg, varav romber beskriver villkor, och pilar beskriver sekvenser och loopar, vilket \"ar vad som beh\"ovs f\"or att beskriva varje t\"ankbar algoritm.



\subsection{\"Ar ett programmeringsspr\r{a}k, t.ex. Python, l\"ampligt f\"or att beskriva algoritmer? Motivera ditt svar!}

\label{q:188:sa:sv:True}

\textbf{Svar}: Ja, f\"or att programmeringsspr\r{a}k har v\"aldefinierade primitiv och regler f\"or hur primitiven kan kombineras. (Nej, f\"or att programmeringsspr\r{a}k kr\"aver att man specificerar m\r{a}nga detaljer.)



\subsection{Vad inneb\"ar top-down metodologin n\"ar man utvecklar (eller uppt\"acker) algoritmer?}

\label{q:189:sa:sv:True}

\textbf{Svar}: Att man startar fr\r{a}n en h\"og abstraktionsniv\r{a} och stegvis arbetar sig ner\r{a}t genom att dela upp problem i mindre delar.



\subsection{Varf\"or \"ar det inte s\r{a} viktigt att f\"olja en strikt syntax i pseudokod?}

\label{q:190:sa:sv:True}

\textbf{Svar}: Eftersom det \"ar m\"anniskor som ska kunna f\"orst\r{a} och l\"asa pseudokod, inte datorer.



\subsection{Varf\"or \"ar det n\"odv\"andigt att veta vilken datatyp en variabel har?}

\label{q:191:sa:sv:True}

\textbf{Svar}: Det \"ar datatypen som anger hur vi skall tolka det bitm\"onster som ligger lagrat i variabeln.



\subsection{Vad \"ar skillnaden mellan k\"allkod och objektkod?}

\label{q:192:sa:sv:True}

\textbf{Svar}: K\"allkod \"ar den programkod som programmeraren skriver. Objektkod \"ar den \"overs\"attning av k\"allkoden som g\r{a}r att k\"ora p\r{a} en dator.



\subsection{Producerar ett syntaktiskt korrekt program alltid korrekta resultat? Motivera ditt svar.}

\label{q:193:sa:sv:True}

\textbf{Svar}: Nej, ett syntaktiskt korrekt program kan inneh\r{a}lla fel, t.ex. logiska fel, som g\"or att programmet ger inkorrekta resultat.



\subsection{Vad k\"annetecknar en datastruktur av typen struct/record (aggregate type)?}

\label{q:194:sa:sv:True}

\textbf{Svar}: Att det \"ar en datastruktur sammansatt av data som kan ha olika typ.



\subsection{Vad inneb\"ar det att en parameter till en subrutin \"overf\"ors som v\"arde (passed by value)?}

\label{q:195:sa:sv:True}

\textbf{Svar}: Passed by value inneb\"ar att parametern \"overf\"ors som ett kopierat v\"arde och att detta kopierade v\"arde lagras i en lokal variabel i subrutinen.



\subsection{Vad inneb\"ar det att en parameter till en subrutin \"overf\"ors som referens (passed by reference)?}

\label{q:196:sa:sv:True}

\textbf{Svar}: Passed by reference inneb\"ar att parametern \"overf\"ors som en referens till en plats d\"ar den ligger lagrad, vilket inneb\"ar att eventuella \"andringar g\"ors i den ursprungliga variabeln.



\subsection{Vad g\"or en assemblator/assemblerare (assembler)?}

\label{q:197:sa:sv:True}

\textbf{Svar}: En assemblator/assemblerare (assembler) omvandlar assemblerkod (assembly code) till maskinkod.



\subsection{Vad k\"annetecknar en datastruktur av typen array?}

\label{q:198:sa:sv:True}

\textbf{Svar}: Att alla element i datastrukturen \"ar av samma datatyp, och att de enskilda elementen n\r{a}s via index.



\subsection{Vilka \"ar de fyra stora programmeringsparadigmerna (programming paradigms)?}

\label{q:199:sa:sv:True}

\textbf{Svar}: Imperativ, funktionell, objektorienterad och deklarativ (logik-) programmering.



\subsection{Ange fyra vanliga primitiva datatyper.}

\label{q:200:sa:sv:True}

\textbf{Svar}: Heltal (integer), flyttal (floating point number), tecken (character), boolean (sanningsv\"arde).



\subsection{Vad g\"or en kompilator (compiler)?}

\label{q:201:sa:sv:True}

\textbf{Svar}: En kompilator \"overs\"atter k\"allkod, skriven i ett programmeringsspr\r{a}k, till exekverbar maskinkod.



\subsection{Ett program kan ge upphov till tre olika typer av fel: syntaktiska fel (syntactic errors), exekveringsfel (runtime errors) och logiska fel (logic errors). Vilken typ av fel \"ar mest allvarliga och varf\"or?}

\label{q:202:sa:sv:True}

\textbf{Svar}: Logiska fel, eftersom de inte ger upphov till n\r{a}got felmeddelande.



\subsection{Ett program kan ge upphov till tre olika typer av fel: syntaktiska fel (syntactic errors), exekveringsfel (runtime errors) och logiska fel (logic errors). Vilket typ av fel \"ar minst allvarliga och varf\"or?}

\label{q:203:sa:sv:True}

\textbf{Svar}: Syntaktiska fel, eftersom de uppt\"acks redan av kompilatorn.



\subsection{Vad \"ar concurrent programming?}

\label{q:204:sa:sv:True}

\textbf{Svar}: Programmering d\"ar man programmerar flera parallella exekveringsv\"agar (tr\r{a}dar) i samma program.



\subsection{Beskriv kortfattat begreppen sekvens, selektion och iteration.}

\label{q:205:sa:sv:True}

\textbf{Svar}: En sekvens \"ar en f\"oljd av instruktioner som utf\"ors i ordning. Selektion \"ar en valm\"ojlighet - att utf\"ora olika instruktioner beroende p\r{a} programmets tillst\r{a}nd (t ex genom if-satser). Iteration inneb\"ar att samma instruktion eller sekvens av instruktioner k\"ors flera g\r{a}nger (t ex med loopar).



\subsection{En variabel pekar p\r{a} ett bitm\"onster i lagrat i minnet; vad beh\"over vi veta f\"or att kunna tolka bitm\"onstret p\r{a} r\"att s\"att?}

\label{q:206:sa:sv:True}

\textbf{Svar}: Vilken datatyp variabelns data har. Det \"ar datatypen som anger hur programmet skall tolka det bitm\"onster som ligger lagrat i variabeln.



\subsection{Vad inneb\"ar begreppen sekvens, selektion och iteration?}

\label{q:207:sa:sv:True}

\textbf{Svar}: Sekvens: en f\"oljd av instruktioner som utf\"ors i ordning. Selektion: olika instruktioner utf\"ors beroende p\r{a} programmets tillst\r{a}nd. Iteration: samma (sekvens av) instruktioner utf\"ors flera g\r{a}nger (beroende p\r{a} programmets tillst\r{a}nd).



\subsection{Ge exempel p\r{a} tv\r{a} olika s\"att att beskriva algoritmer.}

\label{q:208:sa:sv:True}

\textbf{Svar}: Fl\"odesdiagram och pseudokod.



\subsection{Vilken generation av programmeringsspr\r{a}k k\"annetecknas av:- ett-till-ett-f\"orh\r{a}llande mellan spr\r{a}kinstruktioner och maskininstruktioner; - inneboende maskin-beroende?}

\label{q:209:sa:sv:True}

\textbf{Svar}: Andra generationen.



\subsection{Vilken generation av programmeringsspr\r{a}k k\"annetecknas av:- maskinoberoende (vanligtvis);- varje primitiv motsvarar en sekvens av maskinspr\r{a}ksinstruktioner?}

\label{q:210:sa:sv:True}

\textbf{Svar}: Tredje generationen.



\subsection{Vad \"ar en literal i ett programmeringsspr\r{a}k?}

\label{q:211:sa:sv:True}

\textbf{Svar}: Ett explicit v\"arde av en viss datatyp.



\subsection{Vad \"ar en konstant i ett programmeringsspr\r{a}k?}

\label{q:212:sa:sv:True}

\textbf{Svar}: En namngiven literal. / Ett namngivet v\"arde av en viss datatyp.



\subsection{I objektorienterad programmering har man klasser och objekt. Ut\"over detta s\r{a} finns det tre egenskaper som k\"annetecknar objektorienterad programmering, vilka?}

\label{q:213:sa:sv:True}

\textbf{Svar}: Arv, inkapsling och polymorfism.



\subsection{\"Overs\"attningen fr\r{a}n k\"allkod till maskinkod sker i tre steg av tre olika enheter i \"overs\"attaren; vad kallas dessa tre enheter?}

\label{q:214:sa:sv:True}

\textbf{Svar}: Lexikalisk analysator, parser och kodgenerator. Lexical analyzer, parser och code generator.



\subsection{Vad \"ar en tr\r{a}d i concurrent programmering?}

\label{q:215:sa:sv:True}

\textbf{Svar}: En concurrent/parallell exekveringsv\"ag inom samma program/process.



\subsection{Vad \"ar den grundl\"aggande byggstenen i logikprogrammeringsspr\r{a}k?}

\label{q:216:sa:sv:True}

\textbf{Svar}: Logisk formel (predikat).



\subsection{Vad \"ar en variabel i ett programmeringsspr\r{a}k?}

\label{q:217:sa:sv:True}

\textbf{Svar}: En variabel \"ar ett namngivet utrymme i prim\"arminnet (main memory).



\subsection{Vad \"ar syftet med att anv\"anda procedurenheter (subprogram, subrutin, procedur, funktion, metod, predikat etc.) vid programmering?}

\label{q:218:sa:sv:True}

\textbf{Svar}: Anv\"ands f\"or att f\"orenkla programutveckling genom abstraktion.



\subsection{Vad inneb\"ar arv i objektorienterad programmering?}

\label{q:219:sa:sv:True}

\textbf{Svar}: Arv till\r{a}ter en klass att omfatta egenskaper hos en annan klass utan att explicit beh\"ova deklarera dem.



\subsection{Vad \"ar skillnaden mellan en kompilator (compiler) och en interpretator (interpreter)?}

\label{q:220:sa:sv:True}

\textbf{Svar}: En kompilator (compiler) \"overs\"atter programkod/k\"allkod till k\"orbar kod. En interpretator (interpreter) tolkar programkod/k\"allkod vid sj\"alva k\"orningen och utf\"or d\r{a} instruktionerna i programkoden/k\"allkoden.



\subsection{Alla programmeringsspr\r{a}k har tre typer av styrning av programfl\"odet, vilka?}

\label{q:221:sa:sv:True}

\textbf{Svar}: Sekvens, selektion och repetition.



\subsection{Vilka tre saker k\"annetecknar l\"attr\"orliga utvecklingsmodeller (agile development models)?}

\label{q:222:sa:sv:True}

\textbf{Svar}: 1) Inkrementella och iterativa; 2) korta feedback-loopar; 3) utforskande n\"ar det g\"aller krav.



\subsection{Vad \"ar design patterns?}

\label{q:223:sa:sv:True}

\textbf{Svar}: Generella l\"osningar p\r{a} \r{a}terkommande problem.



\subsection{Vad \"ar syftet med use case diagram?}

\label{q:224:sa:sv:True}

\textbf{Svar}: Att beskriva det f\"oreslagna systemet fr\r{a}n anv\"andarens synvinkel.



\subsection{Vad \"ar syftet med klassdiagram (class diagrams)?}

\label{q:225:sa:sv:True}

\textbf{Svar}: Att beskriva strukturen av olika typer av objekt (klasser) och relationen mellan olika typer av objekt (klasser).



\subsection{Vilka \"ar de fyra traditionella utvecklingsfaserna vid programvaruutveckling (the traditional development phases of the software life cycle)?}

\label{q:226:sa:sv:True}

\textbf{Svar}: Kravanalys, design, implementation, testning.



\subsection{Vad \"ar huvudsyftet med att dela upp en programvara i moduler?}

\label{q:227:sa:sv:True}

\textbf{Svar}: F\"or att f\"orenkla programvaruutveckling genom att en enskild modul \"ar \"overblickbar och kan utvecklas oberoende av andra moduler.



\subsection{Vilka \"ar de tre \"onskv\"arda egenskaperna f\"or moduler som man vill uppn\r{a} n\"ar man delar upp en programvara i moduler?}

\label{q:228:sa:sv:True}

\textbf{Svar}: H\"og sammanh\r{a}llning (cohesion), l\r{a}g koppling (coupling), information hiding.



\subsection{Vad \"ar det f\"or skillnad p\r{a} glass-box-testning (glass-box testing) och black-box-testning (black-box testing)?}

\label{q:229:sa:sv:True}

\textbf{Svar}: Glass-box testning inneb\"ar att testaren k\"anner till den interna strukturen av programvaran som skall testas och utnyttjar denna information vid konstruktion av testerna. Detta till skillnad fr\r{a}n black-box testning som inte bygger p\r{a} kunskap om programvarans interna struktur.



\subsection{Beskriv skillnaderna mellan en-till-en- (one-to-one), en-till-m\r{a}nga- (one-to-many) och m\r{a}nga-till- m\r{a}nga- (many-to-many) relationer, g\"arna med hj\"alp av exempel.}

\label{q:230:sa:sv:True}

\textbf{Svar}: Ett exempel p\r{a} en en-till-en-relation \"ar {\textquotedblright}man-hustru{\textquotedblright}, eftersom en man endast kan vara man till en hustru och en kvinna endast kan vara hustru till en man (enligt svenska lagar). Ett exempel p\r{a} en en-till- m\r{a}nga-relation \"ar {\textquotedblright}mamma-barn{\textquotedblright}, eftersom ett barn endast har en (biologisk) mamma men en kvinna kan vara mamma till m\r{a}nga barn. Ett exempel p\r{a} en m\r{a}nga-till-m\r{a}nga-relation \"ar {\textquotedblright}bror-syster{\textquotedblright}, eftersom en pojke kan ha flera systrar och en flicka kan ha flera br\"oder.



\subsection{Vad kallas programvarutekniken som bygger p\r{a} att konstruera programvara genom att kombinera olika f\"ardiga komponenter (ist\"allet f\"or att utveckla egna komponenter)?}

\label{q:231:sa:sv:True}

\textbf{Svar}: Component architecture eller component-based software engineering.



\subsection{Beskriv ett exempel p\r{a} var och en av de olika typerna av relation: en-till-en (one-to-one), en-till- m\r{a}nga (one-to-many) och m\r{a}nga-till-m\r{a}nga (many-to-many)!}

\label{q:232:sa:sv:True}

\textbf{Svar}: Ett exempel p\r{a} en en-till-en-relation \"ar {\textquotedblright}man-hustru{\textquotedblright}, eftersom en man endast kan vara man till en hustru och en kvinna endast kan vara hustru till en man (enligt svenska lagar). Ett exempel p\r{a} en en-till- m\r{a}nga-relation \"ar {\textquotedblright}mamma-barn{\textquotedblright}, eftersom ett barn endast har en (biologisk) mamma men en kvinna kan vara mamma till m\r{a}nga barn. Ett exempel p\r{a} en m\r{a}nga-till-m\r{a}nga-relation \"ar {\textquotedblright}bror-syster{\textquotedblright}, eftersom en pojke kan ha flera systrar och en flicka kan ha flera br\"oder.



\subsection{Vilka \"ar de fyra stegen i traditionell mjukvaruutveckling (med t ex vattenfallsmodellen)?}

\label{q:233:sa:sv:True}

\textbf{Svar}: Kravanalys, design, implementation, testning.



\subsection{Beskriv kortfattat n\r{a}gra f\"ordelar med att dela upp program i moduler?}

\label{q:234:sa:sv:True}

\textbf{Svar}: Att f\"orenkla programvaruutvecklingen genom att en enskild modul \"ar \"overblickbar och kan utvecklas oberoende av andra moduler.



\subsection{Vad inneb\"ar prototyping?}

\label{q:235:sa:sv:True}

\textbf{Svar}: Prototyping inneb\"ar att man utvecklar och utv\"arderar en prototyp (en of\"ardig version av det som skall utvecklas).



\subsection{Beskriv vad en sprint inom agil utveckling med Scrum \"ar?}

\label{q:236:sa:sv:True}

\textbf{Svar}: En sprint \"ar en fas/iteration p\r{a} 2 till 4 veckor som ska leverera n\r{a}got resultat (a sprint is an iteration of 2 to 4 weeks and should have some deliveries).



\subsection{Vad utm\"arker black-box-testning (black-box testing)?}

\label{q:237:sa:sv:True}

\textbf{Svar}: Vid black-box testning har testaren ingen kunskap om programvarans interna struktur, till skillnad fr\r{a}n glass-box testning som inneb\"ar att testaren k\"anner till den interna strukturen av programvaran som skall testas och utnyttjar denna information vid konstruktion av testerna.



\subsection{Vad \"ar design patterns (designm\"onster) och vad \"ar de bra f\"or?}

\label{q:238:sa:sv:True}

\textbf{Svar}: Generella l\"osningar p\r{a} \r{a}terkommande problem. Genom att utg\r{a} fr\r{a}n f\"ardiga och bra l\"osningar kan man snabba upp utvecklingstiden och ocks\r{a} g\"ora program mer robusta d\r{a} designm\"onstren ofta \"ar v\"al bepr\"ovade. Designm\"onster ger ocks\r{a} utvecklare och systemarkitekter en gemensamt vokabul\"ar f\"or att diskutera och referera till olika l\"osningar.



\subsection{F\"orklara begreppen koppling (coupling) och sammanh\r{a}llning (cohesion)?}

\label{q:239:sa:sv:True}

\textbf{Svar}: Koppling (coupling) \"ar ett m\r{a}tt p\r{a} hur sammanfl\"atade olika moduler/komponenter \"ar i ett mjukvarusystem (the linkage between modules). L\r{a}g coupling \"ar bra. Sammanh\r{a}llning (cohesion) \"ar ett m\r{a}tt p\r{a} hur v\"al inneh\r{a}llet i en modul h\"anger ihop, hur fokuserad den \"ar (the internal binding within a module). H\"og cohesion \"ar bra.



\subsection{Vilka tre olika typer av relationer mellan entiteter \"ar viktiga att skilja p\r{a} vid programvaruutveckling?}

\label{q:240:sa:sv:True}

\textbf{Svar}: En-till-en, en-till-m\r{a}nga och m\r{a}nga-till-m\r{a}nga.



\subsection{Vad kallas programvaruutvecklingsmetoder som v\"ardes\"atter:- individer och interaktioner mer \"an processer och verktyg;- fungerande programvara mer \"an omfattande dokumentation; - kundsamarbete mer \"an kontraktsf\"orhandlingar.- lyh\"ordhet f\"or f\"or\"andring mer \"an att f\"olja en plan.}

\label{q:241:sa:sv:True}

\textbf{Svar}: Agila.



\subsection{Vad \"ar en programvarumodul?}

\label{q:242:sa:sv:True}

\textbf{Svar}: En hanterbar enhet av programvaran som endast hanterar en del av det arbete som hela programvaran ska utf\"ora.



\subsection{Vad \"ar syftet med Scrum-m\"otet {\textquotedblleft}sprint retrospective{\textquotedblright}?}

\label{q:243:sa:sv:True}

\textbf{Svar}: Att f\"orb\"attra sj\"alva arbetsprocessen utifr\r{a}n erfarenheterna fr\r{a}n den senaste sprinten.



\subsection{Hur m\r{a}nga medlemmar b\"or ett utvecklingsteam ha enligt Scrum?}

\label{q:244:sa:sv:True}

\textbf{Svar}: 3 - 9.



\subsection{Ge tv\r{a} exempel p\r{a} diagram som anv\"ands vid modellering (vid programvaruutveckling).}

\label{q:245:sa:sv:True}

\textbf{Svar}: Strukturdiagram, datafl\"odesdiagram, anv\"andningsfallsdiagram, klassdiagram.



\subsection{Vad \"ar ett designm\"onster (vid programvaruutveckling)?}

\label{q:246:sa:sv:True}

\textbf{Svar}: En f\"orutvecklad modell f\"or att l\"osa \r{a}terkommande problem.



\subsection{Vad utm\"arker glass-box-testning?}

\label{q:247:sa:sv:True}

\textbf{Svar}: Vid glass-box testning k\"anner testaren till den interna strukturen hos programvaran som skall testas och utnyttjar denna information vid konstruktion av testerna, till skillnad fr\r{a}n black-box testning d\r{a} testaren inte har n\r{a}gon kunskap om programvarans interna struktur.



\subsection{Vilka 3 fr\r{a}gor skall varje team-medlem kort besvara vid Daily Scrum-m\"otena?}

\label{q:248:sa:sv:True}

\textbf{Svar}: Vad gjorde du ig\r{a}r? Vad ska du g\"ora idag? Finns det n\r{a}gra hinder som hindrar dig fr\r{a}n att utf\"ora ditt arbete?



\subsection{Vad kallas Scrum-m\"otet, i slutet av en sprint d\"ar ni diskuterar vad som gick bra under den tidigare sprintprocessen och vad som kan f\"orb\"attras inf\"or n\"asta sprint?}

\label{q:249:sa:sv:True}

\textbf{Svar}: Sprint retrospective.



\subsection{Utvecklingsmetoden Scrum har tre olika roller definierade, vilka?}

\label{q:250:sa:sv:True}

\textbf{Svar}: Product Owner (produkt\"agare), Scrum master och Development team. Development team-rollen delas ofta av flera (3-9) personer.



\subsection{Vad \"ar en abstrakt datatyp (abstract data type)?}

\label{q:251:sa:sv:True}

\textbf{Svar}: En datatyp som inneh\r{a}ller b\r{a}de data och operationer f\"or att hantera datat.



\subsection{Vad k\"annetecknar ett sorterat bin\"art tr\"ad (sorted binary tree) ({\textquotedblright}bin\"art s\"oktr\"ad{\textquotedblright})?}

\label{q:252:sa:sv:True}

\textbf{Svar}: Att varje nod i tr\"adet har tv\r{a} eller f\"arre subtr\"ad (barnnoder), att alla noder i det v\"anstra subtr\"adet inneh\r{a}ller v\"arden l\"agre \"an inneh\r{a}llet i den aktuella noden, och att alla noder i det h\"ogra subtr\"adet inneh\r{a}ller v\"arden h\"ogre \"an inneh\r{a}llet i den aktuella noden.



\subsection{Vilka \"ar de fyra grundl\"aggande datastrukturerna (basic data structures) ut\"over arrayer?}

\label{q:253:sa:sv:True}

\textbf{Svar}: Listor, stackar, k\"oer och tr\"ad.



\subsection{Vad \"ar skillnaden mellan en dynamisk och en statisk datastruktur?}

\label{q:254:sa:sv:True}

\textbf{Svar}: En statisk datastruktur kan inte \"andra form eller storlek, vilket en dynamisk datastruktur kan g\"ora.



\subsection{Vad k\"annetecknar ett bin\"art tr\"ad?}

\label{q:255:sa:sv:True}

\textbf{Svar}: Ett bin\"art tr\"ad \"ar en tr\"adstruktur d\"ar varje nod kan ha maximalt tv\r{a} barn-noder.



\subsection{Vad \"ar skillnaden mellan en statisk (static) och en dynamisk (dynamic) datastruktur (data structure)?}

\label{q:256:sa:sv:True}

\textbf{Svar}: Form och storlek (strukturen) av en statisk datastruktur f\"or\"andras inte \"over tiden, \"aven om inneh\r{a}llet kan f\"or\"andras, medan form och storlek (strukturen) av en dynamisk datastruktur kan f\"or\"andras.



\subsection{Kan en lista implementeras som en statisk eller dynamisk datastruktur, b\r{a}de och, eller varken eller? Motivera ditt svar!}

\label{q:257:sa:sv:True}

\textbf{Svar}: En lista kan b\r{a}de implementeras som en statisk datastruktur, t.ex. som en array, och som en dynamisk datastruktur best\r{a}ende av element och pekare.



\subsection{Beskriv de grundl\"aggande datastrukturerna stack (stack) och k\"o (queue)?}

\label{q:258:sa:sv:True}

\textbf{Svar}: En stack \"ar en lista d\"ar man l\"agger till och tar bort element i samma \"ande enligt principen {\textquotedblright}last-in-first- out{\textquotedblright} (LIFO). En k\"o \"ar en lista d\"ar man l\"agger till i ena \"anden och tar bort i den andra \"anden enligt principen {\textquotedblright}first-in-first-out{\textquotedblright} (FIFO).



\subsection{Kan l\r{a}gniv\r{a}-datastrukturen array anv\"andas f\"or att implementera en k\"o (queue)? Motivera ditt svar!}

\label{q:259:sa:sv:True}

\textbf{Svar}: Ja, d\"ar elementen i arrayen beskriver en cirkul\"ar k\"o, och man har en pekare till k\"ons huvud (start) och en pekare till dess svans (\"ande).



\subsection{Vad \"ar en abstrakt datastruktur? Vad \"ar skillnaden mot en record/struct?}

\label{q:260:sa:sv:True}

\textbf{Svar}: En abstrakt datastruktur beskriver en datatyp och dess operationer, allts\r{a} b\r{a}de vad som lagras och vad man kan g\"ora med det. En record/struct \"ar en sammansatt datastruktur som \"ar en samling av data som kan vara av olika datatyper.



\subsection{Ge ett exempel p\r{a} en datastruktur som anv\"ander principen LIFO och en datastruktur som anv\"ander principen FIFO?}

\label{q:261:sa:sv:True}

\textbf{Svar}: LIFO: Det senast tillagda elementet tas bort f\"orst -> StackFIFO: Det f\"orst tillagda elementet tas bort f\"orst -> K\"o (Queue)



\subsection{Listor kan lagras antingen i sammanh\"angande block i minnet, eller i form av l\"ankade listor. Vilket \"ar att f\"oredra f\"or statiska listor, och vilket \"ar b\"attre f\"or dynamiska listor?}

\label{q:262:sa:sv:True}

\textbf{Svar}: F\"or statiska listor vars inneh\r{a}ll inte f\"or\"andras \"ar sammanh\"angande minnesblock att f\"oredra d\r{a} det ger bra prestanda vid l\"asning och anv\"ander lite minne/lagringskapacitet. F\"or dynamiska listor som kan f\"or\"andras kostar det mycket att l\"agga till och ta bort element i ett sammanh\"angande minnesblock eftersom det kan leda till att mycket data beh\"over flyttas. D\"arf\"or \"ar det vanligtvis b\"attre med l\"ankade listor f\"or dynamiska listor d\r{a} ins\"attning och borttag inte kr\"aver mer \"an att n\r{a}gra pekare \"andras (data beh\"over inte flyttas runt).



\subsection{F\"orklara vad en pekare (pointer) \"ar?}

\label{q:263:sa:sv:True}

\textbf{Svar}: En pekare \"ar en variabel som inneh\r{a}ller minnesadressen till det som den pekar p\r{a}.



\subsection{Tv\r{a} typer av specialiserade listor \"ar stack och k\"o, beskriv hur de skiljer sig fr\r{a}n varandra!}

\label{q:264:sa:sv:True}

\textbf{Svar}: En stack \"ar en lista d\"ar man l\"agger till och tar bort element i samma \"ande enligt principen {\textquotedblright}last-in-first-out{\textquotedblright} (LIFO). En k\"o \"ar en lista d\"ar man l\"agger till i ena \"anden och tar bort i den andra \"anden enligt principen {\textquotedblright}first-in-first-out{\textquotedblright} (FIFO).



\subsection{Vad skiljer en abstrakt datatyp (abstract data type) fr\r{a}n en sammansatt datatyp (aggregate type / struct / record)?}

\label{q:265:sa:sv:True}

\textbf{Svar}: En abstrakt datatyp beskriver en datatyp och dess operationer (metoder, procedurer, funktioner), allts\r{a} b\r{a}de vad som lagras (data) och vad man kan g\"ora med det. En record/struct \"ar en sammansatt datastruktur som \"ar en samling av data som kan vara av olika datatyper, men inneh\r{a}ller inga operationer (metoder, procedurer, funktioner).



\subsection{I en variant av listor l\"agger man till och tar bort element i samma \"ande, vad kallas den datastrukturen? I en annan variant l\"agger man till element i ena \"anden och tar bort i den andra, vad kallas den datastrukturen?}

\label{q:266:sa:sv:True}

\textbf{Svar}: Det f\"orsta \"ar en stack, det andra \"ar en k\"o.



\subsection{Vad k\"annetecknar en aggregattyp (struct/record)?}

\label{q:267:sa:sv:True}

\textbf{Svar}: Ett block av data d\"ar olika element kan vara av olika datatyp, elementen kallas f\"alt och n\r{a}s med namn.



\subsection{Kan en k\"o implementeras som en statisk eller dynamisk datastruktur, b\r{a}de och, eller varken eller? Motivera ditt svar!}

\label{q:268:sa:sv:True}

\textbf{Svar}: En k\"o kan b\r{a}de implementeras som en statisk datastruktur, t.ex. som en array, och som en dynamisk datastruktur best\r{a}ende av element och pekare.



\subsection{Vad k\"annetecknar en dynamisk datastruktur till skillnad fr\r{a}n en statiskt datastruktur?}

\label{q:269:sa:sv:True}

\textbf{Svar}: En dynamisk datastruktur kan \"andra form och storlek \"over tiden, men det kan inte en statisk datastruktur.



\subsection{Vad k\"annetecknar datastrukturen bin\"art tr\"ad?}

\label{q:270:sa:sv:True}

\textbf{Svar}: Att varje nod i tr\"adet har som mest tv\r{a} barn (noder).



\subsection{Vad k\"annetecknar rotnoden i en tr\"ad-datastruktur?}

\label{q:271:sa:sv:True}

\textbf{Svar}: Att den inte har n\r{a}gon f\"or\"alder (f\"or\"aldra-nod).



\subsection{Beskriv en f\"ordel och en nackdel med att lagra en aggregattyp (struct/record) i ett sammanh\"angande block ist\"allet f\"or de olika delarna p\r{a} separata platser utpekade av pekare.}

\label{q:272:sa:sv:True}

\textbf{Svar}: F\"ordel: lite snabbare (direkt) \r{a}tkomst till de olika delarna. Nackdel: de olika delarna har alltid samma storlek, vilket inneb\"ar att det g\r{a}r \r{a}t on\"odigt utrymme och att data som kr\"aver mycket plats kanske inte f\r{a}r plats.



\subsection{Vad k\"annetecknar en array?}

\label{q:273:sa:sv:True}

\textbf{Svar}: Ett block av data d\"ar alla element \"ar av samma datatyp, och elementen n\r{a}s genom index.



\subsection{Vad k\"annetecknar en statiskt datastruktur?}

\label{q:274:sa:sv:True}

\textbf{Svar}: Strukturens form eller storlek kan inte f\"or\"andras \"over tiden.



\subsection{Vad k\"annetecknar en dynamisk datastruktur?}

\label{q:275:sa:sv:True}

\textbf{Svar}: Strukturens form och storlek kan f\"or\"andras \"over tiden.



\subsection{Beskriv en f\"ordel och en nackdel med att lagra de olika delarna av en aggregattyp (struct/record) p\r{a} separata platser utpekade av pekare ist\"allet f\"or i ett sammanh\"angande block.}

\label{q:276:sa:sv:True}

\textbf{Svar}: F\"ordel: varje del kan f\r{a} precis det utrymme som kr\"avs, varken mer eller mindre. Nackdel: \r{a}tkomst till de olika delarna blir lite l\r{a}ngsammare eftersom det inneb\"ar att f\"olja en pekare ist\"allet f\"or att direkt h\"amta v\"ardet.



\subsection{Vad \"ar ett databashanteringssystem (database management system)?}

\label{q:277:sa:sv:True}

\textbf{Svar}: Ett system som sk\"oter databaser genom att utf\"ora kommandon f\"or att uppdatera databaserna och f\"or att h\"amta data fr\r{a}n databaserna. En programvara som hanterar skapande, uppdatering, s\"okning och administration av databaser.



\subsection{Vad inneb\"ar commit och rollback i databas-sammanhang?}

\label{q:278:sa:sv:True}

\textbf{Svar}: En commit inneb\"ar att en transaktion \"ar genomf\"ord och godk\"and av databashanteraren. En rollback inneb\"ar att det har uppkommit n\r{a}got problem under en transaktion och att databashanteraren d\"arf\"or \r{a}terst\"aller databasen i det tillst\r{a}nd den hade f\"ore transaktionen (transaktionen rullas tillbaka).



\subsection{F\"or relationsdabaser finns det tre (3) operationer (relational operations), med vars hj\"alp man kan skapa nya tabeller som utg\"or delm\"angder och/eller kombinationer av befintliga tabeller. Vilka operationer?}

\label{q:279:sa:sv:True}

\textbf{Svar}: Select, project och join.



\subsection{Vad inneb\"ar data mining?}

\label{q:280:sa:sv:True}

\textbf{Svar}: Data mining handlar om att uppt\"acka m\"onster i samlingar av data.



\subsection{Vad \"ar ett data warehouse?}

\label{q:281:sa:sv:True}

\textbf{Svar}: En samling statisk data fr\r{a}n en eller flera k\"allor, avsedd f\"or analys av datat.



\subsection{Vad \"ar en databas (database) i f\"orh\r{a}llande till ett databashanteringssystem (DBMS {\textendash} database management system)?}

\label{q:282:sa:sv:True}

\textbf{Svar}: En databas \"ar en organiserad samling av data (som kan hanteras av ett databashanteringssystem). Ett databashanteringssystem \"ar ett system f\"or att skapa, uppdatera och administrera databaser, samt besvara fr\r{a}gor st\"allda mot databaserna.



\subsection{N\"amn ett vanligt problem som kan uppst\r{a} vid t ex \"overf\"oringar mellan konton som transaktioner skyddar mot.}

\label{q:283:sa:sv:True}

\textbf{Svar}: Ett problem \"ar att pengar dras fr\r{a}n det ena kontot men aldrig s\"atts in p\r{a} det andra (pga avbrott eller fel).



\subsection{Vad \"ar SQL?}

\label{q:284:sa:sv:True}

\textbf{Svar}: Structured Query Language (SQL) \"ar ett deklarativt programmeringsspr\r{a}k som anv\"ands f\"or att h\"amta och manipulera data i relationsdatabaser.



\subsection{P\r{a} vilka tv\r{a} s\"att kan en transaktion avslutas?}

\label{q:285:sa:sv:True}

\textbf{Svar}: En transaktion som lyckas avslutas med en commit, och en transaktion som misslyckas avslutas genom en rollback som g\"or transaktionens arbete ogjort.



\subsection{Vad \"ar en transaktion?}

\label{q:286:sa:sv:True}

\textbf{Svar}: En sekvens av databasoperationer, som alla tillsammans antingen ska lyckas eller misslyckas.



\subsection{En transaktion kan avslutas p\r{a} tv\r{a} olika s\"att, vilka?}

\label{q:287:sa:sv:True}

\textbf{Svar}: Genom en commit eller en roll-back.



\subsection{Vad \"ar ett databas-schema?}

\label{q:288:sa:sv:True}

\textbf{Svar}: En beskrivning av databasens struktur.



\subsection{Vad \"ar en databas?}

\label{q:289:sa:sv:True}

\textbf{Svar}: En organiserad samling av data (hanterad av ett DBMS).



\subsection{Vad \"ar en databasmodell?}

\label{q:290:sa:sv:True}

\textbf{Svar}: En konceptuell vy av databasen.



\subsection{N\"amn tv\r{a} saker som skiljer en objektorienterad databas fr\r{a}n en relationsdatabas?}

\label{q:291:sa:sv:True}

\textbf{Svar}: Varje enhet lagras som ett objekt som kan inneh\r{a}lla metoder. DBMS uppr\"atth\r{a}ller l\"ankar/referenser/pekare mellan objekt.



\subsection{Vad \"ar data mining?}

\label{q:292:sa:sv:True}

\textbf{Svar}: Handlar om att uppt\"acka m\"onster i datasamlingar.



\subsection{Vad representerar en tabell i relationsmodellen f\"or databaser?}

\label{q:293:sa:sv:True}

\textbf{Svar}: En relation.



\subsection{Vad representerar en kolumn i en tabell i relationsmodellen f\"or databaser?}

\label{q:294:sa:sv:True}

\textbf{Svar}: Ett attribut.



\subsection{Vad representerar en rad i en tabell i relationsmodellen f\"or databaser?}

\label{q:295:sa:sv:True}

\textbf{Svar}: En instans.



\subsection{Vilka \"ar de tre relationsoperationerna i relationsmodellen f\"or databaser?}

\label{q:296:sa:sv:True}

\textbf{Svar}: Select, project, join.



\subsection{Vilka \"ar de tre grundl\"aggande relationsoperationerna f\"or att ta fram efterfr\r{a}gat data fr\r{a}n en relationsdatabas?}

\label{q:297:sa:sv:True}

\textbf{Svar}: Select, project och join.



\subsection{Vad \"ar ett schema i samband med ett databassystem?}

\label{q:298:sa:sv:True}

\textbf{Svar}: Ett databasschema \"ar en beskrivning av en databas struktur, vilket f\"or relationsdatabaser \"ar dess tabeller och kolumner.



\subsection{Till vilken programmeringsparadigm h\"or databasfr\r{a}gespr\r{a}ket SQL (structured query language)?}

\label{q:299:sa:sv:True}

\textbf{Svar}: Deklarativa programmeringsspr\r{a}k (declarative programming languages).



\subsection{Varf\"or \"ar det av intresse att k\"anna till en algoritms effektivitetsklass/komplexitetsklass?}

\label{q:300:sa:sv:True}

\textbf{Svar}: F\"or att kunna j\"amf\"ora olika algoritmers effektivitet, och kunna bed\"oma om en algoritm \"ar anv\"andbar f\"or stora m\"angder data.



\subsection{Processen att skapa 3D-grafik best\r{a}r av tre steg, varav det f\"orsta \"ar 3D-modellering (3D modeling), och det tredje \"ar bildvisning (display). Vad kallas det andra steget, och vad g\"ors i det steget?}

\label{q:301:sa:sv:True}

\textbf{Svar}: Rendrering (rendering), vilket handlar om att fastst\"alla hur 3D-modellen skall visas n\"ar den projiceras p\r{a} projektionsplanet (determining how the 3D-modell will appear when projected onto the projection plane).



\subsection{I animationsprojekt utf\"or man arbetet vanligtvis i tre steg, vilka?}

\label{q:302:sa:sv:True}

\textbf{Svar}: Storyboard, Key frames, In-betweening.



\subsection{Tv\r{a} grenar inom omr\r{a}det mekanik har visat sig s\"arskilt anv\"andbara vid simulering av naturliga r\"orelser, vilka?}

\label{q:303:sa:sv:True}

\textbf{Svar}: Dynamik (dynamics) och kinematik (kinematics).



\subsection{N\"amn ett s\"att att ta fram s.k. polygonal meshes vid 3D-modellering!}

\label{q:304:sa:sv:True}

\textbf{Svar}: Matematiska ekvationer; Bezier kurvor och ytor; proceduriella modeller.



\subsection{F\"orklara kortfattat skillnaden mellan lokala ljusmodeller (local lightning model) och globala ljusmodeller (global lightning model). Vilken modell ger mest realistiskt resultat? F\"ordelen med den andra?}

\label{q:305:sa:sv:True}

\textbf{Svar}: En lokal ljusmodell tar inte h\"ansyn till hur olika objekt p\r{a}verkar varandra. En global ljusmodell g\"or det (eller f\"ors\"oker i alla fall). Ray-tracing \"ar ett exempel p\r{a} en algoritm som anv\"ands f\"or att ber\"akna en global ljusmodell. En global modell ger ett b\"attre, mer realistiskt resultat, men en lokal modell \"ar enklare och mindre ber\"akningsintensiv.



\subsection{Inom datorgrafik spelar ljus en viktig roll. Ljus brukar delas in i tre (3) olika sorter, vilka? Vad skiljer dem \r{a}t?}

\label{q:306:sa:sv:True}

\textbf{Svar}: Fr\r{a}gan handlar om olika typer av reflekterande ljus: Speglande ljus (specular light), som reflekteras utan att splittras upp; syns som en ljus skinande punkt p\r{a} ett objekt och beh\r{a}ller ljusk\"allans f\"arg. Syns tydligare p\r{a} sl\"ata blanka ytor. Diffust ljus (diffuse light), som splittras upp och reflekteras \r{a}t m\r{a}nga olika h\r{a}ll pga oj\"amnheter i det belysta objektets yta. Tar (delvis) f\"arg fr\r{a}n den reflekterade ytan. Bakgrundsljus (ambient light) som \"ar ljus som finns i hela bilden och f\"ordelas j\"amnt \"over alla objekt. Har inte n\r{a}gon best\"amd k\"alla.



\subsection{F\"orklara hur begreppen frame, key frame och in-betweening som anv\"ands inom animation h\"anger ihop?}

\label{q:307:sa:sv:True}

\textbf{Svar}: Key frames {\textendash} bildrutor som f\r{a}ngar scenen vid specifika tidpunkter. In-betweening {\textendash} producerar bildrutor f\"or att fylla gapet mellan key frames.



\subsection{Processen att skapa 3D-grafik best\r{a}r av tv\r{a} huvudsteg, vilka?}

\label{q:308:sa:sv:True}

\textbf{Svar}: Modellering och rendering.



\subsection{Vad k\"annetecknar en lokal belysningsmodell (local lighting model) inom datorgrafik?}

\label{q:309:sa:sv:True}

\textbf{Svar}: Den tar inte h\"ansyn till ljusinteraktioner mellan objekt.



\subsection{Vad k\"annetecknar en global belysningsmodell (global lighting model) inom datorgrafik?}

\label{q:310:sa:sv:True}

\textbf{Svar}: Den tar h\"ansyn till ljusinteraktioner mellan objekt, till exempel genom ray tracing.



\subsection{M\r{a}nga sv\r{a}ra problem kan beskrivas som s\"okproblem, vilket inneb\"ar att man s\"oker efter en l\"osning i ett s\"oktr\"ad. F\"or att v\"alja s\"okv\"ag i s\"oktr\"adet anv\"ander man sig av {\textquotedblright}tumregler{\textquotedblright} (rules of thumb). Vad kallas s\r{a}dana tumregler och varf\"or beh\"ovs de?}

\label{q:311:sa:sv:True}

\textbf{Svar}: Heuristik (heuristics), och de beh\"ovs f\"or att s\"oktr\"aden f\"or alla sv\r{a}ra problem blir extremt stor, vilket inneb\"ar att det \"ar om\"ojligt att utforska hela s\"oktr\"adet.



\subsection{Vad \"ar skillnaden mellan svag (weak) AI och stark (strong) AI?}

\label{q:312:sa:sv:True}

\textbf{Svar}: Svag AI {\textendash} datorer kan programmeras f\"or att uppvisa ett intelligent beteende. Stark AI {\textendash} datorer kan programmeras s\r{a} att de f\r{a}r intelligens och medvetande.



\subsection{Ett s\"att att klassificera maskininl\"arningansatser (machine/computer learning approaches) \"ar genom i vilken grad de kr\"aver m\"ansklig inblandning. Vilka tre s\r{a}dana klasser brukar man prata om?}

\label{q:313:sa:sv:True}

\textbf{Svar}: Learning by imitation (l\"arande genom imitation); supervised learning (\"overvakat l\"arande); learning by reinforcement (l\"arande genom f\"orst\"arkning).



\subsection{Vad \"ar ett artificiellt neuralt n\"atverk (artificial neural network) och hur f\"or\"andras ett s\r{a}dant n\"atverk under inl\"arning?}

\label{q:314:sa:sv:True}

\textbf{Svar}: Ett artificiellt neuralt n\"atverk \"ar en ber\"akningsmodell som efterliknar en hj\"arnas n\"atverk av neuroner. Ett artificiellt neuralt n\"atverk l\"ar sig genom att justera vikterna i de olika neuronerna i n\"atverket.



\subsection{Vad \"ar skillnaden mellan \"overvakad inl\"arning (supervised learning) och o\"overvakad inl\"arning (unsupervised learning)?}

\label{q:315:sa:sv:True}

\textbf{Svar}: I \"overvakad inl\"arning tr\"anar man systemet med f\"ardig klassificerad data (tr\"aningsdata). I o\"overvakad inl\"arning f\r{a}r systemet ingen tr\"aningsdata utan f\r{a}r sj\"alv analysera indata och hitta m\"onster.



\subsection{\"Ar reinforcement learning en typ av \"overvakad inl\"arning (supervised learning) eller inte? Varf\"or?}

\label{q:316:sa:sv:True}

\textbf{Svar}: Reinforcement learning \"ar o\"overvakad, och bygger p\r{a} att systemet sj\"alv bed\"omer, utifr\r{a}n en given generell regel, huruvida det har lyckats eller inte.



\subsection{Ett neuralt n\"atverk \"ar en ber\"akningsmodell som inspirerats av hur den m\"anskliga hj\"arnan fungerar. Hur l\"ar sig ett neuralt n\"atverk fr\r{a}n exempeldata?}

\label{q:317:sa:sv:True}

\textbf{Svar}: F\"orenklat kan man s\"aga att ett neuronn\"atverk l\"ar sig fr\r{a}n data genom att justera de vikter som \"ar associerade med olika neuroner.



\subsection{F\"orklara kortfattat begreppen {\textquotedblright}information retrieval{\textquotedblright} och {\textquotedblright}information extraction{\textquotedblright} inom spr\r{a}kanalys (natural language processing)!}

\label{q:318:sa:sv:True}

\textbf{Svar}: Information retrieval behandlar metoder f\"or att identifiera dokument som behandlar en viss s\"okfr\r{a}ga eller ett visst \"amne. Information extraction behandlar metoder f\"or att extrahera information som \"ar anv\"andbar f\"or en viss applikation, t.ex. extrahera telefonnummer eller efternamn.



\subsection{Vilka tre typer av lager (layers) finns i ett neuronn\"atverks (neural network) topologi?}

\label{q:319:sa:sv:True}

\textbf{Svar}: Input layer, hidden layer och output layer.



\subsection{Vad \"ar ett s\"oktr\"ad inom AI?}

\label{q:320:sa:sv:True}

\textbf{Svar}: En tr\"adstruktur av noder d\"ar varje nod representerar ett visst tillst\r{a}nd och en l\"osning \"ar en v\"ag fr\r{a}n rotnoden (som representerar det initiala tillst\r{a}ndet) till en m\r{a}lnod (som representerar det \"onskade tillst\r{a}ndet).



\subsection{Vid behandling av naturligt spr\r{a}k utf\"ors tre olika typer av analyser, vilka?}

\label{q:321:sa:sv:True}

\textbf{Svar}: Syntaktisk analys, semantisk analys och kontextuell analys.



\subsection{Vad \"ar Turing-testet?}

\label{q:322:sa:sv:True}

\textbf{Svar}: Ett test d\"ar en m\"ansklig f\"orh\"orsledare ska f\"ors\"oka skilja p\r{a} om testpersonen \"ar m\"ansklig eller maskinell genom att kommunicera via textmeddelanden.



\subsection{Vad \"ar definitionen av en intelligent agent (inom AI)?}

\label{q:323:sa:sv:True}

\textbf{Svar}: En autonom m\r{a}lstyrd enhet som observerar med sensorer och agerar p\r{a} en milj\"o med hj\"alp av st\"alldon (actuators).



\subsection{Vad k\"annetecknar supervised (machine) learning?}

\label{q:324:sa:sv:True}

\textbf{Svar}: En person identifierar r\"att svar f\"or ett antal exempel och agenten generaliserar fr\r{a}n dessa exempel.



\subsection{Vad k\"annetecknar (machine) learning by reinforcement?}

\label{q:325:sa:sv:True}

\textbf{Svar}: Agenten f\r{a}r en allm\"an regel f\"or att bed\"oma sj\"alv n\"ar den har lyckats eller misslyckats.



\subsection{Vad \"ar definitionen av en intelligent agent?}

\label{q:326:sa:sv:True}

\textbf{Svar}: En autonom m\r{a}linriktad enhet som observerar med hj\"alp av sensorer och agerar p\r{a} en milj\"o med hj\"alp av st\"alldon.



\subsection{Vad \"ar definitionen av en intelligent agent?}

\label{q:327:sa:sv:True}

\textbf{Svar}: En autonom m\r{a}linriktad entitet som observerar genom sensorer och agerar p\r{a} en omgivning genom man\"ovreringsorgan (autonomous goal-directed entity which observes using sensors and acts upon an environment using actuators).



\subsection{Vad \"ar s\"ok-heuristik (search heuristics), och vad k\"annetecknar bra s\"ok-heuristik?}

\label{q:328:sa:sv:True}

\textbf{Svar}: S\"ok-heuristisk \"ar tumregler f\"or att n\r{a} ett \"overgripande s\"okm\r{a}l. En bra heuristik \"ar en tillr\"ackligt bra uppskattning av n\"arheten till s\"okm\r{a}let och f\"orh\r{a}llandevis enkel att ber\"akna.



\subsection{Varf\"or beh\"ovs s\"ok-heurestik n\"ar man s\"oker i ett s\"ok-tr\"ad?}

\label{q:329:sa:sv:True}

\textbf{Svar}: F\"or att s\"oktr\"ad f\"or alla intressanta problem \"ar s\r{a} stora att det \"ar praktiskt om\"ojligt att utforska hela s\"oktr\"adet, och man d\"arf\"or beh\"over heurestik (tumregler) f\"or att v\"agleda s\"okningen.



\subsection{Vad \"ar skillnaden mellan en tillst\r{a}ndsgraf och ett s\"oktr\"ad?}

\label{q:330:sa:sv:True}

\textbf{Svar}: En tillst\r{a}ndsgraf beskriver hur man g\r{a}r mellan alla m\"ojliga olika tillst\r{a}nd, medan ett s\"oktr\"ad beskriver de olika m\"ojliga s\"okv\"agarna i en tillst\r{a}ndsgraf f\"or att n\r{a} ett m\r{a}ltillst\r{a}nd.



\subsection{Vad \"ar stopp-problemet (the halting problem), och varf\"or \"ar det intressant ur ett ber\"akningsteoretiskt perspektiv?}

\label{q:331:sa:sv:True}

\textbf{Svar}: Stopp-problemet: \"Ar det m\"ojligt att inom \"andlig tidsrymd med n\r{a}got program avg\"ora om ett godtyckligt program kommer att avslutas f\"or godtyckliga indata? Stopp-problemet \"ar ol\"osbart, vilket visar att det finns problem som inte g\r{a}r att l\"osa (med algoritmer).



\subsection{Ordna f\"oljande komplexitets-/effektivitetsklasser (complexity/efficiency classes) fr\r{a}n den mest effektiva till den minst effektiva: \ensuremath{\Theta}(n^10), \ensuremath{\Theta}(log n), \ensuremath{\Theta}(n), \ensuremath{\Theta}(2^n).}

\label{q:332:sa:sv:True}

\textbf{Svar}: \ensuremath{\Theta}(log n), \ensuremath{\Theta}(n), \ensuremath{\Theta}(n^10), \ensuremath{\Theta}(2^n)



\subsection{Vad \"ar en Turing-maskin och vad \"ar dess syfte?}

\label{q:333:sa:sv:True}

\textbf{Svar}: En Turing-maskin \"ar en matematisk modell av en dator, och syftet \"ar att studera vilka problem som g\r{a}r att l\"osa med en dator.



\subsection{Ordna f\"oljande komplexitets-/effektivitetsklasser (complexity/efficiency classes) fr\r{a}n den mest effektiva till den minst effektiva: \ensuremath{\Theta}(n^4), \ensuremath{\Theta}(n), \ensuremath{\Theta}(2^n), \ensuremath{\Theta}(log n).}

\label{q:334:sa:sv:True}

\textbf{Svar}: \ensuremath{\Theta}(log n), \ensuremath{\Theta}(n), \ensuremath{\Theta}(n^4), \ensuremath{\Theta}(2^n)



\subsection{Vad inneb\"ar det att ett problem \"ar ett polynomiellt problem (polynomial problem) (tillh\"or klassen polynomiella problem)?}

\label{q:335:sa:sv:True}

\textbf{Svar}: Att det finns en algoritm f\"or att l\"osa problemet inom komplexitetsklass O(nx) f\"or n\r{a}got x.



\subsection{\"Ar klassen av polynomiella problem P mindre eller lika med klassen av icke-deterministiskt polynomiella problem NP? Motivera ditt svar!}

\label{q:336:sa:sv:True}

\textbf{Svar}: Det \"ar ett \"oppet problem. Ingen har lyckats visa vare sig att P \"ar mindre \"an NP, eller att P \"ar lika med NP.



\subsection{Givet att komplexiteten f\"or algoritm A \"ar O(n), algoritm B \"ar O(log n), algoritm C \"ar O(n2) och algoritm D \"ar O(n log n2), lista algoritmerna i ordning fr\r{a}n den mest effektiva till den minst effektiva!}

\label{q:337:sa:sv:True}

\textbf{Svar}: B, A, D, C.



\subsection{Ge exempel p\r{a} tre komplexitetsklasser i O-notation och ordna dessa fr\r{a}n mest effektiv till minst effektiv!}

\label{q:338:sa:sv:True}

\textbf{Svar}: Exempelvis: O(n), O(n2), O(2n).



\subsection{Varf\"or \"ar stopp-problemet (the halting problem) intressant ur ett ber\"akningsteoretiskt perspektiv?}

\label{q:339:sa:sv:True}

\textbf{Svar}: Stopproblemet \"ar ol\"osbart, vilket visar att det finns problem som inte g\r{a}r att l\"osa med algoritmer/program.



\subsection{Vad k\"annetecknar stopp-problemet (inom ber\"akningsteori).}

\label{q:340:sa:sv:True}

\textbf{Svar}: Att det inte \"ar ber\"akningsbart (algoritmiskt l\"osbart).



\subsection{Vad inneb\"ar det att en funktion \"ar ber\"akningsbar?}

\label{q:341:sa:sv:True}

\textbf{Svar}: Att funktionen kan ber\"aknas av n\r{a}gon algoritm.



\subsection{Uppfyller icke-detministiska algoritmer definitionen av en algoritm? Motivera ditt svar!}

\label{q:342:sa:sv:True}

\textbf{Svar}: Nej, f\"or ibland \"ar n\"asta steg inte unikt och helt best\"amt av det aktuella tillst\r{a}ndet.



\subsection{Vad vet vi om f\"orh\r{a}llandet mellan polynomiella problem P och icke-deterministiskt polynomiella problem NP?}

\label{q:343:sa:sv:True}

\textbf{Svar}: Alla problem i P ing\r{a}r ocks\r{a} i NP, men huruvida alla problem i NP ocks\r{a} ing\r{a}r i P \"ar en \"oppen fr\r{a}ga.



\subsection{Vad skiljer en deterministisk algoritm fr\r{a}n en icke-deterministisk?}

\label{q:344:sa:sv:True}

\textbf{Svar}: En deterministisk algoritm ger alltid samma svar givet ett visst indata. En icke-deterministisk algoritm kan ge olika svar f\"or samma indata.



\subsection{Vad \"ar syftet med Turing-maskiner?}

\label{q:345:sa:sv:True}

\textbf{Svar}: Det \"ar ett verktyg f\"or att studera datorers ber\"akningsf\"orm\r{a}ga (algoritmisk bearbetning).



\subsection{P\r{a} vilka tv\r{a} s\"att kan en transaktion avslutas?}

\label{q:346:sa:sv:True}

\textbf{Svar}: En transaktion som lyckas avslutas med en commit, och en transaktion som misslyckas avslutas genom en rollback som g\"or transaktionens arbete ogjort.



\subsection{Antag att vi har f\"oljande bitm\"onster och att de representerar heltal enligt tv\r{a}komplementsnotation (two{\textquoteright}s complement notation): "0111 1111, 1111 1001, 1011 1111, 0010 0100, 1000 0001" - Vilket av dessa bitm\"onster representerar det minsta heltalet?}

\label{q:34800:mc:sv:True}

\begin{itemize}
  \item[$\bigcirc$] 0111 1111
  \item[$\bigcirc$] 1111 1001
  \item[$\bigcirc$] 1011 1111
  \item[$\bigcirc$] 1000 0001
\end{itemize}

\subsection{Antag att vi har f\"oljande bitm\"onster och att de representerar heltal enligt tv\r{a}komplementsnotation (two{\textquoteright}s complement notation): "0111 1111, 1111 1001, 1011 1111, 0010 0100, 1000 0001" - Vilket av dessa bitm\"onster representerar det st\"orsta heltalet?}

\label{q:3480001:mc:sv:True}

\begin{itemize}
  \item[$\bigcirc$] 1111 1001
  \item[$\bigcirc$] 1011 1111
  \item[$\bigcirc$] 1000 0001
  \item[$\bigcirc$] 0010 0100
\end{itemize}



\subsection{Antag att 00FF00 \"ar den hexadecimala notationen f\"or ett bitm\"onster som representerar en pixel enligt RGB-standarden. - Vad har denna pixel f\"or f\"argdjup (color depth)?}

\label{q:34900:sa:sv:True}

\textbf{Svar}: 24 bitar/pixel. _ Gr\"on

\subsection{Antag att 00FF00 \"ar den hexadecimala notationen f\"or ett bitm\"onster som representerar en pixel enligt RGB-standarden. - Vilken av f\"oljande f\"arger har den pixeln?}

\label{q:3490001:mc:sv:True}

\begin{itemize}
  \item[$\bigcirc$] Vit
  \item[$\bigcirc$] Svart
  \item[$\bigcirc$] R\"od
  \item[$\bigcirc$] Bl\r{a}
  \item[$\bigcirc$] Gul
  \item[$\bigcirc$] Cyan
  \item[$\bigcirc$] Magenta
\end{itemize}



\subsection{Antag att vi har f\"oljande bitm\"onster och att de representerar heltal enligt tv\r{a}komplementsnotation (two{\textquoteright}s complement notation):1111 1110 0111 1111 0000 0000 0000 0001 1000 0000 1111 1111 - Vilket av dessa bitm\"onster representerar talet -1 (minus ett)?}

\label{q:35000:mc:sv:True}

\begin{itemize}
  \item[$\bigcirc$] 1111 1110
  \item[$\bigcirc$] 0111 1111
  \item[$\bigcirc$] 0000 0000
  \item[$\bigcirc$] 0000 0001
  \item[$\bigcirc$] 1000 0000
\end{itemize}

\subsection{Antag att vi har f\"oljande bitm\"onster och att de representerar heltal enligt tv\r{a}komplementsnotation (two{\textquoteright}s complement notation):1111 1110 0111 1111 0000 0000 0000 0001 1000 0000 1111 1111 - Vilket av dessa bitm\"onster representerar talet 1 (ett)?}

\label{q:3500001:mc:sv:True}

\begin{itemize}
  \item[$\bigcirc$] 1111 1110
  \item[$\bigcirc$] 0111 1111
  \item[$\bigcirc$] 0000 0000
  \item[$\bigcirc$] 1000 0000
  \item[$\bigcirc$] 1111 1111
\end{itemize}



\subsection{Antag att vi har f\"oljande bitm\"onster och att de representerar heltal enligt tv\r{a}komplementsnotation (two{\textquoteright}s complement notation):1111 0100 0111 0101 0000 1010 0000 1011 1000 1010 1111 0101 - Vilket av dessa bitm\"onster representerar det st\"orsta talet?}

\label{q:35100:mc:sv:True}

\begin{itemize}
  \item[$\bigcirc$] 0111 0011
  \item[$\bigcirc$] 0111 0001
  \item[$\bigcirc$] 0110 1111
\end{itemize}

\subsection{Antag att vi har f\"oljande bitm\"onster och att de representerar heltal enligt tv\r{a}komplementsnotation (two{\textquoteright}s complement notation):1111 0100 0111 0101 0000 1010 0000 1011 1000 1010 1111 0101 - Vilket av dessa bitm\"onster representerar det minsta talet?}

\label{q:3510001:mc:sv:True}

\begin{itemize}
  \item[$\bigcirc$] 1000 1011
  \item[$\bigcirc$] 1001 0010
  \item[$\bigcirc$] 1011 1101
\end{itemize}



\subsection{Antag att vi har f\"oljande bitm\"onster: 1000 0011. - Vilket decimalt naturligt tal (noll eller positivt heltal) (unsigned integer) representerar bitm\"ontret ovan?}

\label{q:35200:sa:sv:True}

\textbf{Svar}: 131. _ -125.

\subsection{Antag att vi har f\"oljande bitm\"onster: 1000 0011. - Vilket decimalt heltal (signed integer) representerar bitm\"ontret ovan enligt tv\r{a}komplementsnotation (two{\textquoteright}s complement notation)?}

\label{q:3520001:sa:sv:True}

\vspace{2cm}

\noindent\makebox[\textwidth]{\hrulefill}

\vspace{1cm}

\textit{Svar}: \autoref{q:3520001:sa:sv:True}



\subsection{Antag att vi har f\"oljande bitm\"onster och att de representerar heltal enligt tv\r{a}komplementsnotation (two{\textquoteright}s complement notation):0111 0100, 0010 1001, 1100 0010, 1100 0100, 0011 0001 - Vilket av dessa bitm\"onster representerar det st\"orsta talet?}

\label{q:35300:mc:sv:True}

\begin{itemize}
  \item[$\bigcirc$] 0010 1001
  \item[$\bigcirc$] 1100 0010
  \item[$\bigcirc$] 1100 0100
  \item[$\bigcirc$] 0011 0001
\end{itemize}

\subsection{Antag att vi har f\"oljande bitm\"onster och att de representerar heltal enligt tv\r{a}komplementsnotation (two{\textquoteright}s complement notation):0111 0100, 0010 1001, 1100 0010, 1100 0100, 0011 0001 - Vilket av dessa bitm\"onster representerar det minsta talet?}

\label{q:3530001:mc:sv:True}

\begin{itemize}
  \item[$\bigcirc$] 0010 1001
  \item[$\bigcirc$] 1100 0010
  \item[$\bigcirc$] 1100 0100
  \item[$\bigcirc$] 0011 0001
\end{itemize}



\subsection{Antag att RGB-f\"argkoden f\"or en pixel \"ar CC3300 p\r{a} hexadecimal form (basen 16), att pixeln ing\r{a}r i ett foto taget med en 6 megapixel-kamera, och att fotot \"ar lagrat som en bitmap (d.v.s. okomprimerad). - Ange pixelns f\"argv\"arden f\"or R (r\"ott), G (gr\"ont) och B (bl\r{a}tt) p\r{a} decimal form (basen 10)?}

\label{q:35400:sa:sv:True}

\textbf{Svar}: CC16 =20410 (12*161 +12*160);3316 =5110 (3*161 +3*160);0016 =010 (0*161 +0*160) d.v.s. R = 204; G = 51; B = 0. _  CC3300 p\r{a} hexadecimal form motsvarar bitm\"onstret 110011000011001100000000

\subsection{Antag att RGB-f\"argkoden f\"or en pixel \"ar CC3300 p\r{a} hexadecimal form (basen 16), att pixeln ing\r{a}r i ett foto taget med en 6 megapixel-kamera, och att fotot \"ar lagrat som en bitmap (d.v.s. okomprimerad). - Vad \"ar pixelns f\"argdjup?}

\label{q:3540001:sa:sv:True}

\textbf{Svar}:  vilket best\r{a}r av 24 bitar

\subsection{Antag att RGB-f\"argkoden f\"or en pixel \"ar CC3300 p\r{a} hexadecimal form (basen 16), att pixeln ing\r{a}r i ett foto taget med en 6 megapixel-kamera, och att fotot \"ar lagrat som en bitmap (d.v.s. okomprimerad). - Hur stor plats tar lagringen av fotot i MB (mega-byte)?}

\label{q:354000102:sa:sv:True}

\textbf{Svar}:  d.v.s. pixelns f\"argdjup \"ar 24 bitar. _ 24 bitar = 3 byte; antag 1k = 1000; d\r{a} g\"aller 3 byte/pixel * 6 000 000 pixlar = 18 000 000 byte = 18 MB (megabyte); eller antag 1k = 1024; d\r{a} g\"aller 3 byte/pixel * 6 291 456 pixlar = 18 874 368 byte = 18 MB (megabyte).



\subsection{Antag att vi har f\"oljande bitm\"onster: 1010 1010, 1100 1100, 1001 0000 och 1001 1111. - Om bitm\"onstren ovan representerar naturliga tal (unsigned integers), vilket bitm\"onster representerar d\r{a} det minsta talet?}

\label{q:35500:mc:sv:True}

\begin{itemize}
  \item[$\bigcirc$] 1010 1010
  \item[$\bigcirc$] 1100 1100
  \item[$\bigcirc$] 1001 1111
\end{itemize}

\subsection{Antag att vi har f\"oljande bitm\"onster: 1010 1010, 1100 1100, 1001 0000 och 1001 1111. - Om bitm\"onstren ovan representerar heltal enligt tv\r{a}komplementsnotation (two{\textquoteright}s complement notation), vilket bitm\"onster representerar d\r{a} det minsta talet?}

\label{q:3550001:mc:sv:True}

\begin{itemize}
  \item[$\bigcirc$] 1010 1010
  \item[$\bigcirc$] 1100 1100
  \item[$\bigcirc$] 1001 1111
\end{itemize}





\subsection{Beskriv det decimala talet 9 som ett bin\"art tal representerat med 8-bitar (8 bit unsigned integer).}

\label{q:357:sa:sv:True}

\textbf{Svar}: 00001001.



\subsection{Beskriv talet -1 (minus ett) som ett 8-bitars bitm\"onster enligt tv\r{a}komplementsnotation (two{\textquoteright}s complement notation).}

\label{q:358:sa:sv:True}

\textbf{Svar}: 11111111.



\subsection{Vilket bitm\"onster motsvarar det hexidecimala uttrycket 7F?}

\label{q:359:sa:sv:True}

\textbf{Svar}: 01111111.



\subsection{Beskriv det decimala talet 3 som ett bin\"art tal representerat med 8-bitar (8 bit unsigned integer).}

\label{q:360:sa:sv:True}

\textbf{Svar}: 00000011.



\subsection{Vilket bitm\"onster motsvarar det hexidecimala uttrycket AB?}

\label{q:361:sa:sv:True}

\textbf{Svar}: 1010 1011.



\subsection{Vilket decimaltal (basen 10) motsvarar det hexadecimala talet A2?}

\label{q:362:sa:sv:True}

\textbf{Svar}: 162.



\subsection{Beskriv talet -3 (minus tre) som ett 8-bitars bitm\"onster enligt tv\r{a}komplementsnotation (two{\textquoteright}s complement notation).}

\label{q:363:sa:sv:True}

\textbf{Svar}: 11111101.



\subsection{Vilket bitm\"onster motsvarar det hexadecimala talet 8F?}

\label{q:364:sa:sv:True}

\textbf{Svar}: 1000 1111.



\subsection{Vilket decimaltal (basen 10) motsvarar det hexadecimala talet B3 ?}

\label{q:365:sa:sv:True}

\textbf{Svar}: 179.



\subsection{Beskriv talet 3 (tre) som ett 8-bitars bitm\"onster enligt tv\r{a}komplementsnotation (two{\textquoteright}s complement notation)!}

\label{q:366:sa:sv:True}

\textbf{Svar}: 0000 0011.



\subsection{Beskriv talet 3 (tre) med tv\r{a} tecken i hexadecimal form!}

\label{q:367:sa:sv:True}

\textbf{Svar}: 03.



\subsection{Beskriv talet -4 (minus fyra) som ett 8-bitars bitm\"onster enligt tv\r{a}komplementsnotation (two{\textquoteright}s complement notation).}

\label{q:368:sa:sv:True}

\textbf{Svar}: 11111100.



\subsection{Beskriv talet 2 (tv\r{a}) som ett 8-bitars bitm\"onster enligt tv\r{a}komplementsnotation (two{\textquoteright}s complement notation)!}

\label{q:369:sa:sv:True}

\textbf{Svar}: 0000 0010.



\subsection{Beskriv talet \ensuremath{-}2 (minus tv\r{a}) som ett 8-bitars bitm\"onster enligt tv\r{a}komplementsnotation (two{\textquoteright}s complement notation)!}

\label{q:370:sa:sv:True}

\textbf{Svar}: 1111 1110.



\subsection{Vad \"ar det positiva decimala heltalet 127 som ett bin\"art tal representerat med 8-bitar enligt tv\r{a}komplementsnotation (two{\textquoteright}s complement notation)?}

\label{q:371:sa:sv:True}

\textbf{Svar}: 0111 1111.



\subsection{Vad \"ar det negativa decimala heltalet \ensuremath{-}127 som ett bin\"art tal representerat med 8-bitar enligt tv\r{a}komplementsnotation (two{\textquoteright}s complement notation)?}

\label{q:372:sa:sv:True}

\textbf{Svar}: 1000 0001.



\subsection{Vilket decimalt naturligt tal (noll eller positivt heltal) (unsigned integer) representerar bitm\"onstret 1010 1010?}

\label{q:373:sa:sv:True}

\textbf{Svar}: 170



\subsection{Vilket decimalt naturligt tal (noll eller positivt heltal) (unsigned integer) representerar bitm\"onstret 1011 1011?}

\label{q:374:sa:sv:True}

\textbf{Svar}: 187



\subsection{Vilket bitm\"onster motsvarar det hexadecimala talet C4?}

\label{q:375:sa:sv:True}

\textbf{Svar}: 1100 0100



\subsection{Vilket bitm\"onster motsvarar det hexadecimala talet B3?}

\label{q:376:sa:sv:True}

\textbf{Svar}: 1011 0011



\subsection{Vilket decimalt heltal (signed integer) representerar bitm\"onstret 1010 enligt tv\r{a}komplementsnotation?}

\label{q:377:sa:sv:True}

\textbf{Svar}: -6



\subsection{Vilket decimalt heltal (signed integer) representerar bitm\"onstret 1011 enligt tv\r{a}komplementsnotation?}

\label{q:378:sa:sv:True}

\textbf{Svar}: -5



\subsection{Vilket decimalt naturligt tal (noll eller positivt heltal) (unsigned integer) representerar bitm\"onstret 1101 1011?}

\label{q:379:sa:sv:True}

\textbf{Svar}: 219



\subsection{Vilket bitm\"onster motsvarar det hexadecimala talet D2?}

\label{q:380:sa:sv:True}

\textbf{Svar}: 1101 0010.



\subsection{Vilket decimalt heltal (signed integer) representerar bitm\"onstret 1101 enligt tv\r{a}komplementsnotation?}

\label{q:381:sa:sv:True}

\textbf{Svar}: -3



\subsection{Vilket bitm\"onster motsvarar det hexadecimala talet A5}

\label{q:382:sa:sv:True}

\textbf{Svar}: 1010 0101.



\subsection{Vilket bitm\"onster motsvarar det hexadecimala talet B4}

\label{q:383:sa:sv:True}

\textbf{Svar}: 1011 0100.



\subsection{Vilket decimaltal (basen 10) motsvarar det hexadecimala talet 4D ?}

\label{q:384:sa:sv:True}

\textbf{Svar}: 77.



\subsection{Vilket decimaltal (basen 10) motsvarar det hexadecimala talet 5C ?}

\label{q:385:sa:sv:True}

\textbf{Svar}: 92.



\subsection{Beskriv talet 4 (fyra) som ett 8-bitars bitm\"onster enligt tv\r{a}komplementsnotation (two{\textquoteright}s complement notation)!}

\label{q:386:sa:sv:True}

\textbf{Svar}: 0000 0100.



\subsection{Beskriv talet \ensuremath{-}5 (minus fem) som ett 8-bitars bitm\"onster enligt tv\r{a}komplementsnotation (two{\textquoteright}s complement notation)!}

\label{q:387:sa:sv:True}

\textbf{Svar}: 1111 1011.



\subsection{Vilket bitm\"onster motsvarar det hexadecimala talet C3?}

\label{q:388:sa:sv:True}

\textbf{Svar}: 1100 0011.



\subsection{Vilket bitm\"onster motsvarar det hexadecimala talet 3C?}

\label{q:389:sa:sv:True}

\textbf{Svar}: 0011 1100.



\subsection{Vilket decimaltal (basen 10) motsvarar det hexadecimala talet 3C?}

\label{q:390:sa:sv:True}

\textbf{Svar}: 60



\subsection{Vilket decimaltal (basen 10) motsvarar det hexadecimala talet C3?}

\label{q:391:sa:sv:True}

\textbf{Svar}: 195



\subsection{Beskriv talet -3 (minus tre) som ett 8-bitars bitm\"onster enligt tv\r{a}komplementsnotation (two{\textquoteright}s complement notation)!}

\label{q:392:sa:sv:True}

\textbf{Svar}: 1111 1101.



\subsection{Beskriv talet -4 (minus fyra) som ett 8-bitars bitm\"onster enligt tv\r{a}komplementsnotation (two{\textquoteright}s complement notation)!}

\label{q:393:sa:sv:True}

\textbf{Svar}: 1111 1100.



\subsection{Vilket bitm\"onster motsvarar det hexadecimala talet BE?}

\label{q:394:sa:sv:True}

\textbf{Svar}: 1011 1110



\subsection{Vilket decimaltal (basen 10) motsvarar det hexadecimala talet 2D?}

\label{q:395:sa:sv:True}

\textbf{Svar}: 45



\subsection{Vilket bitm\"onster motsvarar det hexadecimala talet A2?}

\label{q:396:sa:sv:True}

\textbf{Svar}: 1010 0010.



\subsection{Vilket decimaltal (basen 10) motsvarar det hexadecimala talet D2?}

\label{q:397:sa:sv:True}

\textbf{Svar}: 210.



\subsection{Vilket bitm\"onster motsvarar det hexadecimala talet B1?}

\label{q:398:sa:sv:True}

\textbf{Svar}: 1011 0001



\subsection{Vilket decimaltal (basen 10) motsvarar det hexadecimala talet 5E ?}

\label{q:399:sa:sv:True}

\textbf{Svar}: 0101 1110



\subsection{Vilket bitm\"onster motsvarar det hexadecimala talet 7F?}

\label{q:400:sa:sv:True}

\textbf{Svar}: 0111 1111.



\subsection{Vilket decimaltal (basen 10) motsvarar det hexadecimala talet A6?}

\label{q:401:sa:sv:True}

\textbf{Svar}: 166



\subsection{Beskriv talet -6 (minus sex) som ett 8-bitars bitm\"onster enligt tv\r{a}komplementsnotation (two{\textquoteright}s complement notation).}

\label{q:402:sa:sv:True}

\textbf{Svar}: 11111010.



\subsection{Antag att vi har f\"oljande tv\r{a} bitm\"onster 10000001 och 01111110. Vilket bitm\"onster erh\r{a}ller vi om vi utf\"or den logiska operationen AND p\r{a} dessa bitm\"onster?}

\label{q:403:sa:sv:True}

\textbf{Svar}: 00000000.



\subsection{Antag att vi har f\"oljande tv\r{a} bitm\"onster 10000001 och 01111110. Vilket bitm\"onster erh\r{a}ller vi om vi utf\"or den aritmetiska operationen ADD enligt tv\r{a}komplementsnotation (two{\textquoteright}s complement notation) p\r{a} dessa bitm\"onster som d\r{a} representerar tv\r{a} heltal (signed integers)?}

\label{q:404:sa:sv:True}

\textbf{Svar}: 11111111.



\subsection{Vilket bitm\"onster erh\r{a}ller vi om vi utf\"or operationen OR p\r{a} bitm\"onstren 1011 0011 och 0010 0110?}

\label{q:405:sa:sv:True}

\textbf{Svar}: 1011 0111.



\subsection{Vilket bitm\"onster erh\r{a}ller vi om vi utf\"or operationen XOR p\r{a} bitm\"onstren 1011 0011 och 0010 0110?}

\label{q:406:sa:sv:True}

\textbf{Svar}: 1001 0101.



\subsection{Vilket bitm\"onster erh\r{a}ller vi om vi utf\"or operationen AND p\r{a} bitm\"onstren 1001 1011 och 1000 1110?}

\label{q:407:sa:sv:True}

\textbf{Svar}: 1000 1010.



\subsection{Vilket bitm\"onster erh\r{a}ller vi om vi utf\"or operationen OR p\r{a} bitm\"onstren 1001 1011 och 1000 1110?}

\label{q:408:sa:sv:True}

\textbf{Svar}: 1001 1111.



\subsection{Vad blir resultatet av den logiska operationen AND med bitm\"onstren 1010 0101 och 0111 1110? Ange svaret som ett bitm\"onster.}

\label{q:409:sa:sv:True}

\textbf{Svar}: 00100100



\subsection{Vad blir resultatet av den logiska operationen XOR med bitm\"onstren 10100101 och 01111110? Ange svaret som ett bitm\"onster.}

\label{q:410:sa:sv:True}

\textbf{Svar}: 1101 1011



\subsection{Vad blir resultatet av den logiska operationen XOR med bitm\"onstren 10100001 och 01101010? Ange svaret som ett bitm\"onster.}

\label{q:411:sa:sv:True}

\textbf{Svar}: 1100 1011.



\subsection{Vilket bitm\"onster erh\r{a}ller vi om vi utf\"or operationen OR p\r{a} bitm\"onstren 110011 och 101000 ?}

\label{q:412:sa:sv:True}

\textbf{Svar}: 111011.



\subsection{Vilket bitm\"onster erh\r{a}ller vi om vi utf\"or operationen XOR p\r{a} bitm\"onstren 0110 0011 och 0101 0000?}

\label{q:413:sa:sv:True}

\textbf{Svar}: 0011 0011.



\subsection{Vilket bitm\"onster erh\r{a}ller vi om vi utf\"or operationen OR p\r{a} bitm\"onstren 101011 och 010011?}

\label{q:414:sa:sv:True}

\textbf{Svar}: 111011.



\subsection{Vilket bitm\"onster erh\r{a}ller vi om vi utf\"or operationen AND p\r{a} bitm\"onstren 110011 och 101001?}

\label{q:415:sa:sv:True}

\textbf{Svar}: 100001.



\subsection{Vilket bitm\"onster erh\r{a}ller vi om vi utf\"or operationen XOR p\r{a} bitm\"onstren 0110 0011 och 0101 0001?}

\label{q:416:sa:sv:True}

\textbf{Svar}: 0011 0010.



\subsection{Vilket bitm\"onster erh\r{a}ller vi om vi utf\"or operationen AND p\r{a} bitm\"onstren 1101 1101 och 1111 1001?}

\label{q:417:sa:sv:True}

\textbf{Svar}: 1101 1001.



\subsection{Vilket bitm\"onster erh\r{a}ller vi om vi utf\"or operationen XOR p\r{a} bitm\"onstren 0101 0101 och 1000 1100?}

\label{q:418:sa:sv:True}

\textbf{Svar}: 1101 1001.



\subsection{Vilket bitm\"onster erh\r{a}ller vi om vi utf\"or operationen OR p\r{a} bitm\"onstren 01001000 och 10011001?}

\label{q:419:sa:sv:True}

\textbf{Svar}: 1101 1001.



\subsection{Vad \"ar det minsta antalet g\r{a}nger som satserna i en loop-kropp (loop body) utf\"ors i en iteration med pre-test-villkor?}

\label{q:420:sa:sv:True}

\textbf{Svar}: 0



\subsection{Vilket bitm\"onster motsvarar det hexadecimala talet B7?}

\label{q:421:sa:sv:True}

\textbf{Svar}: 1011 0111



\subsection{Vilket bitm\"onster motsvarar det hexadecimala talet C1?}

\label{q:422:sa:sv:True}

\textbf{Svar}: 1100 0001



\subsection{Vilket bitm\"onster motsvarar det hexadecimala talet E3?}

\label{q:423:sa:sv:True}

\textbf{Svar}: 1110 0011



\subsection{Vilket hexadecimalt tal motsvarar bitm\"onstret 10010101?}

\label{q:424:sa:sv:True}

\textbf{Svar}: 95



\subsection{F\"argen magenta \"ar en blandning av maximalt r\"ott och maximalt bl\r{a}tt. Vilket bitm\"onster representerar en magentaf\"argad pixel kodad enligt RGB-standarden med bitdjupet 24 bitar/pixel? Ange svaret i hexadecimal notation.}

\label{q:425:sa:sv:True}

\textbf{Svar}: FF00FF



\subsection{F\"argen gul \"ar en blandning av maximalt r\"ott och maximalt gr\"ont. Vilket bitm\"onster representerar en gulf\"argad pixel kodad enligt RGB-standarden med bitdjupet 24 bitar/pixel? Ange svaret i hexadecimal notation.}

\label{q:426:sa:sv:True}

\textbf{Svar}: FFFF00



\subsection{Vilket hexadecimalt tal motsvarar bitm\"onstret 1110 0101?}

\label{q:427:sa:sv:True}

\textbf{Svar}: E5.



\subsection{Vilket hexadecimalt tal motsvarar bitm\"onstret 10101101?}

\label{q:428:sa:sv:True}

\textbf{Svar}: AD



\subsection{Vilket hexadecimalt tal motsvarar bitm\"onstret 11010100?}

\label{q:429:sa:sv:True}

\textbf{Svar}: D4



\subsection{Vilket v\"arde kommer register 0 att ha efter tre (3) maskincykler? Ange bitm\"onstret p\r{a} hexadecimalform.}

\label{q:430:sa:sv:True}

\textbf{Svar}: 16.



\subsection{Vilket v\"arde kommer register 1 att ha efter tre (3) maskincykler? Ange bitm\"onstret p\r{a} hexadecimalform.}

\label{q:431:sa:sv:True}

\textbf{Svar}: 0C.



\subsection{Vilket v\"arde kommer register 2 att ha efter tre (3) maskincykler? Ange bitm\"onstret p\r{a} hexadecimalform.}

\label{q:432:sa:sv:True}

\textbf{Svar}: 08.



\subsection{Vilket v\"arde kommer programr\"aknaren (program counter) att ha efter tre (3) maskincykler? Ange bitm\"onstret p\r{a} hexadecimalform.}

\label{q:433:sa:sv:True}

\textbf{Svar}: 06



\subsection{Vem skrev program f\"or "the Analytical Engine" och d\"armed kan betraktas som v\"arldens f\"orsta programmerare?}

\label{q:434:mc:sv:True}

\begin{itemize}
  \item[$\bigcirc$] Charles Babbage, Joseph Marie Jacquard, Alonzo Church, Kurt G\"odel, John von Neumann, Blaise Pascal, Alan Turing
\end{itemize}



\subsection{Vem designade "the Analytical Engine" - v\"arldens f\"orsta programmerbara ber\"akningsmaskin?}

\label{q:435:mc:sv:True}

\begin{itemize}
  \item[$\bigcirc$] Joseph Marie Jacquard, Ada Byron (Lovelace), Alonzo Church, Kurt G\"odel, John von Neumann, Blaise Pascal, Alan Turing
\end{itemize}



\subsection{Vem var den f\"orste att anv\"anda h\r{a}lkort (anv\"andes f\"or att lagra tygm\"onster till automatiska v\"avstolar)?}

\label{q:436:mc:sv:True}

\begin{itemize}
  \item[$\bigcirc$] Charles Babbage, Ada Byron (Lovelace), Alonzo Church, Kurt G\"odel, John von Neumann, Blaise Pascal, Alan Turing
\end{itemize}



\subsection{Vem utvecklade den f\"orsta kugghjulsbaserade maskinen f\"or att utf\"ora addition?}

\label{q:437:mc:sv:True}

\begin{itemize}
  \item[$\bigcirc$] Charles Babbage, Joseph Marie Jacquard, Ada Byron (Lovelace), Alonzo Church, Kurt G\"odel, John von Neumann, Alan Turing
\end{itemize}



\subsection{Vem har publicerat ett ofullst\"andighetsteorem som s\"ager att det i alla matematiska teorier som omfattar v\r{a}rt traditionella aritmetiska system finns p\r{a}st\r{a}enden vars sanning eller falskhet inte kan fastst\"allas med hj\"alp av en algoritm?}

\label{q:438:mc:sv:True}

\begin{itemize}
  \item[$\bigcirc$] Alan Turing, Blaise Pascal, Alonzo Church, Charles Babbage, Tim Berners-Lee, Ada Byron (Lovelace), Joseph Jacquard
\end{itemize}



\subsection{Vem f\"oreslog ett system genom vilket dokument som lagras p\r{a} datorer p\r{a} hela Internet kan l\"ankas samman och producera ett n\"at av l\"ankad information (World Wide Web)?}

\label{q:439:mc:sv:True}

\begin{itemize}
  \item[$\bigcirc$] Alan Turing, Blaise Pascal, Alonzo Church, Kurt G\"odel, Charles Babbage, Ada Byron (Lovelace), Joseph Jacquard
\end{itemize}



\subsection{Vem har gett upphov till namnet p\r{a} den datorarkitektur d\"ar CPU h\"amtar instruktioner fr\r{a}n minne \"over en central bus?}

\label{q:440:mc:sv:True}

\begin{itemize}
  \item[$\bigcirc$] Charles Babbage, Joseph Marie Jacquard, Ada Byron (Lovelace), Alonzo Church, Kurt G\"odel, Blaise Pascal, Alan Turing
\end{itemize}



\subsection{Vem gav upphov till namnet p\r{a} den matematiska modellen f\"or en dator som anv\"ands f\"or att studera kraften i algoritmisk bearbetning?}

\label{q:441:mc:sv:True}

\begin{itemize}
  \item[$\bigcirc$] Charles Babbage, Joseph Marie Jacquard, Ada Byron (Lovelace), Alonzo Church, Kurt G\"odel, John von Neumann, Blaise Pascal
\end{itemize}



\subsection{Tesen att de funktioner som kan ber\"aknas av en Turing-maskin \"ar samma som alla ber\"akningsbara funktioner, \"ar namngiven efter Turing och ytterligare en matematiker som bidragit till tesen, vilken?}

\label{q:442:mc:sv:True}

\begin{itemize}
  \item[$\bigcirc$] Charles Babbage, Joseph Marie Jacquard, Ada Byron (Lovelace), Kurt G\"odel, John von Neumann, Blaise Pascal, Tim Berners-Lee
\end{itemize}



\subsection{Vilket bitm\"onster motsvarar det hexadecimala talet D5?}

\label{q:443:mc:sv:True}

\begin{itemize}
  \item[$\bigcirc$] 1010 0010, 1111 0101, 1101 0101
\end{itemize}



\subsection{Vilket hexadecimalt tal motsvarar bitm\"onstret 10001111?}

\label{q:444:mc:sv:True}

\begin{itemize}
  \item[$\bigcirc$] 8F, 7F, 8C
\end{itemize}

\section{English without Answers}
\label{englishWithoutAnswers}

\subsection{Which data storage technology was first used in 1801 by Joseph Jacquard?}

\label{q:3:sa:en:False}

\vspace{2cm}

\noindent\makebox[\textwidth]{\hrulefill}

\vspace{1cm}

\textit{Answer}: \autoref{q:3:sa:en:True}



\subsection{Which special purpose register contains the memory address to the next instruction?}

\label{q:4:sa:en:False}

\vspace{2cm}

\noindent\makebox[\textwidth]{\hrulefill}

\vspace{1cm}

\textit{Answer}: \autoref{q:4:sa:en:True}



\subsection{Which special purpose register contains the next machine instruction to be executed?}

\label{q:5:sa:en:False}

\vspace{2cm}

\noindent\makebox[\textwidth]{\hrulefill}

\vspace{1cm}

\textit{Answer}: \autoref{q:5:sa:en:True}



\subsection{Which processor architecture has few, simple and fast machine instructions?}

\label{q:6:sa:en:False}

\vspace{2cm}

\noindent\makebox[\textwidth]{\hrulefill}

\vspace{1cm}

\textit{Answer}: \autoref{q:6:sa:en:True}



\subsection{There is a special type of machine instruction that is needed to be able to coordinate different processes' access to common resources, what is it called?}

\label{q:7:sa:en:False}

\vspace{2cm}

\noindent\makebox[\textwidth]{\hrulefill}

\vspace{1cm}

\textit{Answer}: \autoref{q:7:sa:en:True}



\subsection{What is the part of the operating system that maintains a process table called?}

\label{q:8:sa:en:False}

\vspace{2cm}

\noindent\makebox[\textwidth]{\hrulefill}

\vspace{1cm}

\textit{Answer}: \autoref{q:8:sa:en:True}



\subsection{What is it called when a single user in a single-user system can execute several programs "simultaneously"?}

\label{q:10:sa:en:False}

\vspace{2cm}

\noindent\makebox[\textwidth]{\hrulefill}

\vspace{1cm}

\textit{Answer}: \autoref{q:10:sa:en:True}



\subsection{Which of the following options is not part of the operating system?}

\label{q:11:sa:en:False}

\vspace{2cm}

\noindent\makebox[\textwidth]{\hrulefill}

\vspace{1cm}

\textit{Answer}: \autoref{q:11:sa:en:True}



\subsection{What is a flag called that controls access to a critical region to ensure that several processes do not access the critical region at the same time (mutual exclusion)?}

\label{q:12:sa:en:False}

\vspace{2cm}

\noindent\makebox[\textwidth]{\hrulefill}

\vspace{1cm}

\textit{Answer}: \autoref{q:12:sa:en:True}



\subsection{What is the part of the operating system that handles data stored as named units (named separate groups of data) on secondary memory called?}

\label{q:13:sa:en:False}

\vspace{2cm}

\noindent\makebox[\textwidth]{\hrulefill}

\vspace{1cm}

\textit{Answer}: \autoref{q:13:sa:en:True}



\subsection{A computer can simulate that it has more primary memory than its actual physical primary memory. What is this simulated memory called?}

\label{q:14:sa:en:False}

\vspace{2cm}

\noindent\makebox[\textwidth]{\hrulefill}

\vspace{1cm}

\textit{Answer}: \autoref{q:14:sa:en:True}



\subsection{What is the special process needed to start a computer called?}

\label{q:15:sa:en:False}

\vspace{2cm}

\noindent\makebox[\textwidth]{\hrulefill}

\vspace{1cm}

\textit{Answer}: \autoref{q:15:sa:en:True}



\subsection{What is called the part of the operating system that allocates and deallocates main memory (main memory) to different processes?}

\label{q:16:sa:en:False}

\vspace{2cm}

\noindent\makebox[\textwidth]{\hrulefill}

\vspace{1cm}

\textit{Answer}: \autoref{q:16:sa:en:True}



\subsection{What is the part of the operating system that allocates time slices to different processes called?}

\label{q:17:sa:en:False}

\vspace{2cm}

\noindent\makebox[\textwidth]{\hrulefill}

\vspace{1cm}

\textit{Answer}: \autoref{q:17:sa:en:True}



\subsection{What is the name of the protocol used by the World Wide Web application?}

\label{q:18:sa:en:False}

\vspace{2cm}

\noindent\makebox[\textwidth]{\hrulefill}

\vspace{1cm}

\textit{Answer}: \autoref{q:18:sa:en:True}



\subsection{To which Internet software layer does the UDP (user datagram protocol) belong?}

\label{q:19:sa:en:False}

\vspace{2cm}

\noindent\makebox[\textwidth]{\hrulefill}

\vspace{1cm}

\textit{Answer}: \autoref{q:19:sa:en:True}



\subsection{To which Internet software layer does the FTP (file transfer protocol) belong?}

\label{q:20:sa:en:False}

\vspace{2cm}

\noindent\makebox[\textwidth]{\hrulefill}

\vspace{1cm}

\textit{Answer}: \autoref{q:20:sa:en:True}



\subsection{What is a LAN?}

\label{q:21:sa:en:False}

\vspace{2cm}

\noindent\makebox[\textwidth]{\hrulefill}

\vspace{1cm}

\textit{Answer}: \autoref{q:21:sa:en:True}



\subsection{To which Internet software layer does the TCP (transmission control protocol) belong?}

\label{q:22:sa:en:False}

\vspace{2cm}

\noindent\makebox[\textwidth]{\hrulefill}

\vspace{1cm}

\textit{Answer}: \autoref{q:22:sa:en:True}



\subsection{What is the name of the organization responsible for assigning IP numbers (the abbreviation will suffice)?}

\label{q:23:sa:en:False}

\vspace{2cm}

\noindent\makebox[\textwidth]{\hrulefill}

\vspace{1cm}

\textit{Answer}: \autoref{q:23:sa:en:True}



\subsection{To which Internet software layer (Internet software layer) does the IPv6 protocol belong?}

\label{q:24:sa:en:False}

\vspace{2cm}

\noindent\makebox[\textwidth]{\hrulefill}

\vspace{1cm}

\textit{Answer}: \autoref{q:24:sa:en:True}



\subsection{What is it called when a web client asks a specific type of server to translate a domain name into an IP number?}

\label{q:25:sa:en:False}

\vspace{2cm}

\noindent\makebox[\textwidth]{\hrulefill}

\vspace{1cm}

\textit{Answer}: \autoref{q:25:sa:en:True}



\subsection{Which Internet protocol for the transport layer is most reliable?}

\label{q:26:sa:en:False}

\vspace{2cm}

\noindent\makebox[\textwidth]{\hrulefill}

\vspace{1cm}

\textit{Answer}: \autoref{q:26:sa:en:True}



\subsection{Name the language used to create web pages with?}

\label{q:27:sa:en:False}

\vspace{2cm}

\noindent\makebox[\textwidth]{\hrulefill}

\vspace{1cm}

\textit{Answer}: \autoref{q:27:sa:en:True}



\subsection{What is the system called that can be used to inspect, filter and block incoming and outgoid network traffic?}

\label{q:28:sa:en:False}

\vspace{2cm}

\noindent\makebox[\textwidth]{\hrulefill}

\vspace{1cm}

\textit{Answer}: \autoref{q:28:sa:en:True}



\subsection{What is the system called that is a software unit that acts as an intermediary between a client and a server with the goal of shielding the client from adverse actions of the server?}

\label{q:29:sa:en:False}

\vspace{2cm}

\noindent\makebox[\textwidth]{\hrulefill}

\vspace{1cm}

\textit{Answer}: \autoref{q:29:sa:en:True}



\subsection{What is the name of the way to achieve repetition in code that requires more space in the primary memory?}

\label{q:31:sa:en:False}

\vspace{2cm}

\noindent\makebox[\textwidth]{\hrulefill}

\vspace{1cm}

\textit{Answer}: \autoref{q:31:sa:en:True}



\subsection{What is the name of the way to achieve repetition in code that does not require more space in the primary memory?}

\label{q:32:sa:en:False}

\vspace{2cm}

\noindent\makebox[\textwidth]{\hrulefill}

\vspace{1cm}

\textit{Answer}: \autoref{q:32:sa:en:True}



\subsection{What is the most common method of verifying that a program is working correctly?}

\label{q:33:sa:en:False}

\vspace{2cm}

\noindent\makebox[\textwidth]{\hrulefill}

\vspace{1cm}

\textit{Answer}: \autoref{q:33:sa:en:True}



\subsection{What is the basic building block in imperative programming languages called?}

\label{q:34:sa:en:False}

\vspace{2cm}

\noindent\makebox[\textwidth]{\hrulefill}

\vspace{1cm}

\textit{Answer}: \autoref{q:34:sa:en:True}



\subsection{What is the logic derivation technique used in logic programming called?}

\label{q:35:sa:en:False}

\vspace{2cm}

\noindent\makebox[\textwidth]{\hrulefill}

\vspace{1cm}

\textit{Answer}: \autoref{q:35:sa:en:True}



\subsection{In object oriented programming, what are the templates from which objects are constructed called?}

\label{q:36:sa:en:False}

\vspace{2cm}

\noindent\makebox[\textwidth]{\hrulefill}

\vspace{1cm}

\textit{Answer}: \autoref{q:36:sa:en:True}



\subsection{What is the programming paradigm called where you describe what should be done instead of how it should be done?}

\label{q:37:sa:en:False}

\vspace{2cm}

\noindent\makebox[\textwidth]{\hrulefill}

\vspace{1cm}

\textit{Answer}: \autoref{q:37:sa:en:True}



\subsection{What is a program that translates source code into machine code called?}

\label{q:38:sa:en:False}

\vspace{2cm}

\noindent\makebox[\textwidth]{\hrulefill}

\vspace{1cm}

\textit{Answer}: \autoref{q:38:sa:en:True}



\subsection{What is the type of programming called that answers questions whether a fact is deducible from a program or not?}

\label{q:39:sa:en:False}

\vspace{2cm}

\noindent\makebox[\textwidth]{\hrulefill}

\vspace{1cm}

\textit{Answer}: \autoref{q:39:sa:en:True}



\subsection{Give an example of an agile development model?}

\label{q:40:sa:en:False}

\vspace{2cm}

\noindent\makebox[\textwidth]{\hrulefill}

\vspace{1cm}

\textit{Answer}: \autoref{q:40:sa:en:True}



\subsection{What is the name of the role in Scrum that maintains a list of requirements and prioritizes between these requirements?}

\label{q:41:sa:en:False}

\vspace{2cm}

\noindent\makebox[\textwidth]{\hrulefill}

\vspace{1cm}

\textit{Answer}: \autoref{q:41:sa:en:True}



\subsection{What are the short iterations (2 {\textendash} 4 weeks) called in Scrum, which should result in something deliverable to the customer/orderer?}

\label{q:42:sa:en:False}

\vspace{2cm}

\noindent\makebox[\textwidth]{\hrulefill}

\vspace{1cm}

\textit{Answer}: \autoref{q:42:sa:en:True}



\subsection{What is called the role in Scrum that must ensure that the Scrum framework is followed?}

\label{q:43:sa:en:False}

\vspace{2cm}

\noindent\makebox[\textwidth]{\hrulefill}

\vspace{1cm}

\textit{Answer}: \autoref{q:43:sa:en:True}



\subsection{What are the short daily meetings called in Scrum when each project participant should answer three questions?}

\label{q:44:sa:en:False}

\vspace{2cm}

\noindent\makebox[\textwidth]{\hrulefill}

\vspace{1cm}

\textit{Answer}: \autoref{q:44:sa:en:True}



\subsection{What are the meetings called in Scrum when you discuss what has gone well this iteration and what can be improved in the next iteration?}

\label{q:45:sa:en:False}

\vspace{2cm}

\noindent\makebox[\textwidth]{\hrulefill}

\vspace{1cm}

\textit{Answer}: \autoref{q:45:sa:en:True}



\subsection{What is the role of a team that is responsible for the team following the Scrum methodology called?}

\label{q:46:sa:en:False}

\vspace{2cm}

\noindent\makebox[\textwidth]{\hrulefill}

\vspace{1cm}

\textit{Answer}: \autoref{q:46:sa:en:True}



\subsection{What does the abbreviation CASE stand for in terms of software engineering?}

\label{q:47:sa:en:False}

\vspace{2cm}

\noindent\makebox[\textwidth]{\hrulefill}

\vspace{1cm}

\textit{Answer}: \autoref{q:47:sa:en:True}



\subsection{What does the abbreviation IDE stand for in terms of software engineering?}

\label{q:48:sa:en:False}

\vspace{2cm}

\noindent\makebox[\textwidth]{\hrulefill}

\vspace{1cm}

\textit{Answer}: \autoref{q:48:sa:en:True}



\subsection{What is the name of the role in Scrum that is responsible for prioritizing which development is to be carried out during the next sprint?}

\label{q:49:sa:en:False}

\vspace{2cm}

\noindent\makebox[\textwidth]{\hrulefill}

\vspace{1cm}

\textit{Answer}: \autoref{q:49:sa:en:True}



\subsection{What is the Scrum meeting, at the end of a sprint where the completed work of the sprint is evaluated with respect to the sprint goals, called?}

\label{q:50:sa:en:False}

\vspace{2cm}

\noindent\makebox[\textwidth]{\hrulefill}

\vspace{1cm}

\textit{Answer}: \autoref{q:50:sa:en:True}



\subsection{What is the basic data structure called, which consists of a block of data elements of the same data type and size, and where each data element is directly accessed via an index?}

\label{q:51:sa:en:False}

\vspace{2cm}

\noindent\makebox[\textwidth]{\hrulefill}

\vspace{1cm}

\textit{Answer}: \autoref{q:51:sa:en:True}



\subsection{What is the basic data structure called that consists of a block of data elements of usually different data types and sizes, and where the individual data elements are accessed by name?}

\label{q:52:sa:en:False}

\vspace{2cm}

\noindent\makebox[\textwidth]{\hrulefill}

\vspace{1cm}

\textit{Answer}: \autoref{q:52:sa:en:True}



\subsection{What is a variable that contains a memory address instead of data called (used in dynamic data structures)?}

\label{q:53:sa:en:False}

\vspace{2cm}

\noindent\makebox[\textwidth]{\hrulefill}

\vspace{1cm}

\textit{Answer}: \autoref{q:53:sa:en:True}



\subsection{What is the name of the dominant query language used to retrieve data from and manipulate data in a database?}

\label{q:54:sa:en:False}

\vspace{2cm}

\noindent\makebox[\textwidth]{\hrulefill}

\vspace{1cm}

\textit{Answer}: \autoref{q:54:sa:en:True}



\subsection{What is called in a database context, a sequence of operations that are packaged together and where either all operations succeed (performed) or all fail (no one performs) (all operations together either succeed or fail)?}

\label{q:55:sa:en:False}

\vspace{2cm}

\noindent\makebox[\textwidth]{\hrulefill}

\vspace{1cm}

\textit{Answer}: \autoref{q:55:sa:en:True}



\subsection{What is the type of data mining that has made the webshop Amazon so successful called?}

\label{q:56:sa:en:False}

\vspace{2cm}

\noindent\makebox[\textwidth]{\hrulefill}

\vspace{1cm}

\textit{Answer}: \autoref{q:56:sa:en:True}



\subsection{What is the type of analysis within data-mining called, which seeks to discover classes by grouping objects into a number of separate groups (in contrast to class description, which seeks to discover properties of members within classes that are already identified)?}

\label{q:57:sa:en:False}

\vspace{2cm}

\noindent\makebox[\textwidth]{\hrulefill}

\vspace{1cm}

\textit{Answer}: \autoref{q:57:sa:en:True}



\subsection{What is the type of analysis within data-mining called, which tries to identify patterns of behavior over time, for example trends in economic systems such as equity markets or in environmental systems such as climate conditions?}

\label{q:58:sa:en:False}

\vspace{2cm}

\noindent\makebox[\textwidth]{\hrulefill}

\vspace{1cm}

\textit{Answer}: \autoref{q:58:sa:en:True}



\subsection{When rendering, a three-dimensional model must be transferred to a flat surface. What is this flat surface called?}

\label{q:59:sa:en:False}

\vspace{2cm}

\noindent\makebox[\textwidth]{\hrulefill}

\vspace{1cm}

\textit{Answer}: \autoref{q:59:sa:en:True}



\subsection{When rendering 3D graphics, a three-dimensional model must be transferred to a flat surface, what is this flat surface called?}

\label{q:60:sa:en:False}

\vspace{2cm}

\noindent\makebox[\textwidth]{\hrulefill}

\vspace{1cm}

\textit{Answer}: \autoref{q:60:sa:en:True}



\subsection{What is it called when applying the laws of physics to determine the positions of objects, e.g. positions of pool balls after a pool stroke?}

\label{q:61:sa:en:False}

\vspace{2cm}

\noindent\makebox[\textwidth]{\hrulefill}

\vspace{1cm}

\textit{Answer}: \autoref{q:61:sa:en:True}



\subsection{What is the part of machine learning called where a human describes the correct answer for a number of examples and the agent (machine learning algorithm) generalizes based on these examples?}

\label{q:62:sa:en:False}

\vspace{2cm}

\noindent\makebox[\textwidth]{\hrulefill}

\vspace{1cm}

\textit{Answer}: \autoref{q:62:sa:en:True}



\subsection{Give an example of a non-computable function?}

\label{q:63:sa:en:False}

\vspace{2cm}

\noindent\makebox[\textwidth]{\hrulefill}

\vspace{1cm}

\textit{Answer}: \autoref{q:63:sa:en:True}



\subsection{What is the name of the machine that is the simplest possible model of a computer?}

\label{q:64:sa:en:False}

\vspace{2cm}

\noindent\makebox[\textwidth]{\hrulefill}

\vspace{1cm}

\textit{Answer}: \autoref{q:64:sa:en:True}



\subsection{An audio file in CD quality means a sampling frequency of 44100 per second, and a sampling depth of 16 bits per channel. How much space in kilobytes (kB) does an uncompressed audio file in stereo (2 channels) in CD quality with a length of 3 minutes take up?}

\label{q:65:sa:en:False}

\vspace{2cm}

\noindent\makebox[\textwidth]{\hrulefill}

\vspace{1cm}

\textit{Answer}: \autoref{q:65:sa:en:True}



\subsection{Assume that we have previously stored digital images with a color depth of 12 bits per pixel (color depth 12 bits per pixel). If we now want to be able to represent half as many different colors compared to before, what color depth should we use then?}

\label{q:66:sa:en:False}

\vspace{2cm}

\noindent\makebox[\textwidth]{\hrulefill}

\vspace{1cm}

\textit{Answer}: \autoref{q:66:sa:en:True}



\subsection{Assume that we have previously stored digital images with a color depth of 12 bits per pixel (color depth 12 bits per pixel). If we now want to be able to represent twice as many different colors compared to before, what color depth should we use then?}

\label{q:67:sa:en:False}

\vspace{2cm}

\noindent\makebox[\textwidth]{\hrulefill}

\vspace{1cm}

\textit{Answer}: \autoref{q:67:sa:en:True}



\subsection{What is character encoding?}

\label{q:68:sa:en:False}

\vspace{2cm}

\noindent\makebox[\textwidth]{\hrulefill}

\vspace{1cm}

\textit{Answer}: \autoref{q:68:sa:en:True}



\subsection{If 6A38 is the hexadecimal notation for a bit pattern representing one sound sample, what is the sampling depth of this sound sample?}

\label{q:71:sa:en:False}

\vspace{2cm}

\noindent\makebox[\textwidth]{\hrulefill}

\vspace{1cm}

\textit{Answer}: \autoref{q:71:sa:en:True}



\subsection{If 6A36B3 is the hexadecimal notation for a bit pattern representing an RGB-encoded pixel, what is the color depth of this pixel?}

\label{q:72:sa:en:False}

\vspace{2cm}

\noindent\makebox[\textwidth]{\hrulefill}

\vspace{1cm}

\textit{Answer}: \autoref{q:72:sa:en:True}



\subsection{What is color depth in the context of image storage?}

\label{q:73:sa:en:False}

\vspace{2cm}

\noindent\makebox[\textwidth]{\hrulefill}

\vspace{1cm}

\textit{Answer}: \autoref{q:73:sa:en:True}



\subsection{What decimal number (base 10) corresponds to the hexadecimal number 15?}

\label{q:74:sa:en:False}

\vspace{2cm}

\noindent\makebox[\textwidth]{\hrulefill}

\vspace{1cm}

\textit{Answer}: \autoref{q:74:sa:en:True}



\subsection{Assume that we have previously stored digital images with a color depth of 8 bits per pixel (color depth 8 bits per pixel). If we now want to be able to represent twice as many different colors compared to before, which color depth should we use?}

\label{q:76:sa:en:False}

\vspace{2cm}

\noindent\makebox[\textwidth]{\hrulefill}

\vspace{1cm}

\textit{Answer}: \autoref{q:76:sa:en:True}



\subsection{What is sampling frequency (sample rate) in connection with digital storage of sound?}

\label{q:77:sa:en:False}

\vspace{2cm}

\noindent\makebox[\textwidth]{\hrulefill}

\vspace{1cm}

\textit{Answer}: \autoref{q:77:sa:en:True}



\subsection{What decimal number (base 10) corresponds to the hexadecimal number 3F?}

\label{q:78:sa:en:False}

\vspace{2cm}

\noindent\makebox[\textwidth]{\hrulefill}

\vspace{1cm}

\textit{Answer}: \autoref{q:78:sa:en:True}



\subsection{Assume that we have previously stored digital images with a color depth of 8 bits per pixel (color depth 8 bits per pixel). If we now want to be able to represent half as many different colors compared to before, which color depth should we use?}

\label{q:79:sa:en:False}

\vspace{2cm}

\noindent\makebox[\textwidth]{\hrulefill}

\vspace{1cm}

\textit{Answer}: \autoref{q:79:sa:en:True}



\subsection{How many bits (color depth) are needed to represent 16 different colors?}

\label{q:80:sa:en:False}

\vspace{2cm}

\noindent\makebox[\textwidth]{\hrulefill}

\vspace{1cm}

\textit{Answer}: \autoref{q:80:sa:en:True}



\subsection{What is the sample depth of an audio file?}

\label{q:81:sa:en:False}

\vspace{2cm}

\noindent\makebox[\textwidth]{\hrulefill}

\vspace{1cm}

\textit{Answer}: \autoref{q:81:sa:en:True}



\subsection{What is the sample rate of an audio file?}

\label{q:82:sa:en:False}

\vspace{2cm}

\noindent\makebox[\textwidth]{\hrulefill}

\vspace{1cm}

\textit{Answer}: \autoref{q:82:sa:en:True}



\subsection{When describing audio files, what does sample rate refer to?}

\label{q:83:sa:en:False}

\vspace{2cm}

\noindent\makebox[\textwidth]{\hrulefill}

\vspace{1cm}

\textit{Answer}: \autoref{q:83:sa:en:True}



\subsection{Assume that we have previously stored digital images with a color depth of 5 bits per pixel. If we now want to be able to represent twice as many different colors compared to before, what color depth should we use then?}

\label{q:84:sa:en:False}

\vspace{2cm}

\noindent\makebox[\textwidth]{\hrulefill}

\vspace{1cm}

\textit{Answer}: \autoref{q:84:sa:en:True}



\subsection{How many bits are required to represent a 24 different colors?}

\label{q:85:sa:en:False}

\vspace{2cm}

\noindent\makebox[\textwidth]{\hrulefill}

\vspace{1cm}

\textit{Answer}: \autoref{q:85:sa:en:True}



\subsection{How many bits are required to represent a 12 different colors?}

\label{q:86:sa:en:False}

\vspace{2cm}

\noindent\makebox[\textwidth]{\hrulefill}

\vspace{1cm}

\textit{Answer}: \autoref{q:86:sa:en:True}



\subsection{How many bits are required to represent a Boolean value?}

\label{q:87:sa:en:False}

\vspace{2cm}

\noindent\makebox[\textwidth]{\hrulefill}

\vspace{1cm}

\textit{Answer}: \autoref{q:87:sa:en:True}



\subsection{Give example of two logical operations that can be performed on boolean values.}

\label{q:88:sa:en:False}

\vspace{2cm}

\noindent\makebox[\textwidth]{\hrulefill}

\vspace{1cm}

\textit{Answer}: \autoref{q:88:sa:en:True}



\subsection{How many bits are required to represent a 9 different colors?}

\label{q:89:sa:en:False}

\vspace{2cm}

\noindent\makebox[\textwidth]{\hrulefill}

\vspace{1cm}

\textit{Answer}: \autoref{q:89:sa:en:True}



\subsection{How many bits are required to represent a 15 different colors?}

\label{q:90:sa:en:False}

\vspace{2cm}

\noindent\makebox[\textwidth]{\hrulefill}

\vspace{1cm}

\textit{Answer}: \autoref{q:90:sa:en:True}



\subsection{What is a byte?}

\label{q:91:sa:en:False}

\vspace{2cm}

\noindent\makebox[\textwidth]{\hrulefill}

\vspace{1cm}

\textit{Answer}: \autoref{q:91:sa:en:True}



\subsection{What do we call 8 bits?}

\label{q:92:sa:en:False}

\vspace{2cm}

\noindent\makebox[\textwidth]{\hrulefill}

\vspace{1cm}

\textit{Answer}: \autoref{q:92:sa:en:True}



\subsection{What are the three different categories of machine instructions (machine instruction categories)?}

\label{q:93:sa:en:False}

\vspace{2cm}

\noindent\makebox[\textwidth]{\hrulefill}

\vspace{1cm}

\textit{Answer}: \autoref{q:93:sa:en:True}



\subsection{What is a machine language?}

\label{q:94:sa:en:False}

\vspace{2cm}

\noindent\makebox[\textwidth]{\hrulefill}

\vspace{1cm}

\textit{Answer}: \autoref{q:94:sa:en:True}



\subsection{What is stored in the program counter?}

\label{q:95:sa:en:False}

\vspace{2cm}

\noindent\makebox[\textwidth]{\hrulefill}

\vspace{1cm}

\textit{Answer}: \autoref{q:95:sa:en:True}



\subsection{What is stored in the instruction register?}

\label{q:96:sa:en:False}

\vspace{2cm}

\noindent\makebox[\textwidth]{\hrulefill}

\vspace{1cm}

\textit{Answer}: \autoref{q:96:sa:en:True}



\subsection{Describe the difference between RISC and CISC processors.}

\label{q:97:sa:en:False}

\vspace{2cm}

\noindent\makebox[\textwidth]{\hrulefill}

\vspace{1cm}

\textit{Answer}: \autoref{q:97:sa:en:True}



\subsection{What different steps are included in a machine cycle? Enter the steps in the order in which they are performed.}

\label{q:98:sa:en:False}

\vspace{2cm}

\noindent\makebox[\textwidth]{\hrulefill}

\vspace{1cm}

\textit{Answer}: \autoref{q:98:sa:en:True}



\subsection{What bit pattern do we obtain if we perform the ADD operation on the bit patterns 1011 0011 and 0010 0110?}

\label{q:99:sa:en:False}

\vspace{2cm}

\noindent\makebox[\textwidth]{\hrulefill}

\vspace{1cm}

\textit{Answer}: \autoref{q:99:sa:en:True}



\subsection{What are the three main parts that a processor (CPU {\textendash} central processing unit) consists of?}

\label{q:100:sa:en:False}

\vspace{2cm}

\noindent\makebox[\textwidth]{\hrulefill}

\vspace{1cm}

\textit{Answer}: \autoref{q:100:sa:en:True}



\subsection{What bit pattern do we get if we perform the operation OR on the bit patterns 1010 0011 and 0010 0110?}

\label{q:101:sa:en:False}

\vspace{2cm}

\noindent\makebox[\textwidth]{\hrulefill}

\vspace{1cm}

\textit{Answer}: \autoref{q:101:sa:en:True}



\subsection{What is required to be able to interpret a bit pattern as a character? What is required to be able to interpret a bit pattern as a character?}

\label{q:102:sa:en:False}

\vspace{2cm}

\noindent\makebox[\textwidth]{\hrulefill}

\vspace{1cm}

\textit{Answer}: \autoref{q:102:sa:en:True}



\subsection{How do you ensure that processes cannot perform operations that are destructive to other processes on a computer, e.g. to write data into other processes' parts of primary memory?}

\label{q:104:sa:en:False}

\vspace{2cm}

\noindent\makebox[\textwidth]{\hrulefill}

\vspace{1cm}

\textit{Answer}: \autoref{q:104:sa:en:True}



\subsection{What is boot strapping (booting) and why is it needed?}

\label{q:105:sa:en:False}

\vspace{2cm}

\noindent\makebox[\textwidth]{\hrulefill}

\vspace{1cm}

\textit{Answer}: \autoref{q:105:sa:en:True}



\subsection{What does real time processing mean?}

\label{q:106:sa:en:False}

\vspace{2cm}

\noindent\makebox[\textwidth]{\hrulefill}

\vspace{1cm}

\textit{Answer}: \autoref{q:106:sa:en:True}



\subsection{What is virtual memory?}

\label{q:108:sa:en:False}

\vspace{2cm}

\noindent\makebox[\textwidth]{\hrulefill}

\vspace{1cm}

\textit{Answer}: \autoref{q:108:sa:en:True}



\subsection{What is the main function of an operating system?}

\label{q:109:sa:en:False}

\vspace{2cm}

\noindent\makebox[\textwidth]{\hrulefill}

\vspace{1cm}

\textit{Answer}: \autoref{q:109:sa:en:True}



\subsection{What does interactive processing mean?}

\label{q:110:sa:en:False}

\vspace{2cm}

\noindent\makebox[\textwidth]{\hrulefill}

\vspace{1cm}

\textit{Answer}: \autoref{q:110:sa:en:True}



\subsection{What does real time processing mean?}

\label{q:111:sa:en:False}

\vspace{2cm}

\noindent\makebox[\textwidth]{\hrulefill}

\vspace{1cm}

\textit{Answer}: \autoref{q:111:sa:en:True}



\subsection{What is the difference between batch processing and interactive processing?}

\label{q:112:sa:en:False}

\vspace{2cm}

\noindent\makebox[\textwidth]{\hrulefill}

\vspace{1cm}

\textit{Answer}: \autoref{q:112:sa:en:True}



\subsection{What is virtual memory and what can it be good for?}

\label{q:113:sa:en:False}

\vspace{2cm}

\noindent\makebox[\textwidth]{\hrulefill}

\vspace{1cm}

\textit{Answer}: \autoref{q:113:sa:en:True}



\subsection{Name four different components of an operating system kernel (operating system kernel)?}

\label{q:114:sa:en:False}

\vspace{2cm}

\noindent\makebox[\textwidth]{\hrulefill}

\vspace{1cm}

\textit{Answer}: \autoref{q:114:sa:en:True}



\subsection{What is a file in a file management system?}

\label{q:115:sa:en:False}

\vspace{2cm}

\noindent\makebox[\textwidth]{\hrulefill}

\vspace{1cm}

\textit{Answer}: \autoref{q:115:sa:en:True}



\subsection{What are the four basic functions of an operating system (functions of operating systems)?}

\label{q:116:sa:en:False}

\vspace{2cm}

\noindent\makebox[\textwidth]{\hrulefill}

\vspace{1cm}

\textit{Answer}: \autoref{q:116:sa:en:True}



\subsection{An operating system consists of two main components (operating system components), what are they called?}

\label{q:118:sa:en:False}

\vspace{2cm}

\noindent\makebox[\textwidth]{\hrulefill}

\vspace{1cm}

\textit{Answer}: \autoref{q:118:sa:en:True}



\subsection{What conditions are required to be fulfilled in order for a deadlock to occur?}

\label{q:119:sa:en:False}

\vspace{2cm}

\noindent\makebox[\textwidth]{\hrulefill}

\vspace{1cm}

\textit{Answer}: \autoref{q:119:sa:en:True}



\subsection{A process's current state can be described by a set of data, which data?}

\label{q:120:sa:en:False}

\vspace{2cm}

\noindent\makebox[\textwidth]{\hrulefill}

\vspace{1cm}

\textit{Answer}: \autoref{q:120:sa:en:True}



\subsection{What is a program and what is a process?}

\label{q:121:sa:en:False}

\vspace{2cm}

\noindent\makebox[\textwidth]{\hrulefill}

\vspace{1cm}

\textit{Answer}: \autoref{q:121:sa:en:True}



\subsection{What's a file?}

\label{q:122:sa:en:False}

\vspace{2cm}

\noindent\makebox[\textwidth]{\hrulefill}

\vspace{1cm}

\textit{Answer}: \autoref{q:122:sa:en:True}



\subsection{What is a directory?}

\label{q:123:sa:en:False}

\vspace{2cm}

\noindent\makebox[\textwidth]{\hrulefill}

\vspace{1cm}

\textit{Answer}: \autoref{q:123:sa:en:True}



\subsection{What does paging mean?}

\label{q:124:sa:en:False}

\vspace{2cm}

\noindent\makebox[\textwidth]{\hrulefill}

\vspace{1cm}

\textit{Answer}: \autoref{q:124:sa:en:True}



\subsection{What is and what does a boot loader do?}

\label{q:125:sa:en:False}

\vspace{2cm}

\noindent\makebox[\textwidth]{\hrulefill}

\vspace{1cm}

\textit{Answer}: \autoref{q:125:sa:en:True}



\subsection{What does interactive processing mean?}

\label{q:126:sa:en:False}

\vspace{2cm}

\noindent\makebox[\textwidth]{\hrulefill}

\vspace{1cm}

\textit{Answer}: \autoref{q:126:sa:en:True}



\subsection{What does batch processing mean?}

\label{q:127:sa:en:False}

\vspace{2cm}

\noindent\makebox[\textwidth]{\hrulefill}

\vspace{1cm}

\textit{Answer}: \autoref{q:127:sa:en:True}



\subsection{What does the term deadlock mean?}

\label{q:129:sa:en:False}

\vspace{2cm}

\noindent\makebox[\textwidth]{\hrulefill}

\vspace{1cm}

\textit{Answer}: \autoref{q:129:sa:en:True}



\subsection{What is a batch processing job?}

\label{q:130:sa:en:False}

\vspace{2cm}

\noindent\makebox[\textwidth]{\hrulefill}

\vspace{1cm}

\textit{Answer}: \autoref{q:130:sa:en:True}



\subsection{What does multitasking mean?}

\label{q:132:sa:en:False}

\vspace{2cm}

\noindent\makebox[\textwidth]{\hrulefill}

\vspace{1cm}

\textit{Answer}: \autoref{q:132:sa:en:True}



\subsection{What is paging?}

\label{q:133:sa:en:False}

\vspace{2cm}

\noindent\makebox[\textwidth]{\hrulefill}

\vspace{1cm}

\textit{Answer}: \autoref{q:133:sa:en:True}



\subsection{What is the difference between a switch and a router?}

\label{q:134:sa:en:False}

\vspace{2cm}

\noindent\makebox[\textwidth]{\hrulefill}

\vspace{1cm}

\textit{Answer}: \autoref{q:134:sa:en:True}



\subsection{What are the two models of inter-process communication?}

\label{q:135:sa:en:False}

\vspace{2cm}

\noindent\makebox[\textwidth]{\hrulefill}

\vspace{1cm}

\textit{Answer}: \autoref{q:135:sa:en:True}



\subsection{What is an IP address?}

\label{q:136:sa:en:False}

\vspace{2cm}

\noindent\makebox[\textwidth]{\hrulefill}

\vspace{1cm}

\textit{Answer}: \autoref{q:136:sa:en:True}



\subsection{What is DNS?}

\label{q:137:sa:en:False}

\vspace{2cm}

\noindent\makebox[\textwidth]{\hrulefill}

\vspace{1cm}

\textit{Answer}: \autoref{q:137:sa:en:True}



\subsection{What does bus and star mean when it comes to network topology?}

\label{q:138:sa:en:False}

\vspace{2cm}

\noindent\makebox[\textwidth]{\hrulefill}

\vspace{1cm}

\textit{Answer}: \autoref{q:138:sa:en:True}



\subsection{What is cloud computing?}

\label{q:139:sa:en:False}

\vspace{2cm}

\noindent\makebox[\textwidth]{\hrulefill}

\vspace{1cm}

\textit{Answer}: \autoref{q:139:sa:en:True}



\subsection{What is the main difference between IPv4 (IP version 4) and IPv6 (IP version 6)?}

\label{q:140:sa:en:False}

\vspace{2cm}

\noindent\makebox[\textwidth]{\hrulefill}

\vspace{1cm}

\textit{Answer}: \autoref{q:140:sa:en:True}



\subsection{What is a certificate in the context of public key encryption?}

\label{q:141:sa:en:False}

\vspace{2cm}

\noindent\makebox[\textwidth]{\hrulefill}

\vspace{1cm}

\textit{Answer}: \autoref{q:141:sa:en:True}



\subsection{Give an example of a type of malware?}

\label{q:142:sa:en:False}

\vspace{2cm}

\noindent\makebox[\textwidth]{\hrulefill}

\vspace{1cm}

\textit{Answer}: \autoref{q:142:sa:en:True}



\subsection{What does DNS lookup mean?}

\label{q:143:sa:en:False}

\vspace{2cm}

\noindent\makebox[\textwidth]{\hrulefill}

\vspace{1cm}

\textit{Answer}: \autoref{q:143:sa:en:True}



\subsection{What does a (network) hub do?}

\label{q:144:sa:en:False}

\vspace{2cm}

\noindent\makebox[\textwidth]{\hrulefill}

\vspace{1cm}

\textit{Answer}: \autoref{q:144:sa:en:True}



\subsection{To which Internet software layer (Internet software layer) does the SMTP protocol belong?}

\label{q:145:sa:en:False}

\vspace{2cm}

\noindent\makebox[\textwidth]{\hrulefill}

\vspace{1cm}

\textit{Answer}: \autoref{q:145:sa:en:True}



\subsection{What does a web server (web server) do?}

\label{q:146:sa:en:False}

\vspace{2cm}

\noindent\makebox[\textwidth]{\hrulefill}

\vspace{1cm}

\textit{Answer}: \autoref{q:146:sa:en:True}



\subsection{What is the purpose of a URL/URI?}

\label{q:147:sa:en:False}

\vspace{2cm}

\noindent\makebox[\textwidth]{\hrulefill}

\vspace{1cm}

\textit{Answer}: \autoref{q:147:sa:en:True}



\subsection{What are the two common Internet protocols for the transport layer?}

\label{q:148:sa:en:False}

\vspace{2cm}

\noindent\makebox[\textwidth]{\hrulefill}

\vspace{1cm}

\textit{Answer}: \autoref{q:148:sa:en:True}



\subsection{What is the name of the encryption technology that is widely used on the Internet and which means that the parties do not need to have access to a common key in advance?}

\label{q:149:sa:en:False}

\vspace{2cm}

\noindent\makebox[\textwidth]{\hrulefill}

\vspace{1cm}

\textit{Answer}: \autoref{q:149:sa:en:True}



\subsection{What are Internet domains and what is their purpose?}

\label{q:150:sa:en:False}

\vspace{2cm}

\noindent\makebox[\textwidth]{\hrulefill}

\vspace{1cm}

\textit{Answer}: \autoref{q:150:sa:en:True}



\subsection{Give two examples of Internet applications using open (publicly available) protocols?}

\label{q:151:sa:en:False}

\vspace{2cm}

\noindent\makebox[\textwidth]{\hrulefill}

\vspace{1cm}

\textit{Answer}: \autoref{q:151:sa:en:True}



\subsection{What is the difference between the HTTP and HTTPS protocols?}

\label{q:152:sa:en:False}

\vspace{2cm}

\noindent\makebox[\textwidth]{\hrulefill}

\vspace{1cm}

\textit{Answer}: \autoref{q:152:sa:en:True}



\subsection{Briefly explain the difference between the network components hub, switch and router?}

\label{q:153:sa:en:False}

\vspace{2cm}

\noindent\makebox[\textwidth]{\hrulefill}

\vspace{1cm}

\textit{Answer}: \autoref{q:153:sa:en:True}



\subsection{What is transferred with the different protocols FTP, HTTP, SMTP?}

\label{q:154:sa:en:False}

\vspace{2cm}

\noindent\makebox[\textwidth]{\hrulefill}

\vspace{1cm}

\textit{Answer}: \autoref{q:154:sa:en:True}



\subsection{What is a certificate? Can all certificates be trusted equally? Motivate your answer!}

\label{q:155:sa:en:False}

\vspace{2cm}

\noindent\makebox[\textwidth]{\hrulefill}

\vspace{1cm}

\textit{Answer}: \autoref{q:155:sa:en:True}



\subsection{What does a digital signature mean for public key encryption, i.e. that when transferring a file, the identity of the sender can be guaranteed?}

\label{q:156:sa:en:False}

\vspace{2cm}

\noindent\makebox[\textwidth]{\hrulefill}

\vspace{1cm}

\textit{Answer}: \autoref{q:156:sa:en:True}



\subsection{What characterizes a distributed system?}

\label{q:157:sa:en:False}

\vspace{2cm}

\noindent\makebox[\textwidth]{\hrulefill}

\vspace{1cm}

\textit{Answer}: \autoref{q:157:sa:en:True}



\subsection{How many times more addresses can be represented with IPv6 compared to IPv4 (as usual, you do not need to calculate a value, it is enough to set up a correct calculation)?}

\label{q:158:sa:en:False}

\vspace{2cm}

\noindent\makebox[\textwidth]{\hrulefill}

\vspace{1cm}

\textit{Answer}: \autoref{q:158:sa:en:True}



\subsection{What is a distributed system?}

\label{q:159:sa:en:False}

\vspace{2cm}

\noindent\makebox[\textwidth]{\hrulefill}

\vspace{1cm}

\textit{Answer}: \autoref{q:159:sa:en:True}



\subsection{What is HTML used for?}

\label{q:160:sa:en:False}

\vspace{2cm}

\noindent\makebox[\textwidth]{\hrulefill}

\vspace{1cm}

\textit{Answer}: \autoref{q:160:sa:en:True}



\subsection{In public key encryption, the term certificate is used, what is it?}

\label{q:161:sa:en:False}

\vspace{2cm}

\noindent\makebox[\textwidth]{\hrulefill}

\vspace{1cm}

\textit{Answer}: \autoref{q:161:sa:en:True}



\subsection{In public key encryption, the term certificate authority is used, what is it?}

\label{q:162:sa:en:False}

\vspace{2cm}

\noindent\makebox[\textwidth]{\hrulefill}

\vspace{1cm}

\textit{Answer}: \autoref{q:162:sa:en:True}



\subsection{What is the benefit of using TCP instead of UDP? What is a disadvantage?}

\label{q:163:sa:en:False}

\vspace{2cm}

\noindent\makebox[\textwidth]{\hrulefill}

\vspace{1cm}

\textit{Answer}: \autoref{q:163:sa:en:True}



\subsection{What is the benefit of using UDP instead of TCP? What is a disadvantage?}

\label{q:164:sa:en:False}

\vspace{2cm}

\noindent\makebox[\textwidth]{\hrulefill}

\vspace{1cm}

\textit{Answer}: \autoref{q:164:sa:en:True}



\subsection{Name the four Internet software layers?}

\label{q:165:sa:en:False}

\vspace{2cm}

\noindent\makebox[\textwidth]{\hrulefill}

\vspace{1cm}

\textit{Answer}: \autoref{q:165:sa:en:True}



\subsection{Briefly explain the concept of client-server!}

\label{q:166:sa:en:False}

\vspace{2cm}

\noindent\makebox[\textwidth]{\hrulefill}

\vspace{1cm}

\textit{Answer}: \autoref{q:166:sa:en:True}



\subsection{What is SMTP used for?}

\label{q:167:sa:en:False}

\vspace{2cm}

\noindent\makebox[\textwidth]{\hrulefill}

\vspace{1cm}

\textit{Answer}: \autoref{q:167:sa:en:True}



\subsection{If person A wants to send a message to person B, encrypted according to public-key encryption, so that no one other than B can read the message. What does the message then need to be encrypted with before the message is sent from A?}

\label{q:168:sa:en:False}

\vspace{2cm}

\noindent\makebox[\textwidth]{\hrulefill}

\vspace{1cm}

\textit{Answer}: \autoref{q:168:sa:en:True}



\subsection{If person A wants to send a message to person B, encrypted according to public-key encryption, so that no one other than A can have sent the message. What does the message then need to be encrypted with before the message is sent from A?}

\label{q:169:sa:en:False}

\vspace{2cm}

\noindent\makebox[\textwidth]{\hrulefill}

\vspace{1cm}

\textit{Answer}: \autoref{q:169:sa:en:True}



\subsection{Name one advantage of public key encryption over symmetric encryption techniques.}

\label{q:170:sa:en:False}

\vspace{2cm}

\noindent\makebox[\textwidth]{\hrulefill}

\vspace{1cm}

\textit{Answer}: \autoref{q:170:sa:en:True}



\subsection{What is a hub or a router used for?}

\label{q:171:sa:en:False}

\vspace{2cm}

\noindent\makebox[\textwidth]{\hrulefill}

\vspace{1cm}

\textit{Answer}: \autoref{q:171:sa:en:True}



\subsection{Name one advantage of UDP over TCP, and one advantage of TCP over UDP.}

\label{q:172:sa:en:False}

\vspace{2cm}

\noindent\makebox[\textwidth]{\hrulefill}

\vspace{1cm}

\textit{Answer}: \autoref{q:172:sa:en:True}



\subsection{What is recursion?}

\label{q:173:sa:en:False}

\vspace{2cm}

\noindent\makebox[\textwidth]{\hrulefill}

\vspace{1cm}

\textit{Answer}: \autoref{q:173:sa:en:True}



\subsection{Why is binary search better than sequential search on sorted data?}

\label{q:174:sa:en:False}

\vspace{2cm}

\noindent\makebox[\textwidth]{\hrulefill}

\vspace{1cm}

\textit{Answer}: \autoref{q:174:sa:en:True}



\subsection{Is there any difference between iteration and recursion in terms of memory usage?}

\label{q:175:sa:en:False}

\vspace{2cm}

\noindent\makebox[\textwidth]{\hrulefill}

\vspace{1cm}

\textit{Answer}: \autoref{q:175:sa:en:True}



\subsection{What is the difference between an algorithm and a program?}

\label{q:177:sa:en:False}

\vspace{2cm}

\noindent\makebox[\textwidth]{\hrulefill}

\vspace{1cm}

\textit{Answer}: \autoref{q:177:sa:en:True}



\subsection{Which two different methods are used to verify that a program is correct (software verification)?}

\label{q:178:sa:en:False}

\vspace{2cm}

\noindent\makebox[\textwidth]{\hrulefill}

\vspace{1cm}

\textit{Answer}: \autoref{q:178:sa:en:True}



\subsection{What is a program in relation to an algorithm?}

\label{q:179:sa:en:False}

\vspace{2cm}

\noindent\makebox[\textwidth]{\hrulefill}

\vspace{1cm}

\textit{Answer}: \autoref{q:179:sa:en:True}



\subsection{Describe how binary search works! What are the requirements for the data you search in?}

\label{q:180:sa:en:False}

\vspace{2cm}

\noindent\makebox[\textwidth]{\hrulefill}

\vspace{1cm}

\textit{Answer}: \autoref{q:180:sa:en:True}



\subsection{What methods can be used to verify the correctness of a program?}

\label{q:181:sa:en:False}

\vspace{2cm}

\noindent\makebox[\textwidth]{\hrulefill}

\vspace{1cm}

\textit{Answer}: \autoref{q:181:sa:en:True}



\subsection{In what two fundamentally different ways can repetition be achieved in an algorithm?}

\label{q:182:sa:en:False}

\vspace{2cm}

\noindent\makebox[\textwidth]{\hrulefill}

\vspace{1cm}

\textit{Answer}: \autoref{q:182:sa:en:True}



\subsection{When is sequential search preferable to binary search?}

\label{q:183:sa:en:False}

\vspace{2cm}

\noindent\makebox[\textwidth]{\hrulefill}

\vspace{1cm}

\textit{Answer}: \autoref{q:183:sa:en:True}



\subsection{What is a prerequisite for binary search to work? Motivate your answer.}

\label{q:184:sa:en:False}

\vspace{2cm}

\noindent\makebox[\textwidth]{\hrulefill}

\vspace{1cm}

\textit{Answer}: \autoref{q:184:sa:en:True}



\subsection{Is binary search a good choice for searching unsorted data? Motivate your answer.}

\label{q:185:sa:en:False}

\vspace{2cm}

\noindent\makebox[\textwidth]{\hrulefill}

\vspace{1cm}

\textit{Answer}: \autoref{q:185:sa:en:True}



\subsection{Define the term algorithm!}

\label{q:186:sa:en:False}

\vspace{2cm}

\noindent\makebox[\textwidth]{\hrulefill}

\vspace{1cm}

\textit{Answer}: \autoref{q:186:sa:en:True}



\subsection{Can all algorithms be described as a flow chart? Motivate your answer!}

\label{q:187:sa:en:False}

\vspace{2cm}

\noindent\makebox[\textwidth]{\hrulefill}

\vspace{1cm}

\textit{Answer}: \autoref{q:187:sa:en:True}



\subsection{Is a programming language, e.g. Python, suitable for describing algorithms? Motivate your answer!}

\label{q:188:sa:en:False}

\vspace{2cm}

\noindent\makebox[\textwidth]{\hrulefill}

\vspace{1cm}

\textit{Answer}: \autoref{q:188:sa:en:True}



\subsection{What does the top-down methodology mean when developing (or discovering) algorithms?}

\label{q:189:sa:en:False}

\vspace{2cm}

\noindent\makebox[\textwidth]{\hrulefill}

\vspace{1cm}

\textit{Answer}: \autoref{q:189:sa:en:True}



\subsection{Why is it not so important to follow a strict syntax in pseudocode?}

\label{q:190:sa:en:False}

\vspace{2cm}

\noindent\makebox[\textwidth]{\hrulefill}

\vspace{1cm}

\textit{Answer}: \autoref{q:190:sa:en:True}



\subsection{Why is it necessary to know the data type of a variable?}

\label{q:191:sa:en:False}

\vspace{2cm}

\noindent\makebox[\textwidth]{\hrulefill}

\vspace{1cm}

\textit{Answer}: \autoref{q:191:sa:en:True}



\subsection{What is the difference between source code and object code?}

\label{q:192:sa:en:False}

\vspace{2cm}

\noindent\makebox[\textwidth]{\hrulefill}

\vspace{1cm}

\textit{Answer}: \autoref{q:192:sa:en:True}



\subsection{Does a syntactically correct program always produce correct results? Motivate your answer.}

\label{q:193:sa:en:False}

\vspace{2cm}

\noindent\makebox[\textwidth]{\hrulefill}

\vspace{1cm}

\textit{Answer}: \autoref{q:193:sa:en:True}



\subsection{What characterizes a data structure of the type struct/record (aggregate type)?}

\label{q:194:sa:en:False}

\vspace{2cm}

\noindent\makebox[\textwidth]{\hrulefill}

\vspace{1cm}

\textit{Answer}: \autoref{q:194:sa:en:True}



\subsection{What does it mean that a parameter to a subroutine is transferred as value (passed by value)?}

\label{q:195:sa:en:False}

\vspace{2cm}

\noindent\makebox[\textwidth]{\hrulefill}

\vspace{1cm}

\textit{Answer}: \autoref{q:195:sa:en:True}



\subsection{What does it mean that a parameter to a subroutine is transferred as a reference (passed by reference)?}

\label{q:196:sa:en:False}

\vspace{2cm}

\noindent\makebox[\textwidth]{\hrulefill}

\vspace{1cm}

\textit{Answer}: \autoref{q:196:sa:en:True}



\subsection{What does an assembler do?}

\label{q:197:sa:en:False}

\vspace{2cm}

\noindent\makebox[\textwidth]{\hrulefill}

\vspace{1cm}

\textit{Answer}: \autoref{q:197:sa:en:True}



\subsection{What characterizes an array data structure?}

\label{q:198:sa:en:False}

\vspace{2cm}

\noindent\makebox[\textwidth]{\hrulefill}

\vspace{1cm}

\textit{Answer}: \autoref{q:198:sa:en:True}



\subsection{What are the four major programming paradigms?}

\label{q:199:sa:en:False}

\vspace{2cm}

\noindent\makebox[\textwidth]{\hrulefill}

\vspace{1cm}

\textit{Answer}: \autoref{q:199:sa:en:True}



\subsection{List four common primitive data types.}

\label{q:200:sa:en:False}

\vspace{2cm}

\noindent\makebox[\textwidth]{\hrulefill}

\vspace{1cm}

\textit{Answer}: \autoref{q:200:sa:en:True}



\subsection{What does a compiler do?}

\label{q:201:sa:en:False}

\vspace{2cm}

\noindent\makebox[\textwidth]{\hrulefill}

\vspace{1cm}

\textit{Answer}: \autoref{q:201:sa:en:True}



\subsection{A program can give rise to three different types of errors: syntactic errors, runtime errors and logic errors. What type of errors are most serious and why?}

\label{q:202:sa:en:False}

\vspace{2cm}

\noindent\makebox[\textwidth]{\hrulefill}

\vspace{1cm}

\textit{Answer}: \autoref{q:202:sa:en:True}



\subsection{A program can give rise to three different types of errors: syntactic errors, runtime errors and logic errors. What type of errors are least serious and why?}

\label{q:203:sa:en:False}

\vspace{2cm}

\noindent\makebox[\textwidth]{\hrulefill}

\vspace{1cm}

\textit{Answer}: \autoref{q:203:sa:en:True}



\subsection{What is concurrent programming?}

\label{q:204:sa:en:False}

\vspace{2cm}

\noindent\makebox[\textwidth]{\hrulefill}

\vspace{1cm}

\textit{Answer}: \autoref{q:204:sa:en:True}



\subsection{Briefly describe the concepts of sequence, selection and iteration.}

\label{q:205:sa:en:False}

\vspace{2cm}

\noindent\makebox[\textwidth]{\hrulefill}

\vspace{1cm}

\textit{Answer}: \autoref{q:205:sa:en:True}



\subsection{A variable points to a bit pattern in stored in memory; what do we need to know to interpret the bit pattern correctly?}

\label{q:206:sa:en:False}

\vspace{2cm}

\noindent\makebox[\textwidth]{\hrulefill}

\vspace{1cm}

\textit{Answer}: \autoref{q:206:sa:en:True}



\subsection{What do the terms sequence, selection and iteration mean?}

\label{q:207:sa:en:False}

\vspace{2cm}

\noindent\makebox[\textwidth]{\hrulefill}

\vspace{1cm}

\textit{Answer}: \autoref{q:207:sa:en:True}



\subsection{Give examples of two different ways of describing algorithms.}

\label{q:208:sa:en:False}

\vspace{2cm}

\noindent\makebox[\textwidth]{\hrulefill}

\vspace{1cm}

\textit{Answer}: \autoref{q:208:sa:en:True}



\subsection{What generation of programming languages is characterized by: - one-to-one correspondence between language instructions and machine instructions; - inherently machine-dependent?}

\label{q:209:sa:en:False}

\vspace{2cm}

\noindent\makebox[\textwidth]{\hrulefill}

\vspace{1cm}

\textit{Answer}: \autoref{q:209:sa:en:True}



\subsection{What generation of programming languages is characterized by:- machine independent (mostly);- each primitive corresponds to a sequence of machine language instructions?}

\label{q:210:sa:en:False}

\vspace{2cm}

\noindent\makebox[\textwidth]{\hrulefill}

\vspace{1cm}

\textit{Answer}: \autoref{q:210:sa:en:True}



\subsection{What is a literal in a programming language?}

\label{q:211:sa:en:False}

\vspace{2cm}

\noindent\makebox[\textwidth]{\hrulefill}

\vspace{1cm}

\textit{Answer}: \autoref{q:211:sa:en:True}



\subsection{What is a constant in a programming language?}

\label{q:212:sa:en:False}

\vspace{2cm}

\noindent\makebox[\textwidth]{\hrulefill}

\vspace{1cm}

\textit{Answer}: \autoref{q:212:sa:en:True}



\subsection{In object-oriented programming you have classes and objects. In addition to this there are three features that characterize object-oriented programming, which ones?}

\label{q:213:sa:en:False}

\vspace{2cm}

\noindent\makebox[\textwidth]{\hrulefill}

\vspace{1cm}

\textit{Answer}: \autoref{q:213:sa:en:True}



\subsection{The translation of source code to machine code is done in three steps by three different units in the translator; what are these three units called?}

\label{q:214:sa:en:False}

\vspace{2cm}

\noindent\makebox[\textwidth]{\hrulefill}

\vspace{1cm}

\textit{Answer}: \autoref{q:214:sa:en:True}



\subsection{What is a thread in concurrent programming?}

\label{q:215:sa:en:False}

\vspace{2cm}

\noindent\makebox[\textwidth]{\hrulefill}

\vspace{1cm}

\textit{Answer}: \autoref{q:215:sa:en:True}



\subsection{What is the basic building block in logic programming languages?}

\label{q:216:sa:en:False}

\vspace{2cm}

\noindent\makebox[\textwidth]{\hrulefill}

\vspace{1cm}

\textit{Answer}: \autoref{q:216:sa:en:True}



\subsection{What is a variable in a programming language?}

\label{q:217:sa:en:False}

\vspace{2cm}

\noindent\makebox[\textwidth]{\hrulefill}

\vspace{1cm}

\textit{Answer}: \autoref{q:217:sa:en:True}



\subsection{What is the purpose of using procedural units (subprogram, subroutine, procedure, function, method, predicate etc.) in programming?}

\label{q:218:sa:en:False}

\vspace{2cm}

\noindent\makebox[\textwidth]{\hrulefill}

\vspace{1cm}

\textit{Answer}: \autoref{q:218:sa:en:True}



\subsection{What does inheritance mean in object-oriented programming?}

\label{q:219:sa:en:False}

\vspace{2cm}

\noindent\makebox[\textwidth]{\hrulefill}

\vspace{1cm}

\textit{Answer}: \autoref{q:219:sa:en:True}



\subsection{What is the difference between a compiler and an interpreter?}

\label{q:220:sa:en:False}

\vspace{2cm}

\noindent\makebox[\textwidth]{\hrulefill}

\vspace{1cm}

\textit{Answer}: \autoref{q:220:sa:en:True}



\subsection{All programming languages have three types of program control flow, which?}

\label{q:221:sa:en:False}

\vspace{2cm}

\noindent\makebox[\textwidth]{\hrulefill}

\vspace{1cm}

\textit{Answer}: \autoref{q:221:sa:en:True}



\subsection{What three things characterize agile development models?}

\label{q:222:sa:en:False}

\vspace{2cm}

\noindent\makebox[\textwidth]{\hrulefill}

\vspace{1cm}

\textit{Answer}: \autoref{q:222:sa:en:True}



\subsection{What are design patterns?}

\label{q:223:sa:en:False}

\vspace{2cm}

\noindent\makebox[\textwidth]{\hrulefill}

\vspace{1cm}

\textit{Answer}: \autoref{q:223:sa:en:True}



\subsection{What is the purpose of use case diagram?}

\label{q:224:sa:en:False}

\vspace{2cm}

\noindent\makebox[\textwidth]{\hrulefill}

\vspace{1cm}

\textit{Answer}: \autoref{q:224:sa:en:True}



\subsection{What is the purpose of class diagrams?}

\label{q:225:sa:en:False}

\vspace{2cm}

\noindent\makebox[\textwidth]{\hrulefill}

\vspace{1cm}

\textit{Answer}: \autoref{q:225:sa:en:True}



\subsection{What are the four traditional development phases in software development (the traditional development phases of the software life cycle)?}

\label{q:226:sa:en:False}

\vspace{2cm}

\noindent\makebox[\textwidth]{\hrulefill}

\vspace{1cm}

\textit{Answer}: \autoref{q:226:sa:en:True}



\subsection{What is the main purpose of dividing a software into modules?}

\label{q:227:sa:en:False}

\vspace{2cm}

\noindent\makebox[\textwidth]{\hrulefill}

\vspace{1cm}

\textit{Answer}: \autoref{q:227:sa:en:True}



\subsection{What are the three desirable characteristics of modules that one wants to achieve when dividing a software into modules?}

\label{q:228:sa:en:False}

\vspace{2cm}

\noindent\makebox[\textwidth]{\hrulefill}

\vspace{1cm}

\textit{Answer}: \autoref{q:228:sa:en:True}



\subsection{What is the difference between glass-box testing and black-box testing?}

\label{q:229:sa:en:False}

\vspace{2cm}

\noindent\makebox[\textwidth]{\hrulefill}

\vspace{1cm}

\textit{Answer}: \autoref{q:229:sa:en:True}



\subsection{Describe the differences between one-to-one, one-to-many and many-to-many relationships, preferably using example.}

\label{q:230:sa:en:False}

\vspace{2cm}

\noindent\makebox[\textwidth]{\hrulefill}

\vspace{1cm}

\textit{Answer}: \autoref{q:230:sa:en:True}



\subsection{What is the software technology called that is based on constructing software by combining different ready-made components (instead of developing own components)?}

\label{q:231:sa:en:False}

\vspace{2cm}

\noindent\makebox[\textwidth]{\hrulefill}

\vspace{1cm}

\textit{Answer}: \autoref{q:231:sa:en:True}



\subsection{Describe an example of each of the different types of relationship: one-to-one (one-to-one), one-to-many (one-to-many), and many-to-many (many-to- many)!}

\label{q:232:sa:en:False}

\vspace{2cm}

\noindent\makebox[\textwidth]{\hrulefill}

\vspace{1cm}

\textit{Answer}: \autoref{q:232:sa:en:True}



\subsection{What are the four steps in traditional software development (using, for example, the waterfall model)?}

\label{q:233:sa:en:False}

\vspace{2cm}

\noindent\makebox[\textwidth]{\hrulefill}

\vspace{1cm}

\textit{Answer}: \autoref{q:233:sa:en:True}



\subsection{Briefly describe some advantages of dividing programs into modules?}

\label{q:234:sa:en:False}

\vspace{2cm}

\noindent\makebox[\textwidth]{\hrulefill}

\vspace{1cm}

\textit{Answer}: \autoref{q:234:sa:en:True}



\subsection{What does prototyping mean?}

\label{q:235:sa:en:False}

\vspace{2cm}

\noindent\makebox[\textwidth]{\hrulefill}

\vspace{1cm}

\textit{Answer}: \autoref{q:235:sa:en:True}



\subsection{Describe what a sprint in agile development with Scrum is?}

\label{q:236:sa:en:False}

\vspace{2cm}

\noindent\makebox[\textwidth]{\hrulefill}

\vspace{1cm}

\textit{Answer}: \autoref{q:236:sa:en:True}



\subsection{What characterizes black-box testing?}

\label{q:237:sa:en:False}

\vspace{2cm}

\noindent\makebox[\textwidth]{\hrulefill}

\vspace{1cm}

\textit{Answer}: \autoref{q:237:sa:en:True}



\subsection{What are design patterns and what are they good for?}

\label{q:238:sa:en:False}

\vspace{2cm}

\noindent\makebox[\textwidth]{\hrulefill}

\vspace{1cm}

\textit{Answer}: \autoref{q:238:sa:en:True}



\subsection{Explain the concepts of coupling and cohesion?}

\label{q:239:sa:en:False}

\vspace{2cm}

\noindent\makebox[\textwidth]{\hrulefill}

\vspace{1cm}

\textit{Answer}: \autoref{q:239:sa:en:True}



\subsection{What three different types of relationships between entities are important to distinguish in software engineering?}

\label{q:240:sa:en:False}

\vspace{2cm}

\noindent\makebox[\textwidth]{\hrulefill}

\vspace{1cm}

\textit{Answer}: \autoref{q:240:sa:en:True}



\subsection{What are software engineering methods called that value: - individuals and interactions over processes and tools;- working software over comprehensive documentation;- customer collaboration over contract negotiation;- responding to change over following a plan.}

\label{q:241:sa:en:False}

\vspace{2cm}

\noindent\makebox[\textwidth]{\hrulefill}

\vspace{1cm}

\textit{Answer}: \autoref{q:241:sa:en:True}



\subsection{What is a software module?}

\label{q:242:sa:en:False}

\vspace{2cm}

\noindent\makebox[\textwidth]{\hrulefill}

\vspace{1cm}

\textit{Answer}: \autoref{q:242:sa:en:True}



\subsection{What is the purpose of the Scrum meeting "sprint retrospective"?}

\label{q:243:sa:en:False}

\vspace{2cm}

\noindent\makebox[\textwidth]{\hrulefill}

\vspace{1cm}

\textit{Answer}: \autoref{q:243:sa:en:True}



\subsection{How many members should a development team have according to Scrum?}

\label{q:244:sa:en:False}

\vspace{2cm}

\noindent\makebox[\textwidth]{\hrulefill}

\vspace{1cm}

\textit{Answer}: \autoref{q:244:sa:en:True}



\subsection{Give two examples of diagrams used in modeling (in software engineering).}

\label{q:245:sa:en:False}

\vspace{2cm}

\noindent\makebox[\textwidth]{\hrulefill}

\vspace{1cm}

\textit{Answer}: \autoref{q:245:sa:en:True}



\subsection{What is a design pattern (in software engineering)?}

\label{q:246:sa:en:False}

\vspace{2cm}

\noindent\makebox[\textwidth]{\hrulefill}

\vspace{1cm}

\textit{Answer}: \autoref{q:246:sa:en:True}



\subsection{What characterizes glass-box testing?}

\label{q:247:sa:en:False}

\vspace{2cm}

\noindent\makebox[\textwidth]{\hrulefill}

\vspace{1cm}

\textit{Answer}: \autoref{q:247:sa:en:True}



\subsection{What 3 questions should each team member briefly answer at the Daily Scrum meetings?}

\label{q:248:sa:en:False}

\vspace{2cm}

\noindent\makebox[\textwidth]{\hrulefill}

\vspace{1cm}

\textit{Answer}: \autoref{q:248:sa:en:True}



\subsection{What is the Scrum meeting, at the end of a sprint where you discuss what went well during the previous sprint process and what can be improved for the next sprint, called?}

\label{q:249:sa:en:False}

\vspace{2cm}

\noindent\makebox[\textwidth]{\hrulefill}

\vspace{1cm}

\textit{Answer}: \autoref{q:249:sa:en:True}



\subsection{The Scrum development method has three different roles defined, which ones?}

\label{q:250:sa:en:False}

\vspace{2cm}

\noindent\makebox[\textwidth]{\hrulefill}

\vspace{1cm}

\textit{Answer}: \autoref{q:250:sa:en:True}



\subsection{What is an abstract data type?}

\label{q:251:sa:en:False}

\vspace{2cm}

\noindent\makebox[\textwidth]{\hrulefill}

\vspace{1cm}

\textit{Answer}: \autoref{q:251:sa:en:True}



\subsection{What characterizes a sorted binary tree ("binary search tree")?}

\label{q:252:sa:en:False}

\vspace{2cm}

\noindent\makebox[\textwidth]{\hrulefill}

\vspace{1cm}

\textit{Answer}: \autoref{q:252:sa:en:True}



\subsection{What are the four basic data structures in addition to arrays?}

\label{q:253:sa:en:False}

\vspace{2cm}

\noindent\makebox[\textwidth]{\hrulefill}

\vspace{1cm}

\textit{Answer}: \autoref{q:253:sa:en:True}



\subsection{What is the difference between a dynamic and a static data structure?}

\label{q:254:sa:en:False}

\vspace{2cm}

\noindent\makebox[\textwidth]{\hrulefill}

\vspace{1cm}

\textit{Answer}: \autoref{q:254:sa:en:True}



\subsection{What characterizes a binary tree?}

\label{q:255:sa:en:False}

\vspace{2cm}

\noindent\makebox[\textwidth]{\hrulefill}

\vspace{1cm}

\textit{Answer}: \autoref{q:255:sa:en:True}



\subsection{What is the difference between a static and a dynamic data structure?}

\label{q:256:sa:en:False}

\vspace{2cm}

\noindent\makebox[\textwidth]{\hrulefill}

\vspace{1cm}

\textit{Answer}: \autoref{q:256:sa:en:True}



\subsection{Can a list be implemented as a static or dynamic data structure, both, or neither? Motivate your answer!}

\label{q:257:sa:en:False}

\vspace{2cm}

\noindent\makebox[\textwidth]{\hrulefill}

\vspace{1cm}

\textit{Answer}: \autoref{q:257:sa:en:True}



\subsection{Describe the basic data structures stack (stack) and queue (queue)?}

\label{q:258:sa:en:False}

\vspace{2cm}

\noindent\makebox[\textwidth]{\hrulefill}

\vspace{1cm}

\textit{Answer}: \autoref{q:258:sa:en:True}



\subsection{Can the low-level data structure array be used to implement a queue? Motivate your answer!}

\label{q:259:sa:en:False}

\vspace{2cm}

\noindent\makebox[\textwidth]{\hrulefill}

\vspace{1cm}

\textit{Answer}: \autoref{q:259:sa:en:True}



\subsection{What is an abstract data structure? What is the difference to a record/struct?}

\label{q:260:sa:en:False}

\vspace{2cm}

\noindent\makebox[\textwidth]{\hrulefill}

\vspace{1cm}

\textit{Answer}: \autoref{q:260:sa:en:True}



\subsection{Give an example of a data structure that uses the LIFO principle and a data structure that uses the FIFO principle?}

\label{q:261:sa:en:False}

\vspace{2cm}

\noindent\makebox[\textwidth]{\hrulefill}

\vspace{1cm}

\textit{Answer}: \autoref{q:261:sa:en:True}



\subsection{Lists can be stored either in contiguous blocks in memory, or in the form of linked lists. Which is preferred for static lists, and which is better for dynamic lists?}

\label{q:262:sa:en:False}

\vspace{2cm}

\noindent\makebox[\textwidth]{\hrulefill}

\vspace{1cm}

\textit{Answer}: \autoref{q:262:sa:en:True}



\subsection{Explain what a pointer is?}

\label{q:263:sa:en:False}

\vspace{2cm}

\noindent\makebox[\textwidth]{\hrulefill}

\vspace{1cm}

\textit{Answer}: \autoref{q:263:sa:en:True}



\subsection{Two types of specialized lists are stack and queue, describe how they differ from each other!}

\label{q:264:sa:en:False}

\vspace{2cm}

\noindent\makebox[\textwidth]{\hrulefill}

\vspace{1cm}

\textit{Answer}: \autoref{q:264:sa:en:True}



\subsection{What distinguishes an abstract data type (abstract data type) from a composite data type (aggregate type / struct / record)?}

\label{q:265:sa:en:False}

\vspace{2cm}

\noindent\makebox[\textwidth]{\hrulefill}

\vspace{1cm}

\textit{Answer}: \autoref{q:265:sa:en:True}



\subsection{In a variation of lists, you add and remove elements at the same end, what is that data structure called? In another variation, elements are added at one end and removed at the other, what is that data structure called?}

\label{q:266:sa:en:False}

\vspace{2cm}

\noindent\makebox[\textwidth]{\hrulefill}

\vspace{1cm}

\textit{Answer}: \autoref{q:266:sa:en:True}



\subsection{What characterizes an aggregate type (struct/record)?}

\label{q:267:sa:en:False}

\vspace{2cm}

\noindent\makebox[\textwidth]{\hrulefill}

\vspace{1cm}

\textit{Answer}: \autoref{q:267:sa:en:True}



\subsection{Can a queue be implemented as a static or dynamic data structure, both, or neither? Motivate your answer!}

\label{q:268:sa:en:False}

\vspace{2cm}

\noindent\makebox[\textwidth]{\hrulefill}

\vspace{1cm}

\textit{Answer}: \autoref{q:268:sa:en:True}



\subsection{What distinguishes a dynamic data structure from a static data structure?}

\label{q:269:sa:en:False}

\vspace{2cm}

\noindent\makebox[\textwidth]{\hrulefill}

\vspace{1cm}

\textit{Answer}: \autoref{q:269:sa:en:True}



\subsection{What characterizes the data structure binary tree?}

\label{q:270:sa:en:False}

\vspace{2cm}

\noindent\makebox[\textwidth]{\hrulefill}

\vspace{1cm}

\textit{Answer}: \autoref{q:270:sa:en:True}



\subsection{What characterizes the root node in a tree data structure?}

\label{q:271:sa:en:False}

\vspace{2cm}

\noindent\makebox[\textwidth]{\hrulefill}

\vspace{1cm}

\textit{Answer}: \autoref{q:271:sa:en:True}



\subsection{Describe an advantage and a disadvantage of storing an aggregate type (struct/record) in a contiguous block instead of the different parts in separate locations designated by pointers.}

\label{q:272:sa:en:False}

\vspace{2cm}

\noindent\makebox[\textwidth]{\hrulefill}

\vspace{1cm}

\textit{Answer}: \autoref{q:272:sa:en:True}



\subsection{What characterizes an array?}

\label{q:273:sa:en:False}

\vspace{2cm}

\noindent\makebox[\textwidth]{\hrulefill}

\vspace{1cm}

\textit{Answer}: \autoref{q:273:sa:en:True}



\subsection{What characterizes a static data structure?}

\label{q:274:sa:en:False}

\vspace{2cm}

\noindent\makebox[\textwidth]{\hrulefill}

\vspace{1cm}

\textit{Answer}: \autoref{q:274:sa:en:True}



\subsection{What characterizes a dynamic data structure?}

\label{q:275:sa:en:False}

\vspace{2cm}

\noindent\makebox[\textwidth]{\hrulefill}

\vspace{1cm}

\textit{Answer}: \autoref{q:275:sa:en:True}



\subsection{Describe an advantage and a disadvantage of storing the different parts of an aggregate type (struct/record) in separate locations designated by pointers instead of in a contiguous block.}

\label{q:276:sa:en:False}

\vspace{2cm}

\noindent\makebox[\textwidth]{\hrulefill}

\vspace{1cm}

\textit{Answer}: \autoref{q:276:sa:en:True}



\subsection{What is a database management system (DBMS)?}

\label{q:277:sa:en:False}

\vspace{2cm}

\noindent\makebox[\textwidth]{\hrulefill}

\vspace{1cm}

\textit{Answer}: \autoref{q:277:sa:en:True}



\subsection{What do commit and rollback mean in a database context?}

\label{q:278:sa:en:False}

\vspace{2cm}

\noindent\makebox[\textwidth]{\hrulefill}

\vspace{1cm}

\textit{Answer}: \autoref{q:278:sa:en:True}



\subsection{For relational databases, there are three (3) operations (relational operations), with the help of which you can create new tables that constitute subsets and / or combinations of existing tables. What operations?}

\label{q:279:sa:en:False}

\vspace{2cm}

\noindent\makebox[\textwidth]{\hrulefill}

\vspace{1cm}

\textit{Answer}: \autoref{q:279:sa:en:True}



\subsection{What does data mining mean?}

\label{q:280:sa:en:False}

\vspace{2cm}

\noindent\makebox[\textwidth]{\hrulefill}

\vspace{1cm}

\textit{Answer}: \autoref{q:280:sa:en:True}



\subsection{What is a data warehouse?}

\label{q:281:sa:en:False}

\vspace{2cm}

\noindent\makebox[\textwidth]{\hrulefill}

\vspace{1cm}

\textit{Answer}: \autoref{q:281:sa:en:True}



\subsection{What is a database (database) in relation to a database management system (DBMS)?}

\label{q:282:sa:en:False}

\vspace{2cm}

\noindent\makebox[\textwidth]{\hrulefill}

\vspace{1cm}

\textit{Answer}: \autoref{q:282:sa:en:True}



\subsection{Mention a common problem that can arise when, for example, transfers between accounts that transactions protect against.}

\label{q:283:sa:en:False}

\vspace{2cm}

\noindent\makebox[\textwidth]{\hrulefill}

\vspace{1cm}

\textit{Answer}: \autoref{q:283:sa:en:True}



\subsection{What is SQL?}

\label{q:284:sa:en:False}

\vspace{2cm}

\noindent\makebox[\textwidth]{\hrulefill}

\vspace{1cm}

\textit{Answer}: \autoref{q:284:sa:en:True}



\subsection{In what two ways can a transaction be terminated?}

\label{q:285:sa:en:False}

\vspace{2cm}

\noindent\makebox[\textwidth]{\hrulefill}

\vspace{1cm}

\textit{Answer}: \autoref{q:285:sa:en:True}



\subsection{What is a transaction?}

\label{q:286:sa:en:False}

\vspace{2cm}

\noindent\makebox[\textwidth]{\hrulefill}

\vspace{1cm}

\textit{Answer}: \autoref{q:286:sa:en:True}



\subsection{A transaction can be terminated in two different ways, which ones?}

\label{q:287:sa:en:False}

\vspace{2cm}

\noindent\makebox[\textwidth]{\hrulefill}

\vspace{1cm}

\textit{Answer}: \autoref{q:287:sa:en:True}



\subsection{What is a database schema?}

\label{q:288:sa:en:False}

\vspace{2cm}

\noindent\makebox[\textwidth]{\hrulefill}

\vspace{1cm}

\textit{Answer}: \autoref{q:288:sa:en:True}



\subsection{What is a database?}

\label{q:289:sa:en:False}

\vspace{2cm}

\noindent\makebox[\textwidth]{\hrulefill}

\vspace{1cm}

\textit{Answer}: \autoref{q:289:sa:en:True}



\subsection{What is a database model?}

\label{q:290:sa:en:False}

\vspace{2cm}

\noindent\makebox[\textwidth]{\hrulefill}

\vspace{1cm}

\textit{Answer}: \autoref{q:290:sa:en:True}



\subsection{Name two things that differentiate an object-oriented database from a relational database?}

\label{q:291:sa:en:False}

\vspace{2cm}

\noindent\makebox[\textwidth]{\hrulefill}

\vspace{1cm}

\textit{Answer}: \autoref{q:291:sa:en:True}



\subsection{What is data mining?}

\label{q:292:sa:en:False}

\vspace{2cm}

\noindent\makebox[\textwidth]{\hrulefill}

\vspace{1cm}

\textit{Answer}: \autoref{q:292:sa:en:True}



\subsection{What does a table represent in the relational model for databases?}

\label{q:293:sa:en:False}

\vspace{2cm}

\noindent\makebox[\textwidth]{\hrulefill}

\vspace{1cm}

\textit{Answer}: \autoref{q:293:sa:en:True}



\subsection{What does a column in a table represent in the database relational model?}

\label{q:294:sa:en:False}

\vspace{2cm}

\noindent\makebox[\textwidth]{\hrulefill}

\vspace{1cm}

\textit{Answer}: \autoref{q:294:sa:en:True}



\subsection{What does a row in a table represent in the database relational model?}

\label{q:295:sa:en:False}

\vspace{2cm}

\noindent\makebox[\textwidth]{\hrulefill}

\vspace{1cm}

\textit{Answer}: \autoref{q:295:sa:en:True}



\subsection{What are the three relational operations in the relational model for databases?}

\label{q:296:sa:en:False}

\vspace{2cm}

\noindent\makebox[\textwidth]{\hrulefill}

\vspace{1cm}

\textit{Answer}: \autoref{q:296:sa:en:True}



\subsection{What are the three basic relational operations for retrieving requested data from a relational database?}

\label{q:297:sa:en:False}

\vspace{2cm}

\noindent\makebox[\textwidth]{\hrulefill}

\vspace{1cm}

\textit{Answer}: \autoref{q:297:sa:en:True}



\subsection{What is a schema in the context of a database system?}

\label{q:298:sa:en:False}

\vspace{2cm}

\noindent\makebox[\textwidth]{\hrulefill}

\vspace{1cm}

\textit{Answer}: \autoref{q:298:sa:en:True}



\subsection{To which programming paradigm does the database query language SQL (structured query language) belong?}

\label{q:299:sa:en:False}

\vspace{2cm}

\noindent\makebox[\textwidth]{\hrulefill}

\vspace{1cm}

\textit{Answer}: \autoref{q:299:sa:en:True}



\subsection{Why is it of interest to know the efficiency class/complexity class of an algorithm?}

\label{q:300:sa:en:False}

\vspace{2cm}

\noindent\makebox[\textwidth]{\hrulefill}

\vspace{1cm}

\textit{Answer}: \autoref{q:300:sa:en:True}



\subsection{The process of creating 3D graphics consists of three steps, the first of which is 3D modeling (3D modeling), and the third is image display (display). What is the second step called, and what is done in that step?}

\label{q:301:sa:en:False}

\vspace{2cm}

\noindent\makebox[\textwidth]{\hrulefill}

\vspace{1cm}

\textit{Answer}: \autoref{q:301:sa:en:True}



\subsection{In animation projects, the work is usually carried out in three steps, which?}

\label{q:302:sa:en:False}

\vspace{2cm}

\noindent\makebox[\textwidth]{\hrulefill}

\vspace{1cm}

\textit{Answer}: \autoref{q:302:sa:en:True}



\subsection{Two branches of the field of mechanics have proven particularly useful in simulating natural motions, which?}

\label{q:303:sa:en:False}

\vspace{2cm}

\noindent\makebox[\textwidth]{\hrulefill}

\vspace{1cm}

\textit{Answer}: \autoref{q:303:sa:en:True}



\subsection{Name a way to produce so-called polygonal meshes in 3D modeling!}

\label{q:304:sa:en:False}

\vspace{2cm}

\noindent\makebox[\textwidth]{\hrulefill}

\vspace{1cm}

\textit{Answer}: \autoref{q:304:sa:en:True}



\subsection{Briefly explain the difference between local lighting models (local lightning model) and global lighting models (global lightning model). Which model gives the most realistic result? The advantage of the second?}

\label{q:305:sa:en:False}

\vspace{2cm}

\noindent\makebox[\textwidth]{\hrulefill}

\vspace{1cm}

\textit{Answer}: \autoref{q:305:sa:en:True}



\subsection{In computer graphics, light plays an important role. Light is usually divided into three (3) different kinds, which ones? What sets them apart?}

\label{q:306:sa:en:False}

\vspace{2cm}

\noindent\makebox[\textwidth]{\hrulefill}

\vspace{1cm}

\textit{Answer}: \autoref{q:306:sa:en:True}



\subsection{Explain how the terms frame, key frame and in-betweening used in animation are related?}

\label{q:307:sa:en:False}

\vspace{2cm}

\noindent\makebox[\textwidth]{\hrulefill}

\vspace{1cm}

\textit{Answer}: \autoref{q:307:sa:en:True}



\subsection{The process of creating 3D-graphics consists of two main steps, which ones?}

\label{q:308:sa:en:False}

\vspace{2cm}

\noindent\makebox[\textwidth]{\hrulefill}

\vspace{1cm}

\textit{Answer}: \autoref{q:308:sa:en:True}



\subsection{What characterizes a local lighting model in computer graphics?}

\label{q:309:sa:en:False}

\vspace{2cm}

\noindent\makebox[\textwidth]{\hrulefill}

\vspace{1cm}

\textit{Answer}: \autoref{q:309:sa:en:True}



\subsection{What characterizes a global lighting model in computer graphics?}

\label{q:310:sa:en:False}

\vspace{2cm}

\noindent\makebox[\textwidth]{\hrulefill}

\vspace{1cm}

\textit{Answer}: \autoref{q:310:sa:en:True}



\subsection{Many difficult problems can be described as search problems, which means that you search for a solution in a search tree. To select the path in the search tree, use "rules of thumb". What are such rules of thumb called and why are they needed?}

\label{q:311:sa:en:False}

\vspace{2cm}

\noindent\makebox[\textwidth]{\hrulefill}

\vspace{1cm}

\textit{Answer}: \autoref{q:311:sa:en:True}



\subsection{What is the difference between weak AI and strong AI?}

\label{q:312:sa:en:False}

\vspace{2cm}

\noindent\makebox[\textwidth]{\hrulefill}

\vspace{1cm}

\textit{Answer}: \autoref{q:312:sa:en:True}



\subsection{One way to classify machine / computer learning approaches is through the degree to which they require human intervention. What three such classes are we usually talking about?}

\label{q:313:sa:en:False}

\vspace{2cm}

\noindent\makebox[\textwidth]{\hrulefill}

\vspace{1cm}

\textit{Answer}: \autoref{q:313:sa:en:True}



\subsection{What is an artificial neural network and how does such a network change during learning?}

\label{q:314:sa:en:False}

\vspace{2cm}

\noindent\makebox[\textwidth]{\hrulefill}

\vspace{1cm}

\textit{Answer}: \autoref{q:314:sa:en:True}



\subsection{What is the difference between supervised learning and unsupervised learning?}

\label{q:315:sa:en:False}

\vspace{2cm}

\noindent\makebox[\textwidth]{\hrulefill}

\vspace{1cm}

\textit{Answer}: \autoref{q:315:sa:en:True}



\subsection{Is reinforcement learning a type of supervised learning or not? Why?}

\label{q:316:sa:en:False}

\vspace{2cm}

\noindent\makebox[\textwidth]{\hrulefill}

\vspace{1cm}

\textit{Answer}: \autoref{q:316:sa:en:True}



\subsection{A neural network is a computational model inspired by how the human brain works. How does a neural network learn from sample data?}

\label{q:317:sa:en:False}

\vspace{2cm}

\noindent\makebox[\textwidth]{\hrulefill}

\vspace{1cm}

\textit{Answer}: \autoref{q:317:sa:en:True}



\subsection{Briefly explain the concepts "information retrieval" and "information extraction" in language analysis (natural language processing)!}

\label{q:318:sa:en:False}

\vspace{2cm}

\noindent\makebox[\textwidth]{\hrulefill}

\vspace{1cm}

\textit{Answer}: \autoref{q:318:sa:en:True}



\subsection{What three types of layers are there in a neural network's topology?}

\label{q:319:sa:en:False}

\vspace{2cm}

\noindent\makebox[\textwidth]{\hrulefill}

\vspace{1cm}

\textit{Answer}: \autoref{q:319:sa:en:True}



\subsection{What is a search tree in the context of AI?}

\label{q:320:sa:en:False}

\vspace{2cm}

\noindent\makebox[\textwidth]{\hrulefill}

\vspace{1cm}

\textit{Answer}: \autoref{q:320:sa:en:True}



\subsection{In natural language processing, three different types of analysis are performed, which?}

\label{q:321:sa:en:False}

\vspace{2cm}

\noindent\makebox[\textwidth]{\hrulefill}

\vspace{1cm}

\textit{Answer}: \autoref{q:321:sa:en:True}



\subsection{What is the Turing test?}

\label{q:322:sa:en:False}

\vspace{2cm}

\noindent\makebox[\textwidth]{\hrulefill}

\vspace{1cm}

\textit{Answer}: \autoref{q:322:sa:en:True}



\subsection{What is the definition of an intelligent agent (in AI)?}

\label{q:323:sa:en:False}

\vspace{2cm}

\noindent\makebox[\textwidth]{\hrulefill}

\vspace{1cm}

\textit{Answer}: \autoref{q:323:sa:en:True}



\subsection{What characterizes supervised (machine) learning?}

\label{q:324:sa:en:False}

\vspace{2cm}

\noindent\makebox[\textwidth]{\hrulefill}

\vspace{1cm}

\textit{Answer}: \autoref{q:324:sa:en:True}



\subsection{What characterizes (machine) learning by reinforcement?}

\label{q:325:sa:en:False}

\vspace{2cm}

\noindent\makebox[\textwidth]{\hrulefill}

\vspace{1cm}

\textit{Answer}: \autoref{q:325:sa:en:True}



\subsection{What is the definition of an intelligent agent?}

\label{q:326:sa:en:False}

\vspace{2cm}

\noindent\makebox[\textwidth]{\hrulefill}

\vspace{1cm}

\textit{Answer}: \autoref{q:326:sa:en:True}



\subsection{What is the definition of an intelligent agent?}

\label{q:327:sa:en:False}

\vspace{2cm}

\noindent\makebox[\textwidth]{\hrulefill}

\vspace{1cm}

\textit{Answer}: \autoref{q:327:sa:en:True}



\subsection{What are search heuristics, and what characterizes good search heuristics?}

\label{q:328:sa:en:False}

\vspace{2cm}

\noindent\makebox[\textwidth]{\hrulefill}

\vspace{1cm}

\textit{Answer}: \autoref{q:328:sa:en:True}



\subsection{Why is search heuristics needed when searching in a search tree?}

\label{q:329:sa:en:False}

\vspace{2cm}

\noindent\makebox[\textwidth]{\hrulefill}

\vspace{1cm}

\textit{Answer}: \autoref{q:329:sa:en:True}



\subsection{What is the difference between a state graph and a search tree?}

\label{q:330:sa:en:False}

\vspace{2cm}

\noindent\makebox[\textwidth]{\hrulefill}

\vspace{1cm}

\textit{Answer}: \autoref{q:330:sa:en:True}



\subsection{What is the halting problem (stopp-problemet), and why is it interesting from a computational theoretical perspective?}

\label{q:331:sa:en:False}

\vspace{2cm}

\noindent\makebox[\textwidth]{\hrulefill}

\vspace{1cm}

\textit{Answer}: \autoref{q:331:sa:en:True}



\subsection{Arrange the following complexity/efficiency classes from most efficient to least efficient: \ensuremath{\Theta}(n^10), \ensuremath{\Theta}(log n), \ensuremath{\Theta}(n), \ensuremath{\Theta}(2^n).}

\label{q:332:sa:en:False}

\vspace{2cm}

\noindent\makebox[\textwidth]{\hrulefill}

\vspace{1cm}

\textit{Answer}: \autoref{q:332:sa:en:True}



\subsection{What is a Turing machine and what is its purpose?}

\label{q:333:sa:en:False}

\vspace{2cm}

\noindent\makebox[\textwidth]{\hrulefill}

\vspace{1cm}

\textit{Answer}: \autoref{q:333:sa:en:True}



\subsection{Arrange the following complexity/efficiency classes from most efficient to least efficient: \ensuremath{\Theta}(n^4), \ensuremath{\Theta}(n), \ensuremath{\Theta}(2^n), \ensuremath{\Theta}(log n).}

\label{q:334:sa:en:False}

\vspace{2cm}

\noindent\makebox[\textwidth]{\hrulefill}

\vspace{1cm}

\textit{Answer}: \autoref{q:334:sa:en:True}



\subsection{What does it mean that a problem is a polynomial problem (belongs to the class of polynomial problems)?}

\label{q:335:sa:en:False}

\vspace{2cm}

\noindent\makebox[\textwidth]{\hrulefill}

\vspace{1cm}

\textit{Answer}: \autoref{q:335:sa:en:True}



\subsection{Is the class of polynomial problems P less than or equal to the class of nondeterministic polynomial problems NP? Motivate your answer!}

\label{q:336:sa:en:False}

\vspace{2cm}

\noindent\makebox[\textwidth]{\hrulefill}

\vspace{1cm}

\textit{Answer}: \autoref{q:336:sa:en:True}



\subsection{Given that the complexity of Algorithm A is O(n), Algorithm B is O(log n), Algorithm C is O(n2), and Algorithm D is O(n log n2), list the algorithms in order from most to least efficient efficient!}

\label{q:337:sa:en:False}

\vspace{2cm}

\noindent\makebox[\textwidth]{\hrulefill}

\vspace{1cm}

\textit{Answer}: \autoref{q:337:sa:en:True}



\subsection{Give examples of three complexity classes in O-notation and order these from most efficient to least efficient!}

\label{q:338:sa:en:False}

\vspace{2cm}

\noindent\makebox[\textwidth]{\hrulefill}

\vspace{1cm}

\textit{Answer}: \autoref{q:338:sa:en:True}



\subsection{Why is the halting problem interesting from a computational theoretical perspective?}

\label{q:339:sa:en:False}

\vspace{2cm}

\noindent\makebox[\textwidth]{\hrulefill}

\vspace{1cm}

\textit{Answer}: \autoref{q:339:sa:en:True}



\subsection{What characterizes the halting problem (in theory of computation).}

\label{q:340:sa:en:False}

\vspace{2cm}

\noindent\makebox[\textwidth]{\hrulefill}

\vspace{1cm}

\textit{Answer}: \autoref{q:340:sa:en:True}



\subsection{What does it mean that a function is computable?}

\label{q:341:sa:en:False}

\vspace{2cm}

\noindent\makebox[\textwidth]{\hrulefill}

\vspace{1cm}

\textit{Answer}: \autoref{q:341:sa:en:True}



\subsection{Do non-deterministic algorithms meet the definition of an algorithm? Motivate your answer!}

\label{q:342:sa:en:False}

\vspace{2cm}

\noindent\makebox[\textwidth]{\hrulefill}

\vspace{1cm}

\textit{Answer}: \autoref{q:342:sa:en:True}



\subsection{What do we know about the relation between polynomial problems P and non-deterministic polynomial problems NP?}

\label{q:343:sa:en:False}

\vspace{2cm}

\noindent\makebox[\textwidth]{\hrulefill}

\vspace{1cm}

\textit{Answer}: \autoref{q:343:sa:en:True}



\subsection{What distinguishes a deterministic algorithm from a non-deterministic one?}

\label{q:344:sa:en:False}

\vspace{2cm}

\noindent\makebox[\textwidth]{\hrulefill}

\vspace{1cm}

\textit{Answer}: \autoref{q:344:sa:en:True}



\subsection{What is the purpose of Turing machines?}

\label{q:345:sa:en:False}

\vspace{2cm}

\noindent\makebox[\textwidth]{\hrulefill}

\vspace{1cm}

\textit{Answer}: \autoref{q:345:sa:en:True}



\subsection{In what two ways can a transaction be terminated?}

\label{q:346:sa:en:False}

\vspace{2cm}

\noindent\makebox[\textwidth]{\hrulefill}

\vspace{1cm}

\textit{Answer}: \autoref{q:346:sa:en:True}



\subsection{Suppose we have the following bit patterns and that they represent integers in two's complement notation: "0111 1111, 1111 1001, 1011 1111, 0010 0100, 1000 0001" - Which of these bit patterns represents the smallest integer?}

\label{q:34800:mc:en:False}

\begin{itemize}
  \item[$\bigcirc$] 0111 1111
  \item[$\bigcirc$] 1111 1001
  \item[$\bigcirc$] 1011 1111
  \item[$\bigcirc$] 1000 0001
\end{itemize}

\vspace{1cm}

\textit{Answer}: \autoref{q:34800:mc:en:True}

\subsection{Suppose we have the following bit patterns and that they represent integers in two's complement notation: "0111 1111, 1111 1001, 1011 1111, 0010 0100, 1000 0001" - Which of these bit patterns represents the largest integer?}

\label{q:3480001:mc:en:False}

\begin{itemize}
  \item[$\bigcirc$] 1111 1001
  \item[$\bigcirc$] 1011 1111
  \item[$\bigcirc$] 1000 0001
  \item[$\bigcirc$] 0010 0100
\end{itemize}

\vspace{1cm}

\textit{Answer}: \autoref{q:3480001:mc:en:True}



\subsection{Assume that 00FF00 is the hexadecimal notation for a bit pattern representing a pixel according to the RGB standard. - What color depth does this pixel have?}

\label{q:34900:sa:en:False}

\vspace{2cm}

\noindent\makebox[\textwidth]{\hrulefill}

\vspace{1cm}

\textit{Answer}: \autoref{q:34900:sa:en:True}

\subsection{Assume that 00FF00 is the hexadecimal notation for a bit pattern representing a pixel according to the RGB standard. - Which of the following colors is that pixel?}

\label{q:3490001:mc:en:False}

\begin{itemize}
  \item[$\bigcirc$] White
  \item[$\bigcirc$] Black
  \item[$\bigcirc$] Red
  \item[$\bigcirc$] Blue
  \item[$\bigcirc$] Yellow
  \item[$\bigcirc$] Cyan
  \item[$\bigcirc$] Magenta
\end{itemize}

\vspace{1cm}

\textit{Answer}: \autoref{q:3490001:mc:en:True}



\subsection{Suppose we have the following bit patterns and that they represent integers in two's complement notation:1111 1110 0111 1111 0000 0000 0000 0001 1000 0000 1111 1111 - Which of these bit patterns represents the number -1 (minus one)?}

\label{q:35000:mc:en:False}

\begin{itemize}
  \item[$\bigcirc$] 1111 1110
  \item[$\bigcirc$] 0111 1111
  \item[$\bigcirc$] 0000 0000
  \item[$\bigcirc$] 0000 0001
  \item[$\bigcirc$] 1000 0000
\end{itemize}

\vspace{1cm}

\textit{Answer}: \autoref{q:35000:mc:en:True}

\subsection{Suppose we have the following bit patterns and that they represent integers in two's complement notation:1111 1110 0111 1111 0000 0000 0000 0001 1000 0000 1111 1111 - Which of these bit patterns represents the number 1 (one)?}

\label{q:3500001:mc:en:False}

\begin{itemize}
  \item[$\bigcirc$] 1111 1110
  \item[$\bigcirc$] 0111 1111
  \item[$\bigcirc$] 0000 0000
  \item[$\bigcirc$] 1000 0000
  \item[$\bigcirc$] 1111 1111
\end{itemize}

\vspace{1cm}

\textit{Answer}: \autoref{q:3500001:mc:en:True}



\subsection{Suppose we have the following bit patterns and that they represent integers in two's complement notation:1111 0100 0111 0101 0000 1010 0000 1011 1000 1010 1111 0101 - Which of these bit patterns represents the largest number?}

\label{q:35100:mc:en:False}

\begin{itemize}
  \item[$\bigcirc$] 0111 0011
  \item[$\bigcirc$] 0111 0001
  \item[$\bigcirc$] 0110 1111
\end{itemize}

\vspace{1cm}

\textit{Answer}: \autoref{q:35100:mc:en:True}

\subsection{Suppose we have the following bit patterns and that they represent integers in two's complement notation:1111 0100 0111 0101 0000 1010 0000 1011 1000 1010 1111 0101 - Which of these bit patterns represents the smallest number?}

\label{q:3510001:mc:en:False}

\begin{itemize}
  \item[$\bigcirc$] 1000 1011
  \item[$\bigcirc$] 1001 0010
  \item[$\bigcirc$] 1011 1101
\end{itemize}

\vspace{1cm}

\textit{Answer}: \autoref{q:3510001:mc:en:True}



\subsection{Suppose we have the following bit pattern: 1000 0011. - What decimal natural number (zero or positive integer) (unsigned integer) does the bit pattern above represent?}

\label{q:35200:sa:en:False}

\vspace{2cm}

\noindent\makebox[\textwidth]{\hrulefill}

\vspace{1cm}

\textit{Answer}: \autoref{q:35200:sa:en:True}

\subsection{Suppose we have the following bit pattern: 1000 0011. - Which decimal integer (signed integer) represents the bit pattern above according to two's complement notation?}

\label{q:3520001:sa:en:False}

\vspace{2cm}

\noindent\makebox[\textwidth]{\hrulefill}

\vspace{1cm}

\textit{Answer}: \autoref{q:3520001:sa:en:True}



\subsection{Suppose we have the following bit patterns and that they represent integers in two's complement notation:0111 0100, 0010 1001, 1100 0010, 1100 0100, 0011 0001 - Which of these bit patterns represents the largest number?}

\label{q:35300:mc:en:False}

\begin{itemize}
  \item[$\bigcirc$] 0010 1001
  \item[$\bigcirc$] 1100 0010
  \item[$\bigcirc$] 1100 0100
  \item[$\bigcirc$] 0011 0001
\end{itemize}

\vspace{1cm}

\textit{Answer}: \autoref{q:35300:mc:en:True}

\subsection{Suppose we have the following bit patterns and that they represent integers in two's complement notation:0111 0100, 0010 1001, 1100 0010, 1100 0100, 0011 0001 - Which of these bit patterns represents the smallest number?}

\label{q:3530001:mc:en:False}

\begin{itemize}
  \item[$\bigcirc$] 0010 1001
  \item[$\bigcirc$] 1100 0010
  \item[$\bigcirc$] 1100 0100
  \item[$\bigcirc$] 0011 0001
\end{itemize}

\vspace{1cm}

\textit{Answer}: \autoref{q:3530001:mc:en:True}



\subsection{Assume that the RGB color code of a pixel is CC3300 in hexadecimal (base 16) form, that the pixel is included in a photo taken with a 6 megapixel camera, and that the photo is stored as a bitmap (ie, uncompressed). - Enter the pixel color values for R (red), G (green) and B (blue) in decimal form (base 10)?}

\label{q:35400:sa:en:False}

\vspace{2cm}

\noindent\makebox[\textwidth]{\hrulefill}

\vspace{1cm}

\textit{Answer}: \autoref{q:35400:sa:en:True}

\subsection{Assume that the RGB color code of a pixel is CC3300 in hexadecimal (base 16) form, that the pixel is included in a photo taken with a 6 megapixel camera, and that the photo is stored as a bitmap (ie, uncompressed). - What is the color depth of the pixel?}

\label{q:3540001:sa:en:False}

\vspace{2cm}

\noindent\makebox[\textwidth]{\hrulefill}

\vspace{1cm}

\textit{Answer}: \autoref{q:3540001:sa:en:True}

\subsection{Assume that the RGB color code of a pixel is CC3300 in hexadecimal (base 16) form, that the pixel is included in a photo taken with a 6 megapixel camera, and that the photo is stored as a bitmap (ie, uncompressed). - How much storage space does the photo take in MB (megabytes)?}

\label{q:354000102:sa:en:False}

\vspace{2cm}

\noindent\makebox[\textwidth]{\hrulefill}

\vspace{1cm}

\textit{Answer}: \autoref{q:354000102:sa:en:True}



\subsection{Suppose we have the following bit patterns: 1010 1010, 1100 1100, 1001 0000 and 1001 1111. - If the bit patterns above represent natural numbers (unsigned integers), then which bit pattern represents the smallest number?}

\label{q:35500:mc:en:False}

\begin{itemize}
  \item[$\bigcirc$] 1010 1010
  \item[$\bigcirc$] 1100 1100
  \item[$\bigcirc$] 1001 1111
\end{itemize}

\vspace{1cm}

\textit{Answer}: \autoref{q:35500:mc:en:True}

\subsection{Suppose we have the following bit patterns: 1010 1010, 1100 1100, 1001 0000 and 1001 1111. - If the bit patterns above represent integers according to two's complement notation, which bit pattern represents the smallest number?}

\label{q:3550001:mc:en:False}

\begin{itemize}
  \item[$\bigcirc$] 1010 1010
  \item[$\bigcirc$] 1100 1100
  \item[$\bigcirc$] 1001 1111
\end{itemize}

\vspace{1cm}

\textit{Answer}: \autoref{q:3550001:mc:en:True}





\subsection{Write the decimal number 9 as a binary number represented by 8 bits (8 bit unsigned integer).}

\label{q:357:sa:en:False}

\vspace{2cm}

\noindent\makebox[\textwidth]{\hrulefill}

\vspace{1cm}

\textit{Answer}: \autoref{q:357:sa:en:True}



\subsection{Write the decimal number -1 (minus one) as a 8-bit bit pattern according to two{\textquoteright}s complement notation.}

\label{q:358:sa:en:False}

\vspace{2cm}

\noindent\makebox[\textwidth]{\hrulefill}

\vspace{1cm}

\textit{Answer}: \autoref{q:358:sa:en:True}



\subsection{Which bit pattern corresponds to the hexadecimal expression 7F?}

\label{q:359:sa:en:False}

\vspace{2cm}

\noindent\makebox[\textwidth]{\hrulefill}

\vspace{1cm}

\textit{Answer}: \autoref{q:359:sa:en:True}



\subsection{Write the decimal number 3 as a binary number represented by 8 bits (8 bit unsigned integer).}

\label{q:360:sa:en:False}

\vspace{2cm}

\noindent\makebox[\textwidth]{\hrulefill}

\vspace{1cm}

\textit{Answer}: \autoref{q:360:sa:en:True}



\subsection{What bit pattern corresponds to the hexidecimal expression AB?}

\label{q:361:sa:en:False}

\vspace{2cm}

\noindent\makebox[\textwidth]{\hrulefill}

\vspace{1cm}

\textit{Answer}: \autoref{q:361:sa:en:True}



\subsection{Which decimal number (base 10) corresponds to the hexadecimal number  A2?}

\label{q:362:sa:en:False}

\vspace{2cm}

\noindent\makebox[\textwidth]{\hrulefill}

\vspace{1cm}

\textit{Answer}: \autoref{q:362:sa:en:True}



\subsection{Write the number -3 (minus three) as an 8-bit bit pattern according to two's complement notation!}

\label{q:363:sa:en:False}

\vspace{2cm}

\noindent\makebox[\textwidth]{\hrulefill}

\vspace{1cm}

\textit{Answer}: \autoref{q:363:sa:en:True}



\subsection{Which bit pattern corresponds to the hexadecimal number 8F?}

\label{q:364:sa:en:False}

\vspace{2cm}

\noindent\makebox[\textwidth]{\hrulefill}

\vspace{1cm}

\textit{Answer}: \autoref{q:364:sa:en:True}



\subsection{Which decimal number (base 10) corresponds to the hexadecimal number  B3 ?}

\label{q:365:sa:en:False}

\vspace{2cm}

\noindent\makebox[\textwidth]{\hrulefill}

\vspace{1cm}

\textit{Answer}: \autoref{q:365:sa:en:True}



\subsection{Write the number 3 (three) as an 8-bit bit pattern according to two's complement notation!}

\label{q:366:sa:en:False}

\vspace{2cm}

\noindent\makebox[\textwidth]{\hrulefill}

\vspace{1cm}

\textit{Answer}: \autoref{q:366:sa:en:True}



\subsection{Describe the number 3 (three) with two characters in hexadecimal form!}

\label{q:367:sa:en:False}

\vspace{2cm}

\noindent\makebox[\textwidth]{\hrulefill}

\vspace{1cm}

\textit{Answer}: \autoref{q:367:sa:en:True}



\subsection{Write the number -4 (minus four) as an 8-bit bit pattern according to two's complement notation!}

\label{q:368:sa:en:False}

\vspace{2cm}

\noindent\makebox[\textwidth]{\hrulefill}

\vspace{1cm}

\textit{Answer}: \autoref{q:368:sa:en:True}



\subsection{Write the number 2 (two) as an 8-bit bit pattern according to two's complement notation!}

\label{q:369:sa:en:False}

\vspace{2cm}

\noindent\makebox[\textwidth]{\hrulefill}

\vspace{1cm}

\textit{Answer}: \autoref{q:369:sa:en:True}



\subsection{Write the number \ensuremath{-}2 (minus two) as an 8-bit bit pattern according to two's complement notation!}

\label{q:370:sa:en:False}

\vspace{2cm}

\noindent\makebox[\textwidth]{\hrulefill}

\vspace{1cm}

\textit{Answer}: \autoref{q:370:sa:en:True}



\subsection{What is the positive decimal integer 127 as a binary number represented by 8 bits according to two's complement notation?}

\label{q:371:sa:en:False}

\vspace{2cm}

\noindent\makebox[\textwidth]{\hrulefill}

\vspace{1cm}

\textit{Answer}: \autoref{q:371:sa:en:True}



\subsection{What is the negative decimal integer \ensuremath{-}127 as a binary number represented by 8 bits according to two's complement notation?}

\label{q:372:sa:en:False}

\vspace{2cm}

\noindent\makebox[\textwidth]{\hrulefill}

\vspace{1cm}

\textit{Answer}: \autoref{q:372:sa:en:True}



\subsection{Which natural decimal number (zero or positive integer) does the bit pattern 1010 1010 represent?}

\label{q:373:sa:en:False}

\vspace{2cm}

\noindent\makebox[\textwidth]{\hrulefill}

\vspace{1cm}

\textit{Answer}: \autoref{q:373:sa:en:True}



\subsection{Which natural decimal number (zero or positive integer) does the bit pattern 1011 1011 represent?}

\label{q:374:sa:en:False}

\vspace{2cm}

\noindent\makebox[\textwidth]{\hrulefill}

\vspace{1cm}

\textit{Answer}: \autoref{q:374:sa:en:True}



\subsection{Which bit pattern corresponds to the hexadecimal number C4?}

\label{q:375:sa:en:False}

\vspace{2cm}

\noindent\makebox[\textwidth]{\hrulefill}

\vspace{1cm}

\textit{Answer}: \autoref{q:375:sa:en:True}



\subsection{Which bit pattern corresponds to the hexadecimal number B3?}

\label{q:376:sa:en:False}

\vspace{2cm}

\noindent\makebox[\textwidth]{\hrulefill}

\vspace{1cm}

\textit{Answer}: \autoref{q:376:sa:en:True}



\subsection{Which decimal integer (signed integer) represents the bit pattern 1010 according to two{\textquoteright}s complement notation?}

\label{q:377:sa:en:False}

\vspace{2cm}

\noindent\makebox[\textwidth]{\hrulefill}

\vspace{1cm}

\textit{Answer}: \autoref{q:377:sa:en:True}



\subsection{Which decimal integer (signed integer) represents the bit pattern 1011 according to two{\textquoteright}s complement notation?}

\label{q:378:sa:en:False}

\vspace{2cm}

\noindent\makebox[\textwidth]{\hrulefill}

\vspace{1cm}

\textit{Answer}: \autoref{q:378:sa:en:True}



\subsection{Which natural decimal number (zero or positive integer) does the bit pattern 1101 1011 represent?}

\label{q:379:sa:en:False}

\vspace{2cm}

\noindent\makebox[\textwidth]{\hrulefill}

\vspace{1cm}

\textit{Answer}: \autoref{q:379:sa:en:True}



\subsection{What bit pattern corresponds to the hexadecimal number D2?}

\label{q:380:sa:en:False}

\vspace{2cm}

\noindent\makebox[\textwidth]{\hrulefill}

\vspace{1cm}

\textit{Answer}: \autoref{q:380:sa:en:True}



\subsection{Which decimal integer (signed integer) represents the bit pattern 1101 according to two{\textquoteright}s complement notation?}

\label{q:381:sa:en:False}

\vspace{2cm}

\noindent\makebox[\textwidth]{\hrulefill}

\vspace{1cm}

\textit{Answer}: \autoref{q:381:sa:en:True}



\subsection{Which bit pattern corresponds to the hexadecimal number A5 ?}

\label{q:382:sa:en:False}

\vspace{2cm}

\noindent\makebox[\textwidth]{\hrulefill}

\vspace{1cm}

\textit{Answer}: \autoref{q:382:sa:en:True}



\subsection{Which bit pattern corresponds to the hexadecimal number B4 ?}

\label{q:383:sa:en:False}

\vspace{2cm}

\noindent\makebox[\textwidth]{\hrulefill}

\vspace{1cm}

\textit{Answer}: \autoref{q:383:sa:en:True}



\subsection{Which decimal number (base 10) corresponds to the hexadecimal number 4D?}

\label{q:384:sa:en:False}

\vspace{2cm}

\noindent\makebox[\textwidth]{\hrulefill}

\vspace{1cm}

\textit{Answer}: \autoref{q:384:sa:en:True}



\subsection{Which decimal number (base 10) corresponds to the hexadecimal number 5C?}

\label{q:385:sa:en:False}

\vspace{2cm}

\noindent\makebox[\textwidth]{\hrulefill}

\vspace{1cm}

\textit{Answer}: \autoref{q:385:sa:en:True}



\subsection{Write the number 4 (four) as an 8-bit bit pattern according to two's complement notation!}

\label{q:386:sa:en:False}

\vspace{2cm}

\noindent\makebox[\textwidth]{\hrulefill}

\vspace{1cm}

\textit{Answer}: \autoref{q:386:sa:en:True}



\subsection{Write the number -5 (minus five) as an 8-bit bit pattern according to two's complement notation!}

\label{q:387:sa:en:False}

\vspace{2cm}

\noindent\makebox[\textwidth]{\hrulefill}

\vspace{1cm}

\textit{Answer}: \autoref{q:387:sa:en:True}



\subsection{Which bit pattern corresponds to the hexadecimal number C3?}

\label{q:388:sa:en:False}

\vspace{2cm}

\noindent\makebox[\textwidth]{\hrulefill}

\vspace{1cm}

\textit{Answer}: \autoref{q:388:sa:en:True}



\subsection{Which bit pattern corresponds to the hexadecimal number 3C?}

\label{q:389:sa:en:False}

\vspace{2cm}

\noindent\makebox[\textwidth]{\hrulefill}

\vspace{1cm}

\textit{Answer}: \autoref{q:389:sa:en:True}



\subsection{Which decimal number (base 10) corresponds to the hexadecimal number 3C?}

\label{q:390:sa:en:False}

\vspace{2cm}

\noindent\makebox[\textwidth]{\hrulefill}

\vspace{1cm}

\textit{Answer}: \autoref{q:390:sa:en:True}



\subsection{Which decimal number (base 10) corresponds to the hexadecimal number C3?}

\label{q:391:sa:en:False}

\vspace{2cm}

\noindent\makebox[\textwidth]{\hrulefill}

\vspace{1cm}

\textit{Answer}: \autoref{q:391:sa:en:True}



\subsection{Write the number -3 (minus three) as an 8-bit bit pattern according to two's complement notation!}

\label{q:392:sa:en:False}

\vspace{2cm}

\noindent\makebox[\textwidth]{\hrulefill}

\vspace{1cm}

\textit{Answer}: \autoref{q:392:sa:en:True}



\subsection{Write the number -4 (minus four) as an 8-bit bit pattern according to two's complement notation!}

\label{q:393:sa:en:False}

\vspace{2cm}

\noindent\makebox[\textwidth]{\hrulefill}

\vspace{1cm}

\textit{Answer}: \autoref{q:393:sa:en:True}



\subsection{What bit pattern corresponds to the hexadecimal number BE?}

\label{q:394:sa:en:False}

\vspace{2cm}

\noindent\makebox[\textwidth]{\hrulefill}

\vspace{1cm}

\textit{Answer}: \autoref{q:394:sa:en:True}



\subsection{What decimal number (base 10) corresponds to the hexadecimal number 2D?}

\label{q:395:sa:en:False}

\vspace{2cm}

\noindent\makebox[\textwidth]{\hrulefill}

\vspace{1cm}

\textit{Answer}: \autoref{q:395:sa:en:True}



\subsection{Which bit pattern corresponds to the hexadecimal number A2?}

\label{q:396:sa:en:False}

\vspace{2cm}

\noindent\makebox[\textwidth]{\hrulefill}

\vspace{1cm}

\textit{Answer}: \autoref{q:396:sa:en:True}



\subsection{Which decimal number (base 10) corresponds to the hexadecimal number D2?}

\label{q:397:sa:en:False}

\vspace{2cm}

\noindent\makebox[\textwidth]{\hrulefill}

\vspace{1cm}

\textit{Answer}: \autoref{q:397:sa:en:True}



\subsection{Which bit pattern corresponds to the hexadecimal number B1?}

\label{q:398:sa:en:False}

\vspace{2cm}

\noindent\makebox[\textwidth]{\hrulefill}

\vspace{1cm}

\textit{Answer}: \autoref{q:398:sa:en:True}



\subsection{Which decimal number (base 10) corresponds to the hexadecimal number  5E ?}

\label{q:399:sa:en:False}

\vspace{2cm}

\noindent\makebox[\textwidth]{\hrulefill}

\vspace{1cm}

\textit{Answer}: \autoref{q:399:sa:en:True}



\subsection{Which bit pattern corresponds to the hexadecimal number 7F?}

\label{q:400:sa:en:False}

\vspace{2cm}

\noindent\makebox[\textwidth]{\hrulefill}

\vspace{1cm}

\textit{Answer}: \autoref{q:400:sa:en:True}



\subsection{Which decimal number (base 10) corresponds to the hexadecimal number  A6?}

\label{q:401:sa:en:False}

\vspace{2cm}

\noindent\makebox[\textwidth]{\hrulefill}

\vspace{1cm}

\textit{Answer}: \autoref{q:401:sa:en:True}



\subsection{Write the number -6 (minus six) as an 8-bit bit pattern according to two's complement notation!}

\label{q:402:sa:en:False}

\vspace{2cm}

\noindent\makebox[\textwidth]{\hrulefill}

\vspace{1cm}

\textit{Answer}: \autoref{q:402:sa:en:True}



\subsection{Suppose we have the following two bit patterns 10000001 and 01111110. What bit pattern do we obtain if we perform the logical AND operation on these bit patterns?}

\label{q:403:sa:en:False}

\vspace{2cm}

\noindent\makebox[\textwidth]{\hrulefill}

\vspace{1cm}

\textit{Answer}: \autoref{q:403:sa:en:True}



\subsection{Suppose we have the following two bit patterns 10000001 and 01111110. What bit pattern do we obtain if we perform the arithmetic operation ADD according to two's complement notation (two's complement notation) on these bit patterns which then represent two integers (signed integers)?}

\label{q:404:sa:en:False}

\vspace{2cm}

\noindent\makebox[\textwidth]{\hrulefill}

\vspace{1cm}

\textit{Answer}: \autoref{q:404:sa:en:True}



\subsection{What bit pattern do we get if we perform the OR operation on the bit patterns 1011 0011 and 0010 0110?}

\label{q:405:sa:en:False}

\vspace{2cm}

\noindent\makebox[\textwidth]{\hrulefill}

\vspace{1cm}

\textit{Answer}: \autoref{q:405:sa:en:True}



\subsection{What bit pattern do we get if we perform the operation XOR on the bit patterns 1011 0011 and 0010 0110?}

\label{q:406:sa:en:False}

\vspace{2cm}

\noindent\makebox[\textwidth]{\hrulefill}

\vspace{1cm}

\textit{Answer}: \autoref{q:406:sa:en:True}



\subsection{What bit pattern do we get if we perform the operation AND on the bit patterns 1001 1011 och 1000 1110?}

\label{q:407:sa:en:False}

\vspace{2cm}

\noindent\makebox[\textwidth]{\hrulefill}

\vspace{1cm}

\textit{Answer}: \autoref{q:407:sa:en:True}



\subsection{What bit pattern do we get if we perform the operation OR on the bit patterns 1001 1011 och 10001 110?}

\label{q:408:sa:en:False}

\vspace{2cm}

\noindent\makebox[\textwidth]{\hrulefill}

\vspace{1cm}

\textit{Answer}: \autoref{q:408:sa:en:True}



\subsection{What will be the result of the logical operation AND with the bit patterns 10100101 and 01111110? Enter the answer as a bit pattern.}

\label{q:409:sa:en:False}

\vspace{2cm}

\noindent\makebox[\textwidth]{\hrulefill}

\vspace{1cm}

\textit{Answer}: \autoref{q:409:sa:en:True}



\subsection{What will be the result of the logical operation XOR with the bit patterns 10100101 and 01111110? Enter the answer as a bit pattern.}

\label{q:410:sa:en:False}

\vspace{2cm}

\noindent\makebox[\textwidth]{\hrulefill}

\vspace{1cm}

\textit{Answer}: \autoref{q:410:sa:en:True}



\subsection{What will be the result of the logical operation XOR with the bit patterns 10100001 and 01101010? Enter the answer as a bit pattern.}

\label{q:411:sa:en:False}

\vspace{2cm}

\noindent\makebox[\textwidth]{\hrulefill}

\vspace{1cm}

\textit{Answer}: \autoref{q:411:sa:en:True}



\subsection{What bit pattern do we get if we perform the operation OR on the bit patterns 110011 and 101000?}

\label{q:412:sa:en:False}

\vspace{2cm}

\noindent\makebox[\textwidth]{\hrulefill}

\vspace{1cm}

\textit{Answer}: \autoref{q:412:sa:en:True}



\subsection{What bit pattern do we get if we perform the operation XOR on the bit patterns 0110 0011 and 0101 0000?}

\label{q:413:sa:en:False}

\vspace{2cm}

\noindent\makebox[\textwidth]{\hrulefill}

\vspace{1cm}

\textit{Answer}: \autoref{q:413:sa:en:True}



\subsection{What bit pattern do we get if we perform the operation OR on the bit patterns 101011 and 010011?}

\label{q:414:sa:en:False}

\vspace{2cm}

\noindent\makebox[\textwidth]{\hrulefill}

\vspace{1cm}

\textit{Answer}: \autoref{q:414:sa:en:True}



\subsection{What bit pattern do we get if we perform the operation AND on the bit patterns 110011 and 101001?}

\label{q:415:sa:en:False}

\vspace{2cm}

\noindent\makebox[\textwidth]{\hrulefill}

\vspace{1cm}

\textit{Answer}: \autoref{q:415:sa:en:True}



\subsection{What bit pattern do we obtain if we perform the XOR operation on the bit patterns 110011 and 101001?}

\label{q:416:sa:en:False}

\vspace{2cm}

\noindent\makebox[\textwidth]{\hrulefill}

\vspace{1cm}

\textit{Answer}: \autoref{q:416:sa:en:True}



\subsection{What bit pattern do we get if we perform the operation AND on the bit patterns 1101 1101 and 1111 1001?}

\label{q:417:sa:en:False}

\vspace{2cm}

\noindent\makebox[\textwidth]{\hrulefill}

\vspace{1cm}

\textit{Answer}: \autoref{q:417:sa:en:True}



\subsection{What bit pattern do we get if we perform the operation XOR on the bit patterns 0101 0101 and 1000 1100?}

\label{q:418:sa:en:False}

\vspace{2cm}

\noindent\makebox[\textwidth]{\hrulefill}

\vspace{1cm}

\textit{Answer}: \autoref{q:418:sa:en:True}



\subsection{What bit pattern do we get if we perform the operation OR on the bit patterns 01001000 and 10011001?}

\label{q:419:sa:en:False}

\vspace{2cm}

\noindent\makebox[\textwidth]{\hrulefill}

\vspace{1cm}

\textit{Answer}: \autoref{q:419:sa:en:True}



\subsection{What is the minimum number of times the statements in a loop body are executed in an iteration with pre-test conditions?}

\label{q:420:sa:en:False}

\vspace{2cm}

\noindent\makebox[\textwidth]{\hrulefill}

\vspace{1cm}

\textit{Answer}: \autoref{q:420:sa:en:True}



\subsection{Which bit pattern corresponds to the hexadecimal number B7?}

\label{q:421:sa:en:False}

\vspace{2cm}

\noindent\makebox[\textwidth]{\hrulefill}

\vspace{1cm}

\textit{Answer}: \autoref{q:421:sa:en:True}



\subsection{Which bit pattern corresponds to the hexadecimal number C1?}

\label{q:422:sa:en:False}

\vspace{2cm}

\noindent\makebox[\textwidth]{\hrulefill}

\vspace{1cm}

\textit{Answer}: \autoref{q:422:sa:en:True}



\subsection{Which bit pattern corresponds to the hexadecimal number E3?}

\label{q:423:sa:en:False}

\vspace{2cm}

\noindent\makebox[\textwidth]{\hrulefill}

\vspace{1cm}

\textit{Answer}: \autoref{q:423:sa:en:True}



\subsection{Which hexadecimal number corresponds to the bit pattern 10010101?}

\label{q:424:sa:en:False}

\vspace{2cm}

\noindent\makebox[\textwidth]{\hrulefill}

\vspace{1cm}

\textit{Answer}: \autoref{q:424:sa:en:True}



\subsection{The color magenta is a mixture of maximum red and maximum blue. Which bit pattern represents a magenta pixel encoded according to the RGB standard with a bit depth of 24 bits / pixel? Enter the answer in hexadecimal notation.}

\label{q:425:sa:en:False}

\vspace{2cm}

\noindent\makebox[\textwidth]{\hrulefill}

\vspace{1cm}

\textit{Answer}: \autoref{q:425:sa:en:True}



\subsection{The color yellow is a mixture of maximum red and maximum green. Which bit pattern represents a yellow pixel encoded according to the RGB standard with a bit depth of 24 bits / pixel? Enter the answer in hexadecimal notation.}

\label{q:426:sa:en:False}

\vspace{2cm}

\noindent\makebox[\textwidth]{\hrulefill}

\vspace{1cm}

\textit{Answer}: \autoref{q:426:sa:en:True}



\subsection{Which hexadecimal number corresponds to the bit pattern 11100101?}

\label{q:427:sa:en:False}

\vspace{2cm}

\noindent\makebox[\textwidth]{\hrulefill}

\vspace{1cm}

\textit{Answer}: \autoref{q:427:sa:en:True}



\subsection{Which hexadecimal number corresponds to the bit pattern 10101101?}

\label{q:428:sa:en:False}

\vspace{2cm}

\noindent\makebox[\textwidth]{\hrulefill}

\vspace{1cm}

\textit{Answer}: \autoref{q:428:sa:en:True}



\subsection{Which hexadecimal number corresponds to the bit pattern 11010100?}

\label{q:429:sa:en:False}

\vspace{2cm}

\noindent\makebox[\textwidth]{\hrulefill}

\vspace{1cm}

\textit{Answer}: \autoref{q:429:sa:en:True}



\subsection{What value will register 0 (R0) have after four (4) machine cycles? Enter the bit pattern in hexadecimal form.}

\label{q:430:sa:en:False}

\vspace{2cm}

\noindent\makebox[\textwidth]{\hrulefill}

\vspace{1cm}

\textit{Answer}: \autoref{q:430:sa:en:True}



\subsection{What value will register 1 (R1) have after four (4) machine cycles? Enter the bit pattern in hexadecimal form.}

\label{q:431:sa:en:False}

\vspace{2cm}

\noindent\makebox[\textwidth]{\hrulefill}

\vspace{1cm}

\textit{Answer}: \autoref{q:431:sa:en:True}



\subsection{What value will register 2 (R2) have after four (4) machine cycles? Enter the bit pattern in hexadecimal form.}

\label{q:432:sa:en:False}

\vspace{2cm}

\noindent\makebox[\textwidth]{\hrulefill}

\vspace{1cm}

\textit{Answer}: \autoref{q:432:sa:en:True}



\subsection{What value will the program counter have after three (3) machine cycles? Enter the bit pattern in hexadecimal form.}

\label{q:433:sa:en:False}

\vspace{2cm}

\noindent\makebox[\textwidth]{\hrulefill}

\vspace{1cm}

\textit{Answer}: \autoref{q:433:sa:en:True}



\subsection{Who wrote programs for "the Analytical Engine" and can thereby be regarded as the first programmer in the world?}

\label{q:434:mc:en:False}

\begin{itemize}
  \item[$\bigcirc$] Charles Babbage, Joseph Marie Jacquard, Alonzo Church, Kurt G\"odel, John von Neumann, Blaise Pascal, Alan Turing
\end{itemize}

\vspace{1cm}

\textit{Answer}: \autoref{q:434:mc:en:True}



\subsection{Who designed "the Analytical Engine" - the first programmable computational machine in the world?}

\label{q:435:mc:en:False}

\begin{itemize}
  \item[$\bigcirc$] Joseph Marie Jacquard, Ada Byron (Lovelace), Alonzo Church, Kurt G\"odel, John von Neumann, Blaise Pascal, Alan Turing
\end{itemize}

\vspace{1cm}

\textit{Answer}: \autoref{q:435:mc:en:True}



\subsection{Who was the first to use punch cards (used to store fabric patterns for automatic looms)?}

\label{q:436:mc:en:False}

\begin{itemize}
  \item[$\bigcirc$] Charles Babbage, Ada Byron (Lovelace), Alonzo Church, Kurt G\"odel, John von Neumann, Blaise Pascal, Alan Turing
\end{itemize}

\vspace{1cm}

\textit{Answer}: \autoref{q:436:mc:en:True}



\subsection{Who developed the first gear-based machine to perform addition?}

\label{q:437:mc:en:False}

\begin{itemize}
  \item[$\bigcirc$] Charles Babbage, Joseph Marie Jacquard, Ada Byron (Lovelace), Alonzo Church, Kurt G\"odel, John von Neumann, Alan Turing
\end{itemize}

\vspace{1cm}

\textit{Answer}: \autoref{q:437:mc:en:True}



\subsection{Who has published an incompleteness theorem that says that in all mathematical theories that encompass our traditional arithmetic system, there are statements whose truth or falsity cannot be determined using an algorithm?}

\label{q:438:mc:en:False}

\begin{itemize}
  \item[$\bigcirc$] Alan Turing, Blaise Pascal, Alonzo Church, Charles Babbage, Tim Berners-Lee, Ada Byron (Lovelace), Joseph Jacquard
\end{itemize}

\vspace{1cm}

\textit{Answer}: \autoref{q:438:mc:en:True}



\subsection{Who proposed a system by which documents stored on computers throughout the Internet could be linked together producing a web of linked information (the World Wide Web)?}

\label{q:439:mc:en:False}

\begin{itemize}
  \item[$\bigcirc$] Alan Turing, Blaise Pascal, Alonzo Church, Kurt G\"odel, Charles Babbage, Ada Byron (Lovelace), Joseph Jacquard
\end{itemize}

\vspace{1cm}

\textit{Answer}: \autoref{q:439:mc:en:True}



\subsection{Who gave rise to the name of the computer architecture where the CPU retrieves instructions from memory over a central bus?}

\label{q:440:mc:en:False}

\begin{itemize}
  \item[$\bigcirc$] Charles Babbage, Joseph Marie Jacquard, Ada Byron (Lovelace), Alonzo Church, Kurt G\"odel, Blaise Pascal, Alan Turing
\end{itemize}

\vspace{1cm}

\textit{Answer}: \autoref{q:440:mc:en:True}



\subsection{Who gave rise to the name of the mathematical model of a computer used in study of the power of algorithmic processing?}

\label{q:441:mc:en:False}

\begin{itemize}
  \item[$\bigcirc$] Charles Babbage, Joseph Marie Jacquard, Ada Byron (Lovelace), Alonzo Church, Kurt G\"odel, John von Neumann, Blaise Pascal
\end{itemize}

\vspace{1cm}

\textit{Answer}: \autoref{q:441:mc:en:True}



\subsection{The thesis that the functions that can be calculated by a Turing machine are the same as all computable functions, is named after Turing and another mathematician who contributed to the thesis, which one?  Charles Babbage, Joseph Marie Jacquard, Ada Byron (Lovelace), Alonzo Church, Kurt G\"odel, John von Neumann, Blaise Pascal, Tim Berners-Lee}

\label{q:442:mc:en:False}

\begin{itemize}
  \item[$\bigcirc$] Charles Babbage, Joseph Marie Jacquard, Ada Byron (Lovelace), Kurt G\"odel, John von Neumann, Blaise Pascal, Tim Berners-Lee
\end{itemize}

\vspace{1cm}

\textit{Answer}: \autoref{q:442:mc:en:True}



\subsection{Which bit pattern corresponds to the hexadecimal number D5?}

\label{q:443:mc:en:False}

\begin{itemize}
  \item[$\bigcirc$] 1010 0010, 1111 0101, 1101 0101
\end{itemize}

\vspace{1cm}

\textit{Answer}: \autoref{q:443:mc:en:True}



\subsection{Which hexadecimal number corresponds to the bit pattern 10001111?}

\label{q:444:mc:en:False}

\begin{itemize}
  \item[$\bigcirc$] 8F, 7F, 8C
\end{itemize}

\vspace{1cm}

\textit{Answer}: \autoref{q:444:mc:en:True}

\section{English with Answers}
\label{englishWithAnswers}

\subsection{Which data storage technology was first used in 1801 by Joseph Jacquard?}

\label{q:3:sa:en:True}

\textbf{Answer}: Punch cards



\subsection{Which special purpose register contains the memory address to the next instruction?}

\label{q:4:sa:en:True}

\textbf{Answer}: Program Counter



\subsection{Which special purpose register contains the next machine instruction to be executed?}

\label{q:5:sa:en:True}

\textbf{Answer}: Instruction register



\subsection{Which processor architecture has few, simple and fast machine instructions?}

\label{q:6:sa:en:True}

\textbf{Answer}: RISC



\subsection{There is a special type of machine instruction that is needed to be able to coordinate different processes' access to common resources, what is it called?}

\label{q:7:sa:en:True}

\textbf{Answer}: Test-and-set (or compare-and-swap)



\subsection{What is the part of the operating system that maintains a process table called?}

\label{q:8:sa:en:True}

\textbf{Answer}: Scheduler



\subsection{What is it called when a single user in a single-user system can execute several programs "simultaneously"?}

\label{q:10:sa:en:True}

\textbf{Answer}: Multitasking



\subsection{Which of the following options is not part of the operating system?}

\label{q:11:sa:en:True}

\textbf{Answer}: Control unit



\subsection{What is a flag called that controls access to a critical region to ensure that several processes do not access the critical region at the same time (mutual exclusion)?}

\label{q:12:sa:en:True}

\textbf{Answer}: Semaphore



\subsection{What is the part of the operating system that handles data stored as named units (named separate groups of data) on secondary memory called?}

\label{q:13:sa:en:True}

\textbf{Answer}: File manager



\subsection{A computer can simulate that it has more primary memory than its actual physical primary memory. What is this simulated memory called?}

\label{q:14:sa:en:True}

\textbf{Answer}: Virtual memory



\subsection{What is the special process needed to start a computer called?}

\label{q:15:sa:en:True}

\textbf{Answer}: Boot strapping (booting)



\subsection{What is called the part of the operating system that allocates and deallocates main memory (main memory) to different processes?}

\label{q:16:sa:en:True}

\textbf{Answer}: Memory manager



\subsection{What is the part of the operating system that allocates time slices to different processes called?}

\label{q:17:sa:en:True}

\textbf{Answer}: Dispatcher



\subsection{What is the name of the protocol used by the World Wide Web application?}

\label{q:18:sa:en:True}

\textbf{Answer}: Hypertext Transfer Protocol (HTTP)



\subsection{To which Internet software layer does the UDP (user datagram protocol) belong?}

\label{q:19:sa:en:True}

\textbf{Answer}: Transport layer



\subsection{To which Internet software layer does the FTP (file transfer protocol) belong?}

\label{q:20:sa:en:True}

\textbf{Answer}: Application layer



\subsection{What is a LAN?}

\label{q:21:sa:en:True}

\textbf{Answer}: Local Area Network



\subsection{To which Internet software layer does the TCP (transmission control protocol) belong?}

\label{q:22:sa:en:True}

\textbf{Answer}: Transport layer



\subsection{What is the name of the organization responsible for assigning IP numbers (the abbreviation will suffice)?}

\label{q:23:sa:en:True}

\textbf{Answer}: ICANN (Internet Corporation for Assigned Names and Numbers)



\subsection{To which Internet software layer (Internet software layer) does the IPv6 protocol belong?}

\label{q:24:sa:en:True}

\textbf{Answer}: Network layer



\subsection{What is it called when a web client asks a specific type of server to translate a domain name into an IP number?}

\label{q:25:sa:en:True}

\textbf{Answer}: DNS-lookup



\subsection{Which Internet protocol for the transport layer is most reliable?}

\label{q:26:sa:en:True}

\textbf{Answer}: Transmission Control Protocol (TCP)



\subsection{Name the language used to create web pages with?}

\label{q:27:sa:en:True}

\textbf{Answer}: HyperText Markup Language (HTML)



\subsection{What is the system called that can be used to inspect, filter and block incoming and outgoid network traffic?}

\label{q:28:sa:en:True}

\textbf{Answer}: Firewall



\subsection{What is the system called that is a software unit that acts as an intermediary between a client and a server with the goal of shielding the client from adverse actions of the server?}

\label{q:29:sa:en:True}

\textbf{Answer}: Proxy server



\subsection{What is the name of the way to achieve repetition in code that requires more space in the primary memory?}

\label{q:31:sa:en:True}

\textbf{Answer}: Recursion



\subsection{What is the name of the way to achieve repetition in code that does not require more space in the primary memory?}

\label{q:32:sa:en:True}

\textbf{Answer}: Iteration



\subsection{What is the most common method of verifying that a program is working correctly?}

\label{q:33:sa:en:True}

\textbf{Answer}: Testing



\subsection{What is the basic building block in imperative programming languages called?}

\label{q:34:sa:en:True}

\textbf{Answer}: Procedure



\subsection{What is the logic derivation technique used in logic programming called?}

\label{q:35:sa:en:True}

\textbf{Answer}: Resolution



\subsection{In object oriented programming, what are the templates from which objects are constructed called?}

\label{q:36:sa:en:True}

\textbf{Answer}: Class



\subsection{What is the programming paradigm called where you describe what should be done instead of how it should be done?}

\label{q:37:sa:en:True}

\textbf{Answer}: Declarative Programming Language



\subsection{What is a program that translates source code into machine code called?}

\label{q:38:sa:en:True}

\textbf{Answer}: Compiler



\subsection{What is the type of programming called that answers questions whether a fact is deducible from a program or not?}

\label{q:39:sa:en:True}

\textbf{Answer}: Logic programming language



\subsection{Give an example of an agile development model?}

\label{q:40:sa:en:True}

\textbf{Answer}: Scrum



\subsection{What is the name of the role in Scrum that maintains a list of requirements and prioritizes between these requirements?}

\label{q:41:sa:en:True}

\textbf{Answer}: Product owner



\subsection{What are the short iterations (2 {\textendash} 4 weeks) called in Scrum, which should result in something deliverable to the customer/orderer?}

\label{q:42:sa:en:True}

\textbf{Answer}: Sprint



\subsection{What is called the role in Scrum that must ensure that the Scrum framework is followed?}

\label{q:43:sa:en:True}

\textbf{Answer}: Scrum master



\subsection{What are the short daily meetings called in Scrum when each project participant should answer three questions?}

\label{q:44:sa:en:True}

\textbf{Answer}: Daily scrum (stand-up)



\subsection{What are the meetings called in Scrum when you discuss what has gone well this iteration and what can be improved in the next iteration?}

\label{q:45:sa:en:True}

\textbf{Answer}: Sprint retrospective



\subsection{What is the role of a team that is responsible for the team following the Scrum methodology called?}

\label{q:46:sa:en:True}

\textbf{Answer}: Scrum master



\subsection{What does the abbreviation CASE stand for in terms of software engineering?}

\label{q:47:sa:en:True}

\textbf{Answer}: Computer Aided Software Engineering (CASE)



\subsection{What does the abbreviation IDE stand for in terms of software engineering?}

\label{q:48:sa:en:True}

\textbf{Answer}: Integrated Development Environment (IDE)



\subsection{What is the name of the role in Scrum that is responsible for prioritizing which development is to be carried out during the next sprint?}

\label{q:49:sa:en:True}

\textbf{Answer}: Product owner



\subsection{What is the Scrum meeting, at the end of a sprint where the completed work of the sprint is evaluated with respect to the sprint goals, called?}

\label{q:50:sa:en:True}

\textbf{Answer}: Sprint review



\subsection{What is the basic data structure called, which consists of a block of data elements of the same data type and size, and where each data element is directly accessed via an index?}

\label{q:51:sa:en:True}

\textbf{Answer}: Array



\subsection{What is the basic data structure called that consists of a block of data elements of usually different data types and sizes, and where the individual data elements are accessed by name?}

\label{q:52:sa:en:True}

\textbf{Answer}: Aggregate type



\subsection{What is a variable that contains a memory address instead of data called (used in dynamic data structures)?}

\label{q:53:sa:en:True}

\textbf{Answer}: Pointer



\subsection{What is the name of the dominant query language used to retrieve data from and manipulate data in a database?}

\label{q:54:sa:en:True}

\textbf{Answer}: Structured Query Language (SQL)



\subsection{What is called in a database context, a sequence of operations that are packaged together and where either all operations succeed (performed) or all fail (no one performs) (all operations together either succeed or fail)?}

\label{q:55:sa:en:True}

\textbf{Answer}: Transaction



\subsection{What is the type of data mining that has made the webshop Amazon so successful called?}

\label{q:56:sa:en:True}

\textbf{Answer}: Association analysis



\subsection{What is the type of analysis within data-mining called, which seeks to discover classes by grouping objects into a number of separate groups (in contrast to class description, which seeks to discover properties of members within classes that are already identified)?}

\label{q:57:sa:en:True}

\textbf{Answer}: Cluster analysis



\subsection{What is the type of analysis within data-mining called, which tries to identify patterns of behavior over time, for example trends in economic systems such as equity markets or in environmental systems such as climate conditions?}

\label{q:58:sa:en:True}

\textbf{Answer}: Sequential pattern analysis



\subsection{When rendering, a three-dimensional model must be transferred to a flat surface. What is this flat surface called?}

\label{q:59:sa:en:True}

\textbf{Answer}: Projection plane



\subsection{When rendering 3D graphics, a three-dimensional model must be transferred to a flat surface, what is this flat surface called?}

\label{q:60:sa:en:True}

\textbf{Answer}: Projection plane



\subsection{What is it called when applying the laws of physics to determine the positions of objects, e.g. positions of pool balls after a pool stroke?}

\label{q:61:sa:en:True}

\textbf{Answer}: Dynamics



\subsection{What is the part of machine learning called where a human describes the correct answer for a number of examples and the agent (machine learning algorithm) generalizes based on these examples?}

\label{q:62:sa:en:True}

\textbf{Answer}: Supervised learning



\subsection{Give an example of a non-computable function?}

\label{q:63:sa:en:True}

\textbf{Answer}: The halting problem



\subsection{What is the name of the machine that is the simplest possible model of a computer?}

\label{q:64:sa:en:True}

\textbf{Answer}: Turing machine



\subsection{An audio file in CD quality means a sampling frequency of 44100 per second, and a sampling depth of 16 bits per channel. How much space in kilobytes (kB) does an uncompressed audio file in stereo (2 channels) in CD quality with a length of 3 minutes take up?}

\label{q:65:sa:en:True}

\textbf{Answer}: 44100 * 16 * 2 * 180 / 8000 = 31 752 kB



\subsection{Assume that we have previously stored digital images with a color depth of 12 bits per pixel (color depth 12 bits per pixel). If we now want to be able to represent half as many different colors compared to before, what color depth should we use then?}

\label{q:66:sa:en:True}

\textbf{Answer}: 11 bits (212 = 4096, 211 = 2048)



\subsection{Assume that we have previously stored digital images with a color depth of 12 bits per pixel (color depth 12 bits per pixel). If we now want to be able to represent twice as many different colors compared to before, what color depth should we use then?}

\label{q:67:sa:en:True}

\textbf{Answer}: 13 bits (212 = 4096, 213 = 8192)



\subsection{What is character encoding?}

\label{q:68:sa:en:True}

\textbf{Answer}: A description of how bit patterns are translated into characters and vice versa. A mapping from bit patterns to characters and vice versa.



\subsection{If 6A38 is the hexadecimal notation for a bit pattern representing one sound sample, what is the sampling depth of this sound sample?}

\label{q:71:sa:en:True}

\textbf{Answer}: 16 bits.



\subsection{If 6A36B3 is the hexadecimal notation for a bit pattern representing an RGB-encoded pixel, what is the color depth of this pixel?}

\label{q:72:sa:en:True}

\textbf{Answer}: 24-bit color depth.



\subsection{What is color depth in the context of image storage?}

\label{q:73:sa:en:True}

\textbf{Answer}: The number of bits per pixel used to encode the color of the pixel.



\subsection{What decimal number (base 10) corresponds to the hexadecimal number 15?}

\label{q:74:sa:en:True}

\textbf{Answer}: 1 * 16 + 5 * 1 = 21



\subsection{Assume that we have previously stored digital images with a color depth of 8 bits per pixel (color depth 8 bits per pixel). If we now want to be able to represent twice as many different colors compared to before, which color depth should we use?}

\label{q:76:sa:en:True}

\textbf{Answer}: 9 bits per pixel (28 = 256 and 29 = 512).



\subsection{What is sampling frequency (sample rate) in connection with digital storage of sound?}

\label{q:77:sa:en:True}

\textbf{Answer}: The sampling frequency describes the number of samples (readings of the sound wave) made per unit of time (second).



\subsection{What decimal number (base 10) corresponds to the hexadecimal number 3F?}

\label{q:78:sa:en:True}

\textbf{Answer}: The hexadecimal (base 16) number series: 0,1,2,3,4,5,6,7,8,9,A,B,C,D,E,F (where A=10, B=11, C=12, D=13, E =14, F=15). 3*161 +F*160 =48+15=63



\subsection{Assume that we have previously stored digital images with a color depth of 8 bits per pixel (color depth 8 bits per pixel). If we now want to be able to represent half as many different colors compared to before, which color depth should we use?}

\label{q:79:sa:en:True}

\textbf{Answer}: 7 bits per pixel; (8 bits can represent 256 values, to represent half (128) 7 bits are needed because 27 = 128).



\subsection{How many bits (color depth) are needed to represent 16 different colors?}

\label{q:80:sa:en:True}

\textbf{Answer}: 4 bits can represent 16 values.



\subsection{What is the sample depth of an audio file?}

\label{q:81:sa:en:True}

\textbf{Answer}: Describes how many bits are used to represent the information of a sample (measurement point).



\subsection{What is the sample rate of an audio file?}

\label{q:82:sa:en:True}

\textbf{Answer}: Describes the number of samples (measurement points) per time unit.



\subsection{When describing audio files, what does sample rate refer to?}

\label{q:83:sa:en:True}

\textbf{Answer}: The sampling frequency describes the number of samples (readings of the sound wave) made per time unit (second).



\subsection{Assume that we have previously stored digital images with a color depth of 5 bits per pixel. If we now want to be able to represent twice as many different colors compared to before, what color depth should we use then?}

\label{q:84:sa:en:True}

\textbf{Answer}: 6 bits per pixel



\subsection{How many bits are required to represent a 24 different colors?}

\label{q:85:sa:en:True}

\textbf{Answer}: 5 bits can represent 32 values.



\subsection{How many bits are required to represent a 12 different colors?}

\label{q:86:sa:en:True}

\textbf{Answer}: 4 bits can represent 16 values.



\subsection{How many bits are required to represent a Boolean value?}

\label{q:87:sa:en:True}

\textbf{Answer}: One bit.



\subsection{Give example of two logical operations that can be performed on boolean values.}

\label{q:88:sa:en:True}

\textbf{Answer}: Two pieces of AND, OR, XOR and NOT.



\subsection{How many bits are required to represent a 9 different colors?}

\label{q:89:sa:en:True}

\textbf{Answer}: 4 bits can represent 16 values.



\subsection{How many bits are required to represent a 15 different colors?}

\label{q:90:sa:en:True}

\textbf{Answer}: 4 bits can represent 16 values.



\subsection{What is a byte?}

\label{q:91:sa:en:True}

\textbf{Answer}: A collection of 8 bits.



\subsection{What do we call 8 bits?}

\label{q:92:sa:en:True}

\textbf{Answer}: One byte.



\subsection{What are the three different categories of machine instructions (machine instruction categories)?}

\label{q:93:sa:en:True}

\textbf{Answer}: Data transfer instructions (data transfer instructions), arithmetic/logic instructions (arithmetic/logic instructions) and control instructions (control instructions).



\subsection{What is a machine language?}

\label{q:94:sa:en:True}

\textbf{Answer}: A machine language is the set of all machine instructions recognized by a certain CPU (a machine language is the set of all machine instructions recognized by the CPU).



\subsection{What is stored in the program counter?}

\label{q:95:sa:en:True}

\textbf{Answer}: The memory address of the next instruction.



\subsection{What is stored in the instruction register?}

\label{q:96:sa:en:True}

\textbf{Answer}: The machine code instruction to be executed.



\subsection{Describe the difference between RISC and CISC processors.}

\label{q:97:sa:en:True}

\textbf{Answer}: RISC processors have few, simple and fast machine instructions, and CISC processors have many and powerful machine instructions.



\subsection{What different steps are included in a machine cycle? Enter the steps in the order in which they are performed.}

\label{q:98:sa:en:True}

\textbf{Answer}: Fetch, Decode och Execute.



\subsection{What bit pattern do we obtain if we perform the ADD operation on the bit patterns 1011 0011 and 0010 0110?}

\label{q:99:sa:en:True}

\textbf{Answer}: 1101 1001 (1011 0011 (= 179) ADD 0010 0110 (= 38) gives 1101 1001 (= 217))



\subsection{What are the three main parts that a processor (CPU {\textendash} central processing unit) consists of?}

\label{q:100:sa:en:True}

\textbf{Answer}: Arithmetic/logic unit, control unit and registers.



\subsection{What bit pattern do we get if we perform the operation OR on the bit patterns 1010 0011 and 0010 0110?}

\label{q:101:sa:en:True}

\textbf{Answer}: 1010 00110010 0110 =1000 0101.



\subsection{What is required to be able to interpret a bit pattern as a character? What is required to be able to interpret a bit pattern as a character?}

\label{q:102:sa:en:True}

\textbf{Answer}: That you know the character encoding.



\subsection{How do you ensure that processes cannot perform operations that are destructive to other processes on a computer, e.g. to write data into other processes' parts of primary memory?}

\label{q:104:sa:en:True}

\textbf{Answer}: By certain machine instructions, so-called privileged instructions, may only be executed by processes that are in privileged mode, which only operating system processes should be.



\subsection{What is boot strapping (booting) and why is it needed?}

\label{q:105:sa:en:True}

\textbf{Answer}: A special process for booting a computer, which means that the operating system is loaded into primary memory and begins to execute. It is needed because when a computer is started, the primary memory is completely empty, and then the processor has no instructions to follow.



\subsection{What does real time processing mean?}

\label{q:106:sa:en:True}

\textbf{Answer}: Execution of tasks in accordance with deadlines in the surrounding reality.



\subsection{What is virtual memory?}

\label{q:108:sa:en:True}

\textbf{Answer}: The computer simulates that it has more primary memory (through paging) than the actual physical primary memory.



\subsection{What is the main function of an operating system?}

\label{q:109:sa:en:True}

\textbf{Answer}: "To manage the resources of a computer", which means, among other things: i) to supervise the operation of the computer (to oversee the operation of the computer);ii) to store and retrieve files;iii) to schedule programs for execution (to schedule programs for execution);iv) to coordinate the execution of programs."



\subsection{What does interactive processing mean?}

\label{q:110:sa:en:True}

\textbf{Answer}: That the program execution supports interaction with the user.



\subsection{What does real time processing mean?}

\label{q:111:sa:en:True}

\textbf{Answer}: Program execution according to deadlines in the surrounding reality.



\subsection{What is the difference between batch processing and interactive processing?}

\label{q:112:sa:en:True}

\textbf{Answer}: Batch processing is the execution of programs (or more precisely: a batch of jobs) without any interaction with a user. Interactive processing is the execution of programs with some type of interaction with a user.



\subsection{What is virtual memory and what can it be good for?}

\label{q:113:sa:en:True}

\textbf{Answer}: Virtual memory is a memory management technique used to expand the computer's available memory beyond the primary memory. Normally, the virtual memory is created on a secondary storage device, for example a hard drive, and the advantage is that you can work with programs and data that require more memory than the physical primary memory you have. The disadvantage is that it is normally slower (although it depends on the type of media it is saved on).



\subsection{Name four different components of an operating system kernel (operating system kernel)?}

\label{q:114:sa:en:True}

\textbf{Answer}: File manager, device drivers, memory manager, scheduler, dispatcher.



\subsection{What is a file in a file management system?}

\label{q:115:sa:en:True}

\textbf{Answer}: A named separate group of data.



\subsection{What are the four basic functions of an operating system (functions of operating systems)?}

\label{q:116:sa:en:True}

\textbf{Answer}: Oversee the operation of a computer; store and retrieve files; schedule programs for execution; coordinate the execution of programs.



\subsection{An operating system consists of two main components (operating system components), what are they called?}

\label{q:118:sa:en:True}

\textbf{Answer}: User interface and kernel.



\subsection{What conditions are required to be fulfilled in order for a deadlock to occur?}

\label{q:119:sa:en:True}

\textbf{Answer}: Competition for non-sharable resources; resources requested on a partial basis; allocated resources cannot be forcibly retrieved.



\subsection{A process's current state can be described by a set of data, which data?}

\label{q:120:sa:en:True}

\textbf{Answer}: Inneh\r{a}llet i programr\"aknaren, inneh\r{a}llet i general purpose registren, och till processen tillh\"orande delar av prim\"arminnet.



\subsection{What is a program and what is a process?}

\label{q:121:sa:en:True}

\textbf{Answer}: A program is a collection of instructions that are executable by a computer (an executable algorithm), and a process is the activity of executing these instructions.



\subsection{What's a file?}

\label{q:122:sa:en:True}

\textbf{Answer}: A named group of data.



\subsection{What is a directory?}

\label{q:123:sa:en:True}

\textbf{Answer}: A named collection of files and (sub-)directories.



\subsection{What does paging mean?}

\label{q:124:sa:en:True}

\textbf{Answer}: Programs and data are rotated back and forth between main memory and mass storage.



\subsection{What is and what does a boot loader do?}

\label{q:125:sa:en:True}

\textbf{Answer}: A program stored in ROM, which is run when the computer starts and loads the operating system into main memory and then transfers the control to the operating system.



\subsection{What does interactive processing mean?}

\label{q:126:sa:en:True}

\textbf{Answer}: Interactive processing is the execution of programs with some type of interaction with a user.



\subsection{What does batch processing mean?}

\label{q:127:sa:en:True}

\textbf{Answer}: Batch processing is the execution of programs (or more exactly: a batch of jobs) without any interaction with a user.



\subsection{What does the term deadlock mean?}

\label{q:129:sa:en:True}

\textbf{Answer}: That processes block (prevent) each other from continuing.



\subsection{What is a batch processing job?}

\label{q:130:sa:en:True}

\textbf{Answer}: A program to be executed along with its input and output.



\subsection{What does multitasking mean?}

\label{q:132:sa:en:True}

\textbf{Answer}: That multiple programs can be executed "at the same time" by a single user.



\subsection{What is paging?}

\label{q:133:sa:en:True}

\textbf{Answer}: That programs and data are rotated back and forth between primary and secondary memory.



\subsection{What is the difference between a switch and a router?}

\label{q:134:sa:en:True}

\textbf{Answer}: A switch connects several "buses" (and/or computers) to a network. A router connects several different networks to a network of networks (internet).



\subsection{What are the two models of inter-process communication?}

\label{q:135:sa:en:True}

\vspace{2cm}

\noindent\makebox[\textwidth]{\hrulefill}

\vspace{1cm}

\textit{Answer}: \autoref{q:135:sa:en:True}



\subsection{What is an IP address?}

\label{q:136:sa:en:True}

\textbf{Answer}: A unique numerical address for a computer connected to the Internet.



\subsection{What is DNS?}

\label{q:137:sa:en:True}

\textbf{Answer}: The Domain Name System translates domain names into IP addresses.



\subsection{What does bus and star mean when it comes to network topology?}

\label{q:138:sa:en:True}

\textbf{Answer}: Bus topology means that all devices are connected to a common communication line, a so-called bus. The Star topology means that all other units are connected to a central unit, so-called access point (access point).



\subsection{What is cloud computing?}

\label{q:139:sa:en:True}

\textbf{Answer}: Large pools of shared computers that can be assigned for use based on need.



\subsection{What is the main difference between IPv4 (IP version 4) and IPv6 (IP version 6)?}

\label{q:140:sa:en:True}

\textbf{Answer}: IPv4 addresses are 32 bits and IPv6 addresses are 128 bits, which means that there are many more IPv6 addresses than IPv4 addresses.



\subsection{What is a certificate in the context of public key encryption?}

\label{q:141:sa:en:True}

\textbf{Answer}: A package consisting of name/identity and public key (a package of name/identity and public key), which certifies that you are who you claim to be.



\subsection{Give an example of a type of malware?}

\label{q:142:sa:en:True}

\textbf{Answer}: Viruses, worms, trojans, spyware, phishing software (viruses, worms, Trojan horses, spyware, phishing software) (one is sufficient).



\subsection{What does DNS lookup mean?}

\label{q:143:sa:en:True}

\textbf{Answer}: The use of DNS (domain name system) to translate from a domain name to an IP address.



\subsection{What does a (network) hub do?}

\label{q:144:sa:en:True}

\textbf{Answer}: It connects computers to a network.



\subsection{To which Internet software layer (Internet software layer) does the SMTP protocol belong?}

\label{q:145:sa:en:True}

\textbf{Answer}: The application layer (SMTP = simple mail transfer protocol).



\subsection{What does a web server (web server) do?}

\label{q:146:sa:en:True}

\textbf{Answer}: Provides access to various web resources, such as web pages.



\subsection{What is the purpose of a URL/URI?}

\label{q:147:sa:en:True}

\textbf{Answer}: To uniquely identify a web resource, e.g. a web page.



\subsection{What are the two common Internet protocols for the transport layer?}

\label{q:148:sa:en:True}

\textbf{Answer}: TCP (transmission control protocol) and UDP (user datagram protocol).



\subsection{What is the name of the encryption technology that is widely used on the Internet and which means that the parties do not need to have access to a common key in advance?}

\label{q:149:sa:en:True}

\textbf{Answer}: Public key encryption (eg the RSA algorithm).



\subsection{What are Internet domains and what is their purpose?}

\label{q:150:sa:en:True}

\textbf{Answer}: Mnemonic names for numeric IP addresses, which is easier for people to remember and means you can change IP addresses but still keep the same mnemonic address.



\subsection{Give two examples of Internet applications using open (publicly available) protocols?}

\label{q:151:sa:en:True}

\textbf{Answer}: For example: HTTP (hypertext transfer protocol) and FTP (file transfer protocol).



\subsection{What is the difference between the HTTP and HTTPS protocols?}

\label{q:152:sa:en:True}

\textbf{Answer}: The traffic over HTTP is not encrypted, while the traffic over HTTPS is encrypted (public key encryption).



\subsection{Briefly explain the difference between the network components hub, switch and router?}

\label{q:153:sa:en:True}

\textbf{Answer}: A hub connects machines/network devices to a network, and sends all traffic to all connected devices. A switch is a type of smarter hub that connects network devices in the same way as a hub, but only sends traffic between the devices/ports that need to communicate.A router connects several networks to each other and forwards traffic between networks.



\subsection{What is transferred with the different protocols FTP, HTTP, SMTP?}

\label{q:154:sa:en:True}

\textbf{Answer}: FTP transfers files.HTTP transfers various types of resources for e.g. web pages etc. (e.g. html documents, images, sounds etc.).SMTP transfers electronic mail.



\subsection{What is a certificate? Can all certificates be trusted equally? Motivate your answer!}

\label{q:155:sa:en:True}

\textbf{Answer}: A certificate is an electronic document that shows who owns a certain identity or encryption key. How much one trusts a certificate depends on the trust in the person who issued the certificate. Some certificate issuers, e.g. some authorities, enjoy a higher level of trust, while a certificate issued by yourself or a less credible or serious actor is less trustworthy. Compare with ordinary identity documents - a passport issued by the state through the Police Authority is worth significantly more than a passport that you have made at home.



\subsection{What does a digital signature mean for public key encryption, i.e. that when transferring a file, the identity of the sender can be guaranteed?}

\label{q:156:sa:en:True}

\textbf{Answer}: That the file is encrypted with the sender's private key.



\subsection{What characterizes a distributed system?}

\label{q:157:sa:en:True}

\textbf{Answer}: Consists of software units that execute on several different computers.



\subsection{How many times more addresses can be represented with IPv6 compared to IPv4 (as usual, you do not need to calculate a value, it is enough to set up a correct calculation)?}

\label{q:158:sa:en:True}

\textbf{Answer}: 2^128 / 2^32 = 2^96



\subsection{What is a distributed system?}

\label{q:159:sa:en:True}

\textbf{Answer}: Consists of software units that execute on several different computers.



\subsection{What is HTML used for?}

\label{q:160:sa:en:True}

\textbf{Answer}: HTML (hyper text markup language) is a language used to create/describe web pages.



\subsection{In public key encryption, the term certificate is used, what is it?}

\label{q:161:sa:en:True}

\textbf{Answer}: A package consisting of name/identity and a public key, which, if issued by a certificate authority, certifies that you are who you pretend to be.



\subsection{In public key encryption, the term certificate authority is used, what is it?}

\label{q:162:sa:en:True}

\textbf{Answer}: An organization that issues certificates (a package of name/identity and public key), which certifies that you are who you pretend to be.



\subsection{What is the benefit of using TCP instead of UDP? What is a disadvantage?}

\label{q:163:sa:en:True}

\textbf{Answer}: An advantage is that it is more reliable, a disadvantage is that it is slower.



\subsection{What is the benefit of using UDP instead of TCP? What is a disadvantage?}

\label{q:164:sa:en:True}

\vspace{2cm}

\noindent\makebox[\textwidth]{\hrulefill}

\vspace{1cm}

\textit{Answer}: \autoref{q:164:sa:en:True}



\subsection{Name the four Internet software layers?}

\label{q:165:sa:en:True}

\textbf{Answer}: Application, transport, network, link.



\subsection{Briefly explain the concept of client-server!}

\label{q:166:sa:en:True}

\textbf{Answer}: A model where clients request services to be performed by servers that provide the services.



\subsection{What is SMTP used for?}

\label{q:167:sa:en:True}

\textbf{Answer}: SMTP is a protocol for handling electronic mail.



\subsection{If person A wants to send a message to person B, encrypted according to public-key encryption, so that no one other than B can read the message. What does the message then need to be encrypted with before the message is sent from A?}

\label{q:168:sa:en:True}

\textbf{Answer}: Person B's public key.



\subsection{If person A wants to send a message to person B, encrypted according to public-key encryption, so that no one other than A can have sent the message. What does the message then need to be encrypted with before the message is sent from A?}

\label{q:169:sa:en:True}

\textbf{Answer}: Person A's private key.



\subsection{Name one advantage of public key encryption over symmetric encryption techniques.}

\label{q:170:sa:en:True}

\textbf{Answer}: You can use a combination of public and private key for encryption and decryption in such a way that you don't need to exchange any key in advance.



\subsection{What is a hub or a router used for?}

\label{q:171:sa:en:True}

\textbf{Answer}: A hub connects machines/network devices to a network. A router connects several networks to an Internet (network of networks).



\subsection{Name one advantage of UDP over TCP, and one advantage of TCP over UDP.}

\label{q:172:sa:en:True}

\textbf{Answer}: UDP is more efficient, and TCP is more reliable.



\subsection{What is recursion?}

\label{q:173:sa:en:True}

\vspace{2cm}

\noindent\makebox[\textwidth]{\hrulefill}

\vspace{1cm}

\textit{Answer}: \autoref{q:173:sa:en:True}



\subsection{Why is binary search better than sequential search on sorted data?}

\label{q:174:sa:en:True}

\textbf{Answer}: Because with binary search, the number of steps in the search process grows logarithmically with the number of entries, while with sequential search, the number of steps grows linearly with the number of entries, which means that binary search is significantly more efficient.



\subsection{Is there any difference between iteration and recursion in terms of memory usage?}

\label{q:175:sa:en:True}

\textbf{Answer}: Yes, each recursive call in a recursion requires additional memory, unlike an iteration where each turn requires no additional memory.



\subsection{What is the difference between an algorithm and a program?}

\label{q:177:sa:en:True}

\textbf{Answer}: A program is an algorithm coded in such a way that a computer can execute it.



\subsection{Which two different methods are used to verify that a program is correct (software verification)?}

\label{q:178:sa:en:True}

\textbf{Answer}: Static verification or code analysis, and testing.



\subsection{What is a program in relation to an algorithm?}

\label{q:179:sa:en:True}

\textbf{Answer}: A program is an algorithm coded in a programming language, i.e. in such a way that a computer can execute it.



\subsection{Describe how binary search works! What are the requirements for the data you search in?}

\label{q:180:sa:en:True}

\textbf{Answer}: Binary search requires that the data being searched is sorted. At each repetition of the search, the number of entries is halved. For each repetition, examine the entry in the middle position: if the entry sought is arranged before the entry in the middle position, continue the search in the first half; if the entry being sought is ordered after the entry in the middle position, then continue the search in the second half. Continue in a similar manner and end the search when the record is either found or the remaining half is empty.



\subsection{What methods can be used to verify the correctness of a program?}

\label{q:181:sa:en:True}

\textbf{Answer}: Static verification or code analysis, and testing.



\subsection{In what two fundamentally different ways can repetition be achieved in an algorithm?}

\label{q:182:sa:en:True}

\textbf{Answer}: Recursion and iteration.



\subsection{When is sequential search preferable to binary search?}

\label{q:183:sa:en:True}

\textbf{Answer}: Sequential search is preferred for very short lists and when the data is not sorted because binary search requires sorted data.



\subsection{What is a prerequisite for binary search to work? Motivate your answer.}

\label{q:184:sa:en:True}

\textbf{Answer}: Sorted/ordered data.



\subsection{Is binary search a good choice for searching unsorted data? Motivate your answer.}

\label{q:185:sa:en:True}

\textbf{Answer}: No, binary search does not work for unsorted/unordered data.



\subsection{Define the term algorithm!}

\label{q:186:sa:en:True}

\textbf{Answer}: An algorithm is an ordered set of unambiguous, executable steps that defines a terminating process.



\subsection{Can all algorithms be described as a flow chart? Motivate your answer!}

\label{q:187:sa:en:True}

\textbf{Answer}: Yes, rectangles and diamonds describe executable steps, of which diamonds describe conditions, and arrows describe sequences and loops, which is what is needed to describe any conceivable algorithm.



\subsection{Is a programming language, e.g. Python, suitable for describing algorithms? Motivate your answer!}

\label{q:188:sa:en:True}

\textbf{Answer}: Yes, because programming languages have well-defined primitives and rules for how the primitives can be combined. (No, because programming languages require specifying many details.)



\subsection{What does the top-down methodology mean when developing (or discovering) algorithms?}

\label{q:189:sa:en:True}

\textbf{Answer}: That you start from a high level of abstraction and gradually work your way down by dividing problems into smaller parts.



\subsection{Why is it not so important to follow a strict syntax in pseudocode?}

\label{q:190:sa:en:True}

\textbf{Answer}: Because it is people who should be able to understand and read pseudocode, not computers.



\subsection{Why is it necessary to know the data type of a variable?}

\label{q:191:sa:en:True}

\textbf{Answer}: It is the data type that specifies how we should interpret the bit pattern stored in the variable.



\subsection{What is the difference between source code and object code?}

\label{q:192:sa:en:True}

\textbf{Answer}: Source code is the program code that the programmer writes. Object code is the translation of the source code that can be run on a computer.



\subsection{Does a syntactically correct program always produce correct results? Motivate your answer.}

\label{q:193:sa:en:True}

\textbf{Answer}: No, a syntactically correct program can contain errors, e.g. logical errors, which cause the program to give incorrect results.



\subsection{What characterizes a data structure of the type struct/record (aggregate type)?}

\label{q:194:sa:en:True}

\textbf{Answer}: That it is a data structure composed of data that can have different types.



\subsection{What does it mean that a parameter to a subroutine is transferred as value (passed by value)?}

\label{q:195:sa:en:True}

\textbf{Answer}: Passed by value means that the parameter is passed as a copied value and that this copied value is stored in a local variable in the subroutine.



\subsection{What does it mean that a parameter to a subroutine is transferred as a reference (passed by reference)?}

\label{q:196:sa:en:True}

\textbf{Answer}: Passed by reference means that the parameter is transferred as a reference to a location where it is stored, which means that any changes are made to the original variable.



\subsection{What does an assembler do?}

\label{q:197:sa:en:True}

\textbf{Answer}: An assembler converts assembly code into machine code.



\subsection{What characterizes an array data structure?}

\label{q:198:sa:en:True}

\textbf{Answer}: That all elements in the data structure are of the same data type, and that the individual elements are accessed via indexes.



\subsection{What are the four major programming paradigms?}

\label{q:199:sa:en:True}

\textbf{Answer}: Imperative, functional, object-oriented and declarative (logic) programming.



\subsection{List four common primitive data types.}

\label{q:200:sa:en:True}

\textbf{Answer}: Integer, floating point number, character, boolean (truth value).



\subsection{What does a compiler do?}

\label{q:201:sa:en:True}

\textbf{Answer}: A compiler translates source code, written in a programming language, into executable machine code.



\subsection{A program can give rise to three different types of errors: syntactic errors, runtime errors and logic errors. What type of errors are most serious and why?}

\label{q:202:sa:en:True}

\textbf{Answer}: Logical errors, because they do not give rise to an error message.



\subsection{A program can give rise to three different types of errors: syntactic errors, runtime errors and logic errors. What type of errors are least serious and why?}

\label{q:203:sa:en:True}

\textbf{Answer}: Syntax errors, because they are already detected by the compiler.



\subsection{What is concurrent programming?}

\label{q:204:sa:en:True}

\textbf{Answer}: Programming where you program several parallel execution paths (threads) in the same program.



\subsection{Briefly describe the concepts of sequence, selection and iteration.}

\label{q:205:sa:en:True}

\textbf{Answer}: A sequence is a succession of instructions executed in order.Selection is a choice - to execute different instructions depending on the state of the program (e.g. through if statements).Iteration means that the same instruction or sequence of instructions is executed several times (eg with loops).



\subsection{A variable points to a bit pattern in stored in memory; what do we need to know to interpret the bit pattern correctly?}

\label{q:206:sa:en:True}

\textbf{Answer}: What data type the variable's data has. It is the data type that specifies how the program should interpret the bit pattern stored in the variable.



\subsection{What do the terms sequence, selection and iteration mean?}

\label{q:207:sa:en:True}

\textbf{Answer}: Sequence: a sequence of instructions that are executed in order. Selection: different instructions are executed depending on the state of the program. Iteration: the same (sequence of) instructions are executed multiple times (depending on the state of the program).



\subsection{Give examples of two different ways of describing algorithms.}

\label{q:208:sa:en:True}

\textbf{Answer}: Flow charts and pseudo code.



\subsection{What generation of programming languages is characterized by: - one-to-one correspondence between language instructions and machine instructions; - inherently machine-dependent?}

\label{q:209:sa:en:True}

\textbf{Answer}: Second generation.



\subsection{What generation of programming languages is characterized by:- machine independent (mostly);- each primitive corresponds to a sequence of machine language instructions?}

\label{q:210:sa:en:True}

\textbf{Answer}: Third generation.



\subsection{What is a literal in a programming language?}

\label{q:211:sa:en:True}

\textbf{Answer}: An explicit value of a certain data type.



\subsection{What is a constant in a programming language?}

\label{q:212:sa:en:True}

\textbf{Answer}: A named literal. / A named value of a certain data type.



\subsection{In object-oriented programming you have classes and objects. In addition to this there are three features that characterize object-oriented programming, which ones?}

\label{q:213:sa:en:True}

\textbf{Answer}: Inheritance, encapsulation and polymorphism.



\subsection{The translation of source code to machine code is done in three steps by three different units in the translator; what are these three units called?}

\label{q:214:sa:en:True}

\vspace{2cm}

\noindent\makebox[\textwidth]{\hrulefill}

\vspace{1cm}

\textit{Answer}: \autoref{q:214:sa:en:True}



\subsection{What is a thread in concurrent programming?}

\label{q:215:sa:en:True}

\textbf{Answer}: A concurrent/parallel execution path within the same program/process.



\subsection{What is the basic building block in logic programming languages?}

\label{q:216:sa:en:True}

\textbf{Answer}: Logical formula (predicate).



\subsection{What is a variable in a programming language?}

\label{q:217:sa:en:True}

\textbf{Answer}: A variable is a named space in main memory.



\subsection{What is the purpose of using procedural units (subprogram, subroutine, procedure, function, method, predicate etc.) in programming?}

\label{q:218:sa:en:True}

\textbf{Answer}: Used to simplify program development through abstraction.



\subsection{What does inheritance mean in object-oriented programming?}

\label{q:219:sa:en:True}

\textbf{Answer}: Inheritance allows one class to encompass the characteristics of another class without having to explicitly declare them.



\subsection{What is the difference between a compiler and an interpreter?}

\label{q:220:sa:en:True}

\textbf{Answer}: A compiler translates program code/source code into executable code. An interpreter (interpreter) interprets program code/source code during the actual execution and then executes the instructions in the program code/source code.



\subsection{All programming languages have three types of program control flow, which?}

\label{q:221:sa:en:True}

\textbf{Answer}: Sequence, selection and repetition.



\subsection{What three things characterize agile development models?}

\label{q:222:sa:en:True}

\vspace{2cm}

\noindent\makebox[\textwidth]{\hrulefill}

\vspace{1cm}

\textit{Answer}: \autoref{q:222:sa:en:True}



\subsection{What are design patterns?}

\label{q:223:sa:en:True}

\textbf{Answer}: General solutions to recurring problems.



\subsection{What is the purpose of use case diagram?}

\label{q:224:sa:en:True}

\textbf{Answer}: To describe the proposed system from the user's point of view.



\subsection{What is the purpose of class diagrams?}

\label{q:225:sa:en:True}

\textbf{Answer}: To describe the structure of different types of objects (classes) and the relationship between different types of objects (classes).



\subsection{What are the four traditional development phases in software development (the traditional development phases of the software life cycle)?}

\label{q:226:sa:en:True}

\textbf{Answer}: Needs analysis, design, implementation, and evaluation



\subsection{What is the main purpose of dividing a software into modules?}

\label{q:227:sa:en:True}

\textbf{Answer}: To simplify software development by making a single module manageable and can be developed independently of other modules.



\subsection{What are the three desirable characteristics of modules that one wants to achieve when dividing a software into modules?}

\label{q:228:sa:en:True}

\textbf{Answer}: High cohesion, low coupling, information hiding.



\subsection{What is the difference between glass-box testing and black-box testing?}

\label{q:229:sa:en:True}

\textbf{Answer}: Glass-box testing means that the tester knows the internal structure of the software to be tested and uses this information when designing the tests. This is in contrast to black-box testing, which is not based on knowledge of the software's internal structure.



\subsection{Describe the differences between one-to-one, one-to-many and many-to-many relationships, preferably using example.}

\label{q:230:sa:en:True}

\textbf{Answer}: An example of a one-to-one relationship is "husband-wife", since a man can only be the husband of one wife and a woman can only be the wife of one man (according to Swedish law). An example of a one-to-many relationship is "mother-child", because a child only has one (biological) mother, but a woman can be the mother of many children. An example of a many-to-many relationship is "brother-sister", since a boy can have multiple sisters and a girl can have multiple brothers.



\subsection{What is the software technology called that is based on constructing software by combining different ready-made components (instead of developing own components)?}

\label{q:231:sa:en:True}

\textbf{Answer}: Component architecture or component-based software engineering.



\subsection{Describe an example of each of the different types of relationship: one-to-one (one-to-one), one-to-many (one-to-many), and many-to-many (many-to- many)!}

\label{q:232:sa:en:True}

\textbf{Answer}: An example of a one-to-one relationship is "husband-wife", since a man can only be the husband of one wife and a woman can only be the wife of one man (according to Swedish law). An example of a one-to-many relationship is "mother-child", because a child only has one (biological) mother, but a woman can be the mother of many children. An example of a many-to-many relationship is "brother-sister", since a boy can have multiple sisters and a girl can have multiple brothers.



\subsection{What are the four steps in traditional software development (using, for example, the waterfall model)?}

\label{q:233:sa:en:True}

\textbf{Answer}: Needs analysis, design, implementation, testing.



\subsection{Briefly describe some advantages of dividing programs into modules?}

\label{q:234:sa:en:True}

\textbf{Answer}: Simplifying software development by making a single module manageable and can be developed independently of other modules.



\subsection{What does prototyping mean?}

\label{q:235:sa:en:True}

\textbf{Answer}: Prototyping involves developing and evaluating a prototype (an unfinished version of what is to be developed).



\subsection{Describe what a sprint in agile development with Scrum is?}

\label{q:236:sa:en:True}

\textbf{Answer}: A sprint is a phase/iteration of 2 to 4 weeks that should deliver some result (a sprint is an iteration of 2 to 4 weeks and should have some deliveries).



\subsection{What characterizes black-box testing?}

\label{q:237:sa:en:True}

\textbf{Answer}: In black-box testing, the tester has no knowledge of the software's internal structure, unlike glass-box testing, which means that the tester knows the internal structure of the software to be tested and uses this information when designing the tests.



\subsection{What are design patterns and what are they good for?}

\label{q:238:sa:en:True}

\textbf{Answer}: General solutions to recurring problems. By starting from ready-made and good solutions, you can speed up the development time and also make programs more robust, as the design patterns are often well-proven. Design patterns also provide developers and system architects with a common vocabulary to discuss and refer to different solutions.



\subsection{Explain the concepts of coupling and cohesion?}

\label{q:239:sa:en:True}

\textbf{Answer}: Coupling is a measure of how intertwined different modules/components are in a software system (the linkage between modules). Low coupling is good. Cohesion is a measure of how well the content of a module hangs together, how focused it is (the internal binding within a module). High cohesion is good.



\subsection{What three different types of relationships between entities are important to distinguish in software engineering?}

\label{q:240:sa:en:True}

\textbf{Answer}: One-to-one, one-to-many and many-to-many.



\subsection{What are software engineering methods called that value: - individuals and interactions over processes and tools;- working software over comprehensive documentation;- customer collaboration over contract negotiation;- responding to change over following a plan.}

\label{q:241:sa:en:True}

\textbf{Answer}: Agile.



\subsection{What is a software module?}

\label{q:242:sa:en:True}

\textbf{Answer}: A manageable unit of the software that handles only part of the work that the entire software is supposed to do.



\subsection{What is the purpose of the Scrum meeting "sprint retrospective"?}

\label{q:243:sa:en:True}

\textbf{Answer}: To improve the work process itself based on the experiences from the last sprint.



\subsection{How many members should a development team have according to Scrum?}

\label{q:244:sa:en:True}

\textbf{Answer}: 3 - 9.



\subsection{Give two examples of diagrams used in modeling (in software engineering).}

\label{q:245:sa:en:True}

\textbf{Answer}: Structure diagram, data flow diagram, use case diagram, class diagram.



\subsection{What is a design pattern (in software engineering)?}

\label{q:246:sa:en:True}

\textbf{Answer}: En f\"orutvecklad modell f\"or att l\"osa \r{a}terkommande problem.



\subsection{What characterizes glass-box testing?}

\label{q:247:sa:en:True}

\textbf{Answer}: In glass-box testing, the tester knows the internal structure of the software to be tested and uses this information when designing the tests, unlike black-box testing, when the tester has no knowledge of the software's internal structure.



\subsection{What 3 questions should each team member briefly answer at the Daily Scrum meetings?}

\label{q:248:sa:en:True}

\textbf{Answer}: What did you do yesterday? What will you do today? Are there any impediments preventing you from doing your work?



\subsection{What is the Scrum meeting, at the end of a sprint where you discuss what went well during the previous sprint process and what can be improved for the next sprint, called?}

\label{q:249:sa:en:True}

\textbf{Answer}: Sprint retrospective.



\subsection{The Scrum development method has three different roles defined, which ones?}

\label{q:250:sa:en:True}

\textbf{Answer}: Product Owner (product owner), Scrum master and Development team. The development team role is often shared by several (3-9) people.



\subsection{What is an abstract data type?}

\label{q:251:sa:en:True}

\textbf{Answer}: A data type that contains both data and operations to manipulate the data.



\subsection{What characterizes a sorted binary tree ("binary search tree")?}

\label{q:252:sa:en:True}

\textbf{Answer}: That each node in the tree has two or fewer subtrees (child nodes), that all nodes in the left subtree contain values ​​lower than the contents of the current node, and that all nodes in the right subtree contain values ​​greater than the contents of the current node.



\subsection{What are the four basic data structures in addition to arrays?}

\label{q:253:sa:en:True}

\textbf{Answer}: Lists, stacks, queues and trees.



\subsection{What is the difference between a dynamic and a static data structure?}

\label{q:254:sa:en:True}

\textbf{Answer}: A static data structure cannot change form or size, as a dynamic data structure can.



\subsection{What characterizes a binary tree?}

\label{q:255:sa:en:True}

\textbf{Answer}: A binary tree is a tree structure where each node can have a maximum of two child nodes.



\subsection{What is the difference between a static and a dynamic data structure?}

\label{q:256:sa:en:True}

\textbf{Answer}: The shape and size (the structure) of a static data structure do not change over time, although the content may change, whereas the shape and size (the structure) of a dynamic data structure can change.



\subsection{Can a list be implemented as a static or dynamic data structure, both, or neither? Motivate your answer!}

\label{q:257:sa:en:True}

\textbf{Answer}: A list can both be implemented as a static data structure, e.g. as an array, and as a dynamic data structure consisting of elements and pointers.



\subsection{Describe the basic data structures stack (stack) and queue (queue)?}

\label{q:258:sa:en:True}

\textbf{Answer}: A stack is a list where you add and remove elements at the same end according to the "last-in-first-out" (LIFO) principle. A queue is a list where you add at one end and remove at the other end according to the "first-in-first-out" (FIFO) principle.



\subsection{Can the low-level data structure array be used to implement a queue? Motivate your answer!}

\label{q:259:sa:en:True}

\textbf{Answer}: Yes, where the elements of the array describe a circular queue, and you have a pointer to the queue's head (start) and a pointer to its tail (end).



\subsection{What is an abstract data structure? What is the difference to a record/struct?}

\label{q:260:sa:en:True}

\textbf{Answer}: An abstract data structure describes a data type and its operations, i.e. both what is stored and what can be done with it.A record/struct is a composite data structure that is a collection of data that can be of different data types.



\subsection{Give an example of a data structure that uses the LIFO principle and a data structure that uses the FIFO principle?}

\label{q:261:sa:en:True}

\textbf{Answer}: LIFO: The last added element is removed first -> StackFIFO: The first element added is removed first -> Queue



\subsection{Lists can be stored either in contiguous blocks in memory, or in the form of linked lists. Which is preferred for static lists, and which is better for dynamic lists?}

\label{q:262:sa:en:True}

\textbf{Answer}: For static lists whose content does not change, contiguous blocks of memory are preferred as they provide good read performance and use little memory/storage capacity. For dynamic lists that can change, adding and removing elements in a contiguous block of memory is expensive because it can result in a lot of data needing to be moved. Therefore, linked lists are usually better for dynamic lists as insertion and removal do not require more than changing some pointers (data does not need to be moved around).



\subsection{Explain what a pointer is?}

\label{q:263:sa:en:True}

\textbf{Answer}: A pointer is a variable that contains the memory address of what it points to.



\subsection{Two types of specialized lists are stack and queue, describe how they differ from each other!}

\label{q:264:sa:en:True}

\textbf{Answer}: A stack is a list where you add and remove elements at the same end according to the "last-in-first-out" (LIFO) principle. A queue is a list where you add at one end and remove at the other end according to the "first-in-first-out" (FIFO) principle.



\subsection{What distinguishes an abstract data type (abstract data type) from a composite data type (aggregate type / struct / record)?}

\label{q:265:sa:en:True}

\textbf{Answer}: An abstract data type describes a data type and its operations (methods, procedures, functions), i.e. both what is stored (data) and what can be done with it. A record/struct is a composite data structure that is a collection of data that can be of different data types, but does not contain any operations (methods, procedures, functions).



\subsection{In a variation of lists, you add and remove elements at the same end, what is that data structure called? In another variation, elements are added at one end and removed at the other, what is that data structure called?}

\label{q:266:sa:en:True}

\textbf{Answer}: The first is a stack, the second is a queue.



\subsection{What characterizes an aggregate type (struct/record)?}

\label{q:267:sa:en:True}

\textbf{Answer}: A block of data where different elements can be of different data types, the elements are called fields and are accessed by names.



\subsection{Can a queue be implemented as a static or dynamic data structure, both, or neither? Motivate your answer!}

\label{q:268:sa:en:True}

\textbf{Answer}: A queue can be implemented both as a static data structure, e.g. as an array, and as a dynamic data structure consisting of elements and pointers.



\subsection{What distinguishes a dynamic data structure from a static data structure?}

\label{q:269:sa:en:True}

\textbf{Answer}: A dynamic data structure can change form and size over time, but a static data structure cannot.



\subsection{What characterizes the data structure binary tree?}

\label{q:270:sa:en:True}

\textbf{Answer}: Each node in the tree has at most two children (nodes).



\subsection{What characterizes the root node in a tree data structure?}

\label{q:271:sa:en:True}

\textbf{Answer}: It does not have any parent (node).



\subsection{Describe an advantage and a disadvantage of storing an aggregate type (struct/record) in a contiguous block instead of the different parts in separate locations designated by pointers.}

\label{q:272:sa:en:True}

\textbf{Answer}: Advantage: slightly faster (direct) access to the various parts. Disadvantage: the different parts are always the same size, which means that unnecessary space is used and data that requires a lot of space may not fit.



\subsection{What characterizes an array?}

\label{q:273:sa:en:True}

\textbf{Answer}: A block of data where all elements are of the same data type, and the elements are accessed by indexes/indices.



\subsection{What characterizes a static data structure?}

\label{q:274:sa:en:True}

\textbf{Answer}: The form or size of the structure cannot change over time.



\subsection{What characterizes a dynamic data structure?}

\label{q:275:sa:en:True}

\textbf{Answer}: The form and size of the structure can change over time.



\subsection{Describe an advantage and a disadvantage of storing the different parts of an aggregate type (struct/record) in separate locations designated by pointers instead of in a contiguous block.}

\label{q:276:sa:en:True}

\textbf{Answer}: Advantage: each part can get exactly the space required, no more, no less. Disadvantage: accessing the various parts becomes a bit slower because it involves following a pointer instead of directly retrieving the value.



\subsection{What is a database management system (DBMS)?}

\label{q:277:sa:en:True}

\textbf{Answer}: A system that manages databases by executing commands to update the databases and to retrieve data from the databases. A software that handles the creation, updating, searching and administration of databases.



\subsection{What do commit and rollback mean in a database context?}

\label{q:278:sa:en:True}

\textbf{Answer}: A commit means that a transaction has been completed and approved by the database manager. A rollback means that a problem has occurred during a transaction and that the database manager therefore restores the database to the state it had before the transaction (the transaction is rolled back).



\subsection{For relational databases, there are three (3) operations (relational operations), with the help of which you can create new tables that constitute subsets and / or combinations of existing tables. What operations?}

\label{q:279:sa:en:True}

\textbf{Answer}: Select, project och join.



\subsection{What does data mining mean?}

\label{q:280:sa:en:True}

\textbf{Answer}: Data mining is about discovering patterns in collections of data.



\subsection{What is a data warehouse?}

\label{q:281:sa:en:True}

\textbf{Answer}: A collection of static data from one or more sources, intended for analysis of the data.



\subsection{What is a database (database) in relation to a database management system (DBMS)?}

\label{q:282:sa:en:True}

\textbf{Answer}: A database is an organized collection of data (that can be managed by a database management system). A database management system is a system for creating, updating and administering databases, as well as answering questions posed to the databases.



\subsection{Mention a common problem that can arise when, for example, transfers between accounts that transactions protect against.}

\label{q:283:sa:en:True}

\textbf{Answer}: One problem is that money is withdrawn from one account but never deposited into the other (due to interruption or error).



\subsection{What is SQL?}

\label{q:284:sa:en:True}

\textbf{Answer}: Structured Query Language (SQL) is a declarative programming language used to retrieve and manipulate data in relational databases.



\subsection{In what two ways can a transaction be terminated?}

\label{q:285:sa:en:True}

\textbf{Answer}: A transaction that succeeds ends with a commit, and a transaction that fails ends with a rollback that undoes the transaction's work.



\subsection{What is a transaction?}

\label{q:286:sa:en:True}

\textbf{Answer}: A sequence of database operations, which should all together either succeed or fail.



\subsection{A transaction can be terminated in two different ways, which ones?}

\label{q:287:sa:en:True}

\textbf{Answer}: Through a commit or a roll-back.



\subsection{What is a database schema?}

\label{q:288:sa:en:True}

\textbf{Answer}: A description of the structure of the database.



\subsection{What is a database?}

\label{q:289:sa:en:True}

\textbf{Answer}: An organized collection of data (managed by a DBMS).



\subsection{What is a database model?}

\label{q:290:sa:en:True}

\textbf{Answer}: A conceptual view of the database.



\subsection{Name two things that differentiate an object-oriented database from a relational database?}

\label{q:291:sa:en:True}

\textbf{Answer}: Each entity is stored as an object that can contain methods. DBMS maintains inter-object links/references/pointers.



\subsection{What is data mining?}

\label{q:292:sa:en:True}

\textbf{Answer}: Deals with discovering patterns in collections of data.



\subsection{What does a table represent in the relational model for databases?}

\label{q:293:sa:en:True}

\textbf{Answer}: A relation.



\subsection{What does a column in a table represent in the database relational model?}

\label{q:294:sa:en:True}

\textbf{Answer}: An attribute.



\subsection{What does a row in a table represent in the database relational model?}

\label{q:295:sa:en:True}

\textbf{Answer}: An instance.



\subsection{What are the three relational operations in the relational model for databases?}

\label{q:296:sa:en:True}

\textbf{Answer}: Select, project, join.



\subsection{What are the three basic relational operations for retrieving requested data from a relational database?}

\label{q:297:sa:en:True}

\textbf{Answer}: Select, project and join.



\subsection{What is a schema in the context of a database system?}

\label{q:298:sa:en:True}

\textbf{Answer}: A database schema is a description of a database's structure, which for relational databases is its tables and columns.



\subsection{To which programming paradigm does the database query language SQL (structured query language) belong?}

\label{q:299:sa:en:True}

\textbf{Answer}: Declarative programming languages.



\subsection{Why is it of interest to know the efficiency class/complexity class of an algorithm?}

\label{q:300:sa:en:True}

\textbf{Answer}: To be able to compare the effectiveness of different algorithms, and to be able to assess whether an algorithm is useful for large amounts of data.



\subsection{The process of creating 3D graphics consists of three steps, the first of which is 3D modeling (3D modeling), and the third is image display (display). What is the second step called, and what is done in that step?}

\label{q:301:sa:en:True}

\textbf{Answer}: Rendering, which is about determining how the 3D model will appear when projected onto the projection plane.



\subsection{In animation projects, the work is usually carried out in three steps, which?}

\label{q:302:sa:en:True}

\textbf{Answer}: Storyboard, Key frames, In-betweening.



\subsection{Two branches of the field of mechanics have proven particularly useful in simulating natural motions, which?}

\label{q:303:sa:en:True}

\textbf{Answer}: Dynamics and kinematics.



\subsection{Name a way to produce so-called polygonal meshes in 3D modeling!}

\label{q:304:sa:en:True}

\textbf{Answer}: Mathematical equations; Bezier curves and surfaces; procedural models.



\subsection{Briefly explain the difference between local lighting models (local lightning model) and global lighting models (global lightning model). Which model gives the most realistic result? The advantage of the second?}

\label{q:305:sa:en:True}

\textbf{Answer}: A local lighting model does not take into account how different objects affect each other. A global light model does (or at least tries to). Ray tracing is an example of an algorithm used to calculate a global light model. A global model gives a better, more realistic result, but a local model is simpler and less computationally intensive.



\subsection{In computer graphics, light plays an important role. Light is usually divided into three (3) different kinds, which ones? What sets them apart?}

\label{q:306:sa:en:True}

\textbf{Answer}: The question is about different types of reflective light: Specular light, which is reflected without splitting up; appears as a bright shining point on an object and retains the color of the light source. Visible more clearly on smooth shiny surfaces. Diffuse light, which splits up and is reflected in many different directions due to unevenness in the surface of the illuminated object. Takes (partial) color from the reflected surface. Background light (ambient light) which is light that is present throughout the image and is distributed evenly over all objects. Has no definite source.



\subsection{Explain how the terms frame, key frame and in-betweening used in animation are related?}

\label{q:307:sa:en:True}

\textbf{Answer}: Key frames {\textendash} frames capturing the scene at specific points in time. In-betweening {\textendash} producing frames to fill the gap between key frames.



\subsection{The process of creating 3D-graphics consists of two main steps, which ones?}

\label{q:308:sa:en:True}

\textbf{Answer}: Modeling and rendering.



\subsection{What characterizes a local lighting model in computer graphics?}

\label{q:309:sa:en:True}

\textbf{Answer}: It does not take into account light interactions between objects.



\subsection{What characterizes a global lighting model in computer graphics?}

\label{q:310:sa:en:True}

\textbf{Answer}: It takes into account light interactions between objects, for example through ray tracing.



\subsection{Many difficult problems can be described as search problems, which means that you search for a solution in a search tree. To select the path in the search tree, use "rules of thumb". What are such rules of thumb called and why are they needed?}

\label{q:311:sa:en:True}

\textbf{Answer}: Heuristics, and they are needed because the search trees for all hard problems become extremely large, which means that it is impossible to explore the entire search tree.



\subsection{What is the difference between weak AI and strong AI?}

\label{q:312:sa:en:True}

\textbf{Answer}: Weak AI {\textendash} computers can be programmed to exhibit intelligent behavior. Strong AI {\textendash} computers can be programmed to gain intelligence and consciousness.



\subsection{One way to classify machine / computer learning approaches is through the degree to which they require human intervention. What three such classes are we usually talking about?}

\label{q:313:sa:en:True}

\textbf{Answer}: Learning by imitation; supervised learning; learning by reinforcement.



\subsection{What is an artificial neural network and how does such a network change during learning?}

\label{q:314:sa:en:True}

\textbf{Answer}: An artificial neural network is a computational model that mimics a brain's network of neurons. An artificial neural network learns by adjusting the weights in the different neurons in the network.



\subsection{What is the difference between supervised learning and unsupervised learning?}

\label{q:315:sa:en:True}

\textbf{Answer}: In supervised learning, the system is trained with ready classified data (training data). In unsupervised learning, the system receives no training data, but must itself analyze the input data and find patterns.



\subsection{Is reinforcement learning a type of supervised learning or not? Why?}

\label{q:316:sa:en:True}

\textbf{Answer}: Reinforcement learning is unsupervised, and is based on the system itself assessing, based on a given general rule, whether it has succeeded or not.



\subsection{A neural network is a computational model inspired by how the human brain works. How does a neural network learn from sample data?}

\label{q:317:sa:en:True}

\textbf{Answer}: Put simply, a neural network learns from data by adjusting the weights associated with different neurons.



\subsection{Briefly explain the concepts "information retrieval" and "information extraction" in language analysis (natural language processing)!}

\label{q:318:sa:en:True}

\textbf{Answer}: Information retrieval deals with methods for identifying documents that deal with a particular search query or topic. Information extraction deals with methods for extracting information that is useful for a particular application, e.g. extract phone number or last name.



\subsection{What three types of layers are there in a neural network's topology?}

\label{q:319:sa:en:True}

\textbf{Answer}: Input layer, hidden layer och output layer.



\subsection{What is a search tree in the context of AI?}

\label{q:320:sa:en:True}

\textbf{Answer}: A tree structure of nodes where each node represents a particular state and a solution is a path from the root node (representing the initial state) to a goal node (representing the desired state).



\subsection{In natural language processing, three different types of analysis are performed, which?}

\label{q:321:sa:en:True}

\textbf{Answer}: Syntactic analysis, semantic analysis and contextual analysis.



\subsection{What is the Turing test?}

\label{q:322:sa:en:True}

\textbf{Answer}: A test where a human interrogator must try to distinguish whether the test subject is human or machine by communicating via text messages.



\subsection{What is the definition of an intelligent agent (in AI)?}

\label{q:323:sa:en:True}

\textbf{Answer}: An autonomous target-directed unit that observes with sensors and acts on an environment using actuators.



\subsection{What characterizes supervised (machine) learning?}

\label{q:324:sa:en:True}

\textbf{Answer}: A person identifies the correct answer for a number of examples and the agent generalizes from these examples.



\subsection{What characterizes (machine) learning by reinforcement?}

\label{q:325:sa:en:True}

\textbf{Answer}: The agent is given a general rule to judge for itself when it has succeeded or failed.



\subsection{What is the definition of an intelligent agent?}

\label{q:326:sa:en:True}

\textbf{Answer}: An autonomous goal-directed entity which observes using sensors and acts upon an environment using actuators.



\subsection{What is the definition of an intelligent agent?}

\label{q:327:sa:en:True}

\textbf{Answer}: An autonomous goal-directed entity which observes using sensors and acts upon an environment using actuators (autonomous goal-directed entity which observes using sensors and acts upon an environment using actuators).



\subsection{What are search heuristics, and what characterizes good search heuristics?}

\label{q:328:sa:en:True}

\textbf{Answer}: Search heuristics are rules of thumb for reaching an overall search goal. A good heuristic is a sufficiently good estimate of the proximity to the search target and relatively easy to calculate.



\subsection{Why is search heuristics needed when searching in a search tree?}

\label{q:329:sa:en:True}

\textbf{Answer}: Because search trees for all interesting problems are so large that it is practically impossible to explore the entire search tree, and you therefore need rules of thumb to guide the search.



\subsection{What is the difference between a state graph and a search tree?}

\label{q:330:sa:en:True}

\textbf{Answer}: A state graph describes how to go between all possible different states, while a search tree describes the different possible search paths in a state graph to reach a goal state.



\subsection{What is the halting problem (stopp-problemet), and why is it interesting from a computational theoretical perspective?}

\label{q:331:sa:en:True}

\textbf{Answer}: The Halting Problem: Is it possible to determine in finite time with any program whether an arbitrary program will terminate for arbitrary input? The Stop problem is unsolvable, which shows that there are problems that cannot be solved (by algorithms).



\subsection{Arrange the following complexity/efficiency classes from most efficient to least efficient: \ensuremath{\Theta}(n^10), \ensuremath{\Theta}(log n), \ensuremath{\Theta}(n), \ensuremath{\Theta}(2^n).}

\label{q:332:sa:en:True}

\textbf{Answer}: \ensuremath{\Theta}(log n), \ensuremath{\Theta}(n), \ensuremath{\Theta}(n^10), \ensuremath{\Theta}(2^n)



\subsection{What is a Turing machine and what is its purpose?}

\label{q:333:sa:en:True}

\textbf{Answer}: A Turing machine is a mathematical model of a computer, and the purpose is to study which problems can be solved with a computer.



\subsection{Arrange the following complexity/efficiency classes from most efficient to least efficient: \ensuremath{\Theta}(n^4), \ensuremath{\Theta}(n), \ensuremath{\Theta}(2^n), \ensuremath{\Theta}(log n).}

\label{q:334:sa:en:True}

\textbf{Answer}: \ensuremath{\Theta}(log n), \ensuremath{\Theta}(n), \ensuremath{\Theta}(n^4), \ensuremath{\Theta}(2^n)



\subsection{What does it mean that a problem is a polynomial problem (belongs to the class of polynomial problems)?}

\label{q:335:sa:en:True}

\textbf{Answer}: That there is an algorithm to solve the problem within complexity class O(nx) for some x.



\subsection{Is the class of polynomial problems P less than or equal to the class of nondeterministic polynomial problems NP? Motivate your answer!}

\label{q:336:sa:en:True}

\textbf{Answer}: It is an open problem. No one has succeeded in showing either that P is less than NP, or that P is equal to NP.



\subsection{Given that the complexity of Algorithm A is O(n), Algorithm B is O(log n), Algorithm C is O(n2), and Algorithm D is O(n log n2), list the algorithms in order from most to least efficient efficient!}

\label{q:337:sa:en:True}

\textbf{Answer}: B, A, D, C.



\subsection{Give examples of three complexity classes in O-notation and order these from most efficient to least efficient!}

\label{q:338:sa:en:True}

\textbf{Answer}: For example: O(n), O(n2), O(2n).



\subsection{Why is the halting problem interesting from a computational theoretical perspective?}

\label{q:339:sa:en:True}

\textbf{Answer}: The stopping problem is unsolvable, which shows that there are problems that cannot be solved by algorithms/programs.



\subsection{What characterizes the halting problem (in theory of computation).}

\label{q:340:sa:en:True}

\textbf{Answer}: That it is not computable (algorithmically solvable).



\subsection{What does it mean that a function is computable?}

\label{q:341:sa:en:True}

\textbf{Answer}: That the function can be computed by some algorithm.



\subsection{Do non-deterministic algorithms meet the definition of an algorithm? Motivate your answer!}

\label{q:342:sa:en:True}

\textbf{Answer}: No, because sometimes the next step is not uniquely and completely determined by the current state.



\subsection{What do we know about the relation between polynomial problems P and non-deterministic polynomial problems NP?}

\label{q:343:sa:en:True}

\textbf{Answer}: All problems in P are also included in NP, but whether all problems in NP are also included in P is an open question.



\subsection{What distinguishes a deterministic algorithm from a non-deterministic one?}

\label{q:344:sa:en:True}

\textbf{Answer}: A deterministic algorithm always gives the same answer given a certain input. A non-deterministic algorithm can give different answers for the same input.



\subsection{What is the purpose of Turing machines?}

\label{q:345:sa:en:True}

\textbf{Answer}: It is a tool for studying the computing power of computers (algorithmic processing).



\subsection{In what two ways can a transaction be terminated?}

\label{q:346:sa:en:True}

\textbf{Answer}: A transaction that succeeds ends with a commit, and a transaction that fails ends with a rollback that undoes the transaction's work.



\subsection{Suppose we have the following bit patterns and that they represent integers in two's complement notation: "0111 1111, 1111 1001, 1011 1111, 0010 0100, 1000 0001" - Which of these bit patterns represents the smallest integer?}

\label{q:34800:mc:en:True}

\begin{itemize}
  \item[$\bigcirc$] 0111 1111
  \item[$\bigcirc$] 1111 1001
  \item[$\bigcirc$] 1011 1111
  \item[$\bigcirc$] 1000 0001
\end{itemize}

\subsection{Suppose we have the following bit patterns and that they represent integers in two's complement notation: "0111 1111, 1111 1001, 1011 1111, 0010 0100, 1000 0001" - Which of these bit patterns represents the largest integer?}

\label{q:3480001:mc:en:True}

\begin{itemize}
  \item[$\bigcirc$] 1111 1001
  \item[$\bigcirc$] 1011 1111
  \item[$\bigcirc$] 1000 0001
  \item[$\bigcirc$] 0010 0100
\end{itemize}



\subsection{Assume that 00FF00 is the hexadecimal notation for a bit pattern representing a pixel according to the RGB standard. - What color depth does this pixel have?}

\label{q:34900:sa:en:True}

\textbf{Answer}: 24 bits/pixel. _ Green

\subsection{Assume that 00FF00 is the hexadecimal notation for a bit pattern representing a pixel according to the RGB standard. - Which of the following colors is that pixel?}

\label{q:3490001:mc:en:True}

\begin{itemize}
  \item[$\bigcirc$] White
  \item[$\bigcirc$] Black
  \item[$\bigcirc$] Red
  \item[$\bigcirc$] Blue
  \item[$\bigcirc$] Yellow
  \item[$\bigcirc$] Cyan
  \item[$\bigcirc$] Magenta
\end{itemize}



\subsection{Suppose we have the following bit patterns and that they represent integers in two's complement notation:1111 1110 0111 1111 0000 0000 0000 0001 1000 0000 1111 1111 - Which of these bit patterns represents the number -1 (minus one)?}

\label{q:35000:mc:en:True}

\begin{itemize}
  \item[$\bigcirc$] 1111 1110
  \item[$\bigcirc$] 0111 1111
  \item[$\bigcirc$] 0000 0000
  \item[$\bigcirc$] 0000 0001
  \item[$\bigcirc$] 1000 0000
\end{itemize}

\subsection{Suppose we have the following bit patterns and that they represent integers in two's complement notation:1111 1110 0111 1111 0000 0000 0000 0001 1000 0000 1111 1111 - Which of these bit patterns represents the number 1 (one)?}

\label{q:3500001:mc:en:True}

\begin{itemize}
  \item[$\bigcirc$] 1111 1110
  \item[$\bigcirc$] 0111 1111
  \item[$\bigcirc$] 0000 0000
  \item[$\bigcirc$] 1000 0000
  \item[$\bigcirc$] 1111 1111
\end{itemize}



\subsection{Suppose we have the following bit patterns and that they represent integers in two's complement notation:1111 0100 0111 0101 0000 1010 0000 1011 1000 1010 1111 0101 - Which of these bit patterns represents the largest number?}

\label{q:35100:mc:en:True}

\begin{itemize}
  \item[$\bigcirc$] 0111 0011
  \item[$\bigcirc$] 0111 0001
  \item[$\bigcirc$] 0110 1111
\end{itemize}

\subsection{Suppose we have the following bit patterns and that they represent integers in two's complement notation:1111 0100 0111 0101 0000 1010 0000 1011 1000 1010 1111 0101 - Which of these bit patterns represents the smallest number?}

\label{q:3510001:mc:en:True}

\begin{itemize}
  \item[$\bigcirc$] 1000 1011
  \item[$\bigcirc$] 1001 0010
  \item[$\bigcirc$] 1011 1101
\end{itemize}



\subsection{Suppose we have the following bit pattern: 1000 0011. - What decimal natural number (zero or positive integer) (unsigned integer) does the bit pattern above represent?}

\label{q:35200:sa:en:True}

\textbf{Answer}: 131. _ -125.

\subsection{Suppose we have the following bit pattern: 1000 0011. - Which decimal integer (signed integer) represents the bit pattern above according to two's complement notation?}

\label{q:3520001:sa:en:True}

\vspace{2cm}

\noindent\makebox[\textwidth]{\hrulefill}

\vspace{1cm}

\textit{Answer}: \autoref{q:3520001:sa:en:True}



\subsection{Suppose we have the following bit patterns and that they represent integers in two's complement notation:0111 0100, 0010 1001, 1100 0010, 1100 0100, 0011 0001 - Which of these bit patterns represents the largest number?}

\label{q:35300:mc:en:True}

\begin{itemize}
  \item[$\bigcirc$] 0010 1001
  \item[$\bigcirc$] 1100 0010
  \item[$\bigcirc$] 1100 0100
  \item[$\bigcirc$] 0011 0001
\end{itemize}

\subsection{Suppose we have the following bit patterns and that they represent integers in two's complement notation:0111 0100, 0010 1001, 1100 0010, 1100 0100, 0011 0001 - Which of these bit patterns represents the smallest number?}

\label{q:3530001:mc:en:True}

\begin{itemize}
  \item[$\bigcirc$] 0010 1001
  \item[$\bigcirc$] 1100 0010
  \item[$\bigcirc$] 1100 0100
  \item[$\bigcirc$] 0011 0001
\end{itemize}



\subsection{Assume that the RGB color code of a pixel is CC3300 in hexadecimal (base 16) form, that the pixel is included in a photo taken with a 6 megapixel camera, and that the photo is stored as a bitmap (ie, uncompressed). - Enter the pixel color values for R (red), G (green) and B (blue) in decimal form (base 10)?}

\label{q:35400:sa:en:True}

\textbf{Answer}: CC16 =20410 (12*161 +12*160);3316 =5110 (3*161 +3*160);0016 =010 (0*161 +0*160) i.e. R = 204; G = 51; B = 0. _ CC3300 in hexadecimal form corresponds to the bit pattern 110011000011001100000000

\subsection{Assume that the RGB color code of a pixel is CC3300 in hexadecimal (base 16) form, that the pixel is included in a photo taken with a 6 megapixel camera, and that the photo is stored as a bitmap (ie, uncompressed). - What is the color depth of the pixel?}

\label{q:3540001:sa:en:True}

\textbf{Answer}:  which consists of 24 bits

\subsection{Assume that the RGB color code of a pixel is CC3300 in hexadecimal (base 16) form, that the pixel is included in a photo taken with a 6 megapixel camera, and that the photo is stored as a bitmap (ie, uncompressed). - How much storage space does the photo take in MB (megabytes)?}

\label{q:354000102:sa:en:True}

\textbf{Answer}:  i.e. the color depth of the pixel is 24 bits. _ 24 bits = 3 bytes; assume 1k = 1000; then 3 bytes/pixel * 6



\subsection{Suppose we have the following bit patterns: 1010 1010, 1100 1100, 1001 0000 and 1001 1111. - If the bit patterns above represent natural numbers (unsigned integers), then which bit pattern represents the smallest number?}

\label{q:35500:mc:en:True}

\begin{itemize}
  \item[$\bigcirc$] 1010 1010
  \item[$\bigcirc$] 1100 1100
  \item[$\bigcirc$] 1001 1111
\end{itemize}

\subsection{Suppose we have the following bit patterns: 1010 1010, 1100 1100, 1001 0000 and 1001 1111. - If the bit patterns above represent integers according to two's complement notation, which bit pattern represents the smallest number?}

\label{q:3550001:mc:en:True}

\begin{itemize}
  \item[$\bigcirc$] 1010 1010
  \item[$\bigcirc$] 1100 1100
  \item[$\bigcirc$] 1001 1111
\end{itemize}





\subsection{Write the decimal number 9 as a binary number represented by 8 bits (8 bit unsigned integer).}

\label{q:357:sa:en:True}

\textbf{Answer}: 00001001.



\subsection{Write the decimal number -1 (minus one) as a 8-bit bit pattern according to two{\textquoteright}s complement notation.}

\label{q:358:sa:en:True}

\textbf{Answer}: 11111111.



\subsection{Which bit pattern corresponds to the hexadecimal expression 7F?}

\label{q:359:sa:en:True}

\textbf{Answer}: 01111111.



\subsection{Write the decimal number 3 as a binary number represented by 8 bits (8 bit unsigned integer).}

\label{q:360:sa:en:True}

\textbf{Answer}: 00000011.



\subsection{What bit pattern corresponds to the hexidecimal expression AB?}

\label{q:361:sa:en:True}

\textbf{Answer}: 1010 1011.



\subsection{Which decimal number (base 10) corresponds to the hexadecimal number  A2?}

\label{q:362:sa:en:True}

\textbf{Answer}: 162.



\subsection{Write the number -3 (minus three) as an 8-bit bit pattern according to two's complement notation!}

\label{q:363:sa:en:True}

\textbf{Answer}: 11111101.



\subsection{Which bit pattern corresponds to the hexadecimal number 8F?}

\label{q:364:sa:en:True}

\textbf{Answer}: 1000 1111.



\subsection{Which decimal number (base 10) corresponds to the hexadecimal number  B3 ?}

\label{q:365:sa:en:True}

\textbf{Answer}: 179.



\subsection{Write the number 3 (three) as an 8-bit bit pattern according to two's complement notation!}

\label{q:366:sa:en:True}

\textbf{Answer}: 0000 0011.



\subsection{Describe the number 3 (three) with two characters in hexadecimal form!}

\label{q:367:sa:en:True}

\textbf{Answer}: 03.



\subsection{Write the number -4 (minus four) as an 8-bit bit pattern according to two's complement notation!}

\label{q:368:sa:en:True}

\textbf{Answer}: 11111100.



\subsection{Write the number 2 (two) as an 8-bit bit pattern according to two's complement notation!}

\label{q:369:sa:en:True}

\textbf{Answer}: 0000 0010.



\subsection{Write the number \ensuremath{-}2 (minus two) as an 8-bit bit pattern according to two's complement notation!}

\label{q:370:sa:en:True}

\textbf{Answer}: 1111 1110.



\subsection{What is the positive decimal integer 127 as a binary number represented by 8 bits according to two's complement notation?}

\label{q:371:sa:en:True}

\textbf{Answer}: 0111 1111.



\subsection{What is the negative decimal integer \ensuremath{-}127 as a binary number represented by 8 bits according to two's complement notation?}

\label{q:372:sa:en:True}

\textbf{Answer}: 1000 0001.



\subsection{Which natural decimal number (zero or positive integer) does the bit pattern 1010 1010 represent?}

\label{q:373:sa:en:True}

\textbf{Answer}: 170



\subsection{Which natural decimal number (zero or positive integer) does the bit pattern 1011 1011 represent?}

\label{q:374:sa:en:True}

\textbf{Answer}: 187



\subsection{Which bit pattern corresponds to the hexadecimal number C4?}

\label{q:375:sa:en:True}

\textbf{Answer}: 1100 0100



\subsection{Which bit pattern corresponds to the hexadecimal number B3?}

\label{q:376:sa:en:True}

\textbf{Answer}: 1011 0011



\subsection{Which decimal integer (signed integer) represents the bit pattern 1010 according to two{\textquoteright}s complement notation?}

\label{q:377:sa:en:True}

\textbf{Answer}: -6



\subsection{Which decimal integer (signed integer) represents the bit pattern 1011 according to two{\textquoteright}s complement notation?}

\label{q:378:sa:en:True}

\textbf{Answer}: -5



\subsection{Which natural decimal number (zero or positive integer) does the bit pattern 1101 1011 represent?}

\label{q:379:sa:en:True}

\textbf{Answer}: 219



\subsection{What bit pattern corresponds to the hexadecimal number D2?}

\label{q:380:sa:en:True}

\textbf{Answer}: 1101 0010.



\subsection{Which decimal integer (signed integer) represents the bit pattern 1101 according to two{\textquoteright}s complement notation?}

\label{q:381:sa:en:True}

\textbf{Answer}: -3



\subsection{Which bit pattern corresponds to the hexadecimal number A5 ?}

\label{q:382:sa:en:True}

\textbf{Answer}: 1010 0101.



\subsection{Which bit pattern corresponds to the hexadecimal number B4 ?}

\label{q:383:sa:en:True}

\textbf{Answer}: 1011 0100.



\subsection{Which decimal number (base 10) corresponds to the hexadecimal number 4D?}

\label{q:384:sa:en:True}

\textbf{Answer}: 77.



\subsection{Which decimal number (base 10) corresponds to the hexadecimal number 5C?}

\label{q:385:sa:en:True}

\textbf{Answer}: 92.



\subsection{Write the number 4 (four) as an 8-bit bit pattern according to two's complement notation!}

\label{q:386:sa:en:True}

\textbf{Answer}: 0000 0100.



\subsection{Write the number -5 (minus five) as an 8-bit bit pattern according to two's complement notation!}

\label{q:387:sa:en:True}

\textbf{Answer}: 1111 1011.



\subsection{Which bit pattern corresponds to the hexadecimal number C3?}

\label{q:388:sa:en:True}

\textbf{Answer}: 1100 0011.



\subsection{Which bit pattern corresponds to the hexadecimal number 3C?}

\label{q:389:sa:en:True}

\textbf{Answer}: 0011 1100.



\subsection{Which decimal number (base 10) corresponds to the hexadecimal number 3C?}

\label{q:390:sa:en:True}

\textbf{Answer}: 60



\subsection{Which decimal number (base 10) corresponds to the hexadecimal number C3?}

\label{q:391:sa:en:True}

\textbf{Answer}: 195



\subsection{Write the number -3 (minus three) as an 8-bit bit pattern according to two's complement notation!}

\label{q:392:sa:en:True}

\textbf{Answer}: 1111 1101.



\subsection{Write the number -4 (minus four) as an 8-bit bit pattern according to two's complement notation!}

\label{q:393:sa:en:True}

\textbf{Answer}: 1111 1100.



\subsection{What bit pattern corresponds to the hexadecimal number BE?}

\label{q:394:sa:en:True}

\textbf{Answer}: 1011 1110



\subsection{What decimal number (base 10) corresponds to the hexadecimal number 2D?}

\label{q:395:sa:en:True}

\textbf{Answer}: 45



\subsection{Which bit pattern corresponds to the hexadecimal number A2?}

\label{q:396:sa:en:True}

\textbf{Answer}: 1010 0010.



\subsection{Which decimal number (base 10) corresponds to the hexadecimal number D2?}

\label{q:397:sa:en:True}

\textbf{Answer}: 210.



\subsection{Which bit pattern corresponds to the hexadecimal number B1?}

\label{q:398:sa:en:True}

\textbf{Answer}: 1011 0001



\subsection{Which decimal number (base 10) corresponds to the hexadecimal number  5E ?}

\label{q:399:sa:en:True}

\textbf{Answer}: 0101 1110



\subsection{Which bit pattern corresponds to the hexadecimal number 7F?}

\label{q:400:sa:en:True}

\textbf{Answer}: 0111 1111.



\subsection{Which decimal number (base 10) corresponds to the hexadecimal number  A6?}

\label{q:401:sa:en:True}

\textbf{Answer}: 166



\subsection{Write the number -6 (minus six) as an 8-bit bit pattern according to two's complement notation!}

\label{q:402:sa:en:True}

\textbf{Answer}: 11111010.



\subsection{Suppose we have the following two bit patterns 10000001 and 01111110. What bit pattern do we obtain if we perform the logical AND operation on these bit patterns?}

\label{q:403:sa:en:True}

\textbf{Answer}: 00000000.



\subsection{Suppose we have the following two bit patterns 10000001 and 01111110. What bit pattern do we obtain if we perform the arithmetic operation ADD according to two's complement notation (two's complement notation) on these bit patterns which then represent two integers (signed integers)?}

\label{q:404:sa:en:True}

\textbf{Answer}: 11111111.



\subsection{What bit pattern do we get if we perform the OR operation on the bit patterns 1011 0011 and 0010 0110?}

\label{q:405:sa:en:True}

\textbf{Answer}: 1011 0111.



\subsection{What bit pattern do we get if we perform the operation XOR on the bit patterns 1011 0011 and 0010 0110?}

\label{q:406:sa:en:True}

\textbf{Answer}: 1001 0101.



\subsection{What bit pattern do we get if we perform the operation AND on the bit patterns 1001 1011 och 1000 1110?}

\label{q:407:sa:en:True}

\textbf{Answer}: 1000 1010.



\subsection{What bit pattern do we get if we perform the operation OR on the bit patterns 1001 1011 och 10001 110?}

\label{q:408:sa:en:True}

\textbf{Answer}: 1001 1111.



\subsection{What will be the result of the logical operation AND with the bit patterns 10100101 and 01111110? Enter the answer as a bit pattern.}

\label{q:409:sa:en:True}

\textbf{Answer}: 00100100



\subsection{What will be the result of the logical operation XOR with the bit patterns 10100101 and 01111110? Enter the answer as a bit pattern.}

\label{q:410:sa:en:True}

\textbf{Answer}: 1101 1011



\subsection{What will be the result of the logical operation XOR with the bit patterns 10100001 and 01101010? Enter the answer as a bit pattern.}

\label{q:411:sa:en:True}

\textbf{Answer}: 1100 1011.



\subsection{What bit pattern do we get if we perform the operation OR on the bit patterns 110011 and 101000?}

\label{q:412:sa:en:True}

\textbf{Answer}: 111011.



\subsection{What bit pattern do we get if we perform the operation XOR on the bit patterns 0110 0011 and 0101 0000?}

\label{q:413:sa:en:True}

\textbf{Answer}: 0011 0011.



\subsection{What bit pattern do we get if we perform the operation OR on the bit patterns 101011 and 010011?}

\label{q:414:sa:en:True}

\textbf{Answer}: 111011.



\subsection{What bit pattern do we get if we perform the operation AND on the bit patterns 110011 and 101001?}

\label{q:415:sa:en:True}

\textbf{Answer}: 100001.



\subsection{What bit pattern do we obtain if we perform the XOR operation on the bit patterns 110011 and 101001?}

\label{q:416:sa:en:True}

\textbf{Answer}: 0011 0010.



\subsection{What bit pattern do we get if we perform the operation AND on the bit patterns 1101 1101 and 1111 1001?}

\label{q:417:sa:en:True}

\textbf{Answer}: 1101 1001.



\subsection{What bit pattern do we get if we perform the operation XOR on the bit patterns 0101 0101 and 1000 1100?}

\label{q:418:sa:en:True}

\textbf{Answer}: 1101 1001.



\subsection{What bit pattern do we get if we perform the operation OR on the bit patterns 01001000 and 10011001?}

\label{q:419:sa:en:True}

\textbf{Answer}: 1101 1001.



\subsection{What is the minimum number of times the statements in a loop body are executed in an iteration with pre-test conditions?}

\label{q:420:sa:en:True}

\textbf{Answer}: 0



\subsection{Which bit pattern corresponds to the hexadecimal number B7?}

\label{q:421:sa:en:True}

\textbf{Answer}: 1011 0111



\subsection{Which bit pattern corresponds to the hexadecimal number C1?}

\label{q:422:sa:en:True}

\textbf{Answer}: 1100 0001



\subsection{Which bit pattern corresponds to the hexadecimal number E3?}

\label{q:423:sa:en:True}

\textbf{Answer}: 1110 0011



\subsection{Which hexadecimal number corresponds to the bit pattern 10010101?}

\label{q:424:sa:en:True}

\textbf{Answer}: 95



\subsection{The color magenta is a mixture of maximum red and maximum blue. Which bit pattern represents a magenta pixel encoded according to the RGB standard with a bit depth of 24 bits / pixel? Enter the answer in hexadecimal notation.}

\label{q:425:sa:en:True}

\textbf{Answer}: FF00FF



\subsection{The color yellow is a mixture of maximum red and maximum green. Which bit pattern represents a yellow pixel encoded according to the RGB standard with a bit depth of 24 bits / pixel? Enter the answer in hexadecimal notation.}

\label{q:426:sa:en:True}

\textbf{Answer}: FFFF00



\subsection{Which hexadecimal number corresponds to the bit pattern 11100101?}

\label{q:427:sa:en:True}

\textbf{Answer}: E5.



\subsection{Which hexadecimal number corresponds to the bit pattern 10101101?}

\label{q:428:sa:en:True}

\textbf{Answer}: AD



\subsection{Which hexadecimal number corresponds to the bit pattern 11010100?}

\label{q:429:sa:en:True}

\textbf{Answer}: D4



\subsection{What value will register 0 (R0) have after four (4) machine cycles? Enter the bit pattern in hexadecimal form.}

\label{q:430:sa:en:True}

\textbf{Answer}: 16.



\subsection{What value will register 1 (R1) have after four (4) machine cycles? Enter the bit pattern in hexadecimal form.}

\label{q:431:sa:en:True}

\textbf{Answer}: 0C.



\subsection{What value will register 2 (R2) have after four (4) machine cycles? Enter the bit pattern in hexadecimal form.}

\label{q:432:sa:en:True}

\textbf{Answer}: 08.



\subsection{What value will the program counter have after three (3) machine cycles? Enter the bit pattern in hexadecimal form.}

\label{q:433:sa:en:True}

\textbf{Answer}: 06



\subsection{Who wrote programs for "the Analytical Engine" and can thereby be regarded as the first programmer in the world?}

\label{q:434:mc:en:True}

\begin{itemize}
  \item[$\bigcirc$] Charles Babbage, Joseph Marie Jacquard, Alonzo Church, Kurt G\"odel, John von Neumann, Blaise Pascal, Alan Turing
\end{itemize}



\subsection{Who designed "the Analytical Engine" - the first programmable computational machine in the world?}

\label{q:435:mc:en:True}

\begin{itemize}
  \item[$\bigcirc$] Joseph Marie Jacquard, Ada Byron (Lovelace), Alonzo Church, Kurt G\"odel, John von Neumann, Blaise Pascal, Alan Turing
\end{itemize}



\subsection{Who was the first to use punch cards (used to store fabric patterns for automatic looms)?}

\label{q:436:mc:en:True}

\begin{itemize}
  \item[$\bigcirc$] Charles Babbage, Ada Byron (Lovelace), Alonzo Church, Kurt G\"odel, John von Neumann, Blaise Pascal, Alan Turing
\end{itemize}



\subsection{Who developed the first gear-based machine to perform addition?}

\label{q:437:mc:en:True}

\begin{itemize}
  \item[$\bigcirc$] Charles Babbage, Joseph Marie Jacquard, Ada Byron (Lovelace), Alonzo Church, Kurt G\"odel, John von Neumann, Alan Turing
\end{itemize}



\subsection{Who has published an incompleteness theorem that says that in all mathematical theories that encompass our traditional arithmetic system, there are statements whose truth or falsity cannot be determined using an algorithm?}

\label{q:438:mc:en:True}

\begin{itemize}
  \item[$\bigcirc$] Alan Turing, Blaise Pascal, Alonzo Church, Charles Babbage, Tim Berners-Lee, Ada Byron (Lovelace), Joseph Jacquard
\end{itemize}



\subsection{Who proposed a system by which documents stored on computers throughout the Internet could be linked together producing a web of linked information (the World Wide Web)?}

\label{q:439:mc:en:True}

\begin{itemize}
  \item[$\bigcirc$] Alan Turing, Blaise Pascal, Alonzo Church, Kurt G\"odel, Charles Babbage, Ada Byron (Lovelace), Joseph Jacquard
\end{itemize}



\subsection{Who gave rise to the name of the computer architecture where the CPU retrieves instructions from memory over a central bus?}

\label{q:440:mc:en:True}

\begin{itemize}
  \item[$\bigcirc$] Charles Babbage, Joseph Marie Jacquard, Ada Byron (Lovelace), Alonzo Church, Kurt G\"odel, Blaise Pascal, Alan Turing
\end{itemize}



\subsection{Who gave rise to the name of the mathematical model of a computer used in study of the power of algorithmic processing?}

\label{q:441:mc:en:True}

\begin{itemize}
  \item[$\bigcirc$] Charles Babbage, Joseph Marie Jacquard, Ada Byron (Lovelace), Alonzo Church, Kurt G\"odel, John von Neumann, Blaise Pascal
\end{itemize}



\subsection{The thesis that the functions that can be calculated by a Turing machine are the same as all computable functions, is named after Turing and another mathematician who contributed to the thesis, which one?  Charles Babbage, Joseph Marie Jacquard, Ada Byron (Lovelace), Alonzo Church, Kurt G\"odel, John von Neumann, Blaise Pascal, Tim Berners-Lee}

\label{q:442:mc:en:True}

\begin{itemize}
  \item[$\bigcirc$] Charles Babbage, Joseph Marie Jacquard, Ada Byron (Lovelace), Kurt G\"odel, John von Neumann, Blaise Pascal, Tim Berners-Lee
\end{itemize}



\subsection{Which bit pattern corresponds to the hexadecimal number D5?}

\label{q:443:mc:en:True}

\begin{itemize}
  \item[$\bigcirc$] 1010 0010, 1111 0101, 1101 0101
\end{itemize}



\subsection{Which hexadecimal number corresponds to the bit pattern 10001111?}

\label{q:444:mc:en:True}

\begin{itemize}
  \item[$\bigcirc$] 8F, 7F, 8C
\end{itemize}









\section{Final Words}
\label{finalWords}
If you have any questions or issues with the questions and/or answers presented in this document or the document itself, please contact the course administrations.

\pagebreak
\bibliographystyle{IEEEtran}
\bibliography{bibtex}
\bibdata{bibtex}

\end{sloppypar}
\end{document}
